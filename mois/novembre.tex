\dvmonth{Novembre}

%%%%%%%%%%%%%%%
% 1er novembre
%%%%%%%%%%%%%%%

\dvday{Le Mystère du Mal}

\dvquote{
Car déjà le mystère du mal est à l'œuvre ;
 il faut seulement que celui qui le retient maintenant ait disparu.
}{\bibleverse{IITh}(2:7) \NBS}

\dvlettrine{P}{aul a écrit aux Thessaloniciens} pour les assurer du fait
 qu'en dépit des persécutions dont ils étaient victimes,
 ils n'étaient pas entrés dans la période de la grande tribulation.
 Deux choses devront arriver avant que ce jour n'arrive~:
 une défection de la foi arrivera dans l'Église \suggest{Église}
 et l'homme de péché sera révélé.

L'attraction du mal avait déjà commencé à l'époque de Paul.
 Et elle n'a fait qu'augmenter. Pouquoi les gens sont-ils attirés
 par le mal alors qu'il amène toujours au désastre ?
 C'est bien là le mystère.

\dvbox{
Chaque jour des hommes rejettent Jésus-Christ
 \ocadr le seul vrai chemin vers le ciel. 
}

Il est inconcevable que des gens puissent délibérément choisir le mal
 plutôt que le bien \ocadr même s'ils savent que leur choix peut les tuer.
 Pourtant c'est ce qu'ils le font.
 Ça a commencé dans le jardin d'Eden, quand Adam et Eve ont choisi de manger
 de l'arbre dont Dieu avait annoncé qu'il les tuerait.
 C'est un mystère~: les hommes choisissent une chemin qui mène à la mort
 plutôt que le chemin qui conduit à la vie
 \ocadr un chemin qui ne va leur amener que malheur et destruction.
 Pourtant c'est ce qu'ils font.

Le Seigneur lance l'invitation, dans \bibleverse{Is}(1:18),
 à \og venir et raisonner ensemble. \fg{} L'obéissance est raisonnable.
 Accepter le salut que Jésus offre est raisonnable.
 Pourtant les hommes continuent  de jeter la raison aux orties,
 tout cela parce qu'ils aiment les ténèbres plus que la lumière,
 tout cela parce que le mal est pour eux plus attirant que l'obéissance.

Sondez votre cœur aujourd'hui. Demandez à Dieu de vous montrer
 si vous n'êtes pas en train de choisir le mal à l'obéissance.
 Et si vous trouvez que c'est le cas, repentez-vous!

\dvrule

\dvprayer{
Père, aide-nous à prendre en compte Ta Parole
 afin que nous ne soyons pas trompés.
 Puissions-nous remettre nos cœurs et nos vies à Jésus-Christ,
 le Seigneur vivant. 
}{\Amen}


%%%%%%%%%%%%%%%
% 2 novembre
%%%%%%%%%%%%%%%

\dvday{Sa Mission de Salut}

\suggest{Salut avec une majuscule}

\dvquote{
C'est une parole certaine et digne d'être entièrement reçue,
 que le Christ-Jésus est venu dans le monde
 pour sauver les pécheurs, dont je suis, moi,
 le premier.
}{\bibleverse{ITim}(1:15)}

\dvlettrine{U}{n jour où Jésus mangeait} avec des collecteurs d'impôts
 et d'autres pécheurs, Il entendit les Scribes et les Pharisiens
 critiquer le choix de ses compagnons de table.
 \og Jésus, qui avait entendu, leur dit~:
 Ce ne sont pas les bien-portants qui ont besoin de médecin,
 mais les malades. Je ne suis pas venu appeler des justes,
 mais des pécheurs \fg{} \punct{Point en trop}
 (\bibleverse{Mc}(2:17)).

\dvbox{
Nul n'est hors de la portée de l'amour de Dieu. 
}

Si vous vous portez bien, vous évitez le docteur.
 Mais quand vous êtes malades, vous savez que vous avez besoin d'aide.
 L'homme était malade. Il avait choisi un chemin mortel,
 un chemin qui mène à la maladie, la misère et la destruction.
 Dieu a vu la situation désespérée de l'homme et a eu pitié de lui
 en envoyant Son Fils, le Grand Médecin.
 Il n'a pas envoyé Jésus pour condamner le monde.
 Cela n'était pas nécessaire \typo{nécessaire},
 car le monde était déjà condamné. Il a envoyé Jésus
 \og pour que le monde soit sauvé par Lui \fg{} \punct{Point en trop}
 (\bibleverse{Jn}(3:17)).

Notez que Paul se désigne lui-même comme le premier des pécheurs.
 Il savait qui il était. Paul ne cherchait pas à excuser son passé.
 Le message qu'il voulait faire passer, c'est que si Jésus pouvait le sauver
 \ocadr après qu'il ait blasphémé Jésus, approuvé les crimes
 contre les Chrétiens, allant même jusqu'à les pourchasser
 et les tuer \fcadr{} alors le reste d'entre nous n'avons aucun souci
 à nous faire. Jésus va nous pardonner et nous sauver aussi.
 Celui qui est descendu au plus bas pour sauver les pires transgresseurs
 peut venir vous toucher dans la fosse de votre péché
 et vous amener dans Sa lumière.
 Nul n'est hors de la portée de l'amour de Dieu.

\dvrule

\dvprayer{
Père, merci d'avoir envoyé Ton Fils dans le monde
 pour pardonner et sauver ceux qui sont malades du péché. 
}{\DlNdJ}


%%%%%%%%%%%%%%%
% 3 novembre
%%%%%%%%%%%%%%%

\dvday{Un Seul Médiateur}

\dvquote{
Car il y a un seul Dieu, et aussi un seul médiateur entre Dieu,
 et les hommes, le Christ-Jésus homme,
 qui s'est donné lui-même en rançon pour tous~:
 c'est le témoignage rendu en temps voulu\dots{}
}{\bibleverse{ITim}(2:5-6)}

\dvlettrine{C}{'est à la fois arrogant et osé pour l'homme,}
 être limité et pécheur, que de penser qu'il peut se présenter
 effrontément devant le Dieu éternel et saint.
 Il est infiniment pur; nous sommes entachés par le péché.
 Il est lumière; nous demeurons dans les ténèbres.
 Aussi, comment l'homme pécheur peut-il jamais espérer
 se tenir debout devant Dieu?

Il existe une solution, mais une seule seulement.
 Job avait réclamé un pont, un Médiateur qui puisse poser Sa main
 à la fois sur Dieu et sur l'homme. Et nous avons un Médiateur en Jésus.
 Celui qui était Dieu est devenu homme afin qu'Il puisse se tenir
 sur la brèche entre Dieu et nous.

\dvbox{
Aucun saint ne peut vous sauver \ocadr seul Jésus peut vous sauver. 
}

Aucun saint ne peut se tenir sur la brèche pour vous.
 La Vierge Marie ne peut pas le faire pour vous.
 Peu importe combien cette personne a été sainte ou vertueuse,
 ce n'était qu'un homme; ce n'était qu'une femme.
 Vous pouvez adresser des prières à un saint mort toute la journée,
 ce saint restera totalement incapable de bâtir un pont sur le précipice
 qui vous sépare de Dieu. Seul Jésus peut le faire pour vous,
 parce que Lui seul peut à la fois vous toucher et toucher Dieu.
 Il est le seul vrai Médiateur que Dieu a envoyé pour nous amener à Lui.

Par la mort et la résurrection de Jésus-Christ, un chemin a été ouvert
 qu'il nous est possible d'emprunter pour nous approcher du Père infini,
 éternel, saint, pur et majestueux. \suggest{Y a-t-il un nouveau paragraphe ici?}
Nous nous tenons justes devant Lui grâce à la justice de notre Médiateur.

Louange à Dieu pour Son Fils!

\dvrule

\dvprayer{
Père, merci d'avoir offert ce que Job réclamait,
 un médiateur qui puisse nous toucher tous les deux.
Merci pour l'espérance de la vie éternelle. 
}{\DlNdJ}


%%%%%%%%%%%%%%%
% 4 novembre
%%%%%%%%%%%%%%%

\dvday{Le Mystère}

\dvquote{
\dots{} le mystère de la piété est grand~:
 Celui qui a été manifesté en chair, justifié en Esprit,
 est apparu aux anges, a été prêché parmi les nations,
 a été cru dans le monde, a été élevé dans la gloire.
}{\bibleverse{ITim}(3:16)}

\punct{Point et espace manquants}

\dvlettrine{I}{l existe un mystère formidable} relatif à la piété
 \ocadr le fait que l'homme devient comme son dieu.
 Parlant des idoles muettes, sourdes, aveugles, immobiles des païens,
 le psalmiste déclare~: \punct{deux-points}
 \og Ils leur ressemblent, ceux qui les fabriquent,
 Tous ceux qui se confient en elles \fg{} (\bibleverse{Ps}(115:8)).
 C'est une vérité psychologique de base;
 un homme devient comme son dieu.
 Si votre dieu est haineux et amer, alors vous devenez haineux et amer.

\dvbox{
Nous, qui suivons Dieu, sommes transformés de jour en jour
 pour devenir comme Lui. 
}

La Bible nous dit que \og nous \grammar{Pas besoin de majuscule, la phrase continue}
 sommes maintenant enfants de Dieu, et ce que nous serons n'a pas encore
 été manifesté ; mais nous savons que lorsqu'Il sera manifesté,
 nous serons semblables à lui, parce que nous le verrons
 tel qu'Il est \fg{} (\bibleverse{IJn}(3:2)).

Comme j'aime voir ce mystère se dévoiler dans ma propre vie,
 pour voir les changements qu'Il effectue jour après jour
 alors que je marche à Sa suite en Le servant.

David a dit~: \punct{deux-points}
 \og Dès le réveil, je me rassasierai de Ton image. \fg{}
 Et un jour je m'éveillerai et je me retrouverai au ciel
 et je serai juste comme Lui. C'est le mystère de la piété~:
 la vie transformée conformée à Son image.

\dvrule

\dvprayer{
Père, Merci à Toi pour la puissance de Ton Esprit à transformer
 la vie d'une personne et de la conformer à Ton image.
 Sois à l'œuvre dans nos cœurs et nos vies aujourd'hui.
 Aide-nous à céder sous Tes mains,
 sans jamais résister à ce que Tu veux faire en nous.
}{\DlNdJ}


%%%%%%%%%%%%%%%
% 5 novembre
%%%%%%%%%%%%%%%

\dvday{Sois un Modèle}

\dvquote{
Que personne ne méprise ta jeunesse ;
 mais sois un modèle pour les fidèles, en parole, en conduite, en esprit,
 en amour, en foi, en pureté.
}{\bibleverse{ITim}(4:12)}

\dvlettrine{P}{arce que Timothée était jeune,} certains dans l'église
 le regardaient de haut et refusaient de recevoir de lui.
 Aussi Paul a t-il écrit à Timothée en lui disant~: \punct{deux-points}
 \og C'est simple, il te suffit d'être un modèle. \fg{}
 Paul a souligné six domaines dans lesquels Timothée
 devait être un modèle pour les fidèles~:

\suggest{Je serais curieux de voir la mise en page en anglais}

\emph{En parole} \ocadr
 Ceci pouvait être compris de deux façons.
 D'abord, Paul a pu vouloir dire~: \punct{deux-points}
 \og Sois un modèle dans ton langage. \fg{}
 Mais il aurait aussi pu vouloir dire~: \punct{deux-points}
 \og Sois un modèle dans ta connaissance et ta compréhension des Écritures
 \ocadr sois un homme de la Parole. \fg{} Les deux sont importants.

\emph{En conduite} \ocadr
 Que ton style de vie sois un exemple de ce qu'est un croyant.
 Sois un modèle de Christ dans tes actions et tes attitudes.

\emph{En amour} \ocadr
 L'amour que Paul a décrit dans \bibleverse{ICo}(13:) \typo{s à Corinthiens}
 est l'amour qui devrait émaner de la vie de chaque croyant.

\emph{En esprit} \ocadr
 Certaines personnes ont un esprit doux; d'autres ont un esprit mesquin.
 Il n'y a pas de place pour la mesquinerie parmi les croyants.

\emph{En foi} \ocadr
 Ceci peut aussi vouloir dire l'une de deux choses.
 Ou bien nous devons être un modèle dans notre confiance en Dieu,
 ou dans notre propre capacité à être dignes de confiance, ou les deux.

\emph{En pureté} \ocadr
 Timothée était jeune, non marié, et vivait dans une société païenne
 et corrompue. Paul l'exhortait à vivre une vie de pureté,
 une vie au-dessus de tout reproche.

\dvbox{
Le message de Paul est pour tous les âges.
 Nous devons être des modeles de Christ à la fois pour l'Église
 et pour un monde qui s'interroge sur le christianisme. 
}

\dvrule

\dvprayer{
Père, aide-nous à être un exemple digne de Toi dans le monde,
 afin que nous puissions Te porter gloire.
}{\Amen}


