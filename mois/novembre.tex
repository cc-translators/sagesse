\dvmonth{Novembre}

%%%%%%%%%%%%%%%
% 1er novembre
%%%%%%%%%%%%%%%

\dvday{Le Mystère du Mal}

\dvquote{
Car déjà le mystère du mal est à l'œuvre ;
 il faut seulement que celui qui le retient maintenant ait disparu.
}{\ibibleverse{IITh}(2:7) \NBS}

\lettrine{P}{aul a écrit aux Thessaloniciens} pour les assurer du fait
 qu'en dépit des persécutions dont ils étaient victimes,
 ils n'étaient pas entrés dans la période de la grande tribulation.
 Deux choses devront arriver avant que ce jour n'arrive~:
 une défection de la foi arrivera dans l'Église \suggest{Église}
 et l'homme de péché sera révélé.

L'attraction du mal avait déjà commencé à l'époque de Paul.
 Et elle n'a fait qu'augmenter. Pourquoi \typo{Pourquoi}
 les gens sont-ils attirés
 par le mal alors qu'il amène toujours au désastre ?
 C'est bien là le mystère.

\dvbox{
Chaque jour des hommes rejettent Jésus-Christ
 \ocadr le seul vrai chemin vers le ciel. 
}

Il est inconcevable que des gens puissent délibérément choisir le mal
 plutôt que le bien \ocadr même s'ils savent que leur choix peut les tuer.
 Pourtant c'est ce qu'ils le font.
 Ça a commencé dans le jardin d'Éden, quand Adam et Ève ont choisi de manger
 de l'arbre dont Dieu avait annoncé qu'il les tuerait.
 C'est un mystère~: les hommes choisissent une chemin qui mène à la mort
 plutôt que le chemin qui conduit à la vie
 \ocadr un chemin qui ne va leur amener que malheur et destruction.
 Pourtant c'est ce qu'ils font.

Le Seigneur lance l'invitation, dans \ibibleverse{Is}(1:18),
 à \og venir et raisonner ensemble. \fg{} L'obéissance est raisonnable.
 Accepter le salut que Jésus offre est raisonnable.
 Pourtant les hommes continuent  de jeter la raison aux orties,
 tout cela parce qu'ils aiment les ténèbres plus que la lumière,
 tout cela parce que le mal est pour eux plus attirant que l'obéissance.

Sondez votre cœur aujourd'hui. Demandez à Dieu de vous montrer
 si vous n'êtes pas en train de choisir le mal à l'obéissance.
 Et si vous trouvez que c'est le cas, repentez-vous!

\dvrule

\dvprayer{
Père, aide-nous à prendre en compte Ta Parole
 afin que nous ne soyons pas trompés.
 Puissions-nous remettre nos cœurs et nos vies à Jésus-Christ,
 le Seigneur vivant. 
}{\Amen}


%%%%%%%%%%%%%%%
% 2 novembre
%%%%%%%%%%%%%%%

\dvday{Sa Mission de Salut}

\suggest{Salut avec une majuscule}

\dvquote{
C'est une parole certaine et digne d'être entièrement reçue,
 que le Christ-Jésus est venu dans le monde
 pour sauver les pécheurs, dont je suis, moi,
 le premier.
}{\ibibleverse{ITm}(1:15)}

\lettrine{U}{n jour où Jésus mangeait} avec des collecteurs d'impôts
 et d'autres pécheurs, Il entendit les Scribes et les Pharisiens
 critiquer le choix de ses compagnons de table.
 \og Jésus, qui avait entendu, leur dit~:
 Ce ne sont pas les bien-portants qui ont besoin de médecin,
 mais les malades. Je ne suis pas venu appeler des justes,
 mais des pécheurs \fg{} \punct{Point en trop}
 (\ibibleverse{Mc}(2:17)).

\dvbox{
Nul n'est hors de la portée de l'amour de Dieu. 
}

Si vous vous portez bien, vous évitez le docteur.
 Mais quand vous êtes malades, vous savez que vous avez besoin d'aide.
 L'homme était malade. Il avait choisi un chemin mortel,
 un chemin qui mène à la maladie, la misère et la destruction.
 Dieu a vu la situation désespérée de l'homme et a eu pitié de lui
 en envoyant Son Fils, le Grand Médecin.
 Il n'a pas envoyé Jésus pour condamner le monde.
 Cela n'était pas nécessaire \typo{nécessaire},
 car le monde était déjà condamné. Il a envoyé Jésus
 \og pour que le monde soit sauvé par Lui \fg{} \punct{Point en trop}
 (\ibibleverse{Jn}(3:17)).

Notez que Paul se désigne lui-même comme le premier des pécheurs.
 Il savait qui il était. Paul ne cherchait pas à excuser son passé.
 Le message qu'il voulait faire passer, c'est que si Jésus pouvait le sauver
 \ocadr après qu'il ait blasphémé Jésus, approuvé les crimes
 contre les Chrétiens, allant même jusqu'à les pourchasser
 et les tuer \fcadr{} alors le reste d'entre nous n'avons aucun souci
 à nous faire. Jésus va nous pardonner et nous sauver aussi.
 Celui qui est descendu au plus bas pour sauver les pires transgresseurs
 peut venir vous toucher dans la fosse de votre péché
 et vous amener dans Sa lumière.
 Nul n'est hors de la portée de l'amour de Dieu.

\dvrule

\dvprayer{
Père, merci d'avoir envoyé Ton Fils dans le monde
 pour pardonner et sauver ceux qui sont malades du péché. 
}{\DlNdJ}


%%%%%%%%%%%%%%%
% 3 novembre
%%%%%%%%%%%%%%%

\dvday{Un Seul Médiateur}

\dvquote{
Car il y a un seul Dieu, et aussi un seul médiateur entre Dieu,
 et les hommes, le Christ-Jésus homme,
 qui s'est donné lui-même en rançon pour tous~:
 c'est le témoignage rendu en temps voulu\dots{}
}{\ibibleverse{ITm}(2:5-6)}

\lettrine{C}{'est à la fois arrogant et osé pour l'homme,}
 être limité et pécheur, que de penser qu'il peut se présenter
 effrontément devant le Dieu éternel et saint.
 Il est infiniment pur; nous sommes entachés par le péché.
 Il est lumière; nous demeurons dans les ténèbres.
 Aussi, comment l'homme pécheur peut-il jamais espérer
 se tenir debout devant Dieu?

Il existe une solution, mais une seule seulement.
 Job avait réclamé un pont, un Médiateur qui puisse poser Sa main
 à la fois sur Dieu et sur l'homme. Et nous avons un Médiateur en Jésus.
 Celui qui était Dieu est devenu homme afin qu'Il puisse se tenir
 sur la brèche entre Dieu et nous.

\dvbox{
Aucun saint ne peut vous sauver \ocadr seul Jésus peut vous sauver. 
}

Aucun saint ne peut se tenir sur la brèche pour vous.
 La Vierge Marie ne peut pas le faire pour vous.
 Peu importe combien cette personne a été sainte ou vertueuse,
 ce n'était qu'un homme; ce n'était qu'une femme.
 Vous pouvez adresser des prières à un saint mort toute la journée,
 ce saint restera totalement incapable de bâtir un pont sur le précipice
 qui vous sépare de Dieu. Seul Jésus peut le faire pour vous,
 parce que Lui seul peut à la fois vous toucher et toucher Dieu.
 Il est le seul vrai Médiateur que Dieu a envoyé pour nous amener à Lui.

Par la mort et la résurrection de Jésus-Christ, un chemin a été ouvert
 qu'il nous est possible d'emprunter pour nous approcher du Père infini,
 éternel, saint, pur et majestueux. \suggest{Y a-t-il un nouveau paragraphe ici?}
Nous nous tenons justes devant Lui grâce à la justice de notre Médiateur.

Louange à Dieu pour Son Fils!

\dvrule

\dvprayer{
Père, merci d'avoir offert ce que Job réclamait,
 un médiateur qui puisse nous toucher tous les deux.
Merci pour l'espérance de la vie éternelle. 
}{\DlNdJ}


%%%%%%%%%%%%%%%
% 4 novembre
%%%%%%%%%%%%%%%

\dvday{Le Mystère}

\dvquote{
\dots{} le mystère de la piété est grand~:
 Celui qui a été manifesté en chair, justifié en Esprit,
 est apparu aux anges, a été prêché parmi les nations,
 a été cru dans le monde, a été élevé dans la gloire.
}{\ibibleverse{ITm}(3:16)}

\punct{Point et espace manquants}

\lettrine{I}{l existe un mystère formidable} relatif à la piété
 \ocadr le fait que l'homme devient comme son dieu.
 Parlant des idoles muettes, sourdes, aveugles, immobiles des païens,
 le psalmiste déclare~: \punct{deux-points}
 \og Ils leur ressemblent, ceux qui les fabriquent,
 Tous ceux qui se confient en elles \fg{} (\ibibleverse{Ps}(115:8)).
 C'est une vérité psychologique de base;
 un homme devient comme son dieu.
 Si votre dieu est haineux et amer, alors vous devenez haineux et amer.

\dvbox{
Nous, qui suivons Dieu, sommes transformés de jour en jour
 pour devenir comme Lui. 
}

La Bible nous dit que \og nous \grammar{Pas besoin de majuscule, la phrase continue}
 sommes maintenant enfants de Dieu, et ce que nous serons n'a pas encore
 été manifesté ; mais nous savons que lorsqu'Il sera manifesté,
 nous serons semblables à lui, parce que nous le verrons
 tel qu'Il est \fg{} (\ibibleverse{IJn}(3:2)).

Comme j'aime voir ce mystère se dévoiler dans ma propre vie,
 pour voir les changements qu'Il effectue jour après jour
 alors que je marche à Sa suite en Le servant.

David a dit~: \punct{deux-points}
 \og Dès le réveil, je me rassasierai de Ton image. \fg{}
 Et un jour je m'éveillerai et je me retrouverai au ciel
 et je serai juste comme Lui. C'est le mystère de la piété~:
 la vie transformée conformée à Son image.

\dvrule

\dvprayer{
Père, Merci à Toi pour la puissance de Ton Esprit à transformer
 la vie d'une personne et de la conformer à Ton image.
 Sois à l'œuvre dans nos cœurs et nos vies aujourd'hui.
 Aide-nous à céder sous Tes mains,
 sans jamais résister à ce que Tu veux faire en nous.
}{\DlNdJ}


%%%%%%%%%%%%%%%
% 5 novembre
%%%%%%%%%%%%%%%

\dvday{Sois un Modèle}

\dvquote{
Que personne ne méprise ta jeunesse ;
 mais sois un modèle pour les fidèles, en parole, en conduite, en esprit,
 en amour, en foi, en pureté.
}{\ibibleverse{ITm}(4:12)}

\lettrine{P}{arce que Timothée était jeune,} certains dans l'église
 le regardaient de haut et refusaient de recevoir de lui.
 Aussi Paul a t-il écrit à Timothée en lui disant~: \punct{deux-points}
 \og C'est simple, il te suffit d'être un modèle. \fg{}
 Paul a souligné six domaines dans lesquels Timothée
 devait être un modèle pour les fidèles~:

\suggest{Je serais curieux de voir la mise en page en anglais}

\emph{En parole} \ocadr
 Ceci pouvait être compris de deux façons.
 D'abord, Paul a pu vouloir dire~: \punct{deux-points}
 \og Sois un modèle dans ton langage. \fg{}
 Mais il aurait aussi pu vouloir dire~: \punct{deux-points}
 \og Sois un modèle dans ta connaissance et ta compréhension des Écritures
 \ocadr sois un homme de la Parole. \fg{} Les deux sont importants.

\emph{En conduite} \ocadr
 Que ton style de vie sois un exemple de ce qu'est un croyant.
 Sois un modèle de Christ dans tes actions et tes attitudes.

\emph{En amour} \ocadr
 L'amour que Paul a décrit dans \ibibleverse{ICo}(13:) \typo{s à Corinthiens}
 est l'amour qui devrait émaner de la vie de chaque croyant.

\emph{En esprit} \ocadr
 Certaines personnes ont un esprit doux; d'autres ont un esprit mesquin.
 Il n'y a pas de place pour la mesquinerie parmi les croyants.

\emph{En foi} \ocadr
 Ceci peut aussi vouloir dire l'une de deux choses.
 Ou bien nous devons être un modèle dans notre confiance en Dieu,
 ou dans notre propre capacité à être dignes de confiance, ou les deux.

\emph{En pureté} \ocadr
 Timothée était jeune, non marié, et vivait dans une société païenne
 et corrompue. Paul l'exhortait à vivre une vie de pureté,
 une vie au-dessus de tout reproche.

\dvbox{
Le message de Paul est pour tous les âges.
 Nous devons être des modèles de Christ à la fois pour l'Église
 et pour un monde qui s'interroge sur le christianisme. 
}

\typo{modèles}

\dvrule

\dvprayer{
Père, aide-nous à être un exemple digne de Toi dans le monde,
 afin que nous puissions Te porter gloire.
}{\Amen}


%%%%%%%%%%%%%%%
% 6 novembre
%%%%%%%%%%%%%%%

\dvday{Le Péché du Préjugé}

\dvquote{
Je te conjure devant Dieu, devant le Christ-Jésus et devant les anges élus,
 d'observer ces règles sans préjugé et de ne rien faire par favoritisme.
}{\ibibleverse{ITm}(5:21)}

\lettrine{P}{aul a donné des instructions à Timothée}
 sur des sujets tels que, la prise en charge des veuves,
 sa relation avec les anciens de l'église, des règles pour les jeunes femmes,
 et ainsi de suite. Puis il dit~: \punct{deux-points}
 \og Timothée, \punct{virgule}
 tu ne dois pas montrer de traitement préférentiel ou de partialité.
 Sois bien sûr d'observer ces règles sans préférence pour l'un plutôt
 que pour l'autre. \fg{}

Dans toute la Bible, il nous est dit que Dieu est impartial.
 Mais malheureusement, nous, nous faisons souvent preuve de partialité.
 Nous sommes enclins à honorer le riche et à plutôt escamoter le pauvre,
 mais Dieu n'est pas comme ça. Il est tout aussi soucieux de sauver l'âme
 du plus pauvre des hommes à la surface de la terre qu'Il l'est de l'homme
 le plus riche. La situation sociale ne signifie rien pour Dieu,
 nous existons tous au même niveau et au même plan. 

\dvbox{
Si nous nous réclamons de Son Nom, nous devons développer
 Sa perspective sur les préjugés.
}

Peu importe combien de succès vous avez remportés, combien de richesses
 vous avez accumulées, ou qui vous êtes en termes de classement par le monde.
 Peu importe que vous habitiez dans un château ou dans une cabane.
 Indépendamment de la façon dont vous êtes vus par tout le monde,
 vous comptez pour Jésus. 

Nous devons trouver une façon de gommer les différences entre nous.
 Comment pouvons-nous atteindre un monde perdu si nous ne voyons pas
 ceux qui le composent de la façon dont Jésus les voit?
 Nous devons apprendre à aimer comme Il a aimé, et à apprécier les autres
 parce qu'ils comptent tant pour Lui.

\dvrule

\dvprayer{
Père, Merci pour Ton grand amour pour nous.
 Aide-nous à aimer avec Ton amour
 \ocadr sans préférence ni partialité.
}{\DlNdJ}

\punct{tiret dans « Aide-nous »}


%%%%%%%%%%%%%%%
% 7 novembre
%%%%%%%%%%%%%%%

\dvday{Les Vraies Richesses}

\dvquote{
Certes, c'est une grande source de gain que la piété,
 si l'on se contente de ce qu'on a.
}{\ibibleverse{ITm}(6:6)}

\lettrine{L}{es gens disent souvent~:} \punct{deux-points}
 \og Oh si seulement je pouvais gagner au loto! \fg{}
 Mais vous risquez d'être surpris. Des études ont montré
 que très fréquemment, ces gros lots fabuleux gâchent complètement
 la vie des gagnants. Ils passent d'une vie d'insouciance
 à une vie remplie d'anxiété. Bien qu'ils aient pensé
 que l'argent allait leur apporter un grand bonheur,
 dans beaucoup de cas il ne leur apporte que du chagrin.

Les gens se serrent la ceinture et suent à grosses gouttes
 pour amasser leurs richesses, mais au jour du jugement
 ces richesses n'auront aucune valeur, parce qu'il n'est pas possible
 d'acheter Dieu. Les vraies richesses sont, par contre, éternelles.
 Elles durent pour toujours; rien ne peut les diminuer.
 Vous pouvez être pauvres dans ce monde et être, cependant,
 un héritier dans le royaume de Dieu.

\dvbox{
Les richesses terrestres constituent une complète imposture. 
}

Jérémie a dit~: \punct{deux-points}
 \og Ainsi parle l'Éternel~: Que le sage ne se glorifie pas de sa sagesse,
 que le fort ne se glorifie pas de sa force, que le riche ne se glorifie pas
 de sa richesse, mais que celui qui veut se glorifier se glorifie d'avoir
 de l'intelligence et de Me connaître, de savoir que je suis l'Éternel,
 qui exerce la bienveillance, le droit et la justice sur la terre;
 Car c'est à cela que je prends plaisir \fg{} \punct{Point après la référence}
 (\ibibleverse{Jr}(9:22-23)).

\og J'ai appris à me contenter de l'état où je me trouve, disait Paul.
 \punct{Pas besoin de répéter les guillemets ici}
 Je sais vivre dans l'humiliation, et je sais vivre dans l'abondance.
 En tout et partout, j'ai appris à être rassasié et à avoir faim,
 à être dans l'abondance et à être dans la disette \fg{} \punct{Point après la référence}
 (\ibibleverse{Ph}(4:11-12)).

Au lieu de nous évertuer à accumuler toujours davantage, apprenons la valeur
 de savoir se contenter de ce qu'on a. 

\dvrule

\dvprayer{
Père, nous Te remercions de nous avoir offert des richesses aussi vastes.
 Apprends-nous à être satisfaits de ce que Tu nous a donné.
}{\DlNdJ}


%%%%%%%%%%%%%%%
% 8 novembre
%%%%%%%%%%%%%%%

\dvday{La Mort est Abolie}

\dvquote{
\dots{} notre Sauveur Christ-Jésus, qui a réduit à l'impuissance la mort
 et mis en lumière la vie et l'incorruptibilité par l'Évangile\dots{}
}{\ibibleverse{IITm}(1:10)}

\lettrine{U}{ne chose que vous pouvez dire sur la mort,}
 c'est qu'elle est certaine.
 Jusqu'à présent les statistiques sont impressionnantes~:
 cent pour cent des gens meurent. Mais voici un phénomène intéressant~:
 si vous êtes nés une fois, vous mourrez deux fois;
 si vous êtes nés deux fois, vous ne mourrez qu'une seule fois.

À notre mort physique, nous serons transformés.
 Nous subirons une métamorphose. C'est nécessaire, parce qu'en ce moment
 nous existons dans des corps corruptibles. Mais pour exister au ciel,
 nous avons besoin de corps incorruptibles. Et quand ces tentes
 seront dissoutes, quand nos corps retourneront à la poussière,
 nous entrerons dans notre existence éternelle avec Dieu.
 Quelle glorieuse promesse ! Quelle espérance bénie ! 

\dvbox{
Notre éternité avec Dieu~: quelle glorieuse promesse ! 
}

Un de ces jours, il se peut qu'en prenant le journal,
 vous lisiez~: \punct{deux-points}
 \og Chuck Smith, pasteur de Calvary Chapel est mort. \fg{}
 N'allez surtout pas en croire un mot ! Si vous voyez ça,
 sachez avec certitude que je ne serai pas mort \ocadr je serai simplement
 sorti d'une vieille tente usée et entré dans une belle demeure,
 \og \dots{} \punct{Un point en trop}
 dans les cieux, un édifice qui est l'ouvrage de Dieu, une demeure éternelle
 qui n'a pas été faite par la main des hommes \fg{} (\ibibleverse{IICo}(5:1)).
 Aussi, s'il vous plaît, n'allez pas pleurer pour moi
 \grammar{pleurer pour moi}, parce que vous savez que je ne pleurerai pas.
 \og Il y a abondance de joies devant Ta face, des délices éternelles
 à Ta droite \fg{} (\ibibleverse{Ps}(16:11)). 

Dieu veut être en communion avec vous, et Il a envoyé Son Fils pour abolir
 la mort afin que vous puissiez avoir la vie éternelle
 et l'immortalité par Lui. Et c'est la vérité de l'Évangile.
 \suggest{Évangile avec une majuscule}

\dvrule

\dvprayer{
Père, nous Te remercions pour cette vie et pour l'immortalité
 qui est à nous par Jésus-Christ. Nous prions pour ceux
 qui ne Te connaissent pas et Te demandons de parler à leurs cœurs.
}{\DlNdJ}


%%%%%%%%%%%%%%%
% 9 novembre
%%%%%%%%%%%%%%%

\dvday{Vases d'Honneur}

\dvquote{
Si un homme donc se purifie de ces choses, il sera un vase d’honneur,
 sanctifié, et propre au service de son maître,
 et préparé pour toute bonne œuvre.
}{\ibibleverse{IITm}(2:21) \KJF}

\punct{Point manquant}

\lettrine{L}{es vases étaient très communs} dans les temps bibliques.
 Certains, faits d'or ou d'argent, étaient utilisés dans un but décoratif.
 D'autres étaient faits d'argile et utilisés à des fins diverses
 allant de transporter de l'eau jusqu'à contenir des ordures
 ou des eaux usées. Ainsi, certains étaient appelés des vases d'honneur
 et d'autres, des vases de déshonneur.

Prenez un moment et demandez-vous~: \punct{deux-points}
 \og Que contient ma vie? \fg \punct{guillemets fermants manquants}
 Les choses qui remplissent votre vie sont-elles
 pures ou sont-elles souillées comme des eaux usées? 

\dvbox{
Vous avez été créés pour contenir Dieu. 
}

Dieu a choisi de remplir un pot d'argile \ocadr vous \fcadr{}
 de la chose la plus précieuse qui existe \ocadr Lui-Même.
 Il veut vous remplir de Lui-Même afin que vous débordiez de Sa grâce,
 de Sa miséricorde et de Son amour pour désaltérer un monde assoiffé.

Paul a écrit à Timothée au sujet d'hommes qui, dans l'église,
 corrompaient les gens avec un enseignement erroné.
 Ces hommes étaient devenus des vases de déshonneur.
 Dieu ne pouvait pas les utiliser; leur doctrine était impure.
 Ils enseignaient leurs propres idées au lieu des vérités de Dieu,
 et de cette façon, souillaient le contenu de leurs vies
 de leur propre parfum. 

Comme il est facile de devenir un vase de déshonneur !
 Tout ce qu'il suffit de faire est de cesser de purifier ses pensées
 chaque jour. Mais si vous voulez être un vase d'honneur,
 prêt à l'usage du Maître, alors vous devez fuir les mauvais désirs
 de la chair. Il vous faut communier avec Dieu à la fin de chaque journée
 et Lui demander de vous purifier de nouveau. 

\dvrule

\dvprayer{
Dieu, fais de nous des vases d'honneur, afin que nous puissions déverser
 Ta grâce sur le monde qui nous entoure.
 Purifie-nous et utilise-nous, Seigneur. 
}{\Amen}


%%%%%%%%%%%%%%%
% 10 novembre
%%%%%%%%%%%%%%%

\dvday{Inspirée par Dieu}

\dvquote{
Toute Écriture est inspirée de Dieu et utile pour enseigner,
 pour convaincre, pour redresser, pour éduquer dans la justice\dots{}
}{\ibibleverse{IITm}(3:16)}

\lettrine{L}{a Bible parle du Dieu éternel qui a créé l'univers,}
 du Dieu qui existe en dehors du temps et de l'espace.
 Elle nous dit que ce Dieu qui a créé toutes choses,
 aime l'homme et désire vivre en communion avec lui,
 afin que l'homme puisse recevoir les bénédictions de Le connaître.
 Les Écritures sont la Parole de Dieu adressée à l'homme.

\dvbox{
Parce que la Bible est inspirée par Dieu, elle est infaillible et inerrante.
}

Par les prophéties, Dieu prouve qu'Il est en dehors du temps.
 Lui seul connaît la fin depuis le commencement.
 Lui seul peut prédire une chose avant qu'elle n'arrive.
 En prophétisant des choses encore futures, Il prouve qu'Il est vraiment
 l'Auteur de ce Livre, qui est composé à quatre-vingt pour cent de prophétie. 

Quand Jésus parlait à Ses disciples de Sa mort et de sa résurrection,
 Il leur a dit~: \punct{deux-points}
 \og Je m'en vais et je reviendrai vers vous\dots{}
 Je vais vers le Père\dots{} Je vous ai dit ces choses maintenant,
 avant qu'elles n'arrivent, afin que, lorsqu'elles arriveront,
 vous croyiez \fg{} (\ibibleverse{Jn}(14:28-29)).
 Il leur a annoncé à l'avance les choses à venir afin qu'ils puissent
 voir qu'Il était vraiment qui Il disait être \ocadr le Fils de Dieu. 

Dieu nous a donné Sa Parole afin que nous comprenions Sa nature,
 Son plan de rédemption et le chemin de la croissance spirituelle.
 Les vérités qu'elle contient vont bâtir votre foi.
 Puissions-nous l'étudier quotidiennement, en pratiquant la Parole
 et en ne l'écoutant pas seulement.

\dvrule

\dvprayer{
Merci, Père, pour Ta Parole, une lumière sur notre sentier
 qui nous guide dans Ta vérité. 
}{\DlNdJ}


%%%%%%%%%%%%%%%
% 11 novembre
%%%%%%%%%%%%%%%

\dvday{Le Seigneur m'a assisté}

\dvquote{
C'est le Seigneur qui m'a assisté et qui m'a fortifié,
 afin que la prédication soit portée par moi à sa plénitude
 et entendue de tous les païens.
}{\ibibleverse{IITm}(4:17)}

\lettrine{A}{u premier procès de Paul devant Néron,}
 sa situation semblait si sombre que tous ses compagnons l'ont abandonné.
 \og Dans ma première défense, personne ne m'a assisté,
 \punct{Pas besoin de répéter les guillemets}
 disait-il, mais tous m'ont abandonné \fg{}
 (\ibibleverse{IITm}(4:16)).

Comme il est horrible de se sentir abandonné !
 La Bible dit que \og l'ami \typo{Pas besoin de majuscule}
 aime en tout temps \fg{} (\ibibleverse{Pr}(17:17)), mais trop souvent,
 nous découvrons que ceux que nous pensions être de vrais amis,
 n'étaient en fait que des connaissances.
 Même s'il est possible que tous les autres vous laissent tomber,
 Jésus va toujours se tenir à vos côtés.
 Le Seigneur ne va pas abandonner Son peuple.
 Comme David le disait~: \punct{deux-points}
 \og Quand je marche dans la vallée de l'ombre de la mort,
 je ne crains aucun mal, car Tu es avec moi \fg{} (\ibibleverse{Ps}(23:4)).
 \punct{Point manquant}

\dvbox{
Tant que je sais que le Seigneur m'assiste,
 je peux faire face aux épreuves du lendemain et aux incertitudes du futur.
}

Aujourd'hui, vous vous sentez peut-être comme Paul \ocadr délaissé,
 tout seul, faisant face à un futur incertain.
 Dans cette nuit sombre, vous pouvez être certains que Jésus
 se tient à vos côtés.
 \og Et voici, je suis avec vous tous les jours, jusqu'à la fin du monde \fg{}
 (\ibibleverse{Mt}(28:20)).
 Il a dit~: \punct{deux-points}
 \og Je ne te délaisserai pas ni ne t'abandonnerai \fg{}
 (\ibibleverse{He}(13:5)). 

Comme quelqu'un l'a dit~: \punct{deux-points}
 \og Je ne connais \suggest{Ou je ne sais}
 pas ce que demain apportera, mais je connais Celui qui apportera demain,
 et je sais Qui me tient la main. \fg{}
 Il se tiendra à mes côtés et m'aidera à traverser l'épreuve. 

\dvrule

\dvprayer{
Père, nous te remercions que nous avons l'assurance si merveilleuse
 que Tu seras à nos côtés. Soutiens-nous, quand nous pensons
 que nous n'en pouvons plus. Soutiens-nous de Ta puissante main droite. 
}{\DlNdJ}


%%%%%%%%%%%%%%%
% 12 novembre
%%%%%%%%%%%%%%%

\dvday{Professer ou Pratiquer ?}

\dvquote{
Ils font profession de connaître Dieu, mais ils le renient par leurs œuvres.
 Ils sont odieux, rebelles et incapables d'aucune œuvre bonne.
}{\ibibleverse{Tt}(1:16)}

\lettrine{B}{eaucoup de gens pensent que,}
 parce qu'ils disent en se couchant une petite prière
 du genre~: \punct{deux-points}
 \og Je mets ma tête sur l'oreiller, et je te prie, Seigneur,
 sur mon âme de veiller \fg{}, cela signifie qu'ils connaissent Dieu.
 Comme Jésus le disait~: \punct{deux-points}
 \og Pourquoi m'appelez-vous~: Seigneur, Seigneur !
 et ne faites-vous pas ce que je dis? \fg{} (\ibibleverse{Lc}(6:46)).
 \punct{Point manquant} 

À votre avis, qu'est-ce qui intéresse le plus Dieu?
 Que vous professiez ou que vous pratiquiez?
 Jean-Baptiste a dit~: \punct{deux-points}
 \og Produisez donc des fruits dignes de la repentance \fg{}
 (\ibibleverse{Lc}(3:8)).
 Autrement dit~: \punct{deux-points}
 \og Que votre vie soit en cohérence avec vos paroles. \fg{}

\dvbox{
C'est une chose que de connaître des choses sur Dieu,
 c'en est une toute autre que de connaître Dieu.
}

Dire que vous croyez en Dieu n'est pas suffisant;
 vous devez faire suivre vos paroles par une vie en cohérence avec Dieu.
 Les mots ne veulent rien dire par eux-mêmes. Vous pouvez dire n'importe quoi.
 Mais \typo{Mais} comment vivez-vous? 

Paul nous encourage à nous examiner nous-mêmes,
 il a dit en effet~: \punct{deux-points}
 \og Si nous nous jugions nous-mêmes, nous ne serions pas jugés \fg{}
 (\ibibleverse{ICo}(11:31)).
 Est-il possible que vous soyez un de ceux dont Paul parlait
 \ocadr ceux qui sont dans l'église et professent connaître Dieu, mais qui,
 en fait, Le renient par leurs œuvres? Avez-vous laissé d'autres dieux
 supplanter votre amour pour Lui?
 A-t-Il vraiment la première place dans votre vie? 

Ne vous contentez pas seulement de professer votre foi
 \ocadr mettez-la \typo{pas de majuscule après un tiret long}
 aussi en pratique ! 

\dvrule

\dvprayer{
Seigneur, montre-moi si, dans mon travail, je Te renie.
 Montre-moi s'il y a des incohérences entre mes propos et ma pratique.
 Aide-moi, Seigneur, à Te connaître et Te servir en vérité. 
}{\DlNdJ}


%%%%%%%%%%%%%%%
% 13 novembre
%%%%%%%%%%%%%%%

\dvday{La Bienheureuse Espérance}

\dvquote{
En attendant la bienheureuse espérance et la manifestation
 de la gloire de notre grand Dieu et Sauveur, le Christ-Jésus.
}{\ibibleverse{Tt}(2:13)}

\lettrine{L}{'espérance du croyant,} c'est la vie éternelle avec Dieu.
 Pour nous, la mort n'est pas la fin; c'est simplement une métamorphose.
 C'est simplement un changement de corps. Mais si votre espérance n'est pas
 en Jésus-Christ pour la vie éternelle, alors comme Paul l'a dit, vous êtes
 \og sans Christ\dots{} sans espérance et sans Dieu dans le monde \fg{}
 (\ibibleverse{Ep}(2:12)). 

\dvbox{
Si vous n'avez pas Christ, alors vous n'avez pas Dieu\dots{}
 et vous n'avez pas d'espoir.
}

Nous qui sommes croyants, espérons non seulement en la vie éternelle,
 mais nous attendons aussi la manifestation glorieuse de Jésus-Christ.
 Il a promis qu'Il reviendrait. Il a dit à Ses disciples~: \punct{deux-points}
 \og Que votre cœur ne se trouble pas. Croyez en Dieu, croyez aussi en Moi.
 Il y a beaucoup de demeures dans la maison de mon Père.
 Sinon, je vous l'aurais dit ; car Je vais vous préparer une place.
 Donc, si Je m'en vais et vous prépare une place, Je reviendrai
 et Je vous prendrai avec moi, afin que là où Je suis,
 vous y soyez aussi \fg{} (\ibibleverse{Jn}(14:1-3)).
 Et ainsi nous attendons avec espoir la glorieuse manifestation
 de notre Seigneur Jésus-Christ. 

Jésus nous a indiqué beaucoup des signes de Son retour,
 des choses que nous pourrions surveiller pour savoir si ce retour est proche.
 Nos espoirs se renforcent quand nous observons le monde aujourd'hui
 et voyons que toutes les conditions sont réunies \ocadr tout ce dont la Bible
 parle comme devant arriver avant ce jour.
 Plus le monde devient sombre et plus l'espoir de l'enfant de Dieu
 est brillant. 

\dvrule

\dvprayer{
Père, nous Te remercions de ce que dans un monde assombri
 par la détresse et le désespoir,
 nous avons une glorieuse espérance qui nous soutient.
}{\DlNdJ}


%%%%%%%%%%%%%%%
% 14 novembre
%%%%%%%%%%%%%%%

\dvday{Justifiés !}

\dvquote{
\dots{} afin que, justifiés par sa grâce, nous devenions héritiers
 dans l'espérance de la vie éternelle.
}{\ibibleverse{Tt}(3:7)}

\lettrine{P}{ar Sa miséricorde,} Dieu vous pardonne de vos péchés.
 \suggest{vous pardonne vos péchés}
 Mais par Sa grâce, Il vous justifie, et cela signifie qu'Il a décidé
 de laisser tomber les accusations contre vous. Il a effacé l'ardoise.
 \og Il n'y a donc maintenant aucune condamnation pour ceux
 qui sont en Christ-Jésus \fg{} (\ibibleverse{Rm}(8:1)).
 Parce que vous avez reçu Jésus-Christ comme votre Seigneur,
 quand Dieu vous regarde aujourd'hui, Il vous voit comme quelqu'un
 d'absolument innocent. Ainsi, alors que la justice c'est recevoir
 ce qu'on mérite, et que la miséricorde c'est ne pas recevoir
 ce qu'on mérite, la grâce, c'est recevoir ce qu'on ne mérite pas. 

\dvbox{
Étant justifié par Sa grâce, je suis maintenant un enfant de Dieu.
}

Je ne mérite pas l'amour de Dieu. Je ne mérite pas l'effacement total
 de mes péchés, ni la vie éternelle dans Son royaume.
 Mais ce sont des choses que Dieu me donne en raison de la justification
 qui est venue pas Sa grâce. Je mérite la mort; par grâce,
 Dieu me donne la vie. Je mérite l'enfer; par grâce,
 Dieu m'ouvre les portes du ciel. 

En tant que Son enfant, je deviens un héritier de Dieu.
 Le monde compte beaucoup de millionnaires et même des milliardaires.
 Mais la richesse accumulée par ces gens n'est absolument rien,
 comparée aux richesses de Dieu. Nous, qui appartenons à Dieu,
 sommes plus riches que les gens les plus riches sur terre,
 parce que la gloire éternelle du royaume de Dieu nous appartient.
 Dans les temps sans fin à venir, nous qui sommes les enfants de Dieu,
 jouirons des richesses du royaume, de la beauté de Sa présence,
 de la majesté de Son amour et du réconfort de Sa grâce
 et de Sa miséricorde \ocadr pour toujours. 

\dvrule

\dvprayer{
Père, Merci de ce que, par Ta grâce tu as effacé l'ardoise.
 Oh, comme nous sommes heureux en raison de Ta grâce manifestée envers nous!
}{\DlNdJ}


%%%%%%%%%%%%%%%
% 15 novembre
%%%%%%%%%%%%%%%

\dvday{Mets-le sur Mon Compte}

\dvquote{
S'il t'a fait quelque tort, ou s'il te doit quelque chose,
 mets-le sur mon compte.
}{\ibibleverse{Ph}(1:18)}

\lettrine{L}{e livre de Philémon} est en fait une lettre que Paul
 a écrite pour intercéder en faveur d'Onésime, un esclave en cavale.
 De toute évidence, quand il s'est enfui, Onésime avait volé de l'argent
 à Philémon qui habitait à Colosse.
 Utilisant cet argent, il avait fini par arriver à Rome
 \ocadr espérant sans nul doute s'y fondre dans la foule.
 Mais Onésime fut arrêté et mis en prison, et là il rencontra
 l'apôtre Paul. Paul commença à témoigner auprès de lui de la grâce
 salvatrice de Jésus-Christ, et Onésime en vint à naître de nouveau.
 Au bout d'un moment, Paul découvrit qu'Onésime était un esclave en cavale.
 Paul connaissait son maître, Philémon, parce qu'il l'avait aussi
 mené au Seigneur. Paul écrivit donc cette lettre à Philémon
 en lui demandant de recevoir Onésime comme il recevrait Paul lui-même.
 Il implora Philémon d'affranchir son esclave, afin qu'Onésime
 puisse revenir et servir avec lui, Paul, dans l'œuvre de l'Évangile.
 \suggest{Évangile}

\dvbox{
Notre dette a été totalement payée par Jésus-Christ.
}

Paul a écrit~: \punct{deux-points}
 \og S'il \typo{Majuscule}
 te doit quelque chose, mets-le sur mon compte. \fg{}
 \punct{Guillemet après point}
 Paul est prêt à payer la dette d'Onésime,
 ce qui est \typo{« est » manquant}
 exactement ce que Jésus-Christ a fait pour nous.
 Il a payé la dette que nous devions. 

Nous étions tous comme Onésime, en fuite pour échapper à notre Maître.
 Nous étions des esclaves en cavale, avec une dette que nous ne pouvions
 pas payer.
 Mais tout comme Onésime, nous avons nous aussi un Avocat, un Médiateur,
 un Intercesseur. Nous avons le Seigneur Jésus-Christ,
 qui intercède pour nous.
 \og Et l'Éternel a fait retomber sur lui la faute de nous tous \fg{}
 (\ibibleverse{Is}(53:6)). 

La dette a été payée en totalité.
 Nous ne devons plus rien maintenant
 \ocadr si ce n'est notre vie et notre gratitude.

\dvrule

\dvprayer{
Seigneur, comme nous Te sommes reconnaissants
 d'avoir payé notre dette dans sa totalité !
}{\DlNdJ}


%%%%%%%%%%%%%%%
% 16 novembre
%%%%%%%%%%%%%%%

\dvday{Le Message}

\dvquote{
Après avoir autrefois, à plusieurs reprises et de plusieurs manières,
 parlé à nos pères par les prophètes, Dieu nous a parlé par le Fils
 en ces jours qui sont les derniers\dots{}
}{\ibibleverse{He}(1:1-2)}

\lettrine{D}{ieu parle à l'homme} depuis le commencement des temps.
 Il parle par la nature, par Ses prophètes, par Sa Parole, par des visions
 et par des rêves.
 Il parle par Sa petite voix en nous. Quelquefois, Il parle de façon audible.
 Dieu nous parle encore; le problème, c'est que nous ne sommes pas toujours
 à l'écoute. 

Dans la Bible, nous voyons que les hommes ont déformé les messages de Dieu
 en essayant d'interpréter Ses lois à leur façon.
 Les hommes étaient devenus si confus, qu'ils ne connaissaient plus
 la vérité sur Dieu.
 Aussi Dieu a envoyé Son Fils dans le monde
 pour révéler la vérité aux hommes. 

\dvbox{
Dieu parle encore aujourd'hui. Écoutez.
}

Quel est le message fondamental que Dieu nous a donné par Son Fils?
 Tout d'abord, Jésus nous a appris que Dieu est amour.
 \og Car Dieu a tant aimé le monde qu'Il a donné Son Fils unique,
 afin que quiconque croit en Lui ne périsse pas, mais qu'il ait
 la vie éternelle. \fg{}
 Il a enseigné que Dieu est plein de miséricorde, de grâce,
 de compassion et de pardon. Il a enseigné que Dieu est lumière.
 Jésus a aussi enseigné que Dieu veut que nous nous aimions
 les uns les autres, tout comme Il nous aime. 

Puissions-nous entendre clairement la voix de Dieu quand Il nous parle
 de la vie, de l'amour, de notre relation avec Lui et de nos relations
 les uns avec les autres. Puisse Dieu nous aider à vraiment écouter.
 Puissions-nous obéir à Sa voix et aimer les autres comme Il voudrait
 que nous aimions.

\dvrule

\dvprayer{
Dieu, donne-nous des oreilles pour entendre ce que Tu nous dit
 par Ta Parole et par Ton Saint-Esprit.
}{\ESN}


%%%%%%%%%%%%%%%
% 17 novembre
%%%%%%%%%%%%%%%

\dvday{Partir à la Dérive}

\dvquote{
C'est pourquoi nous devons prêter une plus vive attention
 à ce que nous avons entendu, de peur d'aller à la dérive.
}{\ibibleverse{He}(2:1)}

\lettrine{L}{a Lettre aux Hébreux} a été écrite pour mettre en garde
 certains Juifs, qui revenaient lentement de la foi chrétienne au Judaïsme,
 afin qu'ils sachent que les voies religieuses traditionnelles
 n'étaient pas la solution, parce que nous avons maintenant
 la révélation finale de Dieu par Son Fils. 

L'expression \og c'est pourquoi \fg{} renvoie au chapitre précédent
 qui parlait de la façon dont, dans les temps passés,
 Dieu avait parlé aux pères par les prophètes.
 L'auteur a alors ajouté que dans ces temps qui sont les derniers,
 Dieu a parlé par une révélation plus directe
 \ocadr Il a parlé par Son Fils, Jésus.
 Aussi devons-nous prendre en compte les choses que nous avons entendues
 de la bouche de Jésus. 

\dvbox{
Ne pas écouter la parole de Jésus,
 c'est se mettre en danger de partir a la dérive.
}

Qu'a dit Jésus? Il a dit~: \punct{deux-points, majuscule}
 \og Si un homme ne naît de nouveau il ne peut voir le royaume de Dieu \fg{}
 (\ibibleverse{Jn}(3:3)).
 Il a dit~: \punct{deux-points}
 \og Moi, je suis le chemin, la vérité et la vie.
 Nul ne vient au Père que par moi \fg{} (\ibibleverse{Jn}(14:6)).
 Il a dit qu'Il n'est pas venu \og dans le monde pour juger le monde,
 mais pour que le monde soit sauvé par Lui \fg{} (\ibibleverse{Jn}(3:17)).
 Il a dit~: \punct{deux-points}
 \og Car Dieu a tant aimé le monde qu'Il a donné son Fils unique,
 afin que quiconque croit en Lui ne périsse pas,
 mais qu'il ait la vie éternelle \fg{} (\ibibleverse{Jn}(3:16)). 

Partir à la dérive est un processus lent, imperceptible.
 La seule façon de savoir si l'on dérive est d'avoir un cadre de référence
 stationnaire.
 En vous voyant bouger par rapport à ce point stationnaire,
 vous vous rendez alors compte que vous dérivez. 

Fixez vos yeux sur l'Ancre, Jésus-Christ. Tenez compte de Ses paroles.
 Il sera votre référence stationnaire.
 Il vous empêchera de partir à la dérive. 

\dvrule

\dvprayer{
Père, nous avons tendance à nous égarer loin de Toi.
 Garde-nous près de Toi. Ancre-nous dans Ta vérité.
}{\DlNdJ}


%%%%%%%%%%%%%%%
% 18 novembre
%%%%%%%%%%%%%%%

\dvday{Entrer dans le Repos de Dieu}

\dvquote{
J'ai donc juré dans ma colère~:
 Ils n'entreront certainement pas dans mon repos.
}{\ibibleverse{He}(3:11)}

\lettrine{L}{'auteur de la Lettre aux Hébreux}
 cite un passage du \ibibleverse{Ps}(95:) qui parle du jour où le peuple
 n'a pas réussi à entrer en Terre Promise, ni à faire l'expérience
 de la paix et de l'abondance que Dieu voulait lui donner. 

Dieu nous a promis le repos. Jésus a dit~: \punct{deux-points}
 \og Venez à moi, vous tous qui êtes fatigués et chargés,
 et je vous donnerai du repos \fg{} (\ibibleverse{Mt}(11:28)).
 Quand vous recevez Jésus comme votre Sauveur, une des premières expériences
 que vous faites est celle d'une paix profonde.
 Ayant capitulé, vous ne vous battez plus contre Dieu. 

\dvbox{
L'incrédulité peut vous empêcher de jouir de la vie riche et pleine
 que Dieu veut que vous connaissiez en Christ.
}

Dieu veut que vous ayez une vie belle et paisible, sans bouleversement
 \ocadr une vie de paix et de repos.
 Cependant, combien peu de personnes dans le peuple de Dieu connaissent
 et apprécient vraiment cette paix parfaite qui surpasse toute intelligence !
 Peut-être, \suggest{pas de virgule} est-ce parce que nous ne croyons pas
 à Ses promesses.
 Comme les enfants d'Israël, nous laissons les géants de la vie
 nous empêcher de prendre tout ce que Dieu veut nous donner. 

Beaucoup d'entre vous en sont toujours à la traversée du désert
 pour ce qui est de votre expérience chrétienne;
 vous continuez à vous battre avec la chair et vous ne gagnez pas de terrain
 et ne faites aucune conquête. Vous ne vivez pas
 une vie Chrétienne \suggest{chrétienne, adjectif, vs. une Chrétienne}
 victorieuse. 

Dans notre incrédulité, nous ne croyons pas qu'Il va faire les choses
 qu'Il a promises \grammar{choses promises}
 de faire.
 Dieu dit~: \punct{deux-points}
 \og Vous êtes dans la traversée du désert depuis assez longtemps.
 Il est temps pour vous d'entrer dans la Terre Promise.
 Il est temps d'entrer conquérir le territoire. \fg{}
 Saisissez-vous \punct{Saisissez-vous}
 des victoires que Dieu a pour vous et commencez à faire l'expérience
 \typo{expérience}
 de la richesse et de la plénitude de la vie en Jésus-Christ.
 Goûtez à Son repos. 

\dvrule

\dvprayer{
Seigneur, puissions-nous Te suivre dans cet endroit de victoire.
}{\DlNdJ}


%%%%%%%%%%%%%%%
% 19 novembre
%%%%%%%%%%%%%%%

\dvday{La Séduction du Péché}

\dvquote{
Mais exhortez-vous chaque jour, aussi longtemps qu'on peut dire~:
 Aujourd'hui! afin qu'aucun de vous ne s'endurcisse par la séduction
 du péché.
}{\ibibleverse{He}(3:11)}

\lettrine{L}{e péché est séduisant,} diaboliquement séduisant.
 En surface, le péché paraît promettre le plaisir,
 mais la Bible nous prévient que \og telle \typo{Pas de majuscule}
 voie paraît droite devant un homme, mais à la fin,
 c'est la voie de la mort \fg{} (\ibibleverse{Pr}(16:25)).
 Le péché sait se justifier. Vous avez entendu ces justifications.
 \og Les temps ont changé; tout le monde le fait. \fg{}
 \suggest{Ajout de \ocadr au lieu de répéter les guillemets?}
 \og Je ne suis qu'un homme. \fg{}
 \og Juste une fois ne peut pas vous faire mal. \fg{}
 \punct{Point avant guillemet}
 \og Je ne peux pas m'en empêcher, Dieu m'a fait ainsi. \fg{}

\dvbox{
Endurcis par le péché, vous vous retrouvez à faire des choses
 que vous vous étiez jurés de ne jamais faire.
}

Le péché vous séduit en vous endurcissant.
 Si vous vous adonnez à quelque chose assez longtemps,
 vous finirez par l'accepter. 

La solution face à la séduction du péché, c'est de vous entourer
 d'autres croyants, de vrais amis qui vous mettront en garde
 lorsque vous vous égarerez.
 \og Les blessures d'un ami sont dignes de confiance,
 les baisers d'un ennemi sont trompeurs \fg{} (\ibibleverse{Pr}(27:6)).
 La pression exercée par l'entourage peut être une chose positive,
 quand cet entourage aime le Seigneur et qu'il est résolu à Le servir. 

Si l'Esprit de Dieu a parlé à votre cœur d'un péché que vous avez permis
 dans votre vie, il est absolument important de le laisser tomber
 immédiatement.
 N'attendez pas demain ou la semaine prochaine.
 N'attendez pas un moment plus propice. Rappelez-vous~:
 le péché est trompeur.
 Si vous le laissez s'installer,
 le péché va endurcir votre cœur contre Dieu. 

\dvrule

\dvprayer{
Père, aide-nous à écouter Ta Parole et à en tenir compte,
 de peur que nos cœurs ne soient endurcis par la séduction du péché.
 Aide-nous à abandonner le chemin du péché et à marcher
 dans les sentiers de la justice.
}{\Amen} 


%%%%%%%%%%%%%%%
% 20 novembre
%%%%%%%%%%%%%%%

\dvday{Vivante et Efficace}

\dvquote{
La Parole de Dieu est vivante et efficace\dots{}
}{\ibibleverse{He}(4:12a)}

\typo{4:12a ?}

\lettrine{L}{es graines sont vivantes.}
 En raison de l'ADN unique implanté dans les graines, chacune d'elles,
 qu'elle provienne d'un arbre, d'une vigne ou d'une autre plante,
 est capable de se reproduire.
 Dans \ibibleverse{Lc}(8:), quand Jésus a raconté la parabole du semeur
 qui avait semé la semence sur quatre types de sols,
 Il a expliqué~: \punct{deux-points}
 \og La semence, c'est la Parole de Dieu. \fg{}
 Cela signifie que quand vous accueillez la Parole de Dieu,
 elle s'implante dans votre cœur et commence à vous transformer. 

\dvbox{
La Parole de Dieu commence à vous changer
 \ocadr de l'intérieur à l'extérieur \fcadr{}
 en l'image de Jésus.
}

\suggest{De l'intérieur \emph{vers} l'extérieur}

L'auteur de la Lettre aux Hébreux ajoute alors~: \punct{deux-points}
 \og Elle est efficace. \fg{} \punct{Point avant guillemet}
 Elle l'est vraiment, puissamment.
 La puissance de Dieu est tellement impressionnante!
 Dans le \ibibleverse{Ps}(33:6), nous lisons~: \punct{deux-points}
 \og Les cieux ont été faits par la Parole de l'Éternel,
 et toute leur armée par le souffle de Sa bouche. \fg{}
 Rendez-vous compte \ocadr le vaste univers tout entier a été créé
 par la Parole de Dieu, par juste \suggest{juste/simplement par}
 le souffle de Sa bouche.
 Nous lisons dans la Genèse~: \punct{deux-points}
 \og Dieu dit~: Que la lumière soit ! Et la lumière fut. \fg{}
 Dieu a créé tout ce que nous voyons et connaissons
 par la puissance de Sa Parole. 

La puissance de la Parole de Dieu est évidente dans les vies
 qu'elle a transformées. Des gens qui étaient à un moment donné
 le rebut de la société \ocadr des hommes et des femmes considérés
 comme sans espoir et sans valeur \fcadr{} ont été changés et guéris
 par la puissance de la Parole de Dieu.
 Par le simple fait d'accueillir et de méditer les vérités de l'Écriture,
 la dépression a été dissipée, les cœurs ont été guéris,
 les pensées ont été transformées et des vies ont été changées
 \ocadr pour toujours. 

\dvrule

\dvprayer{
Seigneur, conduis-nous à étudier Ta puissante Parole vivante
 afin qu'elle puisse accomplir en nous une transformation glorieuse.
}{\Amen}


%%%%%%%%%%%%%%%
% 21 novembre
%%%%%%%%%%%%%%%

\dvday{La Puissance de La Parole}

\dvquote{
La Parole de Dieu est vivante et efficace,
 plus acérée qu'aucune épée à double tranchant;
 elle pénètre jusqu'à la division de l'âme et de l'esprit,
 des jointures et des moelles; elle est juge des sentiments
 et des pensées du cœur.
}{\ibibleverse{He}(4:12)}

\lettrine{J}{usqu'à ce qu'ils développent l'épée à double tranchant,}
 les Romains avaient combattu dans toutes leurs batailles avec une épée
 à simple tranchant. Deux tranchants leur permirent de faire tournoyer
 l'épée dans les deux sens et de blesser en venant des deux directions.
 C'est cette arme secrète qui permit aux légions romaines \grammar{romaines}
 de conquérir le monde. 

\dvbox{
C'est par la Parole de Dieu que j'accède à la vérité sur Dieu
 \ocadr et sur moi-même.
}

La Parole de Dieu est plus acérée qu'aucune épée à double tranchant,
 assez acérée pour pénètrer jusqu'à la division de l'âme et de l'esprit.
 L'homme est un être en trois parties
 \ocadr le corps, l'âme et l'esprit \fcadr{}
 et cependant elles sont si bien intégrées qu'il est pratiquement impossible
 de complètement séparer les trois.
 Beaucoup de gens vont à l'église et vivent une expérience hautement
 émotionnelle, mais ça ne touche pas leur esprit, et ils repartent
 sans avoir été changés.
 Une simple expérience ne change pas un homme et ne nourrit pas son esprit.
 Il faut la Parole de Dieu pour y arriver. 

La Parole de Dieu est juge des sentiments et des pensées du cœur.
 Si souvent, nous ne connaissons pas nous-mêmes notre propre cœur.
 Pourquoi ai-je fait ceci? Quelles étaient mes intentions réelles?
 Nos motivations sont souvent déguisées.
 Ce qui peut apparaître comme un geste admirable peut être motivé
 par quelque chose de mauvais \ocadr comme le désir d'être reconnu
 voire admiré. C'est pourquoi la Parole de Dieu est si importante pour nous,
 parce qu'elle révèle les vraies intentions de nos cœurs. 

\dvrule

\dvprayer{
Père, nous Te remercions pour Ta Parole.
 Donne-nous faim de la lire et de la méditer.
}{\Amen}


%%%%%%%%%%%%%%%
% 22 novembre
%%%%%%%%%%%%%%%

\dvday{Leçons par la Souffrance}

\dvquote{
\dots{} Il a appris, bien qu'Il fût le Fils,
 l'obéissance par ce qu'Il a souffert.
}{\ibibleverse{He}(5:8)}

\lettrine{J}{ésus a appris l'obéissance} et Son obéissance
 L'a fait souffrir.
 Sa mort sur la croix fut incroyablement atroce.
 Mais Sa souffrance a servi à accomplir le dessein éternel de Dieu.
 Nous pouvons beaucoup apprendre de Lui,
 \og qui \typo{pas de majuscule}
 en vue de la joie qui Lui était proposée, a supporté la croix,
 méprisé la honte \fg{} (\ibibleverse{He}(12:2)). 

À la différence de Jésus, nous apprenons souvent l'obéissance,
 forcés et contraints. Habituellement, nous désobéissons,
 puis nous apprenons que l'obéissance aurait beaucoup mieux valu.
 Nous apprenons l'obéissance par la souffrance que nous éprouvons
 en conséquence de notre désobéissance. 

\dvbox{
La souffrance est un instrument divin.
}

Tous les enfants de Dieu vont souffrir.
 Croire en Jésus-Christ comme Seigneur et Grand Prêtre ne nous confère pas
 une immunité contre la souffrance. Nous aimerions bien penser que,
 parce que nous appartenons à Dieu, Il va nous protéger d'absolument
 toute espèce de douleur.
 Mais sommes-nous plus grands que notre Seigneur?
 Si Jésus a souffert et appris l'obéissance par cette souffrance,
 ne devons-nous pas, à plus forte raison, apprendre cette leçon? 

Dieu utilise la souffrance pour accomplir Ses desseins éternels.
 Il l'utilise aussi pour nous apprendre l'obéissance, la dépendance,
 la foi, la grâce et la patience.
 Nous pouvons croire que Dieu ne va jamais nous laisser souffrir en vain,
 tout comme Il n'a pas laissé Son Fils souffrir en vain. 

Quel que soit le défi auquel Dieu vous demande de répondre,
 obéissez avec joie. Vous ne savez pas comment Dieu va pouvoir utiliser
 cette souffrance pour votre bénéfice et pour Sa gloire. 

\dvrule

\dvprayer{
Père, nous sommes en paix sachant que Tes pensées envers nous sont bonnes
 et non mauvaises et que Tu nous réserves un futur glorieux.
 Aide-nous à obéir joyeusement.
}{\DlNdJ}


%%%%%%%%%%%%%%%
% 23 novembre
%%%%%%%%%%%%%%%

\dvday{Notre Ancre}

\dvquote{
Cette espérance, nous l'avons comme une ancre solide et ferme,
 pour notre âme; elle pénètre au-delà du voile\dots{}
}{\ibibleverse{He}(6:19)}

\lettrine{À}{ moins d'être amarré à quelque chose de solide,}
 il est très facile de partir à la dérive.
 Vous ne remarquerez peut-être aucun déplacement au début,
 mais avec le temps, cette lente dérive peut vous emmenez très loin.
 En vous réveillant un beau matin, vous êtes choqués de voir
 combien vous vous êtes éloignés de votre point de départ. 

Comme le disait le psalmiste~: \punct{deux-points}
 \og Il m'a retiré de la fosse de destruction du fond de la boue;
 il a dressé mes pieds sur le roc. \fg{}

\dvbox{
En ces temps qui changent, nous avons besoin d'être ancrés
 au Christ qui ne change pas.
}

Après avoir décrit l'ancre de nos âmes
 \ocadr qui est l'espérance du retour de Christ \fcadr{}
 l'auteur de La Lettre aux Hébreux, mentionne la présence derrière le voile.
 Le voile faisait référence à l'épaisse séparation entre le Lieu Saint
 dans le temple et le Saint des Saints \ocadr l'endroit le plus sacré.
 Avant Jésus, personne en dehors du grand prêtre \suggest{Grand Prêtre}
 ne pouvait pénétrer
 dans cet endroit saint où demeurait la présence de Dieu.
 Mais quand Jésus a été crucifié, il est écrit que le voile du temple
 a été déchiré en deux du haut jusqu'en bas.

C'est la seule façon dont une brèche pouvait être ouverte
 dans la séparation \ocadr du haut (le ciel) vers le bas (la terre).
 Dieu a lancé une invitation dans cette brèche. La porte est ouverte
 et en raison du sacrifice de Son Fils, vous pouvez avoir accès à Dieu
 à n'importe quel moment.
 \og Approchons-nous donc avec assurance du trône de la grâce,
 afin d'obtenir miséricorde et de trouver grâce,
 en vue d'un secours opportun \fg{} (\ibibleverse{He}(4:16)). 

\dvrule

\dvprayer{
Seigneur, nous sommes tellement reconnaissants pour la croix,
 par laquelle Tu as offert l'espérance de l'éternité et le droit
 d'entrer en Ta présence. Puissions-nous nous accrocher
 à cette Ancre qui est Jésus.
}{\Amen} 

\suggest{Cette ancre qu'est Jésus}


%%%%%%%%%%%%%%%
% 24 novembre
%%%%%%%%%%%%%%%

\dvday{Parfaitement}

\dvquote{
C'est pour cela aussi qu'Il peut sauver parfaitement ceux qui s'approchent
 de Dieu par Lui, étant toujours vivant pour intercéder en leur faveur.
}{\ibibleverse{He}(7:25)}

\lettrine{L}{a phrase~:} \punct{deux-points}
 \og Il peut sauver parfaitement \fg{} comporte quelques ramifications
 très importantes et nous en dit long sur la taille et l'ampleur
 de la puissance de Dieu. Nous apprenons que~: 

Il sauve des personnes de tous les peuples
 \ocadr Le salut de Dieu est inclusif.
 Il est capable de sauver des gens de toutes les nations,
 de toutes les races et de tous les groupes ethniques.
 Qui que vous soyez, d'où que vous veniez,
 vous pouvez être sauvés par notre Grand Prêtre, Jésus-Christ. 

\dvbox{
Vous avez besoin de salut. Jésus est capable de vous sauver.
}

Il sauve le pire des pécheurs \ocadr aussi horrible qu'ait été votre vie,
 aussi malfaisante qu'ait été votre conduite, vous n'êtes pas
 hors d'atteinte du pardon et de la grâce de Dieu.
 Petit pécheur, grand pécheur, peu importe.
 Il est assez grand pour vous sauver. 

Il sauve de circonstances extrêmes \ocadr que vous vous retrouviez poussés
 dans une fosse aux lions, jetés dans une fournaise ardente,
 abandonnés en prison, naufragés en pleine mer,
 traînés devant des magistrats hostiles ou exilés sur une île
 \ocadr Dieu est capable de vous délivrer.
 Il est possible qu'Il vous ait placés dans ces circonstances
 pour une raison précise; Il est possible qu'Il choisisse de vous garder
 dans cette situation pour un certain temps.
 Mais ne doutez jamais de Sa capacité à vous secourir.
 Il est le Dieu qui a le pouvoir de fermer la gueule des lions,
 de marcher avec vous dans la fournaise, de faire s'ouvrir d'elle-même
 la porte de la prison, de vous arracher aux grands fonds,
 de vaincre vos ennemis et de vous délivrer de l'exil. 

Il est le Dieu qui sauve parfaitement. 

\dvrule

\dvprayer{
Père, nous reconnaissons Ta majesté, Ta puissance, Ta force, Ta sagesse
 et Ta justice. Nous sommes si reconnaissants de T'appartenir.
 Merci d'être allé aussi loin pour nous sauver.
}{\Amen}


%%%%%%%%%%%%%%%
% 25 novembre
%%%%%%%%%%%%%%%

\dvday{Non Gravées sur la Pierre}

\dvquote{
Or voici l'alliance que j'établirai avec la maison d'Israël,
 après ces jours-là, dit le Seigneur;
 je mettrai mes lois dans leur intelligence,
 je les inscrirai aussi dans leur cœur;
 je serai leur Dieu et ils seront mon peuple.
}{\ibibleverse{He}(8:10)}

\lettrine{P}{arce qu'Il désirait une relation avec Son peuple,}
 Dieu a établi une alliance avec les enfants d'Israël.
 C'était une alliance merveilleuse et riche et\dots{} elle a échoué.
 Dieu a respecté Sa part du contrat, mais l'homme n'a pas respecté la sienne. 

\dvbox{
Dieu nous donne une nouvelle alliance
 \ocadr une volonté qui désire Lui plaire, Le servir, et Lui obéir.
}

Lors du dernier repas pris avec Ses disciples dans la chambre haute,
 Jésus \og prit du pain ; et après avoir rendu grâces, Il le rompit
 et le leur donna en disant~: Ceci est Mon corps, qui est donné pour vous;
 faites ceci en mémoire de Moi. De même Il prit la coupe, après le repas,
 et la leur donna, en disant~: Cette coupe est la nouvelle alliance
 en Mon sang, qui est répandu pour vous. \fg{} (\ibibleverse{Lc}(22:19-20)). 

Les règles de cette nouvelle alliance, données dans cette chambre haute,
 ne sont pas écrites extérieurement sur des tablettes de pierre;
 elles sont écrites dans nos cœurs. 

Quelle alliance bénie, quelle vérité magnifique~:
 Dieu écrit Ses lois sur nos cœurs, puis nous motive et nous donne
 la puissance de les garder. Que notre Dieu est grand ! 

\dvrule

\dvprayer{
Père, garde-nous complètement dépendants de Toi pour recevoir
 la force et le désir d'obéir à Tes lois.
 Nous te remercions de ce que l'alliance entre nous
 ne dépend pas de nous, mais de Toi. Nous T'aimons, Seigneur.
}{\Amen}


%%%%%%%%%%%%%%%
% 26 novembre
%%%%%%%%%%%%%%%

\dvday{Un Rendez-Vous Inévitable}

\dvquote{
Et comme il est réservé aux hommes de mourir une seule fois,
 après quoi vient le jugement\dots{}
}{\ibibleverse{He}(9:27)}

\lettrine{T}{out comme la mort,} le jugement de Dieu est inévitable.
 Quand nous cesserons de vivre dans ce corps et que nous rencontrerons
 Dieu à notre mort, c'est alors que notre destinée éternelle
 sera déterminée. Dieu est un Dieu, saint, juste et véritable,
 et seuls ceux qui sont saints et justes seront autorisés
 à partager l'éternité avec Lui. 

\dvbox{
Les hommes aimeraient vivre comme si aucun jugement ne les attendait.
}

L'incrédule aimerait croire qu'il n'aura jamais de comptes à rendre à Dieu
 sur la façon dont il aura vécu sa vie. Mais ceux qui pensent ainsi
 seront bien surpris. Parce que la vie n'est pas juste, nous avons besoin
 d'un jugement futur. Des hommes mauvais prospèrent dans cette vie,
 alors que les justes subissent des persécutions. Beaucoup de gens
 s'en sont sortis très bien en dépit de leurs mauvaises actions.
 Pour qu'il y ait équité, il faut qu'il y ait une justice future
 qui fasse payer à ces gens le mal qu'ils ont fait. 

Jean a dit~: \punct{deux-points}
 \og Cela, je vous l'ai écrit, afin que vous sachiez que vous avez
 la vie éternelle, vous qui croyez au nom du Fils de Dieu \fg{}
 (\ibibleverse{IJn}(5:13)).
 Êtes-vous prêts pour votre rendez-vous avec Dieu?
 Êtes-vous certains que vous allez passer l'éternité
 dans le royaume des cieux? 

\dvrule

\dvprayer{
Seigneur, nous prions pour ceux qui aujourd'hui vivent dans l'incertitude
 concernant le futur. Nous demandons que Ton Esprit se saisisse de leurs
 cœurs et les aide à être prêts quand ce rendez-vous arrivera.
}{\DlNdJ}


%%%%%%%%%%%%%%%
% 27 novembre
%%%%%%%%%%%%%%%

\dvday{Sacrifice de Louange}

\dvquote{
Par Lui, offrons sans cesse à Dieu un sacrifice de louange,
 c'est-à-dire le fruit de lèvres qui confessent Son Nom.
 Cependant, n'oubliez pas la bienfaisance et la libéralité,
 car c'est à de tels sacrifices que Dieu prend plaisir.
}{\ibibleverse{He}(13:15-16)}

\lettrine{S}{acrifice signifie donner quelque chose de valeur.}
 Je peux sacrifier de mon temps pour venir vous aider sur un projet.
 Je peux sacrifier le siège que j'occupe afin que quelqu'un d'autre
 puisse s'asseoir. 

L'Écriture parle beaucoup des sacrifices que nous devons offrir à Dieu.
 Nous sommes encouragés à Lui offrir le sacrifice de louange,
 le fruit de nos lèvres. Cela implique que quelquefois, nous n'avons pas envie
 de louer Dieu, mais que nous devons quand même le faire.
 À d'autres moments, notre louange est une réponse spontanée
 à la bonté de Dieu. Il m'arrive de me répandre en louange spontanément
 chaque fois que je contemple tout ce que Dieu a fait.
 Je pense aux bénédictions dans ma vie, je pense à vous et je pense
 à cette œuvre que Dieu a faite. Et mon cœur explose de louange. 

Il nous est aussi demandé de sacrifier nos biens matériels.
 Dieu prend plaisir à chaque fois que nous prenons soin les uns des autres,
 à chaque fois que nous voyons un frère dans le besoin et que nous donnons
 pour satisfaire ce besoin. Il est béni quand nous voyons les autres
 avec Ses yeux \ocadr quand nous remarquons les pauvres
 et que nous leur tendons une main de secours. 

\dvbox{
Puissiez-vous déborder de la joie et de la satisfaction de bénir votre Dieu
 en offrant des sacrifices de louange et de bonnes œuvres à votre Père. 
}

\dvrule

\dvprayer{
Père, nous demandons que Ton Saint-Esprit nous incite à T'offrir la louange,
 et que Tu nous donne beaucoup d'occasions de nous sacrifier pour ceux
 qui sont dans le besoin.
}{\NpDlNdJ}


%%%%%%%%%%%%%%%
% 28 novembre
%%%%%%%%%%%%%%%

\dvday{La Mise à l'Épreuve de Votre Foi}

\dvquote{
Mes frères, considérez comme un sujet de joie complète
 les diverses épreuves que vous pouvez rencontrer,
 sachant que la mise à l'épreuve de votre foi produit la patience.
}{\ibibleverse{Jc}(1:2-3)}

\lettrine{P}{endant la construction de la navette spatiale,}
 les ingénieurs de la NASA savaient qu'à son retour
 dans l'atmosphère terrestre, elle serait soumise à des chaleurs
 et des pressions extrêmes.
 Pour s'opposer à ce stress, ils ont conçu un système de tuiles
 résistantes à la chaleur qui allaient protéger la navette;
 mais ils ne pouvaient pas envoyer la navette dans l'espace
 sans avoir auparavant fait subir à ces tuiles des tests extrêmes.
 Ces tests n'étaient pas conçus pour détruire les tuiles mais,
 au contraire, pour vérifier leur résistance. 

De la même façon, Dieu conçoit des tests pour nous \ocadr des tests
 qui prouvent Sa force, Sa fidélité. Chaque test démontre
 le caractère de Dieu plus clairement pour nous. 

\dvbox{
Chaque test nous aide à compter sur Jésus, toujours un peu plus.
}

Il n'est pas facile d'éprouver de la joie pendant une épreuve,
 mais ce passage de l'Écriture nous dit que nous devons le faire.
 La joie vient de voir, au-delà de l'épreuve, le fruit qui en résultera.
 Paul a dit~: \punct{deux-points}
 \og Bien plus, nous nous glorifions même dans les tribulations,
 sachant que la tribulation produit la persévérance,
 la persévérance une fidélité éprouvée,
 et la fidélité éprouvée l'espérance \fg{} (\ibibleverse{Rm}(5:3-4)). 

Dieu veut vous amener à une maturité spirituelle et à une relation
 plus profonde avec Lui. Aussi la prochaine fois que vous vous trouvez
 à traverser une lourde épreuve, considérez-la \punct{tiret}
 comme un sujet de joie complète.
 Rendez-vous compte que vous vous approchez de l'accomplissement
 des desseins de Dieu en vous. 

\dvrule

\dvprayer{
Père, nous prions pour que Tu nous aides à nous réjouir dans nos épreuves.
 Seigneur, aide-nous à voir, au-delà des circonstances présentes adverses,
 le fruit de justice qui va venir dans nos vies en résultat
 d'avoir été prouvés fidèles.
}{\DlNdJ}


%%%%%%%%%%%%%%%
% 29 novembre
%%%%%%%%%%%%%%%

\dvday{La Foi et les Œuvres}

\dvquote{
Mes frères, à quoi bon dire qu'on a la foi,
 si l'on n'a pas les œuvres ?
 Cette foi peut-elle sauver ?
}{\ibibleverse{Jc}(2:14)}

\lettrine{I}{l faut deux rames pour ramer.}
 Si vous n'en avez qu'une, tout ce que vous arriverez à faire
 avec votre barque, sera de tourner en rond.
 Mais avec deux rames, vous pouvez aller partout où vous le voulez. 

La foi et les œuvres sont des compagnes qui travaillent ensemble.
 La foi produit des œuvres; les œuvres démontrent l'authenticité de la foi.
 Quoi que vous croyiez va se révéler dans les choses que vous ferez.
 Si quelqu'un, par exemple, s'avance dans une réunion pour recevoir
 Jésus-Christ, sa vie va être changée. Si des gens disent s'être avancés,
 mais qu'après avoir quitté la réunion, ils retournent aussitôt au péché
 auquel ils s'adonnaient, vous pouvez voir qu'ils n'ont pas vraiment
 reçu Christ. Leurs paroles ne voulaient rien dire. 

\dvbox{
La vraie foi se révèle dans l'action.
}

Jacques nous dit que \og il en est ainsi de la foi~: si elle n'a pas d'œuvres,
 elle est morte en elle-même \fg{} (\ibibleverse{Jc}(2:17)).
 Il nous donne ensuite l'exemple d'un frère ou d'une sœur dans le besoin.
 Si vous vous contentez de leur souhaiter du bien, mais ne les aidez pas
 à se nourrir ou se vêtir, vos souhaits de bien-être sont dépourvus de sens.
 Vous n'avez pas démontré de la vraie foi. 

La vraie foi produit des œuvres d'amour, de gentillesse et de bonté.
 Vous n'êtes pas sauvés par ces œuvres de justice \punct{virgule en trop};
 elles montrent simplement que vous êtes sauvés. 

La foi qui va vous sauver amènera le fruit de la justice dans votre vie.
 Rappelez-vous ce que Jésus a dit dans \ibibleverse{Mt}(7:16)~: \punct{deux-points}
 \og Vous les reconnaîtrez à leurs fruits. \fg{}

Votre vie produit-elle du bon fruit \ocadr du fruit qui prouve votre foi? 

\dvrule

\dvprayer{
Père, puissent nos actions toujours être en cohérence avec nos paroles.
 Aide-nous à demeurer en Toi, afin que la foi que nous avons produise
 des bonnes œuvres au travers de nous, des œuvres qui démontrent
 l'authenticité de notre engagement envers Jésus-Christ.
}{\ESN}

\suggest{En Son Nom}


%%%%%%%%%%%%%%%
% 30 novembre
%%%%%%%%%%%%%%%

\dvday{La Langue}

\dvquote{
De même, la langue est un petit membre, mais elle a de grandes prétentions.
 Voyez comme un petit feu peut embraser une grande forêt !
}{\ibibleverse{Jc}(3:5)} 

\lettrine{O}{h, les dégâts que la langue peut provoquer!}
 Un membre si petit qui pourtant peut faire tant de mal!
 Jacques nous dit que quand elle se donne un objectif de destruction,
 la langue peut être \og embrasée par l'enfer \fg{} (\ibiblechvs{Jc}(3:6)).
 Il ajoute que toute espèce de bêtes sauvages peuvent être domptées,
 \og mais la langue, aucun homme ne peut la dompter~: c'est un mal
 qu'on ne peut maîtriser ; elle est pleine d'un venin mortel \fg{}
 (\ibiblechvs{Jc}(3:8)).
 Combien de guerres ont commencé à cause de la langue?
 Combien de gens ont eu leur réputation détruite à cause de ragots?
 Combien de cœurs ont été brisés? 

\dvbox{
Combien de vies ont été brisées \ocadr par la seule faute de la langue? 
}

Il est mal d'utiliser la langue à des fins autres
 que celles prévues par Dieu, et il est mal d'utiliser la langue
 de façon incohérente \ocadr d'utiliser la même bouche pour bénir Dieu
 et pour maudire nos frères qui ont été créés à l'image de Dieu. 

La langue représente un tel potentiel pour le mal!
 Cependant, à l'inverse, la langue peut apporter beaucoup d'espoir
 et de réconfort. Elle peut encourager, fortifier et rassurer.
 Elle peut rappeler aux autres qu'ils comptent pour Dieu.
 Elle peut faire part de l'Évangile \suggest{Évangile}
 de Jésus-Christ à un monde qui se meurt.
 Si vous utilisez la langue de la façon voulue par Dieu,
 vous pouvez édifier les autres et les inciter à de bonnes œuvres.
 Vous pouvez amener le salut aux gens qui sont perdus. 

Que Dieu nous aide à utiliser nos langues dans des façons \suggest{de façons}
 qui Le glorifieront beaucoup. 

\dvrule

\dvprayer{
Père, aide-nous à veiller sur nos langues, et à utiliser nos paroles
 pour bénir et non pour maudire, pour édifier et non pour détruire.
}{\DlNdJ}


