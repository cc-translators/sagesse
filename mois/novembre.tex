\dvmonth{Novembre}

%%%%%%%%%%%%%%%
% 1er novembre
%%%%%%%%%%%%%%%

\dvday{Le Mystère du Mal}

\dvquote{
Car déjà le mystère du mal est à l'œuvre ;
 il faut seulement que celui qui le retient maintenant ait disparu.
}{\bibleverse{IITh}(2:7) \NBS}

\dvlettrine{P}{aul a écrit aux Thessaloniciens} pour les assurer du fait
 qu'en dépit des persécutions dont ils étaient victimes,
 ils n'étaient pas entrés dans la période de la grande tribulation.
 Deux choses devront arriver avant que ce jour n'arrive~:
 une défection de la foi arrivera dans l'Église \suggest{Église}
 et l'homme de péché sera révélé.

L'attraction du mal avait déjà commencé à l'époque de Paul.
 Et elle n'a fait qu'augmenter. Pourquoi \typo{Pourquoi}
 les gens sont-ils attirés
 par le mal alors qu'il amène toujours au désastre ?
 C'est bien là le mystère.

\dvbox{
Chaque jour des hommes rejettent Jésus-Christ
 \ocadr le seul vrai chemin vers le ciel. 
}

Il est inconcevable que des gens puissent délibérément choisir le mal
 plutôt que le bien \ocadr même s'ils savent que leur choix peut les tuer.
 Pourtant c'est ce qu'ils le font.
 Ça a commencé dans le jardin d'Eden, quand Adam et Eve ont choisi de manger
 de l'arbre dont Dieu avait annoncé qu'il les tuerait.
 C'est un mystère~: les hommes choisissent une chemin qui mène à la mort
 plutôt que le chemin qui conduit à la vie
 \ocadr un chemin qui ne va leur amener que malheur et destruction.
 Pourtant c'est ce qu'ils font.

Le Seigneur lance l'invitation, dans \bibleverse{Is}(1:18),
 à \og venir et raisonner ensemble. \fg{} L'obéissance est raisonnable.
 Accepter le salut que Jésus offre est raisonnable.
 Pourtant les hommes continuent  de jeter la raison aux orties,
 tout cela parce qu'ils aiment les ténèbres plus que la lumière,
 tout cela parce que le mal est pour eux plus attirant que l'obéissance.

Sondez votre cœur aujourd'hui. Demandez à Dieu de vous montrer
 si vous n'êtes pas en train de choisir le mal à l'obéissance.
 Et si vous trouvez que c'est le cas, repentez-vous!

\dvrule

\dvprayer{
Père, aide-nous à prendre en compte Ta Parole
 afin que nous ne soyons pas trompés.
 Puissions-nous remettre nos cœurs et nos vies à Jésus-Christ,
 le Seigneur vivant. 
}{\Amen}


%%%%%%%%%%%%%%%
% 2 novembre
%%%%%%%%%%%%%%%

\dvday{Sa Mission de Salut}

\suggest{Salut avec une majuscule}

\dvquote{
C'est une parole certaine et digne d'être entièrement reçue,
 que le Christ-Jésus est venu dans le monde
 pour sauver les pécheurs, dont je suis, moi,
 le premier.
}{\bibleverse{ITim}(1:15)}

\dvlettrine{U}{n jour où Jésus mangeait} avec des collecteurs d'impôts
 et d'autres pécheurs, Il entendit les Scribes et les Pharisiens
 critiquer le choix de ses compagnons de table.
 \og Jésus, qui avait entendu, leur dit~:
 Ce ne sont pas les bien-portants qui ont besoin de médecin,
 mais les malades. Je ne suis pas venu appeler des justes,
 mais des pécheurs \fg{} \punct{Point en trop}
 (\bibleverse{Mc}(2:17)).

\dvbox{
Nul n'est hors de la portée de l'amour de Dieu. 
}

Si vous vous portez bien, vous évitez le docteur.
 Mais quand vous êtes malades, vous savez que vous avez besoin d'aide.
 L'homme était malade. Il avait choisi un chemin mortel,
 un chemin qui mène à la maladie, la misère et la destruction.
 Dieu a vu la situation désespérée de l'homme et a eu pitié de lui
 en envoyant Son Fils, le Grand Médecin.
 Il n'a pas envoyé Jésus pour condamner le monde.
 Cela n'était pas nécessaire \typo{nécessaire},
 car le monde était déjà condamné. Il a envoyé Jésus
 \og pour que le monde soit sauvé par Lui \fg{} \punct{Point en trop}
 (\bibleverse{Jn}(3:17)).

Notez que Paul se désigne lui-même comme le premier des pécheurs.
 Il savait qui il était. Paul ne cherchait pas à excuser son passé.
 Le message qu'il voulait faire passer, c'est que si Jésus pouvait le sauver
 \ocadr après qu'il ait blasphémé Jésus, approuvé les crimes
 contre les Chrétiens, allant même jusqu'à les pourchasser
 et les tuer \fcadr{} alors le reste d'entre nous n'avons aucun souci
 à nous faire. Jésus va nous pardonner et nous sauver aussi.
 Celui qui est descendu au plus bas pour sauver les pires transgresseurs
 peut venir vous toucher dans la fosse de votre péché
 et vous amener dans Sa lumière.
 Nul n'est hors de la portée de l'amour de Dieu.

\dvrule

\dvprayer{
Père, merci d'avoir envoyé Ton Fils dans le monde
 pour pardonner et sauver ceux qui sont malades du péché. 
}{\DlNdJ}


%%%%%%%%%%%%%%%
% 3 novembre
%%%%%%%%%%%%%%%

\dvday{Un Seul Médiateur}

\dvquote{
Car il y a un seul Dieu, et aussi un seul médiateur entre Dieu,
 et les hommes, le Christ-Jésus homme,
 qui s'est donné lui-même en rançon pour tous~:
 c'est le témoignage rendu en temps voulu\dots{}
}{\bibleverse{ITim}(2:5-6)}

\dvlettrine{C}{'est à la fois arrogant et osé pour l'homme,}
 être limité et pécheur, que de penser qu'il peut se présenter
 effrontément devant le Dieu éternel et saint.
 Il est infiniment pur; nous sommes entachés par le péché.
 Il est lumière; nous demeurons dans les ténèbres.
 Aussi, comment l'homme pécheur peut-il jamais espérer
 se tenir debout devant Dieu?

Il existe une solution, mais une seule seulement.
 Job avait réclamé un pont, un Médiateur qui puisse poser Sa main
 à la fois sur Dieu et sur l'homme. Et nous avons un Médiateur en Jésus.
 Celui qui était Dieu est devenu homme afin qu'Il puisse se tenir
 sur la brèche entre Dieu et nous.

\dvbox{
Aucun saint ne peut vous sauver \ocadr seul Jésus peut vous sauver. 
}

Aucun saint ne peut se tenir sur la brèche pour vous.
 La Vierge Marie ne peut pas le faire pour vous.
 Peu importe combien cette personne a été sainte ou vertueuse,
 ce n'était qu'un homme; ce n'était qu'une femme.
 Vous pouvez adresser des prières à un saint mort toute la journée,
 ce saint restera totalement incapable de bâtir un pont sur le précipice
 qui vous sépare de Dieu. Seul Jésus peut le faire pour vous,
 parce que Lui seul peut à la fois vous toucher et toucher Dieu.
 Il est le seul vrai Médiateur que Dieu a envoyé pour nous amener à Lui.

Par la mort et la résurrection de Jésus-Christ, un chemin a été ouvert
 qu'il nous est possible d'emprunter pour nous approcher du Père infini,
 éternel, saint, pur et majestueux. \suggest{Y a-t-il un nouveau paragraphe ici?}
Nous nous tenons justes devant Lui grâce à la justice de notre Médiateur.

Louange à Dieu pour Son Fils!

\dvrule

\dvprayer{
Père, merci d'avoir offert ce que Job réclamait,
 un médiateur qui puisse nous toucher tous les deux.
Merci pour l'espérance de la vie éternelle. 
}{\DlNdJ}


%%%%%%%%%%%%%%%
% 4 novembre
%%%%%%%%%%%%%%%

\dvday{Le Mystère}

\dvquote{
\dots{} le mystère de la piété est grand~:
 Celui qui a été manifesté en chair, justifié en Esprit,
 est apparu aux anges, a été prêché parmi les nations,
 a été cru dans le monde, a été élevé dans la gloire.
}{\bibleverse{ITim}(3:16)}

\punct{Point et espace manquants}

\dvlettrine{I}{l existe un mystère formidable} relatif à la piété
 \ocadr le fait que l'homme devient comme son dieu.
 Parlant des idoles muettes, sourdes, aveugles, immobiles des païens,
 le psalmiste déclare~: \punct{deux-points}
 \og Ils leur ressemblent, ceux qui les fabriquent,
 Tous ceux qui se confient en elles \fg{} (\bibleverse{Ps}(115:8)).
 C'est une vérité psychologique de base;
 un homme devient comme son dieu.
 Si votre dieu est haineux et amer, alors vous devenez haineux et amer.

\dvbox{
Nous, qui suivons Dieu, sommes transformés de jour en jour
 pour devenir comme Lui. 
}

La Bible nous dit que \og nous \grammar{Pas besoin de majuscule, la phrase continue}
 sommes maintenant enfants de Dieu, et ce que nous serons n'a pas encore
 été manifesté ; mais nous savons que lorsqu'Il sera manifesté,
 nous serons semblables à lui, parce que nous le verrons
 tel qu'Il est \fg{} (\bibleverse{IJn}(3:2)).

Comme j'aime voir ce mystère se dévoiler dans ma propre vie,
 pour voir les changements qu'Il effectue jour après jour
 alors que je marche à Sa suite en Le servant.

David a dit~: \punct{deux-points}
 \og Dès le réveil, je me rassasierai de Ton image. \fg{}
 Et un jour je m'éveillerai et je me retrouverai au ciel
 et je serai juste comme Lui. C'est le mystère de la piété~:
 la vie transformée conformée à Son image.

\dvrule

\dvprayer{
Père, Merci à Toi pour la puissance de Ton Esprit à transformer
 la vie d'une personne et de la conformer à Ton image.
 Sois à l'œuvre dans nos cœurs et nos vies aujourd'hui.
 Aide-nous à céder sous Tes mains,
 sans jamais résister à ce que Tu veux faire en nous.
}{\DlNdJ}


%%%%%%%%%%%%%%%
% 5 novembre
%%%%%%%%%%%%%%%

\dvday{Sois un Modèle}

\dvquote{
Que personne ne méprise ta jeunesse ;
 mais sois un modèle pour les fidèles, en parole, en conduite, en esprit,
 en amour, en foi, en pureté.
}{\bibleverse{ITim}(4:12)}

\dvlettrine{P}{arce que Timothée était jeune,} certains dans l'église
 le regardaient de haut et refusaient de recevoir de lui.
 Aussi Paul a t-il écrit à Timothée en lui disant~: \punct{deux-points}
 \og C'est simple, il te suffit d'être un modèle. \fg{}
 Paul a souligné six domaines dans lesquels Timothée
 devait être un modèle pour les fidèles~:

\suggest{Je serais curieux de voir la mise en page en anglais}

\emph{En parole} \ocadr
 Ceci pouvait être compris de deux façons.
 D'abord, Paul a pu vouloir dire~: \punct{deux-points}
 \og Sois un modèle dans ton langage. \fg{}
 Mais il aurait aussi pu vouloir dire~: \punct{deux-points}
 \og Sois un modèle dans ta connaissance et ta compréhension des Écritures
 \ocadr sois un homme de la Parole. \fg{} Les deux sont importants.

\emph{En conduite} \ocadr
 Que ton style de vie sois un exemple de ce qu'est un croyant.
 Sois un modèle de Christ dans tes actions et tes attitudes.

\emph{En amour} \ocadr
 L'amour que Paul a décrit dans \bibleverse{ICo}(13:) \typo{s à Corinthiens}
 est l'amour qui devrait émaner de la vie de chaque croyant.

\emph{En esprit} \ocadr
 Certaines personnes ont un esprit doux; d'autres ont un esprit mesquin.
 Il n'y a pas de place pour la mesquinerie parmi les croyants.

\emph{En foi} \ocadr
 Ceci peut aussi vouloir dire l'une de deux choses.
 Ou bien nous devons être un modèle dans notre confiance en Dieu,
 ou dans notre propre capacité à être dignes de confiance, ou les deux.

\emph{En pureté} \ocadr
 Timothée était jeune, non marié, et vivait dans une société païenne
 et corrompue. Paul l'exhortait à vivre une vie de pureté,
 une vie au-dessus de tout reproche.

\dvbox{
Le message de Paul est pour tous les âges.
 Nous devons être des modèles de Christ à la fois pour l'Église
 et pour un monde qui s'interroge sur le christianisme. 
}

\typo{modèles}

\dvrule

\dvprayer{
Père, aide-nous à être un exemple digne de Toi dans le monde,
 afin que nous puissions Te porter gloire.
}{\Amen}


%%%%%%%%%%%%%%%
% 6 novembre
%%%%%%%%%%%%%%%

\dvday{Le Péché du Préjugé}

\dvquote{
Je te conjure devant Dieu, devant le Christ-Jésus et devant les anges élus,
 d'observer ces règles sans préjugé et de ne rien faire par favoritisme.
}{\bibleverse{ITim}(5:21)}

\dvlettrine{P}{aul a donné des instructions à Timothée}
 sur des sujets tels que, la prise en charge des veuves,
 sa relation avec les anciens de l'église, des règles pour les jeunes femmes,
 et ainsi de suite. Puis il dit~: \punct{deux-points}
 \og Timothée, \punct{virgule}
 tu ne dois pas montrer de traitement préférentiel ou de partialité.
 Sois bien sûr d'observer ces règles sans préférence pour l'un plutôt
 que pour l'autre. \fg{}

Dans toute la Bible, il nous est dit que Dieu est impartial.
 Mais malheureusement, nous, nous faisons souvent preuve de partialité.
 Nous sommes enclins à honorer le riche et à plutôt escamoter le pauvre,
 mais Dieu n'est pas comme ça. Il est tout aussi soucieux de sauver l'âme
 du plus pauvre des hommes à la surface de la terre qu'Il l'est de l'homme
 le plus riche. La situation sociale ne signifie rien pour Dieu,
 nous existons tous au même niveau et au même plan. 

\dvbox{
Si nous nous réclamons de Son Nom, nous devons développer
 Sa perspective sur les préjugés.
}

Peu importe combien de succès vous avez remportés, combien de richesses
 vous avez accumulées, ou qui vous êtes en termes de classement par le monde.
 Peu importe que vous habitiez dans un château ou dans une cabane.
 Indépendamment de la façon dont vous êtes vus par tout le monde,
 vous comptez pour Jésus. 

Nous devons trouver une façon de gommer les différences entre nous.
 Comment pouvons-nous atteindre un monde perdu si nous ne voyons pas
 ceux qui le composent de la façon dont Jésus les voit?
 Nous devons apprendre à aimer comme Il a aimé, et à apprécier les autres
 parce qu'ils comptent tant pour Lui.

\dvrule

\dvprayer{
Père, Merci pour Ton grand amour pour nous.
 Aide-nous à aimer avec Ton amour
 \ocadr sans préférence ni partialité.
}{\DlNdJ}

\punct{tiret dans « Aide-nous »}


%%%%%%%%%%%%%%%
% 7 novembre
%%%%%%%%%%%%%%%

\dvday{Les Vraies Richesses}

\dvquote{
Certes, c'est une grande source de gain que la piété,
 si l'on se contente de ce qu'on a.
}{\bibleverse{ITim}(6:6)}

\dvlettrine{L}{es gens disent souvent~:} \punct{deux-points}
 \og Oh si seulement je pouvais gagner au loto! \fg{}
 Mais vous risquez d'être surpris. Des études ont montré
 que très fréquemment, ces gros lots fabuleux gâchent complètement
 la vie des gagnants. Ils passent d'une vie d'insouciance
 à une vie remplie d'anxiété. Bien qu'ils aient pensé
 que l'argent allait leur apporter un grand bonheur,
 dans beaucoup de cas il ne leur apporte que du chagrin.

Les gens se serrent la ceinture et suent à grosses gouttes
 pour amasser leurs richesses, mais au jour du jugement
 ces richesses n'auront aucune valeur, parce qu'il n'est pas possible
 d'acheter Dieu. Les vraies richesses sont, par contre, éternelles.
 Elles durent pour toujours; rien ne peut les diminuer.
 Vous pouvez être pauvres dans ce monde et être, cependant,
 un héritier dans le royaume de Dieu.

\dvbox{
Les richesses terrestres constituent une complète imposture. 
}

Jérémie a dit~: \punct{deux-points}
 \og Ainsi parle l'Éternel~: Que le sage ne se glorifie pas de sa sagesse,
 que le fort ne se glorifie pas de sa force, que le riche ne se glorifie pas
 de sa richesse, mais que celui qui veut se glorifier se glorifie d'avoir
 de l'intelligence et de Me connaître, de savoir que je suis l'Éternel,
 qui exerce la bienveillance, le droit et la justice sur la terre;
 Car c'est à cela que je prends plaisir \fg{} \punct{Point après la référence}
 (\bibleverse{Jr}(9:22-23)).

\og J'ai appris à me contenter de l'état où je me trouve, disait Paul.
 \punct{Pas besoin de répéter les guillemets ici}
 Je sais vivre dans l'humiliation, et je sais vivre dans l'abondance.
 En tout et partout, j'ai appris à être rassasié et à avoir faim,
 à être dans l'abondance et à être dans la disette \fg{} \punct{Point après la référence}
 (\bibleverse{Ph}(4:11-12)).

Au lieu de nous évertuer à accumuler toujours davantage, apprenons la valeur
 de savoir se contenter de ce qu'on a. 

\dvrule

\dvprayer{
Père, nous Te remercions de nous avoir offert des richesses aussi vastes.
 Apprends-nous à être satisfaits de ce que Tu nous a donné.
}{\DlNdJ}


%%%%%%%%%%%%%%%
% 8 novembre
%%%%%%%%%%%%%%%

\dvday{La Mort est Abolie}

\dvquote{
\dots{} notre Sauveur Christ-Jésus, qui a réduit à l'impuissance la mort
 et mis en lumière la vie et l'incorruptibilité par l'Évangile\dots{}
}{\bibleverse{IITim}(1:10)}

\dvlettrine{U}{ne chose que vous pouvez dire sur la mort,}
 c'est qu'elle est certaine.
 Jusqu'à présent les statistiques sont impressionnantes~:
 cent pour cent des gens meurent. Mais voici un phénomène intéressant~:
 si vous êtes nés une fois, vous mourrez deux fois;
 si vous êtes nés deux fois, vous ne mourrez qu'une seule fois.

À notre mort physique, nous serons transformés.
 Nous subirons une métamorphose. C'est nécessaire, parce qu'en ce moment
 nous existons dans des corps corruptibles. Mais pour exister au ciel,
 nous avons besoin de corps incorruptibles. Et quand ces tentes
 seront dissoutes, quand nos corps retourneront à la poussière,
 nous entrerons dans notre existence éternelle avec Dieu.
 Quelle glorieuse promesse ! Quelle espérance bénie ! 

\dvbox{
Notre éternité avec Dieu~: quelle glorieuse promesse ! 
}

Un de ces jours, il se peut qu'en prenant le journal,
 vous lisiez~: \punct{deux-points}
 \og Chuck Smith, pasteur de Calvary Chapel est mort. \fg{}
 N'allez surtout pas en croire un mot ! Si vous voyez ça,
 sachez avec certitude que je ne serai pas mort \ocadr je serai simplement
 sorti d'une vieille tente usée et entré dans une belle demeure,
 \og \dots{} \punct{Un point en trop}
 dans les cieux, un édifice qui est l'ouvrage de Dieu, une demeure éternelle
 qui n'a pas été faite par la main des hommes \fg{} (\bibleverse{IICo}(5:1)).
 Aussi, s'il vous plaît, n'allez pas pleurer pour moi
 \grammar{pleurer pour moi}, parce que vous savez que je ne pleurerai pas.
 \og Il y a abondance de joies devant Ta face, des délices éternelles
 à Ta droite \fg{} (\bibleverse{Ps}(16:11)). 

Dieu veut être en communion avec vous, et Il a envoyé Son Fils pour abolir
 la mort afin que vous puissiez avoir la vie éternelle
 et l'immortalité par Lui. Et c'est la vérité de l'Évangile.
 \suggest{Évangile avec une majuscule}

\dvrule

\dvprayer{
Père, nous Te remercions pour cette vie et pour l'immortalité
 qui est à nous par Jésus-Christ. Nous prions pour ceux
 qui ne Te connaissent pas et Te demandons de parler à leurs cœurs.
}{\DlNdJ}


%%%%%%%%%%%%%%%
% 9 novembre
%%%%%%%%%%%%%%%

\dvday{Vases d\ap{}Honneur}

\dvquote{
Si un homme donc se purifie de ces choses, il sera un vase d’honneur,
 sanctifié, et propre au service de son maître,
 et préparé pour toute bonne œuvre.
}{\bibleverse{IITim}(2:21) \KJF}

\punct{Point manquant}

\dvlettrine{L}{es vases étaient très communs} dans les temps bibliques.
 Certains, faits d'or ou d'argent, étaient utilisés dans un but décoratif.
 D'autres étaient faits d'argile et utilisés à des fins diverses
 allant de transporter de l'eau jusqu'à contenir des ordures
 ou des eaux usées. Ainsi, certains étaient appelés des vases d'honneur
 et d'autres, des vases de déshonneur.

Prenez un moment et demandez-vous~: \punct{deux-points}
 \og Que contient ma vie? \fg \punct{guillemets fermants manquants}
 Les choses qui remplissent votre vie sont-elles
 pures ou sont-elles souillées comme des eaux usées? 

\dvbox{
Vous avez été créés pour contenir Dieu. 
}

Dieu a choisi de remplir un pot d'argile \ocadr vous \fcadr{}
 de la chose la plus précieuse qui existe \ocadr Lui-Même.
 Il veut vous remplir de Lui-Même afin que vous débordiez de Sa grâce,
 de Sa miséricorde et de Son amour pour désaltérer un monde assoiffé.

Paul a écrit à Timothée au sujet d'hommes qui, dans l'église,
 corrompaient les gens avec un enseignement erroné.
 Ces hommes étaient devenus des vases de déshonneur.
 Dieu ne pouvait pas les utiliser; leur doctrine était impure.
 Ils enseignaient leurs propres idées au lieu des vérités de Dieu,
 et de cette façon, souillaient le contenu de leurs vies
 de leur propre parfum. 

Comme il est facile de devenir un vase de déshonneur !
 Tout ce qu'il suffit de faire est de cesser de purifier ses pensées
 chaque jour. Mais si vous voulez être un vase d'honneur,
 prêt à l'usage du Maître, alors vous devez fuir les mauvais désirs
 de la chair. Il vous faut communier avec Dieu à la fin de chaque journée
 et Lui demander de vous purifier de nouveau. 

\dvrule

\dvprayer{
Dieu, fais de nous des vases d'honneur, afin que nous puissions déverser
 Ta grâce sur le monde qui nous entoure.
 Purifie-nous et utilise-nous, Seigneur. 
}{\Amen}


%%%%%%%%%%%%%%%
% 10 novembre
%%%%%%%%%%%%%%%

\dvday{Inspirée par Dieu}

\dvquote{
Toute Écriture est inspirée de Dieu et utile pour enseigner,
 pour convaincre, pour redresser, pour éduquer dans la justice\dots{}
}{\bibleverse{IITim}(3:16)}

\dvlettrine{L}{a Bible parle du Dieu éternel qui a créé l'univers,}
 du Dieu qui existe en dehors du temps et de l'espace.
 Elle nous dit que ce Dieu qui a créé toutes choses,
 aime l'homme et désire vivre en communion avec lui,
 afin que l'homme puisse recevoir les bénédictions de Le connaître.
 Les Écritures sont la Parole de Dieu adressée à l'homme.

\dvbox{
Parce que la Bible est inspirée par Dieu, elle est infaillible et inerrante.
}

Par les prophéties, Dieu prouve qu'Il est en dehors du temps.
 Lui seul connaît la fin depuis le commencement.
 Lui seul peut prédire une chose avant qu'elle n'arrive.
 En prophétisant des choses encore futures, Il prouve qu'Il est vraiment
 l'Auteur de ce Livre, qui est composé à quatre-vingt pour cent de prophétie. 

Quand Jésus parlait à Ses disciples de Sa mort et de sa résurrection,
 Il leur a dit~: \punct{deux-points}
 \og Je m'en vais et je reviendrai vers vous\dots{}
 Je vais vers le Père\dots{} Je vous ai dit ces choses maintenant,
 avant qu'elles n'arrivent, afin que, lorsqu'elles arriveront,
 vous croyiez \fg{} (\bibleverse{Jn}(14:28-29)).
 Il leur a annoncé à l'avance les choses à venir afin qu'ils puissent
 voir qu'Il était vraiment qui Il disait être \ocadr le Fils de Dieu. 

Dieu nous a donné Sa Parole afin que nous comprenions Sa nature,
 Son plan de rédemption et le chemin de la croissance spirituelle.
 Les vérités qu'elle contient vont bâtir votre foi.
 Puissions-nous l'étudier quotidiennement, en pratiquant la Parole
 et en ne l'écoutant pas seulement.

\dvrule

\dvprayer{
Merci, Père, pour Ta Parole, une lumière sur notre sentier
 qui nous guide dans Ta vérité. 
}{\DlNdJ}


%%%%%%%%%%%%%%%
% 11 novembre
%%%%%%%%%%%%%%%

\dvday{Le Seigneur m\ap{}a assisté}

\dvquote{
C'est le Seigneur qui m'a assisté et qui m'a fortifié,
 afin que la prédication soit portée par moi à sa plénitude
 et entendue de tous les païens.
}{\bibleverse{IITim}(4:17)}

\dvlettrine{A}{u premier procès de Paul devant Néron,}
 sa situation semblait si sombre que tous ses compagnons l'ont abandonné.
 \og Dans ma première défense, personne ne m'a assisté,
 \punct{Pas besoin de répéter les guillemets}
 disait-il, mais tous m'ont abandonné \fg{}
 (\bibleverse{IITim}(4:16)).

Comme il est horrible de se sentir abandonné !
 La Bible dit que \og l'ami \typo{Pas besoin de majuscule}
 aime en tout temps \fg{} (\bibleverse{Pr}(17:17)), mais trop souvent,
 nous découvrons que ceux que nous pensions être de vrais amis,
 n'étaient en fait que des connaissances.
 Même s'il est possible que tous les autres vous laissent tomber,
 Jésus va toujours se tenir à vos côtés.
 Le Seigneur ne va pas abandonner Son peuple.
 Comme David le disait~: \punct{deux-points}
 \og Quand je marche dans la vallée de l'ombre de la mort,
 je ne crains aucun mal, car Tu es avec moi \fg{} (\bibleverse{Ps}(23:4)).
 \punct{Point manquant}

\dvbox{
Tant que je sais que le Seigneur m'assiste,
 je peux faire face aux épreuves du lendemain et aux incertitudes du futur.
}

Aujourd'hui, vous vous sentez peut-être comme Paul \ocadr délaissé,
 tout seul, faisant face à un futur incertain.
 Dans cette nuit sombre, vous pouvez être certains que Jésus
 se tient à vos côtés.
 \og Et voici, je suis avec vous tous les jours, jusqu'à la fin du monde \fg{}
 (\bibleverse{Mt}(28:20)).
 Il a dit~: \punct{deux-points}
 \og Je ne te délaisserai pas ni ne t'abandonnerai \fg{}
 (\bibleverse{He}(13:5)). 

Comme quelqu'un l'a dit~: \punct{deux-points}
 \og Je ne connais \suggest{Ou je ne sais}
 pas ce que demain apportera, mais je connais Celui qui apportera demain,
 et je sais Qui me tient la main. \fg{}
 Il se tiendra à mes côtés et m'aidera à traverser l'épreuve. 

\dvrule

\dvprayer{
Père, nous te remercions que nous avons l'assurance si merveilleuse
 que Tu seras à nos côtés. Soutiens-nous, quand nous pensons
 que nous n'en pouvons plus. Soutiens-nous de Ta puissante main droite. 
}{\DlNdJ}


%%%%%%%%%%%%%%%
% 12 novembre
%%%%%%%%%%%%%%%

\dvday{Professer ou Pratiquer ?}

\dvquote{
Ils font profession de connaître Dieu, mais ils le renient par leurs œuvres.
 Ils sont odieux, rebelles et incapables d'aucune œuvre bonne.
}{\bibleverse{Tt}(1:16)}

\dvlettrine{B}{eaucoup de gens pensent que,}
 parce qu'ils disent en se couchant une petite prière
 du genre~: \punct{deux-points}
 \og Je mets ma tête sur l'oreiller, et je te prie, Seigneur,
 sur mon âme de veiller \fg{}, cela signifie qu'ils connaissent Dieu.
 Comme Jésus le disait~: \punct{deux-points}
 \og Pourquoi m'appelez-vous~: Seigneur, Seigneur !
 et ne faites-vous pas ce que je dis? \fg{} (\bibleverse{Lc}(6:46)).
 \punct{Point manquant} 

À votre avis, qu'est-ce qui intéresse le plus Dieu?
 Que vous professiez ou que vous pratiquiez?
 Jean-Baptiste a dit~: \punct{deux-points}
 \og Produisez donc des fruits dignes de la repentance \fg{}
 (\bibleverse{Lc}(3:8)).
 Autrement dit~: \punct{deux-points}
 \og Que votre vie soit en cohérence avec vos paroles. \fg{}

\dvbox{
C'est une chose que de connaître des choses sur Dieu,
 c'en est une toute autre que de connaître Dieu.
}

Dire que vous croyez en Dieu n'est pas suffisant;
 vous devez faire suivre vos paroles par une vie en cohérence avec Dieu.
 Les mots ne veulent rien dire par eux-mêmes. Vous pouvez dire n'importe quoi.
 Mais \typo{Mais} comment vivez-vous? 

Paul nous encourage à nous examiner nous-mêmes,
 il a dit en effet~: \punct{deux-points}
 \og Si nous nous jugions nous-mêmes, nous ne serions pas jugés \fg{}
 (\bibleverse{ICo}(11:31)).
 Est-il possible que vous soyez un de ceux dont Paul parlait
 \ocadr ceux qui sont dans l'église et professent connaître Dieu, mais qui,
 en fait, Le renient par leurs œuvres? Avez-vous laissé d'autres dieux
 supplanter votre amour pour Lui?
 A-t-Il vraiment la première place dans votre vie? 

Ne vous contentez pas seulement de professer votre foi
 \ocadr mettez-la \typo{pas de majuscule après un tiret long}
 aussi en pratique ! 

\dvrule

\dvprayer{
Seigneur, montre-moi si, dans mon travail, je Te renie.
 Montre-moi s'il y a des incohérences entre mes propos et ma pratique.
 Aide-moi, Seigneur, à Te connaître et Te servir en vérité. 
}{\DlNdJ}


%%%%%%%%%%%%%%%
% 13 novembre
%%%%%%%%%%%%%%%

\dvday{La Bienheureuse Espérance}

\dvquote{
En attendant la bienheureuse espérance et la manifestation
 de la gloire de notre grand Dieu et Sauveur, le Christ-Jésus.
}{\bibleverse{Tt}(2:13)}

\dvlettrine{L}{'espérance du croyant,} c'est la vie éternelle avec Dieu.
 Pour nous, la mort n'est pas la fin; c'est simplement une métamorphose.
 C'est simplement un changement de corps. Mais si votre espérance n'est pas
 en Jésus-Christ pour la vie éternelle, alors comme Paul l'a dit, vous êtes
 \og sans Christ\dots{} sans espérance et sans Dieu dans le monde \fg{}
 (\bibleverse{Eph}(2:12)). 

\dvbox{
Si vous n'avez pas Christ, alors vous n'avez pas Dieu\dots{}
 et vous n'avez pas d'espoir.
}

Nous qui sommes croyants, espérons non seulement en la vie éternelle,
 mais nous attendons aussi la manifestation glorieuse de Jésus-Christ.
 Il a promis qu'Il reviendrait. Il a dit à Ses disciples~: \punct{deux-points}
 \og Que votre cœur ne se trouble pas. Croyez en Dieu, croyez aussi en Moi.
 Il y a beaucoup de demeures dans la maison de mon Père.
 Sinon, je vous l'aurais dit ; car Je vais vous préparer une place.
 Donc, si Je m'en vais et vous prépare une place, Je reviendrai
 et Je vous prendrai avec moi, afin que là où Je suis,
 vous y soyez aussi \fg{} (\bibleverse{Jn}(14:1-3)).
 Et ainsi nous attendons avec espoir la glorieuse manifestation
 de notre Seigneur Jésus-Christ. 

Jésus nous a indiqué beaucoup des signes de Son retour,
 des choses que nous pourrions surveiller pour savoir si ce retour est proche.
 Nos espoirs se renforcent quand nous observons le monde aujourd'hui
 et voyons que toutes les conditions sont réunies \ocadr tout ce dont la Bible
 parle comme devant arriver avant ce jour.
 Plus le monde devient sombre et plus l'espoir de l'enfant de Dieu
 est brillant. 

\dvrule

\dvprayer{
Père, nous Te remercions de ce que dans un monde assombri
 par la détresse et le désespoir,
 nous avons une glorieuse espérance qui nous soutient.
}{\DlNdJ}


%%%%%%%%%%%%%%%
% 14 novembre
%%%%%%%%%%%%%%%

\dvday{Justifiés !}

\dvquote{
\dots{} afin que, justifiés par sa grâce, nous devenions héritiers
 dans l'espérance de la vie éternelle.
}{\bibleverse{Tt}(3:7)}

\dvlettrine{P}{ar Sa miséricorde,} Dieu vous pardonne de vos péchés.
 \suggest{vous pardonne vos péchés}
 Mais par Sa grâce, Il vous justifie, et cela signifie qu'Il a décidé
 de laisser tomber les accusations contre vous. Il a effacé l'ardoise.
 \og Il n'y a donc maintenant aucune condamnation pour ceux
 qui sont en Christ-Jésus \fg{} (\bibleverse{Rm}(8:1)).
 Parce que vous avez reçu Jésus-Christ comme votre Seigneur,
 quand Dieu vous regarde aujourd'hui, Il vous voit comme quelqu'un
 d'absolument innocent. Ainsi, alors que la justice c'est recevoir
 ce qu'on mérite, et que la miséricorde c'est ne pas recevoir
 ce qu'on mérite, la grâce, c'est recevoir ce qu'on ne mérite pas. 

\dvbox{
Étant justifié par Sa grâce, je suis maintenant un enfant de Dieu.
}

Je ne mérite pas l'amour de Dieu. Je ne mérite pas l'effacement total
 de mes péchés, ni la vie éternelle dans Son royaume.
 Mais ce sont des choses que Dieu me donne en raison de la justification
 qui est venue pas Sa grâce. Je mérite la mort; par grâce,
 Dieu me donne la vie. Je mérite l'enfer; par grâce,
 Dieu m'ouvre les portes du ciel. 

En tant que Son enfant, je deviens un héritier de Dieu.
 Le monde compte beaucoup de millionnaires et même des milliardaires.
 Mais la richesse accumulée par ces gens n'est absolument rien,
 comparée aux richesses de Dieu. Nous, qui appartenons à Dieu,
 sommes plus riches que les gens les plus riches sur terre,
 parce que la gloire éternelle du royaume de Dieu nous appartient.
 Dans les temps sans fin à venir, nous qui sommes les enfants de Dieu,
 jouirons des richesses du royaume, de la beauté de Sa présence,
 de la majesté de Son amour et du réconfort de Sa grâce
 et de Sa miséricorde \ocadr pour toujours. 

\dvrule

\dvprayer{
Père, Merci de ce que, par Ta grâce tu as effacé l'ardoise.
 Oh, comme nous sommes heureux en raison de Ta grâce manifestée envers nous!
}{\DlNdJ}

