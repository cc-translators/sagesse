\dvmonth{Juillet}

%%%%%%%%%%%%%%
% 1er juillet
%%%%%%%%%%%%%%

\dvday{Prisonniers de l\ap{}Espérance}

\dvquote{
Retournez à la forteresse, vous prisonniers de l'espérance.
 Aujourd'hui même, je le déclare, je te rendrai le double.
}{\ibibleverse{Za}(9:12) \KJF}

\lettrine{D}{ans le verset précédent,} Zacharie s'est adressé à ceux
 qui étaient emprisonnés dans une fosse où il n'y avait pas d'eau.
 Maintenant, il s'adresse à ceux qui sont prisonniers de l'espérance. 

Le croyant est comme \og étreint par \fg{} ou \og prisonnier de \fg{}
 l'espérance de la venue du Messie. Nous ne pouvons pas y échapper.
 Nous ne voulons pas y échapper. C'est ce à quoi nous aspirons,
 c'est ce que nous attendons. Nous savons que quand Jésus reviendra,
 les souffrances de l'humanité cesseront enfin. Les guerres cesseront.
 Les chagrins cesseront. Il gouvernera et règnera dans la paix et la justice.
 Et Il nous libèrera de la fosse où nous nous trouvons peut-être aujourd'hui. 

\dvbox{
Nous nous raccrochons à l'espérance qu'Il va revenir bientôt,
 qu'Il va établir son royaume et qu'Il va amener la justice et la paix
 sur une terre qui en a un besoin criant. 
}

Oh, combien l'espérance est indispensable! Combien nous en avons besoin
 pour nous soutenir. Dans un monde qui s'effondre, un monde dans lequel
 le cœur des hommes leur fait défaut devant la peur,
 qu'il est merveilleux d'avoir cette espérance en Jésus-Christ! 

L'Éternel promet à ceux qui sont prisonniers de l'espérance qu'Il leur rendra
 le double de ce qu'ils ont perdu. Il leur explique dans la dernière partie
 du chapitre comment Il va soumettre leurs ennemis, comment Il va glorifier
 Son peuple et comment Il va en faire comme les pierres d'une couronne,
 \og élevée comme une bannière sur Sa terre. \fg{} (\ibibleverse{Za}(9:16)). 

Nous vivons des temps d'une importance capitale, où les événements
 se conjuguent pour annoncer le retour de Jésus. Aussi longtemps que Dieu
 nous prête vie, nous devons avertir les autres des jours mauvais à venir,
 mais nous devons aussi proclamer le jour glorieux qui arrive
 \ocadr le jour du retour du Seigneur. 

Reviens vite, Seigneur Jésus! 

\dvrule

\dvprayer{
Père, nos regards et nos espérances sont fixés sur Toi.
 Nous avons hâte de voir Ton retour; nous attendons avec anticipation
 le jour où Tu vas venir établir Ton royaume. 
}{\NpDlNdJ}


%%%%%%%%%%%%%%
% 2 juillet
%%%%%%%%%%%%%%

\dvday{Une Alliance devant Dieu}

\dvquote{
Et vous dites~: Pourquoi ? \dots{} Parce que l'Éternel a été témoin
 entre toi et la femme de ta jeunesse que tu as trahie,
 bien qu'elle soit ta compagne et la femme de ton alliance.
}{\ibibleverse{Ml}(2:14)}

\lettrine{C}{e n'était pas faute d'essayer.} Les Israëlites arrosaient
 leurs offrandes de larmes, mais le Seigneur refusait de les accepter.
 Quand ils ont demandé \og Pourquoi? \fg{}, Dieu leur a dit que c'était
 à cause de la façon dont ils traitaient leurs épouses. 

Ils avaient rendu le divorce trop facile. Si un homme s'amourachait
 d'une jolie jeune fille, il pouvait abandonner son épouse en déclarant
 simplement~: \punct{deux-points, majuscule} \og Je divorce d'avec toi \fg{}
 trois fois de suite. Et alors, la compagne de sa jeunesse,
 la femme qui l'avait soutenu pendant les années de vaches maigres
 et avait consenti des sacrifices pour l'aider à démarrer dans la vie,
 la femme qui avait porté ses enfants et tenu sa maison,
 n'avait pas d'autre choix que de s'en aller. 

\dvbox{
Dieu déteste le divorce.
}

Dieu détestait le divorce alors, et Il le déteste tout autant maintenant.
 Et j'ai appris que si Dieu déteste quelque chose,
 je ferais bien de m'assurer que je la déteste aussi. 

L'Écriture dit aux hommes que nous devons aimer notre femme même encore plus
 que nous-mêmes. Nous devons l'aimer comme \og Christ a aimé l'Église
 et s'est livré Lui-même pour elle \fg{} (\ibibleverse{Ep}(5:25)). 

Un grande pression s'exerce aujourd'hui contre le mariage.
 Puissions-nous, nous qui sommes mariés, nous rappeler que nous avons fait
 une alliance l'un avec l'autre devant Dieu, \og pour le meilleur
 et pour le pire, dans la richesse ou dans la pauvreté, dans la maladie
 et la santé, de nous aimer et nous chérir
 jusqu'à ce que la mort nous sépare. \fg{}

\dvrule

\dvprayer{
Père, aide-nous à nous aimer l'un l'autre comme Tu nous a aimés,
 et à nous traiter l'un l'autre avec compassion et gentillesse. 
}{\DlNdJ}


%%%%%%%%%%%%%%
% 3 juillet
%%%%%%%%%%%%%%

\dvday{Le Livre du Souvenir}

\dvquote{
Alors ceux qui craignent l'Éternel se parlèrent l'un à l'autre;
 L'Éternel fut attentif et Il écouta. Et un livre de souvenir
 fut écrit devant Lui pour ceux qui craignent l'Éternel et qui
 respectent Son nom.
}{\ibibleverse{Ml}(3:16)}

\lettrine{A}{vez-vous jamais remarqué} combien vous êtes sensibles
 à votre propre nom? Imaginez que deux personnes soient engagées
 dans une conversation derrière vous, et que l'une d'entre elles
 mentionne votre nom. Ne dressez-vous pas automatiquement l'oreille?
 Vous vous demandez~: \punct{deux-points} \og Que disent-elles de moi? \fg{}

Le Seigneur est tout autant sensible à Son nom. Dès que nos conversations
 se tournent vers Son sujet, Il prête l'oreille.
 \og Que se passe-til? Que disent-ils de Moi? \fg{}

Malheureusement, quelquefois, ce que le Seigneur entend n'est pas très
 plaisant. Israël avait l'habitude de murmurer contre le Seigneur
 et contre les choses qu'Il permettait dans le but de les amener
 à une situation de bénédiction. Au lieu de s'encourager mutuellement
 avec les promesses de Dieu, ils murmuraient contre l'inconfort
 de l'expérience. Et, bien souvent, nous sommes coupables de la même chose. 

\dvbox{
Dieu est, semble-t-il, un teneur de registre remarquable. 
}

Dieu tient un registre de ceux qui le craignent et qui s'entretiennent de Lui.
 C'est une pensée impressionnante que de réaliser que mon nom est inscrit
 dans le livre du souvenir de Dieu! 

Je prie pour que vous vous trouviez à parler souvent de l'amour de Dieu
 et de Sa bonté; et que vous alliez faire part aux autres de ce que
 le Seigneur représente pour vous. Tenons bien occupé cet
 \og ange greffier \fg{} qui écrit dans le livre du souvenir.
 Donnons-lui beaucoup de choses à écrire en nous faisant part les uns
 les autres de la bonté et des bénédictions de notre Seigneur. 


\dvrule

\dvprayer{
Seigneur, puissions-nous vraiment parler souvent, les uns avec les autres,
 de Tes gloires. Merci de nous racheter et de de nous chérir. 
}{\DlNdJ}


%%%%%%%%%%%%%%
% 4 juillet
%%%%%%%%%%%%%%

\dvday{Repentez-vous}

\dvquote{
En ce temps-là parut Jean-Baptiste; il prêchait dans le désert de Judée.
 Il disait~: \og Repentez-vous \dots{} \fg{}
}{\ibibleverse{Mt}(3:1-2)}

\lettrine{Q}{ue signifie \og se repentir \fg{} ?}
 C'est plus que juste dire que vous êtes désolés. La vraie repentance,
 c'est être si désolé, si contrit, que vous ne répétez plus l'offense.
 Si une personne déclare qu'elle s'est repentie d'une certaine action
 ou d'un certain péché et qu'elle continue dans la même conduite,
 il y a de bonnes raisons de douter de l'authenticité de la repentance. 

Pour que Dieu puisse travailler dans une vie, la première étape
 est de vraiment se détourner du péché. Et cela arrive quand nous rencontrons
 la gentillesse, la bonté de Dieu. Comme Paul l'a écrit~: \punct{deux-points}
 \og La bonté de Dieu vous pousse à la repentance \fg{} (\ibibleverse{Rm}(2:4)).
 Quand nous nous rendons compte que Dieu est miséricordieux, plein de grâce,
 plein d'amour et prêt à pardonner nos transgressions, c'est alors
 que nos cœurs sont attendris et conduits à la repentance. 

\dvbox{
Sans vraie repentance, il n'y a pas de pardon réel. 
}

Si vous avez entretenu un péché secret dans votre vie et que vous ne vous
 en êtes pas vraiment détourné \ocadr vous êtes devenu meilleur à cacher
 le péché et plus méticuleux en pensant à tous les détails
 pour ne pas être pris \fcadr{} ne vous y méprenez pas.
 Dieu connaît votre cœur. Il connaît votre péché. Et on ne se moque pas de Lui. 

Méditez sur la bonté de Dieu. Pensez à Son amour pour vous,
 à Son abondante miséricorde et à Sa volonté de non seulement
 vous pardonner mais de changer votre vie et de vous aider à conquérir
 ces désirs destructifs. Abandonnez le péché, détournez-vous définitivement
 de lui. Et recevez le pardon qui nettoie. 

\dvrule

\dvprayer{
Père, que Ton Saint-Esprit nous parle du besoin de vraie repentance.
 Puissions-nous porter des fruits qui démontreront une vraie repentance,
 un vrai demi-tour vis-à-vis du mal, un vrai abandon des choses
 qui détruisent nos vies. 
}{\DlNdJ}


%%%%%%%%%%%%%%
% 5 juillet
%%%%%%%%%%%%%%

\dvday{La Tentation}

\dvquote{
Alors Jésus fut emmené par l'Esprit dans le désert,
 pour être tenté par le diable.
}{\ibibleverse{Mt}(4:1)}

\lettrine{L}{a Bible enseigne que la tentation} \typo{la tentation} tombe généralement
 dans l'un de trois domaines~: la convoitise de la chair,
 la convoitise des yeux ou l'orgueil de la vie. Jésus a été tenté
 dans tous ces domaines. Après quarante jours de jeûne, Il avait faim.
 La faim en-elle même n'est pas un péché. Mais Satan l'a transformée
 en une tentation. \og Si tu es Fils de Dieu, ordonne que ces pierres
 deviennent des pains. \fg{} \punct{Point}
 En d'autres termes, Satan suggérait à Jésus de laisser
 Ses appétits physiques dominer Son esprit. 

\dvbox{
C'est toujours le cœur de la tentation~: laisser la chair dominer l'esprit. 
}

Et ainsi, dans chaque situation, je dois déterminer si je vais céder
 aux désirs de ma chair ou aux désirs de l'esprit. \suggest{de l'Esprit, ou de mon esprit?}

Jésus a utilisé la Parole de Dieu pour répondre à toutes les tentations
 amenées par Satan. C'est pourquoi il est si important pour nous d'enregistrer
 la Parole de Dieu dans nos cœurs. La Parole de Dieu demeurant en vous
 sera le secret de votre force pour surmonter n'importe quelle tentation
 envoyée vers vous par Satan. 

Préparez-vous maintenant, car la tentation va arriver à coup sûr.
 Mettez un plan de défense en place avant d'être confrontés au choix.
 Quand la Parole de Dieu dit une chose et que Satan en dit une autre,
 que ferez-vous? Vous soumettrez-vous à Dieu en suivant Sa Parole
 ou bien laisserez-vous votre chair dominer votre esprit? 

\dvrule

\dvprayer{
Père, nous confessons que, trop souvent, nous avons échoué,
 que nous avons laissé la chair dominer l'esprit. Pardonne-nous Seigneur,
 et purifie-nous. Que nous soyons gouverné par Ton Esprit Saint
 et par Ton Éternelle Parole de vérité. 
}{\DlNdJ}

\suggest{gourvernés?}


