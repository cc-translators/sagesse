\dvmonth{Juillet}

%%%%%%%%%%%%%%
% 1er juillet
%%%%%%%%%%%%%%

\dvday{Prisonniers~de~l'Espérance}

\dvquote{
Retournez à la forteresse, vous prisonniers de l'espérance.
 Aujourd'hui même, je le déclare, je te rendrai le double.
}{\ibibleverse{Za}(9:12) \KJF}

\lettrine{D}{ans le verset précédent,} Zacharie s'est adressé à ceux
 qui étaient emprisonnés dans une fosse où il n'y avait pas d'eau.
 Maintenant, il s'adresse à ceux qui sont prisonniers de l'espérance. 

Le croyant est comme \og étreint par \fg{} ou \og prisonnier de \fg{}
 l'espérance de la venue du Messie. Nous ne pouvons pas y échapper.
 Nous ne voulons pas y échapper. C'est ce à quoi nous aspirons,
 c'est ce que nous attendons. Nous savons que quand Jésus reviendra,
 les souffrances de l'humanité cesseront enfin. Les guerres cesseront.
 Les chagrins cesseront. Il gouvernera et règnera dans la paix et la justice.
 Et Il nous libèrera de la fosse où nous nous trouvons peut-être aujourd'hui. 

\dvbox{
Nous nous raccrochons à l'espérance qu'Il va revenir bientôt,
 qu'Il va établir son royaume et qu'Il va amener la justice et la paix
 sur une terre qui en a un besoin criant. 
}

Oh, combien l'espérance est indispensable! Combien nous en avons besoin
 pour nous soutenir. Dans un monde qui s'effondre, un monde dans lequel
 le cœur des hommes leur fait défaut devant la peur,
 qu'il est merveilleux d'avoir cette espérance en Jésus-Christ! 

L'Éternel promet à ceux qui sont prisonniers de l'espérance qu'Il leur rendra
 le double de ce qu'ils ont perdu. Il leur explique dans la dernière partie
 du chapitre comment Il va soumettre leurs ennemis, comment Il va glorifier
 Son peuple et comment Il va en faire comme les pierres d'une couronne,
 \og élevée comme une bannière sur Sa terre \fg{} (\ibibleverse{Za}(9:16)). 

Nous vivons des temps d'une importance capitale, où les évènements
 se conjuguent pour annoncer le retour de Jésus. Aussi longtemps que Dieu
 nous prête vie, nous devons avertir les autres des jours mauvais à venir,
 mais nous devons aussi proclamer le jour glorieux qui arrive
 \ocadr le jour du retour du Seigneur. 

Reviens vite, Seigneur Jésus! 

\dvrule

\dvprayer{
Père, nos regards et nos espérances sont fixés sur Toi.
 Nous avons hâte de voir Ton retour; nous attendons avec anticipation
 le jour où Tu vas venir établir Ton royaume. 
}{\NpDlNdJ}



%%%%%%%%%%%%%%
% 2 juillet
%%%%%%%%%%%%%%

\dvday{Une Alliance devant Dieu}

\dvquote{
Et vous dites~: Pourquoi ? \dots{} Parce que l'Éternel a été témoin
 entre toi et la femme de ta jeunesse que tu as trahie,
 bien qu'elle soit ta compagne et la femme de ton alliance.
}{\ibibleverse{Ml}(2:14)}

\lettrine{C}{e n'était pas faute d'essayer.} Les Israëlites arrosaient
 leurs offrandes de larmes, mais le Seigneur refusait de les accepter.
 Quand ils ont demandé \og Pourquoi? \fg{}, Dieu leur a dit que c'était
 à cause de la façon dont ils traitaient leurs épouses. 

Ils avaient rendu le divorce trop facile. Si un homme s'amourachait
 d'une jolie jeune fille, il pouvait abandonner son épouse en déclarant
 simplement~: \punct{deux-points, majuscule} \og Je divorce d'avec toi \fg{}
 trois fois de suite. Et alors, la compagne de sa jeunesse,
 la femme qui l'avait soutenu pendant les années de vaches maigres
 et avait consenti des sacrifices pour l'aider à démarrer dans la vie,
 la femme qui avait porté ses enfants et tenu sa maison,
 n'avait pas d'autre choix que de s'en aller. 

\dvbox{
Dieu déteste le divorce.
}

Dieu détestait le divorce alors, et Il le déteste tout autant maintenant.
 Et j'ai appris que si Dieu déteste quelque chose,
 je ferais bien de m'assurer que je la déteste aussi. 

L'Écriture dit aux hommes que nous devons aimer notre femme même encore plus
 que nous-mêmes. Nous devons l'aimer comme \og Christ a aimé l'Église
 et s'est livré Lui-même pour elle \fg{} (\ibibleverse{Ep}(5:25)). 

Un grande pression s'exerce aujourd'hui contre le mariage.
 Puissions-nous, nous qui sommes mariés, nous rappeler que nous avons fait
 une alliance l'un avec l'autre devant Dieu, \og pour le meilleur
 et pour le pire, dans la richesse ou dans la pauvreté, dans la maladie
 et la santé, de nous aimer et nous chérir
 jusqu'à ce que la mort nous sépare. \fg{}

\dvrule

\dvprayer{
Père, aide-nous à nous aimer l'un l'autre comme Tu nous a aimés,
 et à nous traiter l'un l'autre avec compassion et gentillesse. 
}{\DlNdJ}


%%%%%%%%%%%%%%
% 3 juillet
%%%%%%%%%%%%%%

\dvday{Le Livre du Souvenir}

\dvquote{
Alors ceux qui craignent l'Éternel se parlèrent l'un à l'autre;
 L'Éternel fut attentif et Il écouta. Et un livre de souvenir
 fut écrit devant Lui pour ceux qui craignent l'Éternel et qui
 respectent Son nom.
}{\ibibleverse{Ml}(3:16)}

\lettrine{A}{vez-vous jamais remarqué} combien vous êtes sensibles
 à votre propre nom? Imaginez que deux personnes soient engagées
 dans une conversation derrière vous, et que l'une d'entre elles
 mentionne votre nom. Ne dressez-vous pas automatiquement l'oreille?
 Vous vous demandez~: \punct{deux-points} \og Que disent-elles de moi? \fg{}

Le Seigneur est tout autant sensible à Son nom. Dès que nos conversations
 se tournent vers Son sujet, Il prête l'oreille.
 \og Que se passe-til? Que disent-ils de Moi? \fg{}

Malheureusement, quelquefois, ce que le Seigneur entend n'est pas très
 plaisant. Israël avait l'habitude de murmurer contre le Seigneur
 et contre les choses qu'Il permettait dans le but de les amener
 à une situation de bénédiction. Au lieu de s'encourager mutuellement
 avec les promesses de Dieu, ils murmuraient contre l'inconfort
 de l'expérience. Et, bien souvent, nous sommes coupables de la même chose. 

\dvbox{
Dieu est, semble-t-il, un teneur de registre remarquable. 
}

Dieu tient un registre de ceux qui le craignent et qui s'entretiennent de Lui.
 C'est une pensée impressionnante que de réaliser que mon nom est inscrit
 dans le livre du souvenir de Dieu! 

Je prie pour que vous vous trouviez à parler souvent de l'amour de Dieu
 et de Sa bonté; et que vous alliez faire part aux autres de ce que
 le Seigneur représente pour vous. Tenons bien occupé cet
 \og ange greffier \fg{} qui écrit dans le livre du souvenir.
 Donnons-lui beaucoup de choses à écrire en nous faisant part les uns
 les autres de la bonté et des bénédictions de notre Seigneur. 


\dvrule

\dvprayer{
Seigneur, puissions-nous vraiment parler souvent, les uns avec les autres,
 de Tes gloires. Merci de nous racheter et de de nous chérir. 
}{\DlNdJ}


%%%%%%%%%%%%%%
% 4 juillet
%%%%%%%%%%%%%%

\dvday{Repentez-vous}

\dvquote{
En ce temps-là parut Jean-Baptiste; il prêchait dans le désert de Judée.
 Il disait~: \og Repentez-vous \dots{} \fg{}
}{\ibibleverse{Mt}(3:1-2)}

\lettrine{Q}{ue signifie \og se repentir \fg{} ?}
 C'est plus que juste dire que vous êtes désolés. La vraie repentance,
 c'est être si désolé, si contrit, que vous ne répétez plus l'offense.
 Si une personne déclare qu'elle s'est repentie d'une certaine action
 ou d'un certain péché et qu'elle continue dans la même conduite,
 il y a de bonnes raisons de douter de l'authenticité de la repentance. 

Pour que Dieu puisse travailler dans une vie, la première étape
 est de vraiment se détourner du péché. Et cela arrive quand nous rencontrons
 la gentillesse, la bonté de Dieu. Comme Paul l'a écrit~: \punct{deux-points}
 \og La bonté de Dieu vous pousse à la repentance \fg{} (\ibibleverse{Rm}(2:4)).
 Quand nous nous rendons compte que Dieu est miséricordieux, plein de grâce,
 plein d'amour et prêt à pardonner nos transgressions, c'est alors
 que nos cœurs sont attendris et conduits à la repentance. 

\dvbox{
Sans vraie repentance, il n'y a pas de pardon réel. 
}

Si vous avez entretenu un péché secret dans votre vie et que vous ne vous
 en êtes pas vraiment détourné \ocadr vous êtes devenu meilleur à cacher
 le péché et plus méticuleux en pensant à tous les détails
 pour ne pas être pris \fcadr{} ne vous y méprenez pas.
 Dieu connaît votre cœur. Il connaît votre péché. Et on ne se moque pas de Lui. 

Méditez sur la bonté de Dieu. Pensez à Son amour pour vous,
 à Son abondante miséricorde et à Sa volonté de non seulement
 vous pardonner mais de changer votre vie et de vous aider à conquérir
 ces désirs destructifs. Abandonnez le péché, détournez-vous définitivement
 de lui. Et recevez le pardon qui nettoie. 

\dvrule

\dvprayer{
Père, que Ton Saint-Esprit nous parle du besoin de vraie repentance.
 Puissions-nous porter des fruits qui démontreront une vraie repentance,
 un vrai demi-tour vis-à-vis du mal, un vrai abandon des choses
 qui détruisent nos vies. 
}{\DlNdJ}


%%%%%%%%%%%%%%
% 5 juillet
%%%%%%%%%%%%%%

\dvday{La Tentation}

\dvquote{
Alors Jésus fut emmené par l'Esprit dans le désert,
 pour être tenté par le diable.
}{\ibibleverse{Mt}(4:1)}

\lettrine[findent=-0.1em]{L}{a Bible enseigne que la tentation} \typo{la tentation} tombe généralement
 dans l'un de trois domaines~: la convoitise de la chair,
 la convoitise des yeux ou l'orgueil de la vie. Jésus a été tenté
 dans tous ces domaines. Après quarante jours de jeûne, Il avait faim.
 La faim en-elle même n'est pas un péché. Mais Satan l'a transformée
 en une tentation. \og Si tu es Fils de Dieu, ordonne que ces pierres
 deviennent des pains. \fg{} \punct{Point}
 En d'autres termes, Satan suggérait à Jésus de laisser
 Ses appétits physiques dominer Son esprit. 

\dvbox{
C'est toujours le cœur de la tentation~: laisser la chair dominer l'esprit. 
}

Et ainsi, dans chaque situation, je dois déterminer si je vais céder
 aux désirs de ma chair ou aux désirs de l'esprit. \suggest{de l'Esprit, ou de mon esprit?}

Jésus a utilisé la Parole de Dieu pour répondre à toutes les tentations
 amenées par Satan. C'est pourquoi il est si important pour nous d'enregistrer
 la Parole de Dieu dans nos cœurs. La Parole de Dieu demeurant en vous
 sera le secret de votre force pour surmonter n'importe quelle tentation
 envoyée vers vous par Satan. 

Préparez-vous maintenant, car la tentation va arriver à coup sûr.
 Mettez un plan de défense en place avant d'être confrontés au choix.
 Quand la Parole de Dieu dit une chose et que Satan en dit une autre,
 que ferez-vous? Vous soumettrez-vous à Dieu en suivant Sa Parole
 ou bien laisserez-vous votre chair dominer votre esprit? 

\dvrule

\dvprayer{
Père, nous confessons que, trop souvent, nous avons échoué,
 que nous avons laissé la chair dominer l'esprit. Pardonne-nous Seigneur,
 et purifie-nous. Que nous soyons gouverné par Ton Esprit Saint
 et par Ton Éternelle Parole de vérité. 
}{\DlNdJ}

\suggest{gourvernés?}


%%%%%%%%%%%%%%
% 6 juillet
%%%%%%%%%%%%%%

\dvday{Pas de Soucis}

\dvquote{
C'est pourquoi je vous dis~: Ne vous inquiétez pas pour votre vie
 de ce que vous mangerez, ni pour votre corps de quoi vous serez vêtus.
 La vie n'est-elle pas plus que la nourriture,
 et le corps plus que le vêtement ?
}{\ibibleverse{Mt}(6:25)}

\lettrine{L}{a foi et l'inquiétude sont mutuellement exclusives.}
 Si vous avez une foi réelle dans la providence et dans l'attention
 de Dieu pour vous, vous n'allez pas vous inquiéter de vos circonstances.
 Tout souci ressenti est la preuve d'un manque de foi. 

En raison de votre foi en Jésus-Christ, vous êtes maintenant un enfant
 de Dieu. Savoir qu'Il est votre Père \suggest{Père} céleste devrait dissiper
 toute inquiétude sur les problèmes de la vie. Votre Père sait que vous avez
 besoin de nourriture et de vêtements, donc vous n'avez pas à vous en
 inquiéter. Jésus nous a appris que notre Père nous aime tant,
 qu'Il surveille les détails triviaux de nos vies,
 et connaît nos besoins avant même que nous en soyons conscients. 

C'est une question de priorité. Que mettez-vous en premier
 \ocadr la nourriture et les vêtements? Si c'est le cas,
 alors vous ne valez pas mieux que les païens.
 Mais si vous cherchez d'abord le royaume de Dieu et Sa justice,
 alors Dieu va prendre soin de tout le reste. 

\dvbox{
Ne mettez pas en tête de votre liste de priorités les choses
 qui ne doivent pas y être. 
}

Ceux qui perdent leurs vies à rechercher des choses temporelles
 et non productives, n'auront rien d'éternel à montrer pour leurs vies
 quand ils arriveront à la fin du chemin. Que chaque homme
 s'examine donc lui-même. Considérez votre cœur et déterminez
 de mettre Dieu en premier. 

\dvrule

\dvprayer{
Père, aide-nous à Te donner la place de la plus haute priorité dans nos vies.
 Pardonne-nous Seigneur, pour le temps que nous passons à penser
 à des choses physiques qui vont disparaître. 
}{\DlNdJ}


%%%%%%%%%%%%%%
% 7 juillet
%%%%%%%%%%%%%%

\dvday{Perdez-vous}

\dvquote{
Celui qui aura gardé sa vie la perdra, et celui qui aura perdu sa vie
 à cause de moi la retrouvera.
}{\ibibleverse{Mt}(10:39)}

\lettrine{J}{'entends beaucoup de gens} parler aujourd'hui du besoin
 de \og se trouver \fg{}. Mais d'après Jésus, la meilleure chose
 que vous puissiez faire, c'est de vous perdre. 

Parlant un jour à ses adeptes, Jésus a expliqué la première exigence
 pour devenir un disciple~: \og Si quelqu'un veut venir après moi,
 qu'il renonce à lui-même, qu'il se charge chaque jour de sa croix
 et qu'il me suive \fg{} (\ibibleverse{Lc}(9:23)).
 Pour rechercher Jésus, vous devez renoncer à vous-mêmes.
 Vous devez abandonner l'habitude de tout centrer sur vous-mêmes. 

Le plus grand exemple d'altruisme que le monde ait jamais vu
 s'est produit sur la croix \suggest{Croix?}. Bien que Sa préférence ait été
 que cette coupe (la croix) s'éloigne de Lui (\ibibleverse{Mt}(26:39)),
 Jésus a mis de côté Sa volonté pour obéir à celle de Son Père. 

\dvbox{
Perdez-vous en Jésus-Christ. 
}

Celui qui ne vit que pour lui-même et ne pense qu'à lui-même va tout perdre
 en fin de compte. C'est parce que vous ne pouvez garder aucune
 de ces choses que vous faites pour vous-mêmes. Tous les succès,
 toutes les réussites, tous les trophées vont finir par brûler
 dans les feux qui testeront nos œuvres. Les seules choses
 que vous puissiez emmener avec vous \ocadr les seules choses
 qui vont durer pour toujours \fcadr{} ce sont celles
 que vous faites pour Jésus-Christ. 

Vous voulez vous trouver? Vous voulez trouver le sens le plus profond
 possible à la vie? Alors renoncez à vous-mêmes. Perdez-vous
 en Jésus-Christ et ce faisant, vous trouverez la joie et la paix
 qui vont durer pour l'éternité. 

\dvrule

\dvprayer{
Père, aide-nous à vivre par ces principes spirituels que Tu as explicité
 pour nous. Aide-nous à parvenir au point où il n'y a plus rien
 de nous-mêmes et tout de Toi. 
}{\Amen}


%%%%%%%%%%%%%%
% 8 juillet
%%%%%%%%%%%%%%

\dvday{Le Joug Plus Léger}

\dvquote{
Venez à moi, vous tous qui êtes fatigués et chargés,
 et je vous donnerai du repos.
 Prenez mon joug sur vous et recevez mes instructions,
 car je suis doux et humble de cœur,
 et vous trouverez du repos pour vos âmes.
 Car mon joug est aisé, et mon fardeau léger.
}{\ibibleverse{Mt}(11:28-30)}

\lettrine{L}{a vie est pleine de fardeaux.}
 Nous sommes surchargés de responsabilités, d'attentes,
 de nos propres désirs, du besoin d'accumuler des possessions
 ou de rechercher les plaisirs.
 Mais Jésus a offert une invitation à deux volets à ceux
 qui peinent sous ces lourds fardeaux~:
 \og Prenez mon joug sur vous et recevez mes instructions. \fg{}
 \punct{Guillemet fermant après point} 

Le joug était un instrument de la ferme \ocadr un harnais en bois
 que les boeufs portaient pour pouvoir tirer la charrue.
 En utilisant ce terme, Jésus déclare~: \punct{deux-points}
 \og J'ai un travail pour vous. Et je veux prendre les rênes
 de votre vie et commencer à vous diriger. \fg{}

\dvbox{
Il est plus simple de vivre pour Dieu que de vivre pour vous-mêmes. 
}

Jésus a dit alors~: \punct{deux-points} \og Recevez Mes instructions. \fg{}
 Par cette expression, Il invite à un examen. Plus vous allez L'étudier,
 plus vous allez L'aimer en découvrant Son grand amour pour vous. 

Et quel était le fardeau de Jésus? Ses premières paroles rapportées
 étaient~: \punct{deux-points, majuscule} \og Il faut que je m'occupe
 des affaires de mon Père \fg{} (\ibibleverse{Lc}(2:49)).
 Jésus vivait pour faire la volonté de Son Père. En disant~: \punct{deux-points}
 \og Mon fardeau est léger \fg{}, Il déclare qu'Il est bien plus facile
 de plaire à Dieu qu'à vous-mêmes.
 Trouvez un homme qui vit dans la paix et la satisfaction
 et vous aurez trouvé un homme consacré à Jésus-Christ. 

\dvrule

\dvprayer{
Seigneur, tant de gens sont inquiets face aux pressions de cette vie.
 Mais nous te remercions de ce que même en vivant au sein de ce monde
 où tout va trop vite, nous pouvons avoir le repos dans nos âmes
 en nous consacrant à Toi. 
}{\DlNdJ}


%%%%%%%%%%%%%%
% 9 juillet
%%%%%%%%%%%%%%

\dvday{Impardonnable}

\dvquote{
C'est pourquoi je vous dis~:
 Tout péché et tout blasphème sera pardonné aux hommes,
 mais le blasphème contre l'Esprit ne sera point pardonné.
}{\ibibleverse{Mt}(12:31)}

\lettrine{S}{i vous avez peur} d'avoir peut-être commis
 le péché impardonnable, je peux vous assurer que vous ne l'avez pas commis.
 Une fois qu'une personne a commis le péché impardonnable,
 elle s'en moque complètement. 

\dvbox{
Le péché impardonnable consiste à persister dans le rejet du Saint-Esprit. 
}

Quand vous péchez, l'Esprit de Dieu est aux prises avec vous pour
 vous convaincre de ce péché. Nous l'appelons souvent la voix
 de la conscience. Quand nous en tenons compte, confessons nos péchés
 et nous nous détournons d'eux, \og Il est fidèle et juste
 pour nous pardonner nos péchés et nous purifier de toute injustice \fg{}
 (\ibibleverse{IJn}(1:9)).
 Mais Dieu a dit que Son Esprit ne contesterait pas pour toujours
 dans l'homme. Il est possible qu'une personne arrive à un point
 dans sa vie où elle n'entend plus la voix du Saint-Esprit \typo{Saint-Esprit}
 qui convainc de péché. 

Le rejet du témoignage de l'Esprit auprès de votre cœur
 \ocadr qui affirme que Jésus est le Messie, qu'Il est mort pour vos péchés
 et qu'Il est ressuscité \fcadr{}
 est le début du chemin qui mène au péché impardonnable.
 Continuer à rejeter Jésus-Christ comme étant votre Sauveur
 va finir par vous amener au point où Il va cesser de lutter avec vous
 et où vous ne pourrez plus croire. Et si vous mourez \typo{pas futur}
 en ayant rejeté Jésus-Christ comme étant votre Sauveur,
 alors il n'y a plus de pardon, \suggest{possible?}
 ni dans ce monde, ni dans le monde à venir. 

\dvrule

\dvprayer{
Père, nous connaissons des gens autour de nous qui ont dit \og non \fg{}
 à maintes et maintes reprises \ocadr ceux pour qui l'amour des ténèbres
 excède leur amour pour la lumière.
 Mais Tu les aimes en dépit de leur rebellion. Puissent-ils entendre
 et répondre à Ton Esprit avant qu'ils ne soient allés trop loin. 
}{\Amen}


%%%%%%%%%%%%%%
% 10 juillet
%%%%%%%%%%%%%%

\dvday{Fixez vos yeux sur Jésus}

\dvquote{
Pierre sortit de la barque et marcha sur les eaux pour aller vers Jésus.
 Mais en voyant que le vent était fort, il eut peur, et,
 comme il commençait à enfoncer il s'écria~: Seigneur, sauve-moi!
}{\ibibleverse{Mt}(14:29-30)}

\lettrine{A}{près avoir ramé toute la nuit,}
 les disciples n'avaient atteint que le milieu du lac.
 Et maintenant de grosses vagues s'étaient formées et de fortes pluies
 commençaient à s'abattre sur eux. Soudain, au milieu de la tempête,
 ils voient une silhouette marchant sur l'eau dans leur direction.
 Terrifiés, ils se mettent à pousser des cris
 \ocadr \suggest{virgule au lieu de \ocadr}
 mais une voix rassurante leur répond~:
 \og N'ayez pas peur, leur dit Jésus, ce n'est que Moi. \fg{}
 \punct{Pas de répétition des guillemets pour l'incise,
 échange verbe-sujet}

\dvbox{
La peur et la foi sont mutuellement exclusives. 
}

Pierre, intrépide \suggest{intrépide} comme à son habitude, dit~: \punct{deux-points}
 \og Si c'est vraiment Toi, ordonne-moi d'aller vers Toi sur les eaux. \fg{}
 Quand Jésus lui dit de venir, Pierre sort de l'embarcation et commence
 à marcher sur l'eau. Mais ça ne dure que très peu de temps.
 Après seulement quelques pas sur l'eau, Pierre détourne son \suggest{le regard} regard
 de Jésus pour le porter sur les vagues tumultueuses qui mugissent à ses pieds
 \ocadr \suggest{virgule au lieu de \ocadr} et il commence à couler dans l'eau. 

Des situations accablantes peuvent nous faire détourner notre regard de Jésus.
 Quand nous le faisons, nous sombrons souvent dans le désespoir.
 Mais si nous sommes sages, comme Pierre, nous nous rappelons
 de faire appel au Seul qui a le pouvoir de nous aider,
 \og Seigneur, sauve-moi! \fg{}

La peur et la foi sont mutuellement exclusives. La foi élimine la peur
 et la peur élimine la foi. Si vous fixez votre regard sur Jésus,
 votre foi va vous aider à marcher sur les eaux et à surmonter
 les difficultés amenées par la vie. 

\dvrule

\dvprayer{
Seigneur, apprends-nous ce jour à garder nos yeux fixés sur Toi.
 Pour ceux qui sont accablés, qui ont peur d'être en train de sombrer,
 rappelle-leur Ta proximité, Ton amour, et Ton empressement à sauver. 
}{\DlNdJ}


%%%%%%%%%%%%%%
% 11 juillet
%%%%%%%%%%%%%%

\dvday{Pas un Mot}

\dvquote{
Une femme cananéenne qui venait de ces contrées, Lui cria~:
 Aie pitié de moi, Seigneur, Fils de David.
 Ma fille est cruellement tourmentée par le démon.
 Il ne lui répondit pas un mot.
}{\ibibleverse{Mt}(15:22-23)}

\lettrine{L}{e plaidoyer de la femme était passionné,}
 mais Jésus ne lui a pas répondu un seul mot.
 Quand Il s'est finalement mis à parler, c'est aux disciples
 qu'Il s'est adressé. Il leur a déclaré que la femme était en dehors
 de l'alliance conclue par Dieu avec Israël.
 Mais la femme n'a pas été découragée.
 En fait, elle était d'accord avec Jésus~:
 \og C'est vrai Seigneur. Je ne mérite rien.
 Mais je ne demande pas du pain
 \ocadr juste les miettes qui tombent de la table. \fg{}

\dvbox{
Le silence de Dieu ne veut pas toujours dire \og non \fg{}. 
}

Jésus savait dès le début qu'Il allait guérir sa fille.
 Mais Il voulait faire ressortir la foi de la femme dans son expression
 la plus complète. Chaque fois qu'Il faisait un pas en arrière,
 elle en faisait deux en avant. Refusant de se décourager,
 elle a tenu bon jusqu'à ce qu'elle entende les mots
 qu'elle voulait entendre~: \punct{deux-points}
 \og Ô femme, ta foi est grande, qu'il te soit fait comme tu le veux \fg{}
 (\ibibleverse{Mt}(15:28)). 

Quelquefois, Dieu est simplement en train d'attendre.
 Il peut être à l'œuvre dans votre cœur, ou s'attarder dans le but
 de fortifier votre foi. Ne vous découragez pas quand la réponse
 n'est pas immédiate. Tenez bon. Continuez de \suggest{continuez à} demander.
 Démontrez votre foi et attendez les paroles que vous attendez
 avec impatience~: \punct{deux-points}
 \og Qu'il te soit fait comme tu le veux. \fg{}

\dvrule

\dvprayer{
Père, attire-nous toujours plus près de Toi, dans une relation plus intime
 et complète. Apprends-nous la patience pendant que nous attendons
 Ta réponse à nos requêtes. Fortifie notre foi, et donne-nous
 des occasions de Te prouver cette foi.
 Puissions-nous nous reposer dans Ton amour. 
}{\DlNdJ}


%%%%%%%%%%%%%%
% 12 juillet
%%%%%%%%%%%%%%

\dvday{Le Vrai Pardon}

\dvquote{
Alors Pierre s'approcha et Lui dit~:
 Seigneur, combien de fois pardonnerai-je à mon frère,
 lorsqu'il pèchera contre moi? Jusqu'à sept fois ?
 Jésus lui dit~: Ne te dis pas jusqu'à sept fois,
 mais jusqu'à soixante-dix fois sept fois.
}{\ibibleverse{Mt}(18:21-22)}

\lettrine{J}{e soupçonne Pierre} d'avoir espérer impressionner le Seigneur
 avec cette question. Dans sa tête, \og sept fois \fg{} semblait probablement
 comme un effort remarquable qui repoussait les limites ordinaires.
 Aussi quand Jésus l'a corrigé en disant \og soixante dix fois sept \fg{},
 Pierre a probablement pensé~: \og Zut alors ! \fg{}

Le pardon envers les autres est un préalable si nous voulons
 que nos péchés soient pardonnés. Jésus a insisté sur cette vérité
 dans Son enseignement dans \og Le Notre Père \fg{}~: \suggest{Le \emph{Notre Père}}
 \og Pardonne-nous nos offenses comme nous pardonnons aussi
 à ceux qui nous ont offensés. \fg{} \grammar{offensé?}
 (\ibibleverse{Mt}(6:12)). Quelques versets plus loin, Il a ajouté~: \punct{deux-points}
 \og Mais si vous ne pardonnez pas aux hommes,
 votre Père ne vous pardonnera pas non plus vos fautes \fg{} \punct{Point en trop}
 (\ibibleverse{Mt}(6:15)). 

\dvbox{
Le pardon n'est pas une affaire de mathématiques.
 C'est une affaire d'état d'esprit. 
}

Quand Dieu pardonne, Il oublie. Il n'évoque plus jamais la chose.
 Nous ferions bien de L'imiter. Il nous a fait grâce d'une si grande dette
 et pourtant il est étonnant de voir les petites choses
 qui nous restent en travers de la gorge et qui font que nous sommes en colère
 et éprouvons du ressentiment envers les autres.
 Trop souvent, notre attitude est~: \punct{deux-points}
 \og Je vais pardonner \ocadr mais je ne vais pas oublier. \fg{}
 \punct{Point avant guillemet}
 Jésus répondrait~: \og Ce n'est pas du pardon. \fg{}

Pour le croyant, le pardon n'est pas une option, c'est une exigence.
 Mais quoi qu'Il \typo{quoi qu'Il} exige de vous, Dieu va vous rendre
 capable de l'accorder. Si la force de pardonner vous fait défaut,
 tout ce que vous avez à faire est de demander que le Saint-Esprit
 accomplisse ce pardon à travers vous.
 Quelle chose glorieuse quand l'amour de Dieu passe par nous pour guérir
 une autre personne. 

\dvrule

\dvprayer{
Seigneur, sonde-nous et connais nos cœurs.
 S'il nous faut pardonner, aide-nous à le faire par Ton Esprit. 
}{\DlNdJ}



%%%%%%%%%%%%%%
% 13 juillet
%%%%%%%%%%%%%%

\dvday{Amour Refroidi}

\dvquote{
Et en raison des progrès de l'iniquité l'amour
 du plus grand nombre se refroidira.
}{\ibibleverse{Mt}(24:12)}

\lettrine{I}{l y en a certains qui,} à un moment donné,
 ont brûlé d'un zèle sacré pour le Seigneur.
 Ils avaient une vie spirituelle profonde, passant beaucoup de temps
 dans la Parole et dans la prière. Mais leur feu s'est éteint.
 Satan s'est insinué progressivement dans leur vie, détruisant à la fois
 leur témoignage et leur relation avec le Seigneur.
 Bien qu'à un moment ils se soient réjouis d'avoir été libérés
 de la fange et la boue du monde, ils sont retournés, tels des cochons,
 à la même fange et à la même boue.
 Comme Jésus l'avait prédit dans \bibleverse{Mt}(24:),
 leur amour s'est refroidi. 

\dvbox{
Jésus nous a mis en garde contre ces choses qui peuvent
 éteindre le feu dans nos cœurs. 
}

\og Prenez garde à vous-mêmes, de crainte que vos cœurs
 ne s'appesantissent par les excès ou l'ivrognerie,
 et par les soucis de la vie, et que ce jour ne fonde sur vous
 à l'improviste \fg{} (\ibibleverse{Lc}(21:34)).
 Les excès font référence aux appétits de la chair auxquels
 on s'adonne sans retenue. L'ivrognerie ou la stupeur
 \ocadr que ce soit par l'alcool, la drogue ou même, oui,
 par trop de temps passé devant la télé \fcadr{} c'est l'engourdissement
 ou l'obscurcissement de votre cerveau jusqu'au point
 où vous ne pouvez plus penser clairement.
 Les soucis de la vie sont ces choses qui vous occupent
 tant que vous n'avez plus de temps pour les choses de l'Esprit. 

Quelle est la condition de votre cœur aujourd'hui?
 Votre amour pour Jésus y brûle-t-il avec clarté,
 ou votre amour pour Lui s'est-il refroidi? 

\dvrule

\dvprayer{
Père, nous ne voulons pas que notre amour pour Toi se refroidisse.
 Garde le feu de l'amour brûlant dans nos cœurs,
 que notre zèle pour Toi soit un témoignage brillant pour le monde. 
}{\DlNdJ}


%%%%%%%%%%%%%%
% 14 juillet
%%%%%%%%%%%%%%

\dvday{Les Jours de Noé}

\dvquote{
Comme aux jours de Noé ainsi en sera-t-il à l'avènement
 du Fils de l'homme.
}{\ibibleverse{Mt}(4:37)}

\lettrine{À}{l'époque de Noé,} Dieu vit que la méchanceté de l'homme
 était grande sur la terre, et que chaque jour son cœur ne concevait
 que des pensées mauvaises. Il n'est pas difficile de faire
 la comparaison avec la situation d'aujourd'hui. Notre monde est plein
 de violence et de corruption. Les industries de la télévision
 et du cinéma polluent le cerveau des gens au point où les pensées
 et l'imagination du cœur de l'homme sont constamment
 tournées vers le mal. Et, tout comme à l'époque de Noé,
 les gens mangent et boivent, se marient et continuent
 à s'occuper de leurs petites affaires sans une seule pensée
 pour le retour prochain du Seigneur. 

\dvbox{
Le monde ne peut pas continuer comme ça beaucoup plus longtemps,
 vue la façon dont les choses partent en vrille. 
}

L'heure des comptes approche, et rien dans l'arsenal du monde
 ne peut nous sauver. L'éducation ne va pas nous sauver.
 La science, non plus. Le gouvernement ne le pourra pas.
 Les Nations Unies ne le pourront pas. Même Green Peace
 ne peut sauver notre planète. Le seul espoir assuré pour le futur
 est le retour de Jésus-Christ.
 Lui seul a le pouvoir de sauver et libérer. 

Voir la détérioration des conditions du monde autour de nous,
 ne devrait pas nous décourager. Au contraire, cela devrait nous motiver.
 Nous devons relever le défi et vivre des vies justes et saintes
 qui plaisent à Dieu en atttendant le retour de notre Seigneur. 

\dvrule

\dvprayer{
Père, quand nous regardons le monde aujourd'hui, nous voyons les mêmes
 conditions que celles qui existaient aux jours de Noé,
 et nous prenons conscience que le jugement n'est plus très loin.
 Puissions-nous nous humilier, prier et rechercher Ta face,
 et puissions-nous nous détourner de toutes mauvaises choses. 
}{\DlNdJ}


%%%%%%%%%%%%%%
% 15 juillet
%%%%%%%%%%%%%%

\dvday{Le Fidèle Serviteur}

\dvquote{
Son maître lui dit~: Bien, bon et fidèle serviteur,
 tu as été fidèle en peu de choses, je t'établirai sur beaucoup;
 entre dans la joie de ton maître.
}{\ibibleverse{Mt}(25:21)}

\lettrine{N}{ous devons tous procéder} à un inventaire personnel
 et nous demander~: \punct{deux-points}
 \og Qu'est-ce-que le Seigneur m'a confié? \fg{}
 Quand nous comparaîtrons devant le tribunal du Christ,
 chacun d'entre nous devra répondre de ce que nous aurons fait de nos vies,
 de nos talents et de nos ressources. 

\dvbox{
Qu'est-ce que Dieu m'a confié dont Il me demandera un jour des comptes? 
}

Dans cette parabole, Jésus nous parle de l'homme auquel on avait confié
 cinq talents. Au retour de son maître, il avait beaucoup à lui montrer.
 \og Seigneur, a-t-il dit, \punct{Pas de répétition des guillemts dans l'incise}
 tu m'as donné cinq talents et je t'en ai gagné cinq de plus. \fg{}
 Sur quoi son maître a répondu~: \punct{deux-points}
 \og Bien, bon, et fidèle serviteur. \fg{}
 Il a alors promis de confier encore plus de responsabilités à son serviteur,
 et l'a invité à entrer dans \og la joie de son maître. \fg{}

Tout ce que j'ai fait et tout ce que je fais dans ma vie
 a pour but de m'entendre dire un jour les paroles suivantes~: \punct{deux-points}
 \og Bien fait. \fg{} \suggest{Bien joué}
 C'est la motivation derrière tout ce que je fais.
 C'est ce à quoi j'aspire et ce pourquoi je travaille.
 Si Jésus venait aujourd'hui, seriez-vous heureux de Lui montrer
 ce que vous avez fait des choses qu'Il vous a confiées? 

\dvrule

\dvprayer{
Père, nous Te remercions pour ce que Tu as confié à notre bonne garde
 et à nos bons soins. Aide-nous à ne pas être paresseux dans le service
 que nous faisons pour Toi, mais puissions-nous être appliqués
 dans l'usage de ces choses que Tu nous a données
 \ocadr les utilisant, Seigneur, pour Ton but. 
}{\DlNdJ}


%%%%%%%%%%%%%%
% 16 juillet
%%%%%%%%%%%%%%

%\dvday{Le Chemin Qui Mène au Reniement}
\dvday{Le Chemin vers le Reniement}

\dvquote{
Une servante s'approcha de lui et dit~:
 Toi aussi, tu étais avec Jésus le Galiléen.
 Mais il le nia devant tous disant~:
 Je ne sais pas ce que tu veux dire.
}{\ibibleverse{Mt}(26:69-70)}

\lettrine{C}{omment se fait-il qu'un homme} qui avait été un si proche
 compagnon de Jésus ait ainsi pu renier son Seigneur?
 Quelques erreurs seulement ont mené Pierre de l'amitié intime au reniement. 

La première erreur de Pierre a été d'être en désaccord avec Jésus
 (\ibibleverse{Mt}(26:31-35)).
 Si vous vous retrouvez à vous disputer avec le Seigneur,
 sachez bien ceci~: vous avez tort. 

\dvbox{
Quelques erreurs seulement peuvent vous faire passer
 de la communion intime au reniement. 
}

Deuxièmement, Pierre s'est vanté de son engagement.
 \og Quand tu serais pour tous une occasion de chute,
 tu ne le seras jamais pour moi \fg{} (\ibibleverse{Mt}(26:33)).
 Méfiez-vous d'un excès de confiance en vos propres forces. 

Ensuite, Pierre a choisi le sommeil plutôt que la prière
 (\ibibleverse{Mt}(26:40,43)).
 Nous traitons souvent la prière comme une pensée après coup,
 ou comme queque chose d'optionnel.
 Je suis certain que si nous priions plus souvent, nous pécherions moins. 

Puis, Pierre a mis de la distance entre lui et Jésus.
 \punct{Guillemet ouvrant manquant}
 \og Pierre le suivit de loin \fg{} (\ibibleverse{Mt}(26:58)).
 Le secret de la vie chrétienne, \punct{virgule}
 c'est de rester aussi proche de Jésus que possible. 

Finalement, nous retrouvons Pierre en train de se réchauffer
 autour du feu des soldats (\ibibleverse{Mc}(14:54)).
 Il est toujours dangereux de chercher de la chaleur de la part de l'ennemi. 

Pierre a renié le Seigneur, mais sa foi n'a pas défailli.
 Si, comme Pierre, vous avez échoué, Jésus veut vous aider
 à prendre un nouveau départ \ocadr cette fois en Lui tenant bien la main,
 en restant bien à Son côté et en évitant le feu réchauffant de l'ennemi. 

\dvrule

\dvprayer{
Père, nous Te remercions pour Ta miséricorde,
 Ta grâce et Ton pardon de nos péchés. Nous sommes tellement
 reconnaissants de ce que malgré nos échecs,
 Tu peux et Tu veux nous ramener à un point où nous pouvons
 être utiles et servir. 
}{\DlNdJ}


%%%%%%%%%%%%%%
% 17 juillet
%%%%%%%%%%%%%%

\dvday{Pouvoir Suffisant}

\dvquote{
Jésus s'approcha et leur parla ainsi~:
 Tout pouvoir m'a été donné dans le ciel et sur la terre.
}{\ibibleverse{Mt}(28:18)}

\lettrine{L}{e Jésus qui avait le pouvoir} d'amener l'univers
 en existence et le pouvoir de pardonner les péchés,
 et le pouvoir d'exorciser les démons, et le pouvoir de soutenir
 toutes choses par Sa Parole (\ibibleverse{He}(1:3))
 est le même qui a dit à Ses disciples \ocadr et à nous \fcadr{}~: \punct{deux-points}
 \og Allez, faites de toutes les nations des disciples,
 baptisez-les au nom du Père, du Fils et du Saint-Esprit \fg{}
 (\ibibleverse{Mt}(28:19)). \punct{Point après référence}

Il semblerait que, puisque nous sommes faibles, et puisque que Jésus
 a toute cette puissance, peut-être devrait-Il être Celui qui doit y aller.
 Mais Il a aussi dit~: \punct{deux-points}
 \og Et voici, Je suis avec vous tous les jours,
 jusqu'à la fin du monde \fg{} (\ibibleverse{Mt}(28:20)).
 Aussi quand Il nous envoie pour accomplir la Grande Mission,
 Il nous dit en fait~: \punct{deux-points}
 \og Vous y allez, et Je serai avec vous. \fg{}

\dvbox{
Oh, si seulement vous connaissiez la grandeur surabondante
 de la puissance que Dieu a mise à votre disposition! 
}

\grammar{La puissance que Dieu a mise}

Vous n'avez pas à être liés par les désirs de votre chair.
 Vous n'avez pas à vivre dans la défaite. Vous n'avez pas à vivre
 dans l'esclavage. Tout le pouvoir dans l'univers vous est disponible
 par Jésus-Christ \ocadr pouvoir suffisant pour parer à toutes
 les éventualités possibles et imaginables.
 Tout le pouvoir, quel qu'il soit, dont vous avez besoin
 pour répondre à Son appel, pour être conformés à Son image,
 pour être tout ce que Dieu désire est vôtre aujourd'hui en Jésus. 

\dvrule

\dvprayer{
Père, puissions-nous connaître la grandeur surabondante
 de Ta puissance pour nous qui sommes parvenus à la connaissance
 de Jésus-Christ. Puissions-nous aller de l'avant en Ton Nom,
 avec la puissance de Ton Saint-Esprit pour porter l'Évangile
 autour de nous dans un monde qui en a tant besoin. 
}{\Amen}


%%%%%%%%%%%%%%
% 18 juillet
%%%%%%%%%%%%%%

\dvday{Son Toucher}

\dvquote{
Un lépreux vint à lui et, se jetant à genoux, il lui dit d'un ton suppliant~:
 Si tu le veux, tu peux me rendre pur.
 Jésus, ému de compassion, étendit la main, le toucha et dit~:
 Je le veux, sois pur.
}{\ibibleverse{Mc}(1:40-41)}

\lettrine{P}{ersonne ne savait} comment la lèpre se transmettait
 d'une personne à une autre; tout ce que l'on savait, c'était que la maladie
 \ocadr qui putréfiait progressivement la chair \fcadr{}
 était incurable et mortelle. La lèpre se rapproche du péché qui,
 lui aussi, putréfie progressivement ses victimes et mène à leur mort.
 Dans les Écritures, la lèpre est ainsi un symbole du péché. 

La société bannissait le lépreu et exigeait qu'il avertisse
 ceux qui l'approchaient en criant~: \punct{deux-points}
 \og Impur, impur. \fg{}
 Cependant, nous trouvons cet homme à genoux devant Jésus,
 exprimant sa foi dans la capacité et le pouvoir de Jésus
 de le nettoyer de sa lèpre. Pour lui, sa guérison était simplement
 une question de bon vouloir de la part de Jésus. 

\dvbox{
Il n'y a pas d'intouchables pour Jésus. 
}

La compassion a conduit Jésus à étendre la main et à toucher l'homme.
 Ce faisant, Il s'est rendu cérémonialement impur.
 Mais Il a aussi offert un contact plein d'amour à quelqu'un
 qui n'avait probablement pas senti le toucher d'un autre être humain
 depuis des années. 

Puis Jésus a prononcé ces paroles glorieuses~: \punct{deux-points}
 \og Je le veux, sois pur. \fg{}
 Que se passe-t-il quand un pécheur vient à Jésus?
 Animé par la compassion, Jésus tend la main et vient toucher cette vie.
 Peu importe l'état d'avancement de la maladie,
 peu importe combien le péché a déjà rongé, peu importe que personne
 d'autre ne veuille toucher cette vie, Jésus le fait.
 Personne n'est intouchable pour Jésus. 

\dvrule

\dvprayer{
Seigneur, Merci de ce que quand nous tombions en pourriture
 sous les ravages du péché, Tu nous a tendu la main,
 Tu nous a touchés, Tu nous a pardonnés et Tu nous a lavés.
 Nous T'aimons Seigneur. 
}{\Amen}


%%%%%%%%%%%%%%
% 19 juillet
%%%%%%%%%%%%%%

\dvday{Le Complot contre Jésus}

\dvquote{
Les Pharisiens sortirent et se consultèrent aussitôt
 avec les Hérodiens sur les moyens de le faire périr.
}{\ibibleverse{Mc}(3:6)}

\lettrine{L}{a logique vous dit} qu'il est légitime de faire du bien,
 quel que soit le jour de la semaine.
 Mais parce que Jésus avait osé guérir un homme le jour du Sabbat
 \ocadr et qu'Il avait ensuite défendu Ses actions contre leur tradition \fcadr{}
 les Juifs ont cherché à le tuer.

Aujourd'hui même, certains se sentent de devoir détruire Jésus.
 Cherchant à détruire l'influence de Jésus dans notre société,
 l'industrie du film dépeint fréquemment les chrétiens de façon négative.
 Des organisations se battent depuis longtemps pour éliminer
 Jésus du secteur public de notre nation, et malheureusement,
 la Cour Suprême
 \NdT{La plus haute cour de justice aux États-Unis, remplissant les fonctions
 du Conseil d'État et de la Cour de Cassation en France.} les y aide.

\dvbox{
Ne restez pas neutres, ne détournez pas les yeux.
}

Pourquoi est-ce le cas? C'est parce que, comme Jésus l'a dit,
 les hommes ont aimé les ténèbres plus que la lumière.
 Jésus s'est élevé contre l'adultère, la haine, le mensonge
 et la tricherie et a au contraire enseigné le pardon.
 Il a enseigné que nous devons aimer Dieu et nous aimer les uns les autres.
 Mais ceux qui sont remplis de haine envers Dieu et leur prochain
 se sentent de devoir détruire le message de Jésus.
 S'ils y arrivent, alors peut-être peuvent-ils vivre leurs vies
 remplies de péchés sans se sentir coupables.

Je crois que dans les jours à venir nous allons voir de plus en plus
 de nos libertés religieuses retirées et davantage d'oppression
 du christianisme \typo{Pas de majuscule à Christianisme}
 de la part du gouvernement.
 Ceux qui veulent détruire Jésus à tout prix n'auront de cesse
 qu'ils ne voient Son influence éliminée de notre société.

Ne restez pas neutres. Ne détournez pas les yeux.
 Prenez ouvertement position pour le Seigneur partout où Il est attaqué.

\dvrule

\dvprayer{
Père, nous sommes en colère contre ceux qui T'attaquent,
 mais nous savons que Ton royaume arrive. Hâte la venue de ce jour, Seigneur.
 Fais arriver Ton royaume de justice de joie et de paix.
}{\DlNdJ}


%%%%%%%%%%%%%%
% 20 juillet
%%%%%%%%%%%%%%

\dvday{Deux Questions}

\dvquote{
Puis il leur dit~:
 Pourquoi avez-vous tellement peur ?
 Comment n'avez-vous pas de foi ?
}{\ibibleverse{Mc}(4:40)}

\lettrine{L}{es disciples avaient de bonnes raisons d'avoir peur}~:
 une violente tempête venait de s'abattre sur eux,
 les vagues entraient au-dessus de la proue et leur embarcation
 commençait à couler. Et cependant, le Seigneur dormait pendant la tempête.

Quand les problèmes nous tombent dessus, qu'ils nous accablent
 par leur intensité et qu'ils menacent de nous faire couler,
 il peut quelquefois sembler que le Seigneur ne s'inquiète pas beaucoup
 de notre situation difficile. Nos appels sont sans réponse
 et nous nous demandons pourquoi le Seigneur n'aide pas,
 pourquoi notre situation ne change pas.

Nous ne faisons jamais face à une tempête tous seuls.
 Le Seigneur est avec nous \ocadr toujours.
 Et quand Jésus est à bord, il n'y a pas besoin d'avoir peur.

Jésus avait deux questions. Après avoir demandé~: \punct{deux-points}
 \og Pourquoi avez-vous tellement peur? \fg{}, Il leur a alors demandé~: \punct{deux-points}
 \og Comment n'avez-vous pas de foi? \fg{}
 Quand nos problèmes semblent énormes,\punct{virgule} notre foi peut faiblir.
 C'est la peur qui a fait perdre la foi aux disciples.

\dvbox{
La peur est un signe que nous ne croyons pas Dieu capable
 de prendre soin de notre situation.
}

Il est merveilleux de savoir que notre Père céleste veille sur nous,
 qu'Il nous aime, qu'Il s'occupe de nous et qu'Il maîtrise tout.
 Quand nous ne comprenons pas ce qui se passe autour de nous,
 nous pouvons toujours faire confiance à Jésus.
 Nous n'avons pas besoin d'avoir peur. Jésus est à bord.
 Il \typo{Majuscule} est capable de calmer la tempête,
 et Il va nous conduire sans encombre jusqu'à notre port éternel.

\dvrule

\dvprayer{
Père, nous Te remercions pour la paix de Christ qui dépasse
 l'entendement humain, pour ce qu'au milieu de la tempête,
 nos cœurs peuvent être en paix parce que notre confiance est en Toi.
}{\DlPNdJ}


%%%%%%%%%%%%%%
% 21 juillet
%%%%%%%%%%%%%%

\dvday{Toucher Jésus}

\dvquote{
Ayant entendu parler de Jésus, elle vint dans la foule par derrière
 et toucha son vêtement. Car elle disait~:
 Si je puis seulement toucher ses vêtements, je serai guérie.
}{\ibibleverse{Mc}(5:27-28)}

\lettrine{C}{ette femme croyait que Jésus avait du pouvoir.}
 Elle était convaincue que si elle pouvait toucher l'ourlet de Son vêtement,
 elle serait guérie de son tourment. Cependant, des obstacles
 lui bloquaient le chemin. La foule se pressait contre Lui,
 poussant, bousculant, emportant Jésus.
 Mais son propre désespoir et sa détermination poussèrent la femme
 à se frayer un chemin jusqu'à ce qu'elle soit assez près de Lui.
 Tendant la main, elle se saisit de la chose qui était à sa portée
 \ocadr l'ourlet de Son vêtement.

Soudain Jésus s'est arrêté et a demandé~: \punct{deux-points}
 \og Qui a touché Mes vêtements? \fg{}

\dvbox{
Beaucoup de personnes dans la foule s'étaient pressées contre Jésus,
 mais une seule a établi le contact avec Lui.
}

Peut-être faites-vous partie de la foule. Vous êtes pressés contre Lui,
 mais vous ne le touchez pas vraiment. Ce n'est pas suffisant.
 Ce n'est pas assez.

Quand Jésus vous touche, Il amène l'amour, la guérison, la délivrance,
 la puissance et la vie. Il veut vous donner toutes ces choses.
 Il veut vous \typo{vous apporter} apporter ce dont vous avez besoin
 et ce à quoi vous aspirez. Il veut guérir vos blessures,
 vous remplir de Son amour, vous délivrer de la puissance de l'ombre
 et vous permettre de vivre la vie que vous avez été créés pour vivre.

Oh, comme nous avons besoin d'être touchés par Jésus dans nos vies !
 Je vous encourage à vous frayer un chemin à travers la foule
 et à entrer en contact avec Jésus aujourd'hui.
 Tendez-Lui \typo{Majuscule à Lui} vos bras et touchez-Le \typo{tiret}
 maintenant. Laissez-Le  \typo{tiret} vous toucher en retour.
 Au moment où vous vous mettez en contact avec Lui,
 vous allez trouver la guérison, la délivrance et l'aide dont vous avez besoin.

\dvrule

\dvprayer{
Jésus, puissions-nous Te toucher aujourd'hui et être délivrés des choses
 qui nous tourmentent. Touche-nous, Seigneur.
}{\Amen}


%%%%%%%%%%%%%%
% 22 juillet
%%%%%%%%%%%%%%

\dvday{Incrédulité}

\dvquote{
Et il ne put faire là aucun miracle, sinon guérir quelques malades
 en leur imposant les mains. Et il s'étonna de leur incrédulité.
}{\ibibleverse{Mc}(6:5-6)}

\lettrine{D}{ans toute la région de la Galilée,}
 Jésus a rencontré des foules de gens. Partout où Il allait,
 Il amenait des bénédictions, Il amenait la guérison.
 Il réconfortait ceux qui avaient le cœur brisé.
 Il guérissait les aveugles et les infirmes.
 Il prêchait la Bonne Nouvelle aux pauvres.
 Au moment de retourner à Nazareth, une multitude de croyants
 marchaient à sa suite.

Mais Nazareth n'a pas reçu Jésus comme les autres communautés
 l'avaient fait. On ne lui pas amené de malades.
 Des gens blessés ne sont pas venus chercher du réconfort.
 Au lieu de ça, les incrédules l'attendaient au tournant,
 le surveillant, se moquant et demandant~:
 \og Où ce garçon du coin a-t-il trouvé cette sagesse? \fg{}
 \og Où a-t-il obtenu cette puissance? \fg{}

\dvbox{
Nos limitations résultent de notre refus de recevoir la puissance de Dieu.
}

Jésus aimait les gens de Sa ville natale autant que les gens
 de partout où il allait. Il y avait beaucoup de choses
 qu'Il voulait faire pour eux, mais leur incrédulité
 les a empêchés de recevoir. Sa puissance n'était pas limitée;
 la puissance de Dieu n'est limitée par rien.
 La limite est venue de leur refus de recevoir Sa puissance.
 À cause de leur incrédulité, ils n'ont pas amené les infirmes,
 les aveugles ni les malades. Ils n'ont pas donné à Jésus l'occasion de servir.

Jésus veut vous bénir aujourd'hui. Il veut vous guérir, vous soulager,
 vous enseigner, mais quand vous oubliez qu'Il est là et qu'Il est capable,
 quand vous le tenez à distance à cause de votre incrédulité,
 vous vous privez vous-mêmes de la bénédiction.

Soyez bien conscients du danger de l'incrédulité. Gardez vos cœurs.
 Entretenez votre foi en Jésus,
 pour ne jamais limiter Son œuvre dans votre vie.

\dvrule

\dvprayer{
Père, nous prions pour ne pas être coupables d'incrédulité.
 Aide-nous à te faire confiance pour accomplir cette œuvre dans nos vies.
}{\DlNdJ}


%%%%%%%%%%%%%%
% 23 juillet
%%%%%%%%%%%%%%

\dvday{Pardonner}

\dvquote{
Et lorsque vous êtes debout en prière, si vous avez quelque chose
 contre quelqu'un, pardonnez, afin que votre Père
 qui est dans les cieux vous pardonne aussi vos fautes.
}{\ibibleverse{Mc}(11:25)}

\lettrine{N}{os corps sont étonnants.} À l'intérieur du laboratoire
 chimique que vous êtes, différentes émotions causent différentes
 réactions chimiques. Comme le Livre des Proverbes nous le dit~: \punct{deux-points}
 \og Un cœur joyeux est un bon remède \fg{} (\ibiblechvs{Pr}(17:22)).
 Le rire et la joie ont des vertus curatives.
 Mais l'inverse est également vrai. L'amertume et la colère
 créent des composés chimiques destructeurs qui attaquent
 notre santé et notre bien-être.

\dvbox{
Le ressentiment est destructeur.
}

Quand nous ressassons nos blessures et offenses, nos corps en pâtissent.
 Peu importe que vous soyez justifiés dans votre colère.
 Peu importe que l'autre personne mérite votre ressentiment.
 La vérité demeure que vous allez vous faire du mal quand vous vous
 attachez à des sentiments de rancœur.
 L'offense peut sembler impardonnable,
 mais il n'empêche qu'il vous faut pardonner.

Certaines blessures sont si profondes qu'il nous semble impossible
 de pardonner, mais ce n'est pas vrai. Votre Père peut vous aider.
 Présentez-Lui \grammar{Impératif} l'offense et demandez \grammar{impératif}
 au Seigneur de vous libérer de la colère que vous éprouvez
 et du désir de revanche. Demandez au Seigneur de vous aider à pardonner
 l'auteur des torts causés \ocadr tout comme Il vous a pardonnés
 quand vous ne le méritiez pas, que vous n'étiez pas dignes de Son pardon.
 Rappelez-vous que Jésus a fait plus qu'enseigner le pardon, Il l'a pratiqué.
 Alors qu'Il était cloué sur la croix, Il a prié~: \punct{deux-points}
 \og Père pardonne-leur, car ils ne savent pas ce qu'ils font \fg{}
 (\ibibleverse{Lc}(23:34)).

Ne laissez pas l'amertume détruire votre vie.
 Si vous souhaitez bénir Jésus, les autres et vous-mêmes, pardonnez.

\dvrule

\dvprayer{
Père, bien que nous ayons péché contre Toi à maintes et maintes reprises,
 Tu nous a pardonnés à maintes et maintes reprises. Aide-nous à t'imiter.
 Aide-nous à pardonner comme nous avons été pardonnés.
}{}

\fixme{Pas de fin de prière}


%%%%%%%%%%%%%%
% 24 juillet
%%%%%%%%%%%%%%

\dvday{Porter des Fruits}

\dvquote{
La saison venue, il envoya un serviteur vers les vignerons
 pour recevoir de leur part des fruits de la vigne.
}{\ibibleverse{Mc}(12:2)}

\lettrine{D}{ieu aime arpenter Sa vigne.}
 Pourquoi? Parce qu'Il recherche du fruit. \\[1ex]
Dans \ibibleverse{Jn}(15:8), Jésus dit~: \punct{deux-points}
 \og Mon Père est glorifié en ceci~: que vous portiez beaucoup de fruit. \fg{}
 Paul continue dans \ibibleverse{Ga}(5:22-23) en expliquant que le fruit
 de l'Esprit est \og amour, joie, paix, patience, bonté, bienveillance,
 fidélité, douceur, maîtrise de soi. \fg{}
 C'est le fruit que Dieu attendait de la nation d'Israël,
 et c'est ce que le Seigneur attend de nous.

\dvbox{
Quel fruit votre vie produit-elle? 
}

Le contraire des fruits,\punct{virgule} ce sont les œuvres.
 Le fruit se développe naturellement comme le résultat d'une relation,
 alors que les œuvres sont ces choses qui sont produites par les efforts,
 l'organisation et la coordination.

Le Seigneur n'est pas intéressé de venir dans une usine,
 d'entendre le bruit assourdissant des machines et de l'acier qu'on travaille,
 de voir la saleté habituellement trouvée dans des endroits
 débordant d'autant \typo{débordant d'autant} d'activités.
 Le Seigneur veut venir dans Son jardin, pour goûter et apprécier
 les fruits qu'Il y trouve.

Quel fruit votre vie produit-elle? Les œuvres de la chair
 ou le fruit de l'Esprit? Si vous voulez porter du fruit pour Dieu
 \ocadr facilement et naturellement \fcadr{} alors vous devez prêter
 attention à votre relation avec Lui. Vous devez Le chercher,
 penser à Lui et étudier Son caractère.

Le fruit n'arrivera jamais par pure volonté ou détermination.
 Il viendra en marchant simplement avec le Jardinier
 et en demeurant dans Son amour.

\dvrule

\dvprayer{
Père, laisse-nous connaître et vivre Ton amour au quotidien,
 le partager avec ceux que nous rencontrons.
 Puissent-ils voir cet amour et se rendre compte
 que nous sommes Tes disciples.
}{\Amen}


%%%%%%%%%%%%%%
% 25 juillet
%%%%%%%%%%%%%%

\dvday{Le Retour du Roi}

\dvquote{
Alors on verra le Fils de l'homme venir sur les nuées
 avec beaucoup de puissance et de gloire.
}{\ibibleverse{Mc}(13:6)}

\lettrine{Q}{uand Jésus est venu la première fois,}
 c'était en tant que serviteur.
 \og Car je suis descendu du ciel pour faire, non ma volonté,
 mais la volonté de celui qui M'a envoyé \fg{} (\ibibleverse{Jn}(6:38)).
 Sa première venue a été pleine de peine, de rejet et de douleur.
 Sa seconde venue sera totalement différente.
 Il va revenir dans la puissance et la gloire, tel
 \og le \typo{minuscule à \og le \fg} Roi des rois \fg{} \punct{espace}
 qui vient établir le règne de Dieu sur la terre.

Les Juifs étaient dans la confusion par rapport à Jésus,
 parce qu'ils se focalisaient seulement sur une moitié des prophéties
 de l'Ancien Testament concernant le Messie à venir.
 Ils attendaient l'arrivée d'un royaume glorieux;
 une utopie sur terre sans plus aucun chagrin ni maladie.
 D'autres versets parlaient du Messie méprisé, rejeté, percé et battu.
 Ces passages prophétisaient qu'Il arriverait dans l'humilité montant un âne.
 Mais parce qu'ils n'arrivaient pas à réconcilier ces deux portraits
 du Messie, les Juifs L'ont rejeté quand Il est apparu.
 Ils n'ont pas compris que ces descriptions étaient toutes les deux vraies,
 et qu'elles allaient trouver leur accomplissement en deux venues distinctes~:
 une pour faire un sacrifice pour le péché des hommes;
 une autre pour régner dans la gloire, la majesté et la puissance.

\dvbox{
Jésus a résumé ce passage des Écritures par un seul mot~: \og Veillez ! \fg{}
}

Nous devrions guetter le retour de notre Seigneur.
 Nous devrions veiller pleins d'attente, jusqu'à la minute de Son retour
 \ocadr de sorte que s'Il venait aujourd'hui
 \fixme{problème avec cette phrase\dots}
 nous il ne nous reste aucun travail encore inachevé.

\dvrule

\dvprayer{
Père, merci pour l'espoir glorieux que nous avons du très prochain
 retour de Jésus-Christ. Nous attendons ce jour avec des soupirs
 d'impatience et de l'anticipation, désireux de voir Ton royaume
 établi dans la justice et la paix.
}{\DlNdJ}


%%%%%%%%%%%%%%
% 26 juillet
%%%%%%%%%%%%%%

\dvday{Force dans l'Esprit}

\dvquote{
Veillez et priez, afin de ne pas entrer en tentation;
 l'esprit est bien disposé, mais la chair est faible.
}{\ibibleverse{Mc}(14:38)}

\lettrine{N}{ous oublions souvent l'humanité de Jésus.}
 Nous nous concentrons tellement sur Sa divinité que nous en oublions
 qu'Il était à la fois Dieu et homme. Il avait bien un côté humain,
 et nous en voyons la preuve dans le jardin de Gethsémané
 \ocadr où Il fit face au plus grand défi de Sa vie.
 Et nous devons Le \typo{Majuscule} louer pour ce qu'Il a enduré.
 Jésus a revêtu un corps de chair pour acquérir une connaissance
 compassionnelle de nos faiblesses. Parce qu'Il s'est battu avec Son humanité,
 là dans le jardin, Il comprend les combats auxquels nous faisons
 face aujourd'hui.

Nos c\oe{}urs veulent faire ce qui est bien.
 Nos c\oe{}urs veulent plaire à Dieu, mais la chair est si faible.
 Sommes-nous toujours condamnés à l'échec? Non, Dieu merci,
 Il a pris des dispositions pour remédier à la faiblesse de notre chair.
 Jésus a dit~: \punct{deux-points}
 \og Vous recevrez une puissance, celle du Saint-Esprit survenant sur vous \fg{}
 (\ibibleverse{Ac}(1:8)).
 L'antidote de Dieu à la faiblesse de notre chair
 est la puissance conférée par l'Esprit Saint.

\dvbox{
La puissance conférée par l'Esprit Saint peut triompher de la chair.
}

Quand vous comptez sur le Saint-Esprit, les domaines où vous êtes
 le plus faible peuvent devenir les domaines où vous êtes le plus fort.
 À l'inverse, quand vous ne comptez que sur vos propres forces,
 vous croyant complètement capables de tout gérer par vous-mêmes,
 c'est précisément là que vous êtes en danger.

On ne peut pas faire confiance à la chair.
 L'Esprit de Dieu, par contre, peut combler vos lacunes,
 vous fortifier là où vous êtes faibles, vous équiper pour travailler,
 vous guider au travers des difficultés et vous transformer
 à l'image de Christ. Sur lequel des deux préfèreriez-vous compter?


\dvrule

\dvprayer{
Père, nous Te remercions pour l'aide qui est à nous au travers
 de la puissance du Saint-Esprit.
 Seigneur, aide-nous à compter sur Toi plutôt que sur nous.
}{\DlNdJ}


%%%%%%%%%%%%%%
% 27 juillet
%%%%%%%%%%%%%%

\dvday{La Tragédie de l'Incrédulité}

\dvquote{
Mais quand ils entendirent qu'elle disait~:
 \og Jésus est vivant, je l'ai vu ! \fg{},
 ils ne la crurent pas.
}{\ibibleverse{Mc}(16:11)}

\lettrine{L}{e tombeau était vide.} La pierre avait été déplacée.
 Et Jésus \ocadr qui avait triomphé de l'enfer et de la tombe \fcadr{}
 était ressuscité.

Ils auraient dû s'attendre à Sa résurrection.
 Ne leur avait-Il pas dit à maintes et maintes reprises
 qu'Il allait être crucifié, mais qu'Il allait ressusciter le troisième jour ?
 On penserait qu'en ce troisième jour, ils auraient été euphoriques
 et bouillants \suggest{bouillonants} d'anticipation.
 Mais quand Marie vient à eux avec ces nouvelles absolument
 bouleversantes\dots{} ils ne la croient pas!
 Le doute les garde dans les pleurs. Et ils continuent de porter
 le deuil de Celui qui n'est plus dans la tombe.

\dvbox{
L'incrédulité amène la tragédie.
}
\suggest{conduit à}

Adam n'a pas cru à ce que Dieu avait affirmé, aussi a-t-Il mangé
 le fruit et introduit le péché et la mort dans le monde.
 Les gens de l'époque de Noé n'ont pas cru à ses avertissements,
 et ainsi le déluge est arrivé et ils furent détruits.
 Parce qu'ils ne croyaient pas que Dieu allait chasser les habitants
 hors de la Terre Promise, les enfants d'Israël ont péri dans le désert.

Que vous coûte votre incrédulité aujourd'hui?
 La tranquillité d'esprit? Un cœur joyeux?
 Êtes-vous tracassés par vos circonstances ou par le sentiment
 que votre vie est hors de contrôle, même si les Écritures vous disent
 de ne pas vous faire de soucis? Ne croyez-vous pas aux promesses de Dieu,
 à la puissance de Dieu et à l'amour de Dieu pour vous?

Puisse Dieu nous donner la foi pour faire confiance et croire
 \ocadr et ce sans tenir compte des circonstances,
 et ce sans tenir compte de nos sentiments.

\dvrule

\dvprayer{
Père, donne-nous la foi de compter sur Tes promesses et Ton caractère,
 sachant que Tu es souverain, fort, puissant et plein d'amour.
 Rappelle-nous au quotidien que Tu es toujours sur le trône.
}{\Amen}


%%%%%%%%%%%%%%
% 28 juillet
%%%%%%%%%%%%%%

\dvday{La Grandeur de Jésus}

\dvquote{
Voici~: tu deviendras enceinte, tu enfanteras un fils,
 et tu l'appelleras du nom de Jésus. Il sera grand et sera appelé
 Fils du Très-Haut, et le Seigneur Dieu lui donnera le trône de David,
 son père.
}{\ibibleverse{Lc}(1:31-32)}

\lettrine{Q}{uelle est la grandeur du Fils?}
 Il est si grand que Jean a écrit~: \punct{deux-points}
 \og Au commencement était la Parole, et la Parole était avec Dieu,
 et la Parole était Dieu [\dots{}] Tout a été fait par elle,
 et rien de ce qui a été fait n'a été fait sans elle \fg{}
 (\ibibleverse{Jn}(1:1,3)). Il est si grand que Paul a écrit en donnant
 plus de détails~: \punct{deux-points}
 \og Car en lui tout a été créé dans les cieux et sur la terre,
 ce qui est visible et ce qui est invisible, trônes, souverainetés,
 principautés, pouvoirs. Tout a été créé par lui et pour lui \fg{}
 (\ibibleverse{Col}(1:16)). Il n'est pas seulement le Créateur,
 Il est l'objet central de la création.
 Vous avez été créés pour Lui et pour Son bon plaisir.

Il nous est dit que \og Lui, dont la condition était celle de Dieu,
 n'a pas estimé comme une proie à arracher d'être égal avec Dieu,
 mais il s'est dépouillé Lui-même [\dots{}] Il s'est humilié lui-même
 en devenant obéissant jusqu'à la mort, la mort sur la croix.
 C'est pourquoi aussi Dieu L'a souverainement élevé et Lui a donné le nom
 qui est au-dessus de tout nom, afin qu'au nom de Jésus tout genou
 fléchisse [\dots{}] et que toute langue confesse que Jésus-Christ
 est Seigneur, à la gloire de Dieu le Père \fg{} (\ibibleverse{Ph}(2:6-11)).

\dvbox{
Comme nous sommes bénis d'aimer et d'être aimés par un Dieu si grand!
}

\dvrule

\dvprayer{
Père, nous sommes en admiration devant Ta nature \ocadr Ta majesté,
 puissance, gloire, sagesse, patience \fcadr{} et en admiration
 devant Ta beauté. Puisses-Tu vite venir pour nous, Seigneur Jésus.
}{\DlPNdJ}


%%%%%%%%%%%%%%
% 29 juillet
%%%%%%%%%%%%%%

\dvday{Préparez le Chemin du Seigneur}

\dvquote{
Préparez le chemin du Seigneur, Rendez droits ses sentiers.
 Toute vallée sera comblée, toute montagne et toute colline seront abaissées;
 les passages tortueux deviendront droits, et les chemins raboteux
 seront nivelés, et toute chair verra le salut de Dieu.
}{\ibibleverse{Lc}(3:4-6)}

\lettrine{L}{e Roi arrivait.} Et Jean-Baptiste avait été appelé
 à être Son précurseur \ocadr pour annoncer que Son Royaume
 était proche et pour exhorter les gens à se préparer par la repentance.
 Jean était essentiellement appelé à aplanir le chemin
 pour l'arrivée du Seigneur.

\dvbox{
Une certaine préparation est nécessaire avant qu'une vie
 ne rencontre Dieu. Nous avons tous besoin d'être dégrossis.
}

Jésus \ocadr le saint et juste Roi \fcadr{} arrivait à un moment
 de grand délabrement moral et spirituel.
 L'atmosphère était hostile à la justice et aux choses du Seigneur.
 Jean allait donc de partout exhorter les gens à se repentir.
 \og Déjà même la cognée est mise à la racine des arbres, Jean les avertissait,
 tout arbre donc qui ne produit pas de bon fruit est coupé et jeté au feu \fg{}
 (\ibibleverse{Lc}(3:9)). \punct{Pas de répétition des guillemets dans l'incise}
 Quand les gens lui demandaient~: \punct{deux-points}
 \og Que ferons-nous donc ? \fg{}, Jean répondait~: \punct{deux-points}
 \og Que celui qui a deux tuniques partage avec celui qui n'en a pas,
 et que celui qui a de quoi manger fasse de même \fg{} (\ibibleverse{Lc}(3:11)).
 Autrement dit, cessez de penser à vous-mêmes. Pensez aux autres.
 Redressez ces chemins tordus dans votre vie. Préparez-vous pour le Roi.

Notre Roi est en route ! Avec David, prions~: \punct{deux-points}
 \og Sonde-moi, ô Dieu, et connais mon cœur !
 Éprouve-moi, et connais mes préoccupations !
 Regarde si je suis sur une mauvaise voie \fg{} (\ibibleverse{Ps}(139:23-24)).

\dvrule

\dvprayer{
Seigneur, nous T'invitons à faire disparaître les aspérités en nous.
 Que Ta volonté soit faite dans nos cœurs, Père.
}{\Amen}


%%%%%%%%%%%%%%
% 30 juillet
%%%%%%%%%%%%%%

\dvday{Conduit}

\dvquote{
Jésus, rempli d'Esprit Saint, revint du Jourdain et fut conduit
 par l'Esprit dans le désert.
}{\ibibleverse{Lc}(4:1)}

\lettrine{P}{arlant de Jésus,} Jean a dit~: \punct{deux-points}
 \og Celui qui déclare demeurer en Lui, doit marcher aussi comme Lui
 (le Seigneur) a marché \fg{} (\ibibleverse{IJn}(2:6)). \punct{Point en trop}
 La question est~: \punct{deux-points} \og Comment Jésus a-t-Il marché? \fg{}
 \punct{Point en trop}
 La réponse est qu'Il a marché selon l'Esprit. 

Quand votre vie est dirigée par l'Esprit, cela veut dire que Dieu
 va quelquefois interrompre vos plans. Quand cela arrive
 \ocadr quand votre journée prend une direction non prévue,
 ou que des invités inattendus passent à l'improviste \fcadr{}
 vous devriez vous arrêter et demander~: \punct{deux-points}
 \og Seigneur, qu'as-Tu en tête? \fg{} \typo{Majuscule}

\dvbox{
La main de Dieu est derrière chaque interruption et chaque détour.
}

Notez que Jésus a été conduit par l'Esprit dans le désert pour être tenté
 par Satan. Nous avons parfois la fausse impression qu'une vie conduite
 par l'Esprit est une vie toute rose sans jamais aucun problème.
 Mais ça n'est pas la vérité! Une vie conduite par l'Esprit
 peut vous mener dans des directions que vous n'aviez pas prévues.
 Lui ferez-vous confiance même quand vous ne comprenez pas?

L'Esprit de Dieu est avec vous. Il fera concourir toutes choses à votre bien.
 Tout comme Jésus était rempli de l'Esprit, conduit par l'Esprit
 et rendu puissant par l'Esprit, nous aussi avons besoin d'être remplis,
 conduits et rendus puissants. Confiez-Lui votre vie
 \ocadr peu importe combien de fois Il vous interrompt,
 peu importe où Il choisit de vous emmener.

\dvrule

\dvprayer{
Père, nous T'offrons notre temps et nos plans.
 Invite-Toi dans notre journée, Père.
 Amène ces conversations, ces situations où Tu veux amener Ta lumière.
 Puissions-nous vivre chaque minute pour Ta gloire.
}{\DlNdJ}


%%%%%%%%%%%%%%
% 31 juillet
%%%%%%%%%%%%%%

\dvday{Puissance pour Vaincre la Tentation}

\dvquote{
Jésus [\dots{}] fut conduit par l'Esprit dans le désert,
 où il fut tenté par le diable pendant quarante jours.
 Il ne mangea rien durant ces jours-là et, quand ils furent achevés,
 il eut faim.
}{\ibibleverse{Lc}(4:1-2)}

\lettrine{S}{atan aime nous attaquer} ou nous tenter
 quand nous sommes dans une condition physique affaiblie.
 Ce fut le cas pour Jésus. Sachant qu'après quarante jours sans nourriture,
 Jésus était faible et affamé, Satan a suggéré qu'Il utilise
 Ses pouvoirs divins pour répondre aux besoins de Sa chair.
 Autrement dit~: \punct{deux-points, ou virgule}
 \og Que le spirituel soit au service du physique. \fg{}

Aucun de nous n'a de pouvoir divin, mais nous faisons face
 à cette même tentation. Satan essaye constamment de nous faire mettre
 le niveau physique de la vie au-dessus du niveau spirituel;
 de faire que nous nous laissions contrôlés par la chair et non par l'Esprit.

Jésus a répondu à la tentation de Satan par la Parole.
 C'est toujours la meilleure façon de s'y prendre avec Satan.
 \og Il est écrit, a dit Jésus, L'homme ne vivra pas de pain seulement,
 mais de toute parole qui sort de la bouche de Dieu \fg{}
 \punct{pas de répétition des guillemets, guillemet en fin}
 (\ibiblephantom{Lc}(4:4)Luc \& \ibibleverse{Mt}(4:4)).
 Jésus confirmait que la vie spirituelle est toujours supérieure
 à la vie physique.

\dvbox{
Jésus a triomphé de la tentation et peut vous aider a remporter
 aussi la victoire.
}

Jésus sait que nous sommes tentés de privilégier le pain aux dépens
 de la Parole. Il a été tenté de la même façon pour qu'Il puisse
 nous comprendre et nous aider dans nos combats.

Quand Satan essaye de nous éloigner, Jésus peut venir à nos côtés
 et dire~: \punct{deux-points}
 \og Je sais ce que tu penses. Je sais que Tu es tenté. Je comprends. \fg{}
 Mais parce qu'Il a triomphé de la tentation, Il peut aussi vous aider
 à remporter la victoire.

\dvrule

\dvprayer{
Père, combien nous sommes reconnaissants de ce que par Ta victoire
 sur la tentation, Tu es capable de nous aider à vaincre.
 Seigneur, nous sommes impuissants sans Toi. Sois notre force.
}{\DlNdJ}

