\dvmonth{Juillet}

%%%%%%%%%%%%%%
% 1er juillet
%%%%%%%%%%%%%%

\dvday{Prisonniers de l\ap{}Espérance}

\dvquote{
Retournez à la forteresse, vous prisonniers de l'espérance.
 Aujourd'hui même, je le déclare, je te rendrai le double.
}{\ibibleverse{Za}(9:12) \KJF}

\lettrine{D}{ans le verset précédent,} Zacharie s'est adressé à ceux
 qui étaient emprisonnés dans une fosse où il n'y avait pas d'eau.
 Maintenant, il s'adresse à ceux qui sont prisonniers de l'espérance. 

Le croyant est comme \og étreint par \fg{} ou \og prisonnier de \fg{}
 l'espérance de la venue du Messie. Nous ne pouvons pas y échapper.
 Nous ne voulons pas y échapper. C'est ce à quoi nous aspirons,
 c'est ce que nous attendons. Nous savons que quand Jésus reviendra,
 les souffrances de l'humanité cesseront enfin. Les guerres cesseront.
 Les chagrins cesseront. Il gouvernera et règnera dans la paix et la justice.
 Et Il nous libèrera de la fosse où nous nous trouvons peut-être aujourd'hui. 

\dvbox{
Nous nous raccrochons à l'espérance qu'Il va revenir bientôt,
 qu'Il va établir son royaume et qu'Il va amener la justice et la paix
 sur une terre qui en a un besoin criant. 
}

Oh, combien l'espérance est indispensable! Combien nous en avons besoin
 pour nous soutenir. Dans un monde qui s'effondre, un monde dans lequel
 le cœur des hommes leur fait défaut devant la peur,
 qu'il est merveilleux d'avoir cette espérance en Jésus-Christ! 

L'Éternel promet à ceux qui sont prisonniers de l'espérance qu'Il leur rendra
 le double de ce qu'ils ont perdu. Il leur explique dans la dernière partie
 du chapitre comment Il va soumettre leurs ennemis, comment Il va glorifier
 Son peuple et comment Il va en faire comme les pierres d'une couronne,
 \og élevée comme une bannière sur Sa terre. \fg{} (\ibibleverse{Za}(9:16)). 

Nous vivons des temps d'une importance capitale, où les événements
 se conjuguent pour annoncer le retour de Jésus. Aussi longtemps que Dieu
 nous prête vie, nous devons avertir les autres des jours mauvais à venir,
 mais nous devons aussi proclamer le jour glorieux qui arrive
 \ocadr le jour du retour du Seigneur. 

Reviens vite, Seigneur Jésus! 

\dvrule

\dvprayer{
Père, nos regards et nos espérances sont fixés sur Toi.
 Nous avons hâte de voir Ton retour; nous attendons avec anticipation
 le jour où Tu vas venir établir Ton royaume. 
}{\NpDlNdJ}


%%%%%%%%%%%%%%
% 2 juillet
%%%%%%%%%%%%%%

\dvday{Une Alliance devant Dieu}

\dvquote{
Et vous dites~: Pourquoi ? \dots{} Parce que l'Éternel a été témoin
 entre toi et la femme de ta jeunesse que tu as trahie,
 bien qu'elle soit ta compagne et la femme de ton alliance.
}{\ibibleverse{Ml}(2:14)}

\lettrine{C}{e n'était pas faute d'essayer.} Les Israëlites arrosaient
 leurs offrandes de larmes, mais le Seigneur refusait de les accepter.
 Quand ils ont demandé \og Pourquoi? \fg{}, Dieu leur a dit que c'était
 à cause de la façon dont ils traitaient leurs épouses. 

Ils avaient rendu le divorce trop facile. Si un homme s'amourachait
 d'une jolie jeune fille, il pouvait abandonner son épouse en déclarant
 simplement~: \punct{deux-points, majuscule} \og Je divorce d'avec toi \fg{}
 trois fois de suite. Et alors, la compagne de sa jeunesse,
 la femme qui l'avait soutenu pendant les années de vaches maigres
 et avait consenti des sacrifices pour l'aider à démarrer dans la vie,
 la femme qui avait porté ses enfants et tenu sa maison,
 n'avait pas d'autre choix que de s'en aller. 

\dvbox{
Dieu déteste le divorce.
}

Dieu détestait le divorce alors, et Il le déteste tout autant maintenant.
 Et j'ai appris que si Dieu déteste quelque chose,
 je ferais bien de m'assurer que je la déteste aussi. 

L'Écriture dit aux hommes que nous devons aimer notre femme même encore plus
 que nous-mêmes. Nous devons l'aimer comme \og Christ a aimé l'Église
 et s'est livré Lui-même pour elle \fg{} (\ibibleverse{Ep}(5:25)). 

Un grande pression s'exerce aujourd'hui contre le mariage.
 Puissions-nous, nous qui sommes mariés, nous rappeler que nous avons fait
 une alliance l'un avec l'autre devant Dieu, \og pour le meilleur
 et pour le pire, dans la richesse ou dans la pauvreté, dans la maladie
 et la santé, de nous aimer et nous chérir
 jusqu'à ce que la mort nous sépare. \fg{}

\dvrule

\dvprayer{
Père, aide-nous à nous aimer l'un l'autre comme Tu nous a aimés,
 et à nous traiter l'un l'autre avec compassion et gentillesse. 
}{\DlNdJ}



