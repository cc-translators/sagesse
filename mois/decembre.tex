\dvmonth{Décembre}

%%%%%%%%%%%%%%%
% 1er decembre
%%%%%%%%%%%%%%%

\dvday{Attendre Son Retour}

\dvquote{
Prenez donc patience, frères, jusqu'à l'avènement du Seigneur.
}{\bibleverse{Jc}(5:7)}

\dvlettrine{J}{ésus} va revenir \ocadr pas comme un petit bébé dans une crèche,
 mais comme le Roi des rois et le Seigneur des seigneurs.
 Il vient pour régner en justice sur toute la terre. Et Il va revenir vite.
 À trois reprises dans le dernier chapitre de la Bible, Jésus a utilisé le mot
 \og vite \fg{} pour qualifier Son retour.

\og Vite \fg{}. C'est bien ce qu'Il a dit.
 Mais 2000 ans ont passé et Il n'est toujours pas encore revenu.
 C'est pour cette raison que les gens se moquent souvent de l'annonce
 de Son retour. Pierre a parlé de ces moqueurs dans le chapitre 3
 de sa deuxième lettre et il a expliqué~:
 \og Mais il est un point que vous ne devez pas oublier, bien-aimés~:
 c'est que, devant le Seigneur, un jour est comme mille ans et mille ans
 sont comme un jour; le Seigneur ne retarde pas l'accomplissement
 de sa promesse, comme quelques-uns le pensent.
 Il use de patience envers vous, il ne veut pas qu'aucun périsse,
 mais il veut que tous arrivent à la repentance \fg{}
 (\bibleverse{IIP}(3:8-9)).

\dvbox{
Le Seigneur a une bonne raison de retarder Son retour.
}

Jacques nous dit que Dieu attend le précieux fruit de la terre
 et qu'il est plein de patience à son égard.
 J'espérais que le Seigneur revienne en 1978, mais je suis très heureux
 qu'Il ne l'ait pas fait. Pensez simplement où certains d'entre vous seraient
 aujourd'hui s'Il était revenu alors. Il attend encore du monde.

J'imagine qu'Il est aussi impatient de nous ramener à la maison
 que nous le sommes d'y aller, mais Il donne encore à d'autres l'opportunité
 d'intégrer Sa famille. Attendez patiemment en affermissant votre c\oe{}ur,
 parce que le Seigneur revient bientôt.

\dvrule

\dvprayer{
Père, merci pour la glorieuse espérance du retour de Jésus-Christ.
 Pendant ce temps, donne-nous, s'il-Te-plaît, la patience d'attendre.
}{\DlNdJ}


%%%%%%%%%%%%%%%
% 2 decembre
%%%%%%%%%%%%%%%

\dvday{Plus Précieux Que l\ap{}Or}

\dvquote{
Vous en tressaillez d'allégresse, quoique vous soyez maintenant,
 pour un peu de temps, puisqu'il le faut, affligés par diverses épreuves,
 afin que votre foi éprouvée \ocadr bien plus précieuse que l'or périssable,
 cependant éprouvé par le feu \fcadr{} se trouve être un sujet de louange,
 de gloire et d'honneur, lors de la révélation de Jésus-Christ\dots{}
}{\bibleverse{IP}(1:6-7)}

\dvlettrine{Q}{uand nous passons} par des épreuves, Dieu ne nous refuse pas
 le droit d'éprouver nos émotions naturelles. Pour un temps, nous sommes
 dans la douleur. Pendant un temps, nous faisons l'expérience du fardeau
 de notre déception, de notre chagrin. Mais quand cette saison de deuil
 est passée, nous faisons quelque chose d'étrange aux yeux du monde~:
 nous nous réjouissons.
 Nous pouvons le faire seulement parce que nous avons les glorieuses
 promesses de Dieu. Même si nous nous sentons accablés,
 même si nous avons souffert une perte ou éprouvé de la douleur,
 nos âmes sont capables de se réjouir.

\dvbox{
Dieu va continuer à retirer les imperfections en nous
 jusqu'à ce qu'Il puisse nous regarder et voir sa réflexion.
}

Pierre a écrit que l'authenticité de votre foi est bien plus précieuse que l'or.
 L'or est connu comme un des métaux précieux, mais il va finir par se détériorer.
 Et tout comme l'orfèvre fait chauffer l'or jusqu'à ce que toute l'écume
 soit brûlée et que l'or soit devenu si pur qu'il puisse y voir sa réflexion,
 Dieu allume un feu sous notre foi. Il permet aux épreuves de briser l'écume
 impure de nos vies et de la faire remonter à la surface.
 C'est à ce moment-là que nous serons purifiés.

Rappelez-vous que les épreuves font mûrir notre foi.
 Puissions-nous ne pas résister à ce travail de purification de Dieu,
 mais nous en réjouir.

\dvrule

\dvprayer{
Père, Merci pour la mise à l'épreuve de notre foi, et pour le feu de l'épreuve
 que Tu amènes pour nous faire grandir et pour nous purifier.
}{\DlNdJ}


%%%%%%%%%%%%%%%
% 3 decembre
%%%%%%%%%%%%%%%

\dvday{Des Ténèbres à la Lumière}

\dvquote{
Vous, par contre, vous êtes une race élue, un sacerdoce royal,
 une nation sainte, un peuple racheté, afin d'annoncer les vertus
 de Celui qui vous a appelés des ténèbres à Son admirable lumière.
}{\bibleverse{IP}(2:9)}

\punct{Point manquant dans la citation}

\dvlettrine{D}{ieu avait prévu} qu'Israël soit un exemple pour le monde
 des bénédictions qu'Il déverse sur ceux qui font de Lui leur Dieu.
 Il a choisi Israël pour être Sa lumière pour le monde.
 Mais les Juifs ont abandonné les voies de Dieu et ont suivi d'autres dieux.
 Ils n'ont pas amené la lumière de Dieu au monde, mais se sont au contraire
 glorifiés d'avoir été choisis par Dieu.

Les bénédictions que Dieu avait offertes à Israël vous sont maintenant offertes,
 vous qui êtes l'église. \suggest{Église}
 Nous sommes Sa prêtrise royale, et tout comme la responsabilité des prêtres
 était d'amener les hommes à Dieu et Dieu aux hommes,
 notre responsabilité est de représenter Dieu auprès des hommes.

\dvbox{
Nous devons laisser briller notre lumière afin que ceux qui vivent
 dans les ténèbres trouvent leur chemin jusqu'à Lui.
}

Dieu nous a sortis des ténèbres jusqu'à la lumière, aussi devons-nous proclamer
 Ses louanges à ceux qui observent nos vies et s'interrogent sur notre force,
 notre espérance et notre joie. Jésus a déclaré être
 \og la lumière du monde \fg{} (\bibleverse{Jn}(8:12)).
 Et Il a fait de nous \og la lumière du monde \fg{} (\bibleverse{Mt}(5:14)).
 Nous sommes appelés à être Ses ambassadeurs et Ses témoins.

Quelle bénédiction que d'appartenir au Père des lumières!
 Et quel privilège que d'attester de Sa bonté à un monde qui est si désespéré
 de recevoir amour et pardon! \punct{Un point en trop avant le point d'exclamation}

\dvrule

\dvprayer{
Père, puissions-nous vraiment laisser briller notre lumière de telle façon que
 quand les hommes voient les bonnes \oe{}uvres que Tu produis en nous,
 ils Te glorifient.
}{\Amen}


%%%%%%%%%%%%%%%
% 4 decembre
%%%%%%%%%%%%%%%

\dvday{La Bonne Vie}

\dvquote{
Si, en effet, quelqu'un veut aimer la vie et voir des jours heureux,
 qu'il préserve sa langue du mal et ses lèvres des paroles trompeuses;
 qu'il s'éloigne du mal et fasse le bien,
 qu'il recherche la paix et la poursuive.
}{\bibleverse{IP}(3:10-11)}

\dvlettrine{S}{i vous souhaitez} vivre une bonne vie,
 voici selon Pierre, les règles à respecter~: \\[1ex]
Règle numéro un~: préservez votre langue du mal.
 Le commérage est du mal à l'état pur. Il n'y a ni amour ni rien de productif
 dans le commérage. Il ne peut que faire du mal.

Règle numéro deux~: préservez vos lèvres des paroles trompeuses.
 Autrement dit, ne trompez pas par votre langage. Les paroles trompeuses
 déforment ou omettent la vérité. C'est un discours qui joue sur les mots.
 Ceux qui sont malhonnêtes dans leurs propos perdent vite
 crédibilité et confiance.

\dvbox{
Obéissez à Dieu et Il vous comblera de joie, de bonheur, de paix,
 et de prospérité.
}

Règle numéro trois~: fuyez ce qui est mal.
 Non seulement, nous faut-il éviter le mal, mais il nous faut aussi rechercher
 le bien. Comme Paul l'écrivait aux Éphésiens~: \punct{deux-points}
 \og dépouillez-vous\dots{} de la vieille nature qui se corrompt
 par les convoitises trompeuses \punct{retrait virgule avant deux-points?}
 \dots{} et revêtez la nature nouvelle, créée selon Dieu dans une justice
 et une sainteté que produit la vérité \fg{} \punct{Point en trop, guillemet fermant manquant}
 (\bibleverse{Ep}(4:22,24)).
 Puis Paul leur a donné des exemples précis.
 Il ne fallait pas seulement arrêter de mentir,
 il fallait commencer à dire la vérité. Il ne fallait pas seulement arrêter
 de voler, il fallait trouver du travail afin de pouvoir aider les nécessiteux.
 Il ne fallait pas seulement arrêter de calomnier, il fallait s'édifier
 les uns les autres avec des paroles. Éviter le mal signifie choisir
 de faire ce qui est bien.

Et quel est le résultat si l'on suit ces règles? Rien que des bénédictions!

\dvrule

\dvprayer{
Père, nous Te remercions de tant nous aimer que Tu désires nous bénir.
 Aide-nous à suivre les règles que Tu as établies pour bien vivre.
}{\DlNdJ}


%%%%%%%%%%%%%%%
% 5 decembre
%%%%%%%%%%%%%%%

\dvday{Gloire à Dieu}

\dvquote{
\dots{} afin qu'en toutes choses Dieu soit glorifié par Jésus-Christ,
 à qui appartiennent la gloire et la puissance aux siècles des siècles.
 Amen !
}{\bibleverse{IP}(4:11)}

\dvlettrine{C}{omme expression} de Son grand et généreux amour,
 Dieu a donné des dons à chacun d'entre nous.
 À certains, Il a donné le don de miséricorde, à d'autres le don d'exhortation,
 à d'autres le don de sagesse ou de prophétie, d'enseignement ou de libéralité.
 Chacun de nous peut être assuré qu'il nous a été donné au moins un don
 \ocadr un domaine dans lequel nous avons une compétence surnaturelle.

\dvbox{
Les dons nous ont été donnés non pas pour notre gloire, mais pour celle de Dieu.
}

Il est triste de voir des gens essayer de retirer de ces dons venus de Dieu,
 notoriété ou adoration. Il est beau, en revanche, de voir quelqu'un donner
 à Dieu le crédit pour ces capacités.
 Un des plus grands compositeurs de l'histoire, Jean-Sébastien Bach,
 avait pour habitude d'inscrire sur chaque partition~: \punct{deux-points}
 \og À la Gloire de Dieu! \fg{}

Comme Paul l'a dit~: \punct{deux-points}
 \og Qu'as-tu que tu n'aies reçu ? Et si tu l'as reçu,
 pourquoi te glorifies-tu, comme si tu ne l'avais pas reçu?
 (\bibleverse{ICo}(4:7)).
 Chaque don trouve sa plus haute et sa plus grande valeur
 quand il sert à amener la grâce de Dieu aux autres.

Savez-vous quel est votre don spirituel? Il est important de découvrir le don
 (ou dons) que Dieu nous a donné et de le mettre au service des autres
 pour glorifier Dieu. Le faites-vous? Bénissez-vous Dieu en lui redonnant
 ainsi les dons qu'Il vous a donnés?

\dvrule

\dvprayer{
Père, nous sommes si impressionnés et émerveillés par Ta générosité
 et Ton amour. Aide-nous Seigneur, à utiliser les dons que Tu nous as confiés
 pour Ton bénéfice et pour Ta gloire.
}{\DlNdJ}


%%%%%%%%%%%%
% 6 decembre
%%%%%%%%%%%%

\dvday{Il prend soin de vous}

\dvquote{
Déchargez-vous sur lui de tous vos soucis, car il prend soin de vous.
}{\bibleverse{IP}(5:7)}

\dvlettrine{A}{nne,} comme l'Ancien Testament nous le rapporte,
 était tourmentée par son incapacité à concevoir un enfant.
 Quand elle essayait d'aborder le sujet avec son mari,
 cela ne faisait que déchirer leur mariage. Pour couronner le tout,
 l'autre épouse de son mari, Peninna, la narguait et se moquait d'elle
 au point où la pression était devenue telle, qu'Anne avait perdu
 tout appétit et ne pouvait s'arrêter de pleurer.
 Alors qu'elle se rendait à Silo avec son mari pour l'une des grandes fêtes,
 elle confia ses problèmes à l'Éternel.
 Elle dit à Éli, le grand prêtre~: \punct{deux-points, majuscule}
 \og Je suis une femme à l'esprit affligé mais j'épanchais mon âme
 devant l'Éternel. \fg{}
 Eli lui répondit~: \punct{deux-points}
 \og Va en paix, et que le Seigneur te donne ce que tu lui as demandé! \fg{}

\dvbox{
Le Seigneur est proche \ocadr et Il veut et peut prendre
 et porter mes fardeaux à ma place.
}

Son mari ne l'avait pas soulagée dans son chagrin,
 mais quand Anne a apporté son fardeau à Dieu,
 elle a aussitôt reçu l'assurance qu'Il s'en occuperait.

Satan veut vous faire croire que vous êtes tout seul face à votre problème.
 Il veut vous tromper en vous faisant croire que personne ne vous comprend
 et que personne ne s'intéresse à vous. Mais Dieu comprend. Dieu s'intéresse.
 Et parce qu'Il se soucie autant de vous, Il vous dit de vous décharger
 sur Lui de tous vos soucis.

Même si j'essuie échec après échec, même si je provoque des problèmes
 dans ma vie, Dieu m'a promis de ne jamais me laisser ni m'abandonner.
 Oh, combien cette pensée m'encourage! Je ne suis pas seul.

\dvrule

\dvprayer{
Père, puissions-nous te confier rapidement tous les soucis et inquiétudes
 qui nous accablent. Nous te remercions d'enlever cette lourde charge.
}{\Amen}


%%%%%%%%%%%%
% 7 decembre
%%%%%%%%%%%%

\dvday{Conformés à Son image}

\dvquote{
Sa divine puissance nous a donné tout ce qui contribue à la vie et à la piété,
 en nous faisant connaître celui qui nous a appelés par Sa propre gloire
 et par Sa vertu.
}{\bibleverse{IIP}(1:3)}

\dvlettrine{D}{ieu désire} que nous soyons semblables à Lui.
 \og Vous serez saints, car je suis saint \fg{} (\bibleverse{IP}(1:16)).
 Ce devrait donc être le but et le désir de chaque croyant d'être conformé
 à l'image de Jésus-Christ \ocadr d'être saint comme Il est saint,
 pur comme Il est pur, parfait (ou complet, mature)
 comme Il est parfait et complet.

Dans la Genèse, nous lisons que Dieu a créé l'homme à Son image.
 Mais quand l'homme a péché et que l'image de Dieu s'est comme déchirée en lui,
 son esprit est mort. C'est le désir de Dieu que de restaurer l'homme
 à Son image. Et d'après ce passage de l'Écriture,
 Dieu nous a donné tout ce dont nous avons besoin
 pour vivre une vie de vertu glorieuse.

\dvbox{
Il ne suffit pas seulement d'avoir un désir de vivre une vie qui plait à Dieu.
}

Si Dieu ne vous confère pas Sa puissance, il est impossible de vivre une vie
 qui plaît à Dieu. C'est par l'intermédiaire de la relation que nous avons
 avec Jésus, que \fixme{majuscule en trop}
 nous devenons comme Lui. Et c'est par la puissance de la Parole,
 implantée profondément dans le sol fertile de nos c\oe{}urs,
 que nous sommes transformés en l'image de Jésus-Christ.

Un beau jour, la Parole de Dieu et l'Esprit de Dieu auront achevé leur \oe{}uvre
 en nous, et nous serons de nouveau semblable à l'image de Dieu.
 \og Je suis persuadé que celui qui a commencé en vous une œuvre bonne,
 en poursuivra l'achèvement jusqu'au jour du Christ-Jésus \fg{}
 (\bibleverse{Phm}(1:6)).

\dvrule

\dvprayer{
Père, enseigne-nous à marcher près de Toi et à aimer Ta Parole,
 afin que nous soyons conformés de nouveau à Ton image.
}{\DlNdJ}


%%%%%%%%%%%%
% 8 decembre
%%%%%%%%%%%%

\dvday{Comment donc devrions-nous vivre ?}

\dvquote{
Puisque tout cela est en voie de dissolution,
 combien votre conduite et votre piété doivent être saintes !
}{\bibleverse{IIP}(3:11)}

\dvlettrine{É}{tant donné qu'un jour} l'univers matériel va complètement
 disparaître, comment devrais-je donc vivre? Seul un fou consacrerait
 tout son temps, toutes ses valeurs et toute son énergie à des choses purement
 matérielles, des choses destinées à brûler. Paul a écrit~: \punct{deux-points}
 \og Aussi nous regardons, non point aux choses visibles,
 mais à celles qui sont invisibles ; car les choses visibles sont momentanées,
 et les invisibles sont éternelles \fg{} (\bibleverse{IICo}(4:18)).
 Si j'évalue ma richesse et mon trésor par mes possessions matérielles,
 quand toutes ces choses seront détruites, il ne me restera absolument rien.

Aussi, quel genre de personnes devrions nous être? Au verset 14,
 Pierre déclare~: \punct{deux-points}
 \og C'est pourquoi, bien-aimés, dans cette attente,
 efforcez-vous d'être trouvés par lui sans tache et sans défaut
 dans la paix. \fg{}
 Nous devons vivre une vie consacrée aux choses de l'Esprit.
 Nous devons être appliqués dans notre démarche avec le Seigneur,
 vivre en harmonie et en paix avec la volonté de Dieu,
 vivant sans tache et sans défaut.

\dvbox{
Si le Seigneur revenait aujourd'hui, seriez-vous prêts à Le rencontrer?
}

Amos, s'est exclamé~: \punct{deux-points}
 \og Prépare-toi à la rencontre de ton Dieu, ô Israël ! \fg{}
 (\bibleverse{Am}(4:12)).
 Faites le point sur votre vie. Êtes-vous en paix avec Dieu,
 vivez-vous en harmonie avec Lui, occupés par le travail du royaume?
 Avez-vous réglé les problèmes dans votre vie? Si vous ne pouvez pas répondre
 \og oui \fg{} à ces questions, commencez aujourd'hui. Préparez votre c\oe{}ur.

\dvrule

\dvprayer{
Père, nous Te remercions pour l'espérance du royaume de Dieu.
 Nous prions pour que, quand Tu reviendras, Tu nous trouves comme
 des serviteurs fidèles occupés à faire Ta volonté.
}{\DlNdJ}


%%%%%%%%%%%%
% 9 decembre
%%%%%%%%%%%%

\dvday{Pas de Ténèbres}

\dvquote{
Voici le message que nous avons entendu de Lui et que nous vous annonçons~:
 Dieu est lumière, il n'y a pas en Lui de ténèbres.
}{\bibleverse{IJn}(1:5)}

\dvlettrine{P}{ersonne} n'a vraiment vu le soleil. Vous voyez la brillance
 qui rayonne du soleil, mais vous n'avez pas vraiment vu le soleil lui-même.
 En fait, c'est en raison de cette brillance que vous ne pouvez pas voir
 la source de la lumière. De la même façon,
 \og personne \fixme{majuscule?}
 n'a jamais vu Dieu; Dieu le Fils unique, qui est dans le sein du Père, lui,
 l'a fait connaître \fg{} (\bibleverse{Jn}(1:18)).
 Jésus est la radiance du Père.

\dvbox{
La lumière et l'obscurité sont mutuellement exclusives.
}

Il y a, soit de la lumière, soit de l'obscurité, mais vous ne pouvez pas
 avoir un jour mi-lumière, mi-obscurité au même moment parce que la lumière
 chasse les ténèbres. Et parce que Dieu est lumière, Sa présence chasse
 les ténèbres. En Lui, Il n'y a pas du tout de ténèbres.
 Comme Paul le demandait~: \punct{deux-points, guillemets en trop}
 \og Quelle communion y-a-t-il entre la lumière et les ténèbres? \fg{}
 (\bibleverse{IICo}(6:14)).

L'obscurité du péché éloigne l'homme de Dieu. Dieu vous aime et aspire
 à la communion avec vous. Mais ne vous y trompez pas.
 Si vous marchez dans les ténèbres, vous ne pouvez pas avoir
 la communion avec Dieu.
 \og Si nous disons que nous sommes en communion avec Lui,
 et que nous marchions dans les ténèbres, nous mentons et nous ne pratiquons
 pas la vérité. Mais si nous marchons dans la lumière,
 comme Il est lui-même dans la lumière, nous sommes en communion
 les uns avec les autres \fg{} (\bibleverse{IJn}(1:6-7)).

Marchez avec Lui aujourd'hui. Renoncez à l'obscurité du péché;
 choisissez la Lumière.


\dvrule

\dvprayer{
Père, comme nous sommes reconnaissants de ce que la lumière de l'évangile
 a brillé jusqu'à nous, afin que nous puissions marcher dans la lumière,
 comme Tu es dans la lumière, et que nous apprécions la communion avec Toi.
}{\DlNdJ}

\grammar{appréciions?}


%%%%%%%%%%%%
% 10 decembre
%%%%%%%%%%%%

\dvday{La Vie qui Triomphe}

\dvquote{
Je vous ai écrit, jeunes gens, parce que vous êtes forts,
 que la parole de Dieu demeure en vous et que vous avez vaincu le Malin.
}{\bibleverse{IJn}(2:14)}

\dvlettrine{N}{e vous y trompez pas~:} si vous êtes chrétiens,
 vous êtes engagés dans un combat. L'enjeu de ce combat est la maîtrise
 de vos pensées et donc de votre vie. Dieu veut régner sur votre vie
 pour vous bénir et vivre en communion avec vous.
 Satan veut régner sur votre vie pour la détruire.
 Usant de mensonges et de promesses d'immédiate satisfaction,
 il essaye de vous séduire par la convoitise de votre chair.
 Mais quel que soit le gain qu'il vous accorde,
 ce n'est qu'un plaisir temporaire.

\dvbox{
Dieu veut que vous triomphiez des tentations de Satan.
}

Dieu veut que vous connaissiez la victoire. Mais vous ne pouvez pas connaître
 la victoire sur la base de votre propre force. Vous ne pouvez vaincre
 que par Jésus-Christ qui peut vous conférer la puissance nécessaire pour
 confronter les ruses du diable.

Quel est le secret de la victoire? Le passage d'aujourd'hui dit que la force
 des jeunes gens venait de la Parole de Dieu demeurant en eux.
 Comme David le disait~: \punct{deux-points}
 \og Je serre ta promesse dans mon cœur, afin de ne pas pécher contre toi \fg{}
 (\bibleverse{Ps}(119:11)).
 Jésus a triomphé des séductions de Satan avec les Écritures, en répondant
 à chaque mensonge tentateur de Satan par une vérité de la Parole de Dieu.
 Si la Parole de Dieu est dans nos c\oe{}urs, alors nous ne serons pas
 vulnérables aux mensonges de Satan.
 Quand il amènera des tentations, nous aurons en nous la force dont nous
 avons besoin pour contrebalancer ses mensonges avec les vérités bénies \grammar{bénies}
 de notre Père.


\dvrule

\dvprayer{
Père, merci de ce que nous pouvons triompher en suivant Tes pas
 \ocadr en vivant une vie qui Te plaît, une vie qui est gouvernée
 par Ton Esprit.
}{\DlNdJ}


%%%%%%%%%%%%
% 11 decembre
%%%%%%%%%%%%

\dvday{Enfants de Dieu}

\dvquote{
Bien-aimés, nous sommes maintenant enfants de Dieu,
 et ce que nous serons n'a pas encore été manifesté ;
 mais nous savons que lorsqu'Il sera manifesté,
 nous serons semblables à Lui,
 parce que nous le verrons tel qu'Il est.
}{\bibleverse{IJn}(3:2)}

\dvlettrine{P}{arce que nous sommes Ses enfants,}
 Dieu nous réserve un glorieux futur.
 Comme Sa Parole nous le dit~: \punct{deux-points}
 \og Ce que l'œil n'a pas vu, Ce que l'oreille n'a pas entendu,
 et ce qui n'est pas monté au cœur de l'homme,
 c'est tout ce que Dieu a préparé pour ceux qui L'aiment \fg{}
 (\bibleverse{ICo}(2:9)). 

Jésus a prié pour que nous puissions partager Son royaume avec Lui.
 \og Père, Je veux que là où Je suis, ceux que Tu m'as donnés
 soient aussi avec Moi, afin qu'ils contemplent Ma gloire,
 celle que Tu m'as donnée \fg{} (\bibleverse{Jn}(17:24)). 

\dvbox{
Un jour nous passerons au travers du voile appelé la mort
 et nous verrons le visage de notre Sauveur. 
}

Quand nous quitterons ce corps et commencerons notre vie avec Dieu,
 Il nous révèlera les profondeurs, les richesses, la plénitude de Son amour
 \ocadr un amour si grand qu'il faudra tout l'éternité pour le comprendre.
 Et même alors, je ne crois pas que nous serons capables d'appréhender
 pleinement la totalité de Son amour. 

Nous habiterons dans la glorieuse lumière de Sa présence et nous vivrons
 pour toujours dans Son royaume, découvrant jour après jour,
 année après année, milliard d'années après milliard d'années,
 la richesse de Son amour, de Sa grâce et de Sa miséricorde envers nous. 

\og Voyez de quel amour le Père nous a aimés, puisque nous sommes appelés
 enfants de Dieu! \fg{} (\bibleverse{IJn}(3:1)).
 Comme nous sommes bénis d'être à Lui! 

\dvrule

\dvprayer{
Père, nous attendons avec impatience ce royaume éternel
 quand Tu nous révèleras la richesse surabondante de Ta grâce,
 de Ton amour et de Ta miséricorde en Jésus-Christ.
}{\CeSNqnp}


%%%%%%%%%%%%
% 12 decembre
%%%%%%%%%%%%

\dvday{Pourquoi est-Il venu?}

\dvquote{
Le Fils de Dieu est apparu,
 afin de détruire les œuvres du diable.
}{\bibleverse{IJn}(3:8)} 

\dvlettrine{Q}{uelles sont} les œuvres du diable?
 Ce sont les œuvres qui ont séparé l'homme qui vivait
 au milieu des tombeaux (\bibleverse{Mc}(5:2)) de sa famille
 et du reste de la société, et qui le tourmentaient tant
 qu'il se meurtrissait avec des pierres.
 Les œuvres du diable, ce sont ces démons qui habitaient le jeune garçon
 du chapitre 9 de Marc et qui le faisaient grincer des dents,
 entrer en convulsions et baver \ocadr ces esprits qui le jetaient
 de façon répétée dans l'eau et le feu, essayant ainsi de le tuer. 

\dvbox{
Jésus est venu pour détruire les œuvres du diable. 
}

Ne vous y trompez pas; Satan a de la haine pour vous et veut vous détruire.
 Il va essayer par tous les moyens de vous attirer sur un chemin
 qui mène à la mort \ocadr un chemin de rébellion contre Dieu.
 Jésus nous a averti que le diable est venu pour
 \og voler et tuer et détruire \fg{} (\bibleverse{Jn} 10:10)).
 Il va vous dérober la vie bénie. Il va tuer votre relation avec Dieu
 et avec les autres. Il va détruire votre réputation et finalement votre vie.
 Bien que ses méthodes de recrutement soient tentantes,
 ceux qui rejoignent son camp partageront sa destinée. 

L'une des œuvres du diable est la puissance de la mort.
 Mais Jésus a détruit cette puissance en nous amenant la vie éternelle; \punct{point, ou pas de majuscule avec point-virgule}
 par opposition aux buts destructeurs de Satan à votre égard,
 Jésus est venu \og afin que les brebis aient la vie
 et qu'elles l'aient en abondance \fg{} (\bibleverse{Jn}(10:10)). 

\dvrule

\dvprayer{
Père, nous Te sommes tellement reconnaissants de Ton amour.
 Merci d'avoir envoyé Ton Fils dans le monde pour détruire
 les œuvres du diable, d'avoir envoyé Jésus dans le monde
 pour nous donner la vie éternelle dans Ton Royaume.
}{\Amen} 


%%%%%%%%%%%%
% 13 decembre
%%%%%%%%%%%%

\dvday{L\ap{}Amour de Dieu}

\dvquote{
Et nous, nous avons connu l'amour que Dieu a pour nous,
 et nous y avons cru.
}{\bibleverse{IJn}(4:16)} 

\dvlettrine{B}{ien que Son amour soit immense,}
 certains refusent de croire que Dieu les aime.
 En dépit de toutes les preuves qu'Il a données de Son grand amour,
 certaines personnes nient cet amour parce que, disent-elles,
 Il les a laissées tomber.
 \og J'ai prié et demandé certaines choses, disent-elles, \punct{Pas de répétition des guillemets dans ce genre d'incise}
 mais Dieu n'a pas répondu à ces prières. \fg{}
 Pour ces personnes, la preuve de l'amour de Dieu
 \ocadr et dans certains cas, l'existence même de Dieu \fcadr{}
 seraient non seulement des prières exaucées,
 mais des prières exaucées selon leurs désirs. 

Qu'il est insensé de faire reposer votre reconnaissance de l'amour de Dieu
 pour vous sur ce seul test!
 Donnons-nous à nos enfants tout ce qu'ils demandent? Bien sûr que non.
 Quelquefois ils demandent des choses qui ne sont pas bonnes pour eux.
 La prière n'est pas un test de l'amour de Dieu.
 La prière n'est pas un moyen par lequel votre propre volonté est accomplie.
 La prière est un moyen par lequel Dieu accomplit Sa volonté. 

Si donc la prière n'est pas le test de l'amour de Dieu, quel est le vrai test?
 Comment pouvons-nous savoir que nous sommes aimés?
 Jean nous a donné la réponse à ces questions au chapitre précedent.
 \og À ceci, nous avons connu l'amour~:
 c'est qu'Il a donné sa vie pour nous \fg{} (\bibleverse{IJn}(3:16)). 

\dvbox{
La preuve de l'amour de Dieu, c'est Jésus. 
}

La preuve, c'est le Fils de Dieu allant de son plein gré à la croix,
 prenant sur Lui les péchés du monde et mourant pour nous racheter.
 Quand vous avez rencontré Jésus, vous avez aussi rencontré
 l'Amour en personne, et vous pouvez dire avec Jean~: \punct{deux-points}
 \og Nous avons connu l'amour que Dieu a pour nous, et nous y avons cru. \fg{}

\dvrule

\dvprayer{
Père, merci pour ton grand amour pour nous.
 Que Ton amour sois rendu parfait dans nos vies
 et qu'il efface toutes nos peurs.
}{\Amen}


%%%%%%%%%%%%
% 14 decembre
%%%%%%%%%%%%

\dvday{La Vérité}

\dvquote{
Je me suis beaucoup réjoui de trouver de tes enfants
 qui marchent dans la vérité, selon le commandement
 que nous avons reçu du Père.
}{\bibleverse{IIJn}(1:4)}

\fixme{La référence biblique est imcomplète, il manque le numéro de chapitre}

Qu'est-ce que la vérité? La philosophie d'aujourd'hui affirme que \punct{pas besoin de virgule pour la citation}
 \og puisque la vérité est relative, \fg{} \punct{besoin d'une virgule dans la phrase}
 tout ce que vous croyez être vrai est vrai.
 Mais ceux qui suivent ce raisonnement ont rejeté la vérité authentique.
 Paul a expliqué que Dieu S'est clairement révélé à l'homme par la création,
 mais l'homme \og a remplacé la vérité de Dieu par le mensonge
 et a adoré et servi la créature au lieu du Créateur \fg{}
 (\bibleverse{Rm}(1:25)). 

\dvbox{
Dieu est vérité. Il est l'autorité finale pour ce qui est de la vérité. 
}

Dieu ne se découvre pas par quête intellectuelle, mais par révélation écrite
 \ocadr par l'intermédiaire de la Bible.
 La Bible est la Parole de Dieu et par conséquent la vérité de Dieu.
 Quand Il priait pour les disciples, Jésus a dit~: \punct{deux-points}
 \og Sanctifie-les par Ta vérité~: Ta Parole est la vérité \fg{} \punct{Point en trop, guillemet manquant}
 (\bibleverse{Jn}(17:17)). \punct{Point manquant}

Qu'est-ce que la vérité? La vérité, c'est que vous avez été créés
 par un Créateur doué d'une sagesse parfaite.
 Vous n'êtes pas le fruit d'un long processus d'évolution
 fait de circonstances accidentelles, mais vous avez été créés
 dans un but bien spécifique. La vérité, c'est que le Dieu éternel
 tout-puissant qui vous a créés, vous aime et veut le meilleur pour vous;
 Il vous a donc offert un moyen d'obtenir le salut
 \ocadr Son Fils, Jésus-Christ \fcadr{} et un moyen d'avoir une vie abondante
 \ocadr Ses préceptes de vie, la Bible.
 Et si vous soumettez votre vie au Fils de Dieu et que vous suivez
 les préceptes de Dieu, vous ferez l'expérience d'une vie joyeuse
 sur la terre et celle de la vie éternelle au ciel. 

\dvrule

\dvprayer{
Père, comme nous Te sommes reconnaissants de nous avoir donné Ta Parole,
 par laquelle nous pouvons connaître Ta vérité
 \ocadr la seule vérité qui soit!
 Donne-nous un amour pour Ta Parole et le désir de nous y soumettre.
}{DlNdJ}


%%%%%%%%%%%%
% 15 decembre
%%%%%%%%%%%%

\dvday{Imiter Jésus}

\dvquote{
Bien-aimé, n'imite pas le mal, mais le bien.
 Celui qui fait le bien est de Dieu ;
 celui qui fait le mal n'a pas vu Dieu.
}{\bibleverse{IIIJohn}(1:11)}

\fixme{La référence biblique est imcomplète, il manque le numéro de chapitre}

\dvlettrine{C}{ette exhortation de Jean} se trouve entre les références
 à un homme mauvais, Diotrèphe, et à un homme bien, \fixme{Un homme bon, on un homme de bien}
 Démétrius.
 En comparant les deux, Jean nous exhorte à imiter ce qui est bien, \fixme{Ou bon?}
 non pas ce qui est mauvais. 

Jean dit de Diotrèphe qu'il \og aime à être le premier parmi eux. \fg{}
 Jésus avait dit~: \punct{deux-points}
 \fg{} Les rois des nations les dominent\dots{}
 Il n'en est pas de même pour vous. Mais que le plus grand parmi vous
 soit comme le plus jeune, et celui qui gouverne comme celui qui sert \fg{}
 (\bibleverse{Lc}(22:25-26)).
 Si Dieu vous a donné une position d'autorité dans l'église,
 cela signifie que Dieu vous a placé là pour être un serviteur de l'église.
 \suggest{S'agit-il de l'église ou de l'Église?}

\dvbox{
Que ça vous plaise ou non, vous exercez une influence sur quelqu'un
 \ocadr en bien ou en mal. 
}

Peu importe où vous vous trouvez, quelqu'un vous considère comme un modèle
 et suit votre exemple. Votre enfant apprend à être un parent en vous
 regardant faire. Les jeunes chrétiens apprennent ce que suivre Christ
 signifie en vous observant. Comme il est important de prendre
 cette responsabilité au sérieux! Qu'est-ce-que les gens voient en vous? 

Si vous n'aimez pas la réponse, vous pouvez commencer à la changer.
 \og Sois un modèle pour les fidèles, en parole, en conduite,
 en amour, en foi, en pureté \fg{} (\bibleverse{ITm}(4:12)). 

\dvrule

\dvprayer{
Père, aide-nous à imiter les bons exemples que nous voyons dans Ta Parole,
 à tirer les leçons des mauvais, et à vivre, attentifs et bien conscients
 que les autres nous observent et apprennent de nous.
}{\DlNdJ}


%%%%%%%%%%%%
% 16 decembre
%%%%%%%%%%%%

\dvday{Maintenez-vous dans l'Amour de Dieu}

\dvquote{
Maintenez-vous dans l'amour de Dieu\dots{}
}{\bibleverse{Jude}(1:21)} 

\fixme{La référence biblique est imcomplète, il manque le numéro de chapitre}

\dvlettrine{J}{ude veut-il dire} que je dois être toujours \suggest{ou toujours être?}
 si gentil et si merveilleux que Dieu ne va pas avoir
 d'autre choix que de m'aimer?
 Si c'était le cas, je serais vraiment dans le pétrin.
 Heureusement, ce n'est pas ce que Jude voulait dire.
 Il voulait dire que nous devons nous maintenir
 dans une position de bénédiction
 \ocadr une position dans laquelle Dieu peut faire toutes les choses
 qu'Il a très envie de faire pour nous. 

Les géants de la Terre Promise avaient tellement effrayé
 les enfants d'Israël que ceux-ci refusèrent d'y entrer.
 Bien qu'ils aient été délivrés d'Égypte, \punct{espace manquant}
 ils périrent dans le désert à cause de leur incrédulité.
 Ils se sont privés de la plénitude des bénédictions
 que Dieu leur réservait.

\dvbox{
La meilleure façon de vous maintenir dans l'amour de Dieu
 est de vous rendre compte que Jésus revient bientôt. 
}

Beaucoup d'entre vous ont été délivrés de l'esclavage du péché,
 mais vous continuez à vivre dans le désert.
 Vous n'êtes pas entrés dans cette vie pleine et riche en Jésus,
 parce que vous vous accrochez encore au monde. 

Jésus revient bientôt. Quelque chose dans cette idée vous fait
 relâcher votre étreinte des choses du monde.
 Elle vous donne le désir de tendre les mains vers le ciel.
 Elle vous donne le désir de vous purifier et de changer vos priorités.
 Elle vous donne un sentiment de l'urgence à faire part du message
 de l'évangile à un monde qui se meurt. 

Maintenez-vous dans l'amour de Dieu. Croyez en Ses promesses,
 soumettez-vous à Lui et obéissez à tout ce qu'Il vous demande de faire.
 Puis préparez-vous à toutes les bénédictions qu'Il va amener. 

\dvrule

\dvprayer{
Père, nous prions pour que Tu nous aides à nous maintenir
 dans cette position de bénédiction,
 en nous souvenant que Tu reviens bientôt.
}{\DlNdJ}

\suggest{« pour » nécessaire?}
\fixme{sous souvenant}


%%%%%%%%%%%%
% 17 decembre
%%%%%%%%%%%%

\dvday{Jésus Revient}

\dvquote{
Voici qu'Il vient avec les nuées. Tout homme Le verra,
 même ceux qui L'ont percé ; et toutes les tribus de la terre
 se lamenteront à Son sujet. Oui, Amen!
}{\bibleverse{Ap}(1:7)} 

\dvlettrine[ante=\og]{V}{oici qu'Il vient \fg{}} \suggest{virgule?}
 a dit Jean.
 Bien que beaucoup de versets de l'Écriture annoncent
 l'enlèvement de l'église \suggest{Église}
 \ocadr cet évènement qui se passera en un clin d'œil \fcadr{} \typo{œil}
 cette prophétie ne fait pas référence à l'enlèvement.
 Cette prophétie prédit le retour de Jésus-Christ.
 Ces deux évènements \ocadr l'enlèvement et le retour \fcadr{}
 sont des évènements séparés, bien distincts.
 Il ne faut pas les confondre. 

L'enlèvement concerne l'église. \suggest{Église}
 C'est ce moment où Jésus reviendra chercher Son épouse.
 \suggest{C'est à ce moment que}
 Personne ne sait quand cela se passera; tout ce que nous savons,
 c'est que chacun des signes que Jésus nous a donnés pour nous préparer
 à cet évènement s'est déjà produit. 

\dvbox{
Tous les signes se sont accomplis, ce qui signifie que l'enlèvement
 pourrait survenir à n'importe quel moment.
 Ce pourrait même être aujourd'hui. 
}

Quelque \grammar{Quelque} sept années après l'enlèvement,
 Jésus reviendra avec Son église pour juger la terre
 et établir le royaume de Dieu. C'est le retour de Jésus.
 Il surviendra après un temps de grande tribulation
 telle que le monde n'en a encore jamais vue
 \ocadr une période de terreur et de dévastation. 

Vous ne voulez pas être ici quand le jugement de Dieu interviendra.
 La chose merveilleuse, c'est que vous n'avez pas à y être.
 Jésus revient d'abord pour recevoir et accueillir Son église. \suggest{Église}
 Si l'enlèvement devait arriver aujourd'hui
 \ocadr et cela se pourrait très bien \fcadr{} êtes-vous prêts? 

\dvrule

\dvprayer{
Seigneur, nous voulons être prêts quand Tu reviendras.
 Aide-nous à saisir chaque opportunité pour grandir
 dans l'intimité avec Toi.
 Aide-nous à continuellement scruter le ciel en T'attendant.
}{\Amen}



