\dvmonth{Décembre}

%%%%%%%%%%%%%%%
% 1er decembre
%%%%%%%%%%%%%%%

\dvday{Attendre Son Retour}

\dvquote{
Prenez donc patience, frères, jusqu'à l'avènement du Seigneur.
}{\bibleverse{James}(5:7)}

\dvlettrine{J}{ésus} va revenir \ocadr pas comme un petit bébé dans une crèche,
 mais comme le Roi des rois et le Seigneur des seigneurs.
 Il vient pour régner en justice sur toute la terre. Et Il va revenir vite.
 À trois reprises dans le dernier chapitre de la Bible, Jésus a utilisé le mot
 \og vite \fg{} pour qualifier Son retour.

\og Vite \fg{}. C'est bien ce qu'Il a dit.
 Mais 2000 ans ont passé et Il n'est toujours pas encore revenu.
 C'est pour cette raison que les gens se moquent souvent de l'annonce
 de Son retour. Pierre a parlé de ces moqueurs dans le chapitre 3
 de sa deuxième lettre et il a expliqué~:
 \og Mais il est un point que vous ne devez pas oublier, bien-aimés~:
 c'est que, devant le Seigneur, un jour est comme mille ans et mille ans
 sont comme un jour; le Seigneur ne retarde pas l'accomplissement
 de sa promesse, comme quelques-uns le pensent.
 Il use de patience envers vous, il ne veut pas qu'aucun périsse,
 mais il veut que tous arrivent à la repentance \fg{}
 (\bibleverse{IIPet}(3:8-9)).

\dvbox{
Le Seigneur a une bonne raison de retarder Son retour.
}

Jacques nous dit que Dieu attend le précieux fruit de la terre
 et qu'il est plein de patience à son égard.
 J'espérais que le Seigneur revienne en 1978, mais je suis très heureux
 qu'Il ne l'ait pas fait. Pensez simplement où certains d'entre vous seraient
 aujourd'hui s'Il était revenu alors. Il attend encore du monde.

J'imagine qu'Il est aussi impatient de nous ramener à la maison
 que nous le sommes d'y aller, mais Il donne encore à d'autres l'opportunité
 d'intégrer Sa famille. Attendez patiemment en affermissant votre c\oe{}ur,
 parce que le Seigneur revient bientôt.

\dvrule

\dvprayer{
Père, merci pour la glorieuse espérance du retour de Jésus-Christ.
 Pendant ce temps, donne-nous, s'il-Te-plaît, la patience d'attendre.
}{\DlNdJ}


%%%%%%%%%%%%%%%
% 2 decembre
%%%%%%%%%%%%%%%

\dvday{Plus Précieux Que l'Or}

\dvquote{
Vous en tressaillez d'allégresse, quoique vous soyez maintenant,
 pour un peu de temps, puisqu'il le faut, affligés par diverses épreuves,
 afin que votre foi éprouvée \ocadr bien plus précieuse que l'or périssable,
 cependant éprouvé par le feu \fcadr{} se trouve être un sujet de louange,
 de gloire et d'honneur, lors de la révélation de Jésus-Christ\dots{}
}{\bibleverse{IPet}(1:6-7)}

\dvlettrine{Q}{uand nous passons} par des épreuves, Dieu ne nous refuse pas
 le droit d'éprouver nos émotions naturelles. Pour un temps, nous sommes
 dans la douleur. Pendant un temps, nous faisons l'expérience du fardeau
 de notre déception, de notre chagrin. Mais quand cette saison de deuil
 est passée, nous faisons quelque chose d'étrange aux yeux du monde~:
 nous nous réjouissons.
 Nous pouvons le faire seulement parce que nous avons les glorieuses
 promesses de Dieu. Même si nous nous sentons accablés,
 même si nous avons souffert une perte ou éprouvé de la douleur,
 nos âmes sont capables de se réjouir.

\dvbox{
Dieu va continuer à retirer les imperfections en nous
 jusqu'à ce qu'Il puisse nous regarder et voir sa réflexion.
}

Pierre a écrit que l'authenticité de votre foi est bien plus précieuse que l'or.
 L'or est connu comme un des métaux précieux, mais il va finir par se détériorer.
 Et tout comme l'orfèvre fait chauffer l'or jusqu'à ce que toute l'écume
 soit brûlée et que l'or soit devenu si pur qu'il puisse y voir sa réflexion,
 Dieu allume un feu sous notre foi. Il permet aux épreuves de briser l'écume
 impure de nos vies et de la faire remonter à la surface.
 C'est à ce moment-là que nous serons purifiés.

Rappelez-vous que les épreuves font mûrir notre foi.
 Puissions-nous ne pas résister à ce travail de purification de Dieu,
 mais nous en réjouir.

\dvrule

\dvprayer{
Père, Merci pour la mise à l'épreuve de notre foi, et pour le feu de l'épreuve
 que Tu amènes pour nous faire grandir et pour nous purifier.
}{\DlNdJ}


%%%%%%%%%%%%%%%
% 3 decembre
%%%%%%%%%%%%%%%

\dvday{Des Ténèbres à la Lumière}

\dvquote{
Vous, par contre, vous êtes une race élue, un sacerdoce royal,
 une nation sainte, un peuple racheté, afin d'annoncer les vertus
 de Celui qui vous a appelés des ténèbres à Son admirable lumière.
}{\bibleverse{IPet}(2:9)}

\punct{Point manquant dans la citation}

\dvlettrine{D}{ieu avait prévu} qu'Israël soit un exemple pour le monde
 des bénédictions qu'Il déverse sur ceux qui font de Lui leur Dieu.
 Il a choisi Israël pour être Sa lumière pour le monde.
 Mais les Juifs ont abandonné les voies de Dieu et ont suivi d'autres dieux.
 Ils n'ont pas amené la lumière de Dieu au monde, mais se sont au contraire
 glorifiés d'avoir été choisis par Dieu.

Les bénédictions que Dieu avait offertes à Israël vous sont maintenant offertes,
 vous qui êtes l'église. \suggest{Église}
 Nous sommes Sa prêtrise royale, et tout comme la responsabilité des prêtres
 était d'amener les hommes à Dieu et Dieu aux hommes,
 notre responsabilité est de représenter Dieu auprès des hommes.

\dvbox{
Nous devons laisser briller notre lumière afin que ceux qui vivent
 dans les ténèbres trouvent leur chemin jusqu'à Lui.
}

Dieu nous a sortis des ténèbres jusqu'à la lumière, aussi devons-nous proclamer
 Ses louanges à ceux qui observent nos vies et s'interrogent sur notre force,
 notre espérance et notre joie. Jésus a déclaré être
 \og la lumière du monde \fg{} (\bibleverse{Jn}(8:12)).
 Et Il a fait de nous \og la lumière du monde \fg{} (\bibleverse{Matt}(5:14)).
 Nous sommes appelés à être Ses ambassadeurs et Ses témoins.

Quelle bénédiction que d'appartenir au Père des lumières!
 Et quel privilège que d'attester de Sa bonté à un monde qui est si désespéré
 de recevoir amour et pardon! \punct{Un point en trop avant le point d'exclamation}

\dvrule

\dvprayer{
Père, puissions-nous vraiment laisser briller notre lumière de telle façon que
 quand les hommes voient les bonnes \oe{}uvres que Tu produis en nous,
 ils Te glorifient.
}{\Amen}


%%%%%%%%%%%%%%%
% 4 decembre
%%%%%%%%%%%%%%%

\dvday{La Bonne Vie}

\dvquote{
Si, en effet, quelqu'un veut aimer la vie et voir des jours heureux,
 qu'il préserve sa langue du mal et ses lèvres des paroles trompeuses;
 qu'il s'éloigne du mal et fasse le bien,
 qu'il recherche la paix et la poursuive.
}{\bibleverse{IPet}(3:10-11)}

\dvlettrine{S}{i vous souhaitez} vivre une bonne vie,
 voici selon Pierre, les règles à respecter~: \\[1ex]
Règle numéro un~: préservez votre langue du mal.
 Le commérage est du mal à l'état pur. Il n'y a ni amour ni rien de productif
 dans le commérage. Il ne peut que faire du mal.

Règle numéro deux~: préservez vos lèvres des paroles trompeuses.
 Autrement dit, ne trompez pas par votre langage. Les paroles trompeuses
 déforment ou omettent la vérité. C'est un discours qui joue sur les mots.
 Ceux qui sont malhonnêtes dans leurs propos perdent vite
 crédibilité et confiance.

\dvbox{
Obéissez à Dieu et Il vous comblera de joie, de bonheur, de paix,
 et de prospérité.
}

Règle numéro trois~: fuyez ce qui est mal.
 Non seulement, nous faut-il éviter le mal, mais il nous faut aussi rechercher
 le bien. Comme Paul l'écrivait aux Éphésiens~: \punct{deux-points}
 \og dépouillez-vous\dots{} de la vieille nature qui se corrompt
 par les convoitises trompeuses \punct{retrait virgule avant deux-points?}
 \dots{} et revêtez la nature nouvelle, créée selon Dieu dans une justice
 et une sainteté que produit la vérité \fg{} \punct{Point en trop, guillemet fermant manquant}
 (\bibleverse{Eph}(4:22,24)).
 Puis Paul leur a donné des exemples précis.
 Il ne fallait pas seulement arrêter de mentir,
 il fallait commencer à dire la vérité. Il ne fallait pas seulement arrêter
 de voler, il fallait trouver du travail afin de pouvoir aider les nécessiteux.
 Il ne fallait pas seulement arrêter de calomnier, il fallait s'édifier
 les uns les autres avec des paroles. Éviter le mal signifie choisir
 de faire ce qui est bien.

Et quel est le résultat si l'on suit ces règles? Rien que des bénédictions!

\dvrule

\dvprayer{
Père, nous Te remercions de tant nous aimer que Tu désires nous bénir.
 Aide-nous à suivre les règles que Tu as établies pour bien vivre.
}{\DlNdJ}



