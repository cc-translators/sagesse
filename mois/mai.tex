\dvmonth{Mai}

%%%%%%%%%%%%
% 1er mai
%%%%%%%%%%%%

\dvday{Le Monde Merveilleux de Dieu}

\dvquote{
Que Tes œuvres sont en grand nombre, ô Éternel!
 Tu les as toutes faites avec sagesse.
 La terre est remplie de ce que Tu possèdes.
}{\bibleverse{Ps}(104:24)}

\dvlettrine{S}{ubmergé} par l'émotion devant la beauté créée par Dieu,
 le psalmiste utilise son langage le plus pittoresque pour essayer
 de décrire l'ouvrage de Dieu.

Il décrit le cycle de la pluie ou comment Dieu fait monter l'eau
 jusqu'aux sommets des montagnes d'où elle redescend dans les cours d'eau
 et comment les sources jaillissent alors dans les vallées pour que les animaux
 sauvages puissent y étancher leur soif, et comment bien que l'eau coule
 jusqu'à la mer, la mer ne déborde jamais.

En tout cela, le psalmiste voit la Sagesse de Dieu.
 Et en contemplant l'agencement spectaculaire et la beauté des choses que Dieu
 a créées, le psalmiste se met à adorer et à louer Dieu.

Il y a aujourd'hui des gens qui regardent la nature et voit le même agencement,
 la même beauté que le psalmiste voyait \ocadr{}mais au lieu d'adorer Dieu pour
 Sa création, ils adorent la création elle-même. Ils adorent la nature.
 Il est vraiment insensé de sentir une rose,
 de toucher la douceur de ses pétales, d'admirer la complexité de son agencement
 et la beauté de sa coloration et de conclure \og la rose est Dieu \fg{}.
 C'est totalement irrationnel. 

\dvbox{
La seule chose rationnelle à faire devant la beauté d'une rose,
 c'est d'en respirer le parfum et de dire~:
 \og voila une création de mon dieu. \fg{}
}

Et puis, comme le psalmiste, nous devrions louer le Dieu responsable
 de toute cette beauté \ocadr{}le Dieu des nuages, et du torrent de montagne,
 et de la mer\dots{} et de la rose. 

\dvrule

\dvprayer{
Père, nous T'adorons et Te louons pour la beauté de Ta création,
 dans laquelle nous nous émerveillons de Ta sagesse et de Ta bonté.
 Nous sommes stupéfaits devant l'œuvre de Tes mains.
 Puisses-Tu créer quelque chose de beau en chacun de nous.
}{\Amen}


%%%%%%%%%%%%
% 2 mai
%%%%%%%%%%%%

\dvday{Satisfaction de l'Âme Affamée }

\dvquote{
Que les hommes célèbrent l'Éternel pour Sa bienveillance et pour Ses merveilles
 en faveur des humains !
 Car Il a rassasié l'âme avide, a comblé de biens l'âme affamée.
}{\bibleverse{Ps}(107:8-9)}

\dvlettrine{V}{ous} est-il déjà arrivé de quitter la table d'un repas
 trop copieux en vous promettant~:
 \og Je ne mangerai plus jamais une autre bouchée de nourriture aussi longtemps
 que je vivrai \fg{} ?
 \`A ce moment-là vous le pensez vraiment sincèrement.
 Vous ne voulez vraiment plus d'une seule autre bouchée de nourriture.
 Mais arrive le soir\dots{} et vous vous retrouvez en train de mettre
 de la de crème chantilly sur une part de tarte qui restait.
 C'est parce que le corps exige d'être constamment nourri.

\dvbox{
La chair ne peut jamais être satisfaite.
}

Nourrissez votre chair, c'est-à-dire votre vieille nature,
 autant que vous le pouvez, elle en voudra toujours davantage.
 En fait si vous cédez à ses demandes dans un domaine,
 au lieu d'être satisfaite, votre vieille nature vous en redemandera
 toujours davantage au point où vous vous retrouvez esclaves.

Tout comme les hommes ont une faim et une soif physique,
 ils ont aussi une faim et soif spirituelles.
 Il y a problème dès que vous essayez de satisfaire un besoin spirituel
 par une expérience charnelle. Cela ne peut pas marcher.

Avez vous faim de Dieu? Soif de paix et de justice ?
 Dans \bibleverse{Jn}(6:35), Jésus a dit~:
 \og Moi, Je suis le pain de vie. Celui qui vient à Moi n'aura jamais faim,
 et celui qui croit en Moi n'aura jamais soif. \fg{}

Jésus est la réponse à toutes vos aspirations. 

\dvrule

\dvprayer{
Père, nous Te remercions que par Ton Fils, notre faim et notre soif
 ont été satisfaites. Pour ceux qui essayent encore de satisfaire
 leur besoin spirituel par des expériences physiques,
 nous demandons que Ton Saint Esprit les ramène à Jésus. 
}{C'est en Son Nom, que nous prions, \Amen}


%%%%%%%%%%%%
% 3 mai
%%%%%%%%%%%%

\dvday{Que Puis-je Donner à Dieu ?}

\dvquote{
Comment rendrai-je à l'Éternel tous ses bienfaits envers moi ?
}{\bibleverse{Ps}(116:12)}

\dvlettrine{S}{achant} que leur Grand-Maman aime les fleurs, nos petites-filles
 vont parfois cueillir pour elle des fleurs dans notre jardin.
 Elles ne font pas trop attention en les cueillant.
 D'habitude, elles ne gardent pas une tige assez longue.
 Le plus souvent, dans leur quête des fleurs parfaites, elles laissent derrière
 elles un sillage de plantes écrasées par leurs petits pieds.

Le jardin n'est pas le leur. C'est le nôtre. Elles vont donc dans notre jardin
 y cueillir nos fleurs pour les donner à Kay.
 Mais vous savez quoi ? Nous aimons ces cadeaux.

Je pense que Dieu reçoit nos cadeaux avec la même attitude.
 Il n'est pas une seule chose matérielle dont Il ait besoin.
 Et, parce que \og la Terre est au Seigneur, et tout ce qu'elle renferme \fg{}
 (\bibleverse{ICor}(10:26)), tout ce que nous pouvons Lui offrir matériellement
 Lui appartient, en fait, déjà. Quelquefois, quand nous traversons Son jardin
 à la recherche de fleurs à Lui offrir, nous y mettons un peu la pagaille.
 Cependant, Il reçoit ce que nous lui apportons avec une attitude
 pleine de grâce et d'amour.

Quand vous considérez tout ce que Dieu a fait pour vous
 et tout ce qu'Il a offert \ocadr le salut, la purification,
 l'espérance du ciel \fcadr{} cela vous donne envie de lui offrir
 quelque chose en retour.

\dvbox{
Que pourrions-nous bien donner à Dieu qui ait une quelconque valeur pour Lui?
}

Il n'a qu'une seule chose qu'Il désire vraiment de votre part.
 Une seule chose. Il veut votre c\oe{}ur. 

\dvrule

\dvprayer{
Père, quand nous pensons à tout ce que Tu nous a donné
 \ocadr le salut, la vie éternelle, notre pain quotidien \fcadr{}
 nous nous demandons ce que nous pourrions bien Te donner.
 Aussi Seigneur, désireux de t'offrir un sacrifice de remerciement,
 nous Te donnons nos c\oe{}urs et nos vies. 
}{}


%%%%%%%%%%%%
% 4 mai
%%%%%%%%%%%%

\dvday{Recherché~: Mort et Vif}

\dvquote{
Détourne mes yeux de la vue des choses vaines, fais-moi vivre dans ta voie!
}{\bibleverse{Ps}(119:37)}

\dvlettrine{I}{l} est intéressant de réaliser comment les vaines choses
 du monde peuvent progressivement nous faire tomber sous leur charme.

Dieu a placé Adam et Eve dans le jardin et leur a dit~:
 \og Vous pouvez manger librement de tous les arbres du jardin.
 Il y a seulement un arbre dont vous ne devez pas manger et c'est l'arbre
 qui est au milieu du jardin. Le jour où vous en mangerez,
 il est sûr que vous mourrez. \fg{}
 Et quel est l'arbre qu'Eve a trouvé le plus attrayant?
 Par lequel s'est-elle sentie attirée? L'arbre interdit.

Le mal est séduisant, attirant et\dots{} totalement trompeur.
 Les gens sont attirés par ce qui est interdit parce qu'ils croient
 à tort que cela va leur amener la satisfaction et le contentement.
 \og Si seulement je pouvais avoir cela,
 alors je serais complètement heureux. \fg{}
 Pourtant, la réalité, c'est que quand nous courrons après ce qui est interdit,
 nous nous retrouvons encore plus vides qu'avant.

\dvbox{
L'excitation des choses temporelles ne dure jamais longtemps.
}

Dieu respecte nos choix. Si nous choisissons de poursuivre des expériences
 charnelles dans l'espoir d'y trouver la satisfaction,
 Dieu va respecter cette décision.
 Mais quand nous sommes prêts à laisser tomber les poursuites charnelles
 pour nous mettre en quête des choses de l'Esprit,
 Dieu va respecter ce choix tout autant.

Sage est l'homme qui, comme David, demande l'aide de Dieu dans la prière.
 \og Protège mes yeux Seigneur.
 Rends-moi insensible aux tentations de la chair.
 Fais-moi vivre dans Tes voies. \fg{}

\dvrule

\dvprayer{
Père, détourne-nous de la poursuite des choses vaines qui n'offrent
 qu'une fausse promesse de satisfaction.
 Puissions-nous être morts à la chair et vivants à Toi,
 pour que nous puissions communier avec Toi et briller comme des étoiles
 pour toujours et à jamais. 
}{\DlNdJnp}


%%%%%%%%%%%%
% 5 mai
%%%%%%%%%%%%

\dvday{L'Esclavage du Mal}

\dvquote{
Affermis mes pas par ta Parole et ne laisse aucun mal me dominer
}{\bibleverse{Ps}(119:133) \NKJV}

\dvlettrine{U}{ne mouche}
 \ocadr incapable de voir le piège tendu juste devant elle \fcadr{}
 se fait prendre dans la toile de l'araignée.
 Au début, elle livre un combat plutôt prometteur.
 Avec l'énergie du désespoir, elle s'agite, se débat,
 essayant de s'échapper.
 Mais plus elle s'agite, plus elle s'englue dans la toile
 pour finalement se retrouver complètement prisonnière. 

Tout comme la mouche, nous nous faisons parfois attraper
 par le péché.
 Que ce soit l'alcool, la drogue, le jeu, la pornographie,
 la débauche, l'adultère, la capacité addictive du péché
 agit comme une toile d'araignée gluante, qui nous enferme
 de plus en plus jusqu'à ce que nous nous retrouvions
 esclaves de la corruption. 

\dvbox{
Dès que vous vous adonnez au péché,
 vous vous retrouvez sous la puissance de Satan. 
}

\og Tout m'est permis, mais tout n'est pas utile,
 tout m'est permis, mais je ne me laisserai pas asservir
 par quoi que ce soit \fg{} (\bibleverse{ICor}(6:12)).
 Jésus est venu dans le monde pour détruire
 la puissance dominatrice de Satan.
 Il est venu pour vous libérer,
 et \og pour proclamer aux captifs la délivrance \fg{} (\bibleverse{Lc}(4:18)).

Sage est l'homme qui use de sa liberté pour éviter le piège du mal.
 Mais si vous avez été capturés, sachez que vous pouvez trouver la liberté
 par l'intermédiaire de Christ.
 Vous n'avez pas à être asservi au péché qui menace de vous détruire.
 Christ peut vous libérer. Confessez votre péché et laissez Le briser
 l'emprise de Satan sur vous. 

\dvrule

\dvprayer{
Seigneur, nous Te remercions d'avoir l'autorité sur les ténèbres.
 Nous Te prions de nous fortifier afin que nous ne nous faisions pas
 prendre dans une toile de péché,
 mais que Tu affermisses nos pas par Ta Parole. 
}{\DlNdJ}


%%%%%%%%%%%%
% 6 mai
%%%%%%%%%%%%

\dvday{Plaire à Dieu}

\dvquote{
Car l'Éternel prend plaisir à son peuple,
 Il donne aux humbles le salut pour parure.
}{\bibleverse{Ps}(149:4)}

\dvlettrine{Q}{uand} nous voyons le plaisir éprouvé par quelqu'un
 qui restaure un objet cassé et le remet à neuf,
 nous nous approchons probablement du plaisir éprouvé par le c\oe{}ur de Dieu.
 Il aime prendre des vies qui ont été ruinées par le péché
 et les restaurer à l'image de Son Fils. 

Quand Dieu vous regarde, Il voit l'\oe{}uvre de Sa grâce en vous.
 Il voit que des cendres d'une vie ratée est sorti quelque chose
 de valeur et de beau.
 Dieu est aussi heureux quand Il vous voit choisir de vivre par l'Esprit,
 et que vous Lui remettez votre vie pour qu'Il puisse, par Son Esprit,
 la rendre conforme à l'image de Son Fils. 

\dvbox{
Quand Dieu vous regarde, Il éprouve du plaisir. 
}

D'autres choses qu'Il peut voir, ne Lui plaisent cependant pas.
 La méchanceté ne lui plaît pas (\bibleverse{Ps}(5:4)).
 Ceux qui reculent dans leur foi ne Lui plaisent pas (\bibleverse{Heb}(10:38)).
 Et ceux qui vivent selon la chair ne Lui plaisent pas non plus
 (\bibleverse{Rom}(8:8)).
 Dieu n'est pas heureux quand vous choisissez de vivre selon la chair
 car Il sait que cela pourrait vous détruire. 

Demandez-vous honnêtement~:
 Mon choix est-il de me plaire ou de plaire à Dieu?
 Qu'en est-il des activités auxquelles vous avez décidé de vous adonner
 \ocadr sont-elles plaisantes à Dieu ou à vous seulement?
 Rappelez-vous les paroles de Jésus, qui disait en parlant du Père~:
 \og Je fais toujours ce qui lui est agréable \fg{} (\bibleverse{Jn}(8:29)).
 Quand vous choisirez de faire de même,
 votre vie sera plus gratifiante et plus riche. 

Vivez pour plaire à Dieu. 

\dvrule

\dvprayer{
Père, aide nous à vivre des vies qui T'honorent.
 Puissions-nous nous consacrer à Te plaire. 
}{\NpDlNdJ}


%%%%%%%%%%%%
% 7 mai
%%%%%%%%%%%%

\dvday{Découvrir la Volonté de Dieu}

\dvquote{
Confie-toi en l'Éternel de tout ton cœur,
 et ne t'appuie pas sur ton intelligence;
 reconnais-Le dans toutes tes voies,
 et c'est Lui qui aplanira tes sentiers.
}{\bibleverse{Prov}(3:5-6)}

\dvlettrine{N}{ous existons} pour glorifier Dieu et pour faire Sa volonté.
 Mais comment savoir ce qu'Il veut que nous fassions?
 Dans ce verset, Salomon nous indique trois étapes pour découvrir
 la volonté de Dieu pour nos vies. 

Premièrement~: \og Confie-toi en l'Éternel de tout ton c\oe{}ur. \fg{}
 Vous n'allez pas toujours comprendre ce que Dieu fait, ni pourquoi.
 Il vous faut d'abord mettre votre pleine et entière confiance en Lui. 

Deuxièmement~: \og Ne t'appuie pas sur ton intelligence. \fg{}
 Je suis toujours stupéfait de voir toutes ces séances de stratégie
 convoquées par les conseils d'église pour essayer d'organiser
 des programmes d'évangélisation de la façon la plus logique
 et la plus attrayante possible.
 La volonté de Dieu n'est pas découverte dans des séances
 de planification mais dans des réunions de prière. 

Troisièmement~: \og Reconnais-Le dans toutes tes voies. \fg{}
 Ne prenez jamais une décision sans avoir d'abord consulté le Seigneur. 

\dvbox{
Dieu dirige de façon très simple et naturelle. 
}

Bien que notre préférence serait d'entendre d'abord la volonté de Dieu
 \ocadr pour que nous puissions décider si elle nous plaît ou pas \fg{}
 ce n'est pas la façon dont Dieu dirige.
 Vous n'allez généralement pas entendre un ange qui chante
 et qui vous fait signe de le suivre dans une certaine direction. 

Vivez simplement en Lui étant consacré; faites-Lui confiance
 et reconnaissez-Le quand vous ressentez une impulsion de le suivre
 dans une direction ou une autre.
 Quand vous ferez cela, alors \og Il aplanira vos sentiers \fg{}
 ou comme le traduisent d'autres versions \og Il guidera vos pas. \fg{}

\dvrule

\dvprayer{
Seigneur, nous offrons nos c\oe{}urs et nos vies
 comme des instruments par lesquels Tu peux \oe{}uvrer
 pour accomplir Ta volonté sur la terre.
}{\DlNdJnp}



