\dvmonth{Mai}

\dvday{Le Monde Merveilleux de Dieu}

\dvquote{
Que Tes œuvres sont en grand nombre, ô Éternel!
 Tu les as toutes faites avec sagesse.
 La terre est remplie de ce que Tu possèdes.
}{\bibleverse{Ps}(104:24)}

\dvlettrine{S}{ubmergé} par l'émotion devant la beauté créée par Dieu,
 le psalmiste utilise son langage le plus pittoresque pour essayer
 de décrire l'ouvrage de Dieu.

Il décrit le cycle de la pluie ou comment Dieu fait monter l'eau
 jusqu'aux sommets des montagnes d'où elle redescend dans les cours d'eau
 et comment les sources jaillissent alors dans les vallées pour que les animaux
 sauvages puissent y étancher leur soif, et comment bien que l'eau coule
 jusqu'à la mer, la mer ne déborde jamais.

En tout cela, le psalmiste voit la Sagesse de Dieu.
 Et en contemplant l'agencement spectaculaire et la beauté des choses que Dieu
 a créées, le psalmiste se met à adorer et à louer Dieu.

Il y a aujourd'hui des gens qui regardent la nature et voit le même agencement,
 la même beauté que le psalmiste voyait \ocadr{}mais au lieu d'adorer Dieu pour
 Sa création, ils adorent la création elle-même. Ils adorent la nature.
 Il est vraiment insensé de sentir une rose,
 de toucher la douceur de ses pétales, d'admirer la complexité de son agencement
 et la beauté de sa coloration et de conclure \og la rose est Dieu \fg{}.
 C'est totalement irrationnel. 

\dvbox{
La seule chose rationnelle à faire devant la beauté d'une rose,
 c'est d'en respirer le parfum et de dire~:
 \og voila une création de mon dieu. \fg{}
}

Et puis, comme le psalmiste, nous devrions louer le Dieu responsable
 de toute cette beauté \ocadr{}le Dieu des nuages, et du torrent de montagne,
 et de la mer\dots{} et de la rose. 

\dvrule

\dvprayer{
Père, nous T'adorons et Te louons pour la beauté de Ta création,
 dans laquelle nous nous émerveillons de Ta sagesse et de Ta bonté.
 Nous sommes stupéfaits devant l'œuvre de Tes mains.
 Puisses-Tu créer quelque chose de beau en chacun de nous.
}{\Amen}

