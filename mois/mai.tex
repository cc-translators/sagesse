\dvmonth{Mai}

%%%%%%%%%%%%
% 1er mai
%%%%%%%%%%%%

\dvday{Le Monde Merveilleux de Dieu}

\dvquote{
Que Tes œuvres sont en grand nombre, ô Éternel!
 Tu les as toutes faites avec sagesse.
 La terre est remplie de ce que Tu possèdes.
}{\bibleverse{Ps}(104:24)}

\dvlettrine{S}{ubmergé} par l'émotion devant la beauté créée par Dieu,
 le psalmiste utilise son langage le plus pittoresque pour essayer
 de décrire l'ouvrage de Dieu.

Il décrit le cycle de la pluie ou comment Dieu fait monter l'eau
 jusqu'aux sommets des montagnes d'où elle redescend dans les cours d'eau
 et comment les sources jaillissent alors dans les vallées pour que les animaux
 sauvages puissent y étancher leur soif, et comment bien que l'eau coule
 jusqu'à la mer, la mer ne déborde jamais.

En tout cela, le psalmiste voit la Sagesse de Dieu.
 Et en contemplant l'agencement spectaculaire et la beauté des choses que Dieu
 a créées, le psalmiste se met à adorer et à louer Dieu.

Il y a aujourd'hui des gens qui regardent la nature et voit le même agencement,
 la même beauté que le psalmiste voyait \ocadr{}mais au lieu d'adorer Dieu pour
 Sa création, ils adorent la création elle-même. Ils adorent la nature.
 Il est vraiment insensé de sentir une rose,
 de toucher la douceur de ses pétales, d'admirer la complexité de son agencement
 et la beauté de sa coloration et de conclure \og la rose est Dieu \fg{}.
 C'est totalement irrationnel. 

\dvbox{
La seule chose rationnelle à faire devant la beauté d'une rose,
 c'est d'en respirer le parfum et de dire~:
 \og voila une création de mon dieu. \fg{}
}

Et puis, comme le psalmiste, nous devrions louer le Dieu responsable
 de toute cette beauté \ocadr{}le Dieu des nuages, et du torrent de montagne,
 et de la mer\dots{} et de la rose. 

\dvrule

\dvprayer{
Père, nous T'adorons et Te louons pour la beauté de Ta création,
 dans laquelle nous nous émerveillons de Ta sagesse et de Ta bonté.
 Nous sommes stupéfaits devant l'œuvre de Tes mains.
 Puisses-Tu créer quelque chose de beau en chacun de nous.
}{\Amen}


%%%%%%%%%%%%
% 2 mai
%%%%%%%%%%%%

\dvday{Satisfaction de l'Âme Affamée }

\dvquote{
Que les hommes célèbrent l'Éternel pour Sa bienveillance et pour Ses merveilles
 en faveur des humains !
 Car Il a rassasié l'âme avide, a comblé de biens l'âme affamée.
}{\bibleverse{Ps}(107:8-9)}

\dvlettrine{V}{ous} est-il déjà arrivé de quitter la table d'un repas
 trop copieux en vous promettant~:
 \og Je ne mangerai plus jamais une autre bouchée de nourriture aussi longtemps
 que je vivrai \fg{} ?
 \`A ce moment-là vous le pensez vraiment sincèrement.
 Vous ne voulez vraiment plus d'une seule autre bouchée de nourriture.
 Mais arrive le soir\dots{} et vous vous retrouvez en train de mettre
 de la de crème chantilly sur une part de tarte qui restait.
 C'est parce que le corps exige d'être constamment nourri.

\dvbox{
La chair ne peut jamais être satisfaite.
}

Nourrissez votre chair, c'est-à-dire votre vieille nature,
 autant que vous le pouvez, elle en voudra toujours davantage.
 En fait si vous cédez à ses demandes dans un domaine,
 au lieu d'être satisfaite, votre vieille nature vous en redemandera
 toujours davantage au point où vous vous retrouvez esclaves.

Tout comme les hommes ont une faim et une soif physique,
 ils ont aussi une faim et soif spirituelles.
 Il y a problème dès que vous essayez de satisfaire un besoin spirituel
 par une expérience charnelle. Cela ne peut pas marcher.

Avez vous faim de Dieu? Soif de paix et de justice ?
 Dans \bibleverse{Jn}(6:35), Jésus a dit~:
 \og Moi, Je suis le pain de vie. Celui qui vient à Moi n'aura jamais faim,
 et celui qui croit en Moi n'aura jamais soif. \fg{}

Jésus est la réponse à toutes vos aspirations. 

\dvrule

\dvprayer{
Père, nous Te remercions que par Ton Fils, notre faim et notre soif
 ont été satisfaites. Pour ceux qui essayent encore de satisfaire
 leur besoin spirituel par des expériences physiques,
 nous demandons que Ton Saint Esprit
 les ramène à Jésus. 
}{C'est en Son Nom, que nous prions, \Amen}

\suggest{ou Saint-Esprit?}


%%%%%%%%%%%%
% 3 mai
%%%%%%%%%%%%

\dvday{Que Puis-je Donner à Dieu ?}

\dvquote{
Comment rendrai-je à l'Éternel tous ses bienfaits envers moi ?
}{\bibleverse{Ps}(116:12)}

\dvlettrine{S}{achant} que leur Grand-Maman aime les fleurs, nos petites-filles
 vont parfois cueillir pour elle des fleurs dans notre jardin.
 Elles ne font pas trop attention en les cueillant.
 D'habitude, elles ne gardent pas une tige assez longue.
 Le plus souvent, dans leur quête des fleurs parfaites, elles laissent derrière
 elles un sillage de plantes écrasées par leurs petits pieds.

Le jardin n'est pas le leur. C'est le nôtre. Elles vont donc dans notre jardin
 y cueillir nos fleurs pour les donner à Kay.
 Mais vous savez quoi ? Nous aimons ces cadeaux.

Je pense que Dieu reçoit nos cadeaux avec la même attitude.
 Il n'est pas une seule chose matérielle dont Il ait besoin.
 Et, parce que \og la Terre est au Seigneur, et tout ce qu'elle renferme \fg{}
 (\bibleverse{ICor}(10:26)), tout ce que nous pouvons Lui offrir matériellement
 Lui appartient, en fait, déjà. Quelquefois, quand nous traversons Son jardin
 à la recherche de fleurs à Lui offrir, nous y mettons un peu la pagaille.
 Cependant, Il reçoit ce que nous lui apportons avec une attitude
 pleine de grâce et d'amour.

Quand vous considérez tout ce que Dieu a fait pour vous
 et tout ce qu'Il a offert \ocadr le salut, la purification,
 l'espérance du ciel \fcadr{} cela vous donne envie de lui offrir
 quelque chose en retour.

\dvbox{
Que pourrions-nous bien donner à Dieu qui ait une quelconque valeur pour Lui?
}

Il n'a qu'une seule chose qu'Il désire vraiment de votre part.
 Une seule chose. Il veut votre c\oe{}ur. 

\dvrule

\dvprayer{
Père, quand nous pensons à tout ce que Tu nous a donné
 \ocadr le salut, la vie éternelle, notre pain quotidien \fcadr{}
 nous nous demandons ce que nous pourrions bien Te donner.
 Aussi Seigneur, désireux de t'offrir un sacrifice de remerciement,
 nous Te donnons nos c\oe{}urs et nos vies. 
}

\missing{Fin de prière?}


%%%%%%%%%%%%
% 4 mai
%%%%%%%%%%%%

\dvday{Recherché~: Mort et Vif}

\dvquote{
Détourne mes yeux de la vue des choses vaines, fais-moi vivre dans ta voie!
}{\bibleverse{Ps}(119:37)}

\dvlettrine{I}{l} est intéressant de réaliser comment les vaines choses
 du monde peuvent progressivement nous faire tomber sous leur charme.

Dieu a placé Adam et Eve dans le jardin et leur a dit~:
 \og Vous pouvez manger librement de tous les arbres du jardin.
 Il y a seulement un arbre dont vous ne devez pas manger et c'est l'arbre
 qui est au milieu du jardin. Le jour où vous en mangerez,
 il est sûr que vous mourrez. \fg{}
 Et quel est l'arbre qu'Eve a trouvé le plus attrayant?
 Par lequel s'est-elle sentie attirée? L'arbre interdit.

Le mal est séduisant, attirant et\dots{} totalement trompeur.
 Les gens sont attirés par ce qui est interdit parce qu'ils croient
 à tort que cela va leur amener la satisfaction et le contentement.
 \og Si seulement je pouvais avoir cela,
 alors je serais complètement heureux. \fg{}
 Pourtant, la réalité, c'est que quand nous courrons après ce qui est interdit,
 nous nous retrouvons encore plus vides qu'avant.

\dvbox{
L'excitation des choses temporelles ne dure jamais longtemps.
}

Dieu respecte nos choix. Si nous choisissons de poursuivre des expériences
 charnelles dans l'espoir d'y trouver la satisfaction,
 Dieu va respecter cette décision.
 Mais quand nous sommes prêts à laisser tomber les poursuites charnelles
 pour nous mettre en quête des choses de l'Esprit,
 Dieu va respecter ce choix tout autant.

Sage est l'homme qui, comme David, demande l'aide de Dieu dans la prière.
 \og Protège mes yeux Seigneur.
 Rends-moi insensible aux tentations de la chair.
 Fais-moi vivre dans Tes voies. \fg{}

\dvrule

\dvprayer{
Père, détourne-nous de la poursuite des choses vaines qui n'offrent
 qu'une fausse promesse de satisfaction.
 Puissions-nous être morts à la chair et vivants à Toi,
 pour que nous puissions communier avec Toi et briller comme des étoiles
 pour toujours et à jamais. 
}{\DlNdJnp}


%%%%%%%%%%%%
% 5 mai
%%%%%%%%%%%%

\dvday{L'Esclavage du Mal}

\dvquote{
Affermis mes pas par ta Parole et ne laisse aucun mal me dominer
}{\bibleverse{Ps}(119:133) \NKJV}

\punct{Faut-il un point final à la citation?}
\fixme{S'agit-il vraiment de la NKJV?}

\dvlettrine{U}{ne mouche}
 \ocadr incapable de voir le piège tendu juste devant elle \fcadr{}
 se fait prendre dans la toile de l'araignée.
 Au début, elle livre un combat plutôt prometteur.
 Avec l'énergie du désespoir, elle s'agite, se débat,
 essayant de s'échapper.
 Mais plus elle s'agite, plus elle s'englue dans la toile
 pour finalement se retrouver complètement prisonnière. 

Tout comme la mouche, nous nous faisons parfois attraper
 par le péché.
 Que ce soit l'alcool, la drogue, le jeu, la pornographie,
 la débauche, l'adultère, la capacité addictive du péché
 agit comme une toile d'araignée gluante, qui nous enferme
 de plus en plus jusqu'à ce que nous nous retrouvions
 esclaves de la corruption. 

\dvbox{
Dès que vous vous adonnez au péché,
 vous vous retrouvez sous la puissance de Satan. 
}

\og Tout m'est permis, mais tout n'est pas utile,
 tout m'est permis, mais je ne me laisserai pas asservir
 par quoi que ce soit \fg{} (\bibleverse{ICor}(6:12)).
 Jésus est venu dans le monde pour détruire
 la puissance dominatrice de Satan.
 Il est venu pour vous libérer,
 et \og pour proclamer aux captifs la délivrance \fg{} (\bibleverse{Lc}(4:18)).

Sage est l'homme qui use de sa liberté pour éviter le piège du mal.
 Mais si vous avez été capturés, sachez que vous pouvez trouver la liberté
 par l'intermédiaire de Christ.
 Vous n'avez pas à être asservi au péché qui menace de vous détruire.
 Christ peut vous libérer. Confessez votre péché et laissez Le briser
 l'emprise de Satan sur vous. 

\dvrule

\dvprayer{
Seigneur, nous Te remercions d'avoir l'autorité sur les ténèbres.
 Nous Te prions de nous fortifier afin que nous ne nous faisions pas
 prendre dans une toile de péché,
 mais que Tu affermisses nos pas par Ta Parole. 
}{\DlNdJ}


%%%%%%%%%%%%
% 6 mai
%%%%%%%%%%%%

\dvday{Plaire à Dieu}

\dvquote{
Car l'Éternel prend plaisir à son peuple,
 Il donne aux humbles le salut pour parure.
}{\bibleverse{Ps}(149:4)}

\dvlettrine{Q}{uand} nous voyons le plaisir éprouvé par quelqu'un
 qui restaure un objet cassé et le remet à neuf,
 nous nous approchons probablement du plaisir éprouvé par le c\oe{}ur de Dieu.
 Il aime prendre des vies qui ont été ruinées par le péché
 et les restaurer à l'image de Son Fils. 

Quand Dieu vous regarde, Il voit l'\oe{}uvre de Sa grâce en vous.
 Il voit que des cendres d'une vie ratée est sorti quelque chose
 de valeur et de beau.
 Dieu est aussi heureux quand Il vous voit choisir de vivre par l'Esprit,
 et que vous Lui remettez votre vie pour qu'Il puisse, par Son Esprit,
 la rendre conforme à l'image de Son Fils. 

\dvbox{
Quand Dieu vous regarde, Il éprouve du plaisir. 
}

D'autres choses qu'Il peut voir, ne Lui plaisent cependant pas.
 La méchanceté ne lui plaît pas (\bibleverse{Ps}(5:4)).
 Ceux qui reculent dans leur foi ne Lui plaisent pas (\bibleverse{Heb}(10:38)).
 Et ceux qui vivent selon la chair ne Lui plaisent pas non plus
 (\bibleverse{Rom}(8:8)).
 Dieu n'est pas heureux quand vous choisissez de vivre selon la chair
 car Il sait que cela pourrait vous détruire. 

Demandez-vous honnêtement~:
 Mon choix est-il de me plaire ou de plaire à Dieu?
 Qu'en est-il des activités auxquelles vous avez décidé de vous adonner
 \ocadr sont-elles plaisantes à Dieu ou à vous seulement?
 Rappelez-vous les paroles de Jésus, qui disait en parlant du Père~:
 \og Je fais toujours ce qui lui est agréable \fg{} (\bibleverse{Jn}(8:29)).
 Quand vous choisirez de faire de même,
 votre vie sera plus gratifiante et plus riche. 

Vivez pour plaire à Dieu. 

\dvrule

\dvprayer{
Père, aide nous à vivre des vies qui T'honorent.
 Puissions-nous nous consacrer à Te plaire. 
}{\NpDlNdJ}


%%%%%%%%%%%%
% 7 mai
%%%%%%%%%%%%

\dvday{Découvrir la Volonté de Dieu}

\dvquote{
Confie-toi en l'Éternel de tout ton cœur,
 et ne t'appuie pas sur ton intelligence;
 reconnais-Le dans toutes tes voies,
 et c'est Lui qui aplanira tes sentiers.
}{\bibleverse{Prov}(3:5-6)}

\dvlettrine{N}{ous existons} pour glorifier Dieu et pour faire Sa volonté.
 Mais comment savoir ce qu'Il veut que nous fassions?
 Dans ce verset, Salomon nous indique trois étapes pour découvrir
 la volonté de Dieu pour nos vies. 

Premièrement~: \og Confie-toi en l'Éternel de tout ton c\oe{}ur. \fg{}
 Vous n'allez pas toujours comprendre ce que Dieu fait, ni pourquoi.
 Il vous faut d'abord mettre votre pleine et entière confiance en Lui. 

Deuxièmement~: \og Ne t'appuie pas sur ton intelligence. \fg{}
 Je suis toujours stupéfait de voir toutes ces séances de stratégie
 convoquées par les conseils d'église pour essayer d'organiser
 des programmes d'évangélisation de la façon la plus logique
 et la plus attrayante possible.
 La volonté de Dieu n'est pas découverte dans des séances
 de planification mais dans des réunions de prière. 

Troisièmement~: \og Reconnais-Le dans toutes tes voies. \fg{}
 Ne prenez jamais une décision sans avoir d'abord consulté le Seigneur. 

\dvbox{
Dieu dirige de façon très simple et naturelle. 
}

Bien que notre préférence serait d'entendre d'abord la volonté de Dieu
 \ocadr pour que nous puissions décider si elle nous plaît ou pas \fg{}
 ce n'est pas la façon dont Dieu dirige.
 Vous n'allez généralement pas entendre un ange qui chante
 et qui vous fait signe de le suivre dans une certaine direction. 

Vivez simplement en Lui étant consacré; faites-Lui confiance
 et reconnaissez-Le quand vous ressentez une impulsion de le suivre
 dans une direction ou une autre.
 Quand vous ferez cela, alors \og Il aplanira vos sentiers \fg{}
 ou comme le traduisent d'autres versions \og Il guidera vos pas. \fg{}

\dvrule

\dvprayer{
Seigneur, nous offrons nos c\oe{}urs et nos vies
 comme des instruments par lesquels Tu peux \oe{}uvrer
 pour accomplir Ta volonté sur la terre.
}{\DlNdJnp}


%%%%%%%%%%%%
% 8 mai
%%%%%%%%%%%%

\dvday{Le Sentier de la Justice}

\dvquote{
La vie est dans le sentier de la justice,
 et cette voie est un sentier
 qui ne mène pas à la mort.
}{\bibleverse{Prov}(12:28)}

Afin que nous sachions ce que c'est qu'être juste,
 Dieu a donné à Moïse les Dix Commandements.
 Les quatre premiers commandements déclarent ce qu'il fallait faire
 pour être juste avec Dieu.
 Les six commandements suivants énumèrent ce qui est nécessaire
 pour être juste avec les hommes. 

Si l'histoire en était restée là, nous serions dans un sacré pétrin,
 parce qu'aucun de nous ne pourrait jamais être juste.
 Mais parce qu'Il savait que l'homme n'est pas capable d'arriver
 à la hauteur de ces exigences de justice, Dieu a établi un autre critère.
 Si nous mettons simplement notre confiance en Dieu,
 Il va compter cette foi comme justice. 

En quoi placez-vous votre confiance ?
 Vous mettez votre confiance en ce que le Dieu éternel,
 le Créateur de l'univers, a tant aimé le monde (et ça vous inclut)
 qu'Il a envoyé Son Fils unique mourir pour vos péchés
 et effacer votre culpabilité. 

\dvbox{
Jésus a promis que ceux qui croiraient en Lui n'allaient jamais mourir. 
}

Jésus ne promet pas que nous allons vivre pour toujours
 dans nos corps présents, mais que nous allons vivre pour toujours
 dans la présence de Dieu
 \ocadr pour ne jamais connaître la séparation d'avec Lui. 

Écrivant aux Éphésiens, Paul a déclaré~:
 \og Et il vous a rendus à la vie, vous qui étiez morts par vos fautes
 et par vos péchés \fg{} (\bibleverse{Eph}(2:1)).
 En plaçant votre confiance en Jésus-Christ, vous pouvez être rendus
 spirituellement vivants aujourd'hui.
 Vous pouvez prendre conscience de Dieu, de Son amour, et de Ses plan
 et dessein pour votre vie. 

\dvrule

\dvprayer{
Père, nous Te remercions de la façon par laquelle Tu nous permets
 d'être considérés comme justes.
 Aide-nous, Seigneur à vivre dans cette justice,
 dans le chemin de laquelle il n'y a pas de mort. 
}{\DlNdJ}


%%%%%%%%%%%%
% 9 mai
%%%%%%%%%%%%

\dvday{Le Mauvais Chemin}

\dvquote{
\og Telle voie paraît droite devant un homme, mais à la fin,
 c'est la voie de la mort. \fg{}
}{\bibleverse{Prov}(14:12)}

\punct{Faut-il des guillemets autour de cette citation
 (non rapportée dans le contexte)?}

\dvlettrine{A}{u bout du compte,} il n'y a que deux destinations dans la vie.
 Vous êtes, soit sur le chemin qui vous mène au ciel,
 soit sur le chemin qui vous mène à l'enfer.

Parlant de ces deux chemins, Jésus a dit~: \punct{deux-points}
 \og Entrez par la porte étroite car large est la porte et spacieux le chemin
 qui mènent à la perdition,
 et il y en a beaucoup qui entrent par là \fg{}
 \punct{Point avant la référence}
 (\bibleverse{Matt}(7:13)).

\dvbox{
Les sages marchent avec un but en tête.
}

Les sages regardent vers l'avenir et déterminent des buts pour leurs vies
 \ocadr des choses qu'ils espèrent avoir accomplies quand ils auront atteint
 la fin du chemin de leur vie terrestre.
 Ils pensent au moment futur où ils se retrouveront devant Jésus.
 Ils vivent pour mériter de s'entendre dire~: 
 \og Bien, bon et fidèle serviteur! \fg{} (\bibleverse{Matt}(25:21)).

Les insensés, par contre, errent sans but précis.
 Croyant au mensonge de Satan selon lequel
 \og toutes les routes mènent à Dieu \fg{},
 ils ignorent les avertissements rencontrés en chemin.
 Toutes les routes ne mènent pas au vrai Dieu vivant.
 Certaines routes mènent à la mort
 \ocadr et à la séparation d'avec l'unique vrai Dieu.

Considérez votre vie. Examinez le chemin suivi par vos pieds.
 Avez-vous choisi le chemin qui mène à Jésus?
 Et à quoi faites-vous confiance pour déterminer votre réponse?
 \typo{Majuscule en début de phrase}
 À là parole de l'homme, ou à la Parole de Dieu?

\dvrule

\dvprayer{
Père, nous demandons que Ton Saint Esprit
 nous guide sur le chemin de la vie qui mène à Ta demeure.
 Nous sommes si reconnaissants, Seigneur, que Ton Fils ait défriché
 un chemin pour nous \ocadr une route qui mène directement à Toi.
 Puissions-nous marcher dans Son chemin, dans Sa vérité et dans Sa lumière.
}{\DlNdJnp}

\suggest{ou Saint-Esprit?}


%%%%%%%%%%%%
% 10 mai
%%%%%%%%%%%%

\dvday{Notre Forte Tour}

\dvquote{
Le nom de l'Éternel est une tour forte;
 Le juste y court et s'y trouve hors d'atteinte.
}{\bibleverse{Prov}(18:10)}

\dvlettrine{D}{ans l'antiquité,} les hommes bâtissaient de hautes tours
 le long des murailles qui entouraient leurs cités.
 Depuis ces tours, une personne pouvait voir très loin et pouvait facilement
 repérer des ennemis qui approchaient.
 Quand les combats se déclenchaient, les tours offraient
 l'avantage de la hauteur. Les lances ne pouvaient pas vous atteindre,
 mais vos lances pouvaient facilement être projetées plus bas
 vers vos attaquants.
 Pour ces raisons, les gens courraient se réfugier dans ces tours
 quand le danger approchait. 

Nous aussi, avons une tour forte \ocadr un lieu de refuge vers lequel
 nous pouvons courir quand les batailles de la vie menacent. 

\dvbox{
Notre refuge? C'est le nom du Seigneur. 
}

D'innombrables fois, quand j'ai eu le plus besoin de Lui,
 j'ai couru vers le nom de Jésus et j'ai trouvé la sécurité.
 Dans les moments de détresse, les moments de peur,
 les moments d'incertitude, les moments où je me sentais désespéré,
 désarmé et faible, j'ai couru vers cet endroit où je sais
 que je vais trouver le réconfort et la sécurité. J'ai couru à Jésus. 

Il est Jéhovah-Jireh, le Dieu qui pourvoit; Il \fixme{Majuscule à \og il \fg}
 a pourvu à mon besoin de salut. Il est Jéhovah-Nissi, ma bannière.
 Sa bannière au-dessus de moi, c'est l'amour. Il est Jéhovah-Shalom, \fixme{Espace en trop avant le tiret}
 ma paix. Il est Jéhovah-Tsidkenu, ma justice. Il est Jéhovah-Shammah \fixme{Espace manquant avant le tiret long}
 \ocadr où que je sois, Il est là aussi. 

Jésus est notre Sauveur, notre Rocher, notre Refuge. Il est notre Forte Tour. 

Vous sentez-vous débordés aujourd'hui? Inquiets \fixme{Majuscule à \og inquiets \fg}
 ou incertains? Courez-à Jésus. 

\dvrule

\dvprayer{
Père, nous Te remercions pour l'aide, la force, la confiance,
 que nous recevons par et dans le nom de Jésus,
 le nom au-dessus de tous les autres noms. 
}{\EsN}

\fixme{Point en trop}


%%%%%%%%%%%%
% 11 mai
%%%%%%%%%%%%

\dvday{L\ap{}Ami Parfait}

\dvquote{
L'homme qui a des amis doit se montrer aimable; et il y a un ami qui s'attache
 plus qu'un frère.
}{\bibleverse{Prov}(18:24) \KJF}

\dvlettrine{N}{ous ne sommes pas} toujours faciles à aimer.
 C'est parce que nous sommes sujets à des sautes d'humeur.
 Je peux être facile à aimer à un moment donné,
 et plutôt l'opposé le moment d'après.

\dvbox{
Un bon ami vous aime quelle que soit votre humeur.
}

Il peut être difficile de trouver un tel ami. Mais en fait,
 j'ai vraiment un Ami qui m'aime indépendamment de mon humeur.
 Il connaît tous les recoins de mon caractère
 \ocadr tous les bons et tous les mauvais.
 Et pourtant, Il continue de m'aimer.

Il est Celui qui a initié notre amitié. Et Il est Celui qui l'entretient.
 Il reste fidèle dans les bons et les mauvais moments et quand les autres
 m'ont abandonné, Il reste à mes côtés. Il a promis que jamais
 Il ne me délaissera, que jamais Il ne m'abandonnera (\bibleverse{Heb}(13:5)).

J'ai laissé tomber mes amis d'innombrables fois, mais Lui ne m'a jamais laissé
 tomber. Il pardonne chacune de mes offenses. Il me donne des conseils
 quand j'en ai besoin, et je n'ai encore jamais trouvé qu'Il se soit trompé.
 Ses pensées sont fixées sur moi. \fixme{Manque majuscule}
 Il ne semble jamais rechercher \grammar{er au lieu de é}
 Son propre bien-être, mais Il est toujours préoccupé du mien.
 En fait, Il m'aime tant, qu'Il a offert Sa vie pour moi.

Il l'a fait, pour vous aussi.

\dvrule

\dvprayer{
Père, nous Te remercions pour Ton Fils unique qui a offert Sa vie
 pour que nous puissions être Tes amis.
 Seigneur, nous prions que comme Abraham a été appelé Ton ami,
 nous puissions être aussi appelés les amis de Dieu.
 Nous voulons Te connaître intimement.
}{\DlNdJnp}


%%%%%%%%%%%%
% 12 mai
%%%%%%%%%%%%

\dvday{La Peur du Seigneur}

\suggest{La crainte?}

\dvquote{
Que ton cœur n'envie pas les pécheurs,
 mais que tout le jour il craigne l'Éternel.
}{\bibleverse{Prov}(23:17)}

\dvlettrine{S}{atan dépeint} toujours la vie du péché
 comme excitante et agréable.
 Si vous prenez cette image pour argent comptant,
 il est facile d'envier ceux qui sont engagés
 dans les plaisirs du péché. Mais il faut regarder plus loin. 

\dvbox{
Vous devez comparer le résultat final d'une vie d'iniquité
 au résultat final d'une vie de justice. 
}

La vie peut finir par la mort et la séparation éternelle d'avec Dieu,
 ou elle peut être récompensée par la vie éternelle au ciel avec Dieu.
 Ce proverbe nous dit qu'au lieu d'envier les pécheurs,
 nous devons marcher dans la peur \suggest{ou la crainte?}
 du Seigneur tout au long de la journée.
 Certains ont une fausse idée de la peur du Seigneur.
 Ils pensent qu'il s'agit de la même peur que celle qui saisit votre cœur
 quand vous remarquer soudain dans votre rétroviseur les éclats de lumière
 d'un gyrophare. Mais il ne s'agit pas de ce genre de peur. 

Le bon genre de peur est ce que vous ressentiriez si vous deviez donner
 un discours devant le Président de la République.
 Vous vous demanderiez~: \punct{deux-points, majuscule}
 \og Mon costume est-il bien repassé? \punct{majuscule}
 Mon discours est-il approprié? \fg{}

Si nous avons ce genre de respect intimidé en paraissant devant le Président,
 à combien plus forte raison devrions nous être remplis de cette crainte
 révérencieuse en présence du Dieu qui a créé l'univers ?
 Sage est l'homme ou la femme qui a la crainte du Seigneur
 \ocadr et utilise cette peur pour vivre une vie qui Lui plaise. 

Il l'a fait, pour vous aussi.

\dvrule

\dvprayer{
Père, aide-nous à avoir assez de sagesse pour voir où la route
 que nous suivons nous conduit.
 Donne-nous la sagesse de nous détourner de notre péché
 et de consacrer nos vies à suivre Jésus. 
}{\EsN}

\suggest{En Son Nom, comme utilisé plus tôt?}


%%%%%%%%%%%%
% 13 mai
%%%%%%%%%%%%

\dvday{L\ap{}Incertitude de Demain}

\dvquote{
Ne te félicite pas du lendemain,
 car tu ne sais pas ce qu'un jour peut enfanter.
}{\bibleverse{Prov}(27:1)}

\dvlettrine{N}{ous ne savons pas} grand chose sur le futur,
 mais nous savons au moins une chose~: il est incertain. 

Le chrétien, cependant, sait quelque chose d'autre.
 Le chrétien sait que, quoi que demain amène, Dieu sera là.
 Jésus a promis~:
 \og Jamais je ne te délaisserai, jamais, je ne t'abandonnerai \fg{}
 (\bibleverse{Heb}(13:5)).
 Ceci signifie que si la douleur, le chagrin ou la tragédie arrivent \grammar{conjugaison arrivent}
 demain, vous serez capable de vous en sortir
 parce que le Seigneur sera avec vous. 

\dvbox{
La volonté de Dieu est le facteur inconnu de demain. 
}

Jacques nous dit qu'il n'est pas mal en soi de faire des plans,
 mais que quand nous les faisons, nous devons les faire en tenant compte
 d'un facteur contingent.
 \og Vous maintenant qui dites~: Aujourd'hui ou demain nous irons
 dans telle ville, nous y passerons une année, nous y ferons des affaires
 et nous réaliserons un gain! Vous qui ne savez pas ce que votre vie
 sera demain! Vous êtes une vapeur qui paraît pour un peu de temps,
 et qui ensuite disparaît.
 Vous devriez dire au contraire~: Si le Seigneur le veut,
 nous vivrons et nous ferons ceci ou cela \fg{} (\bibleverse{James}(4:13-15)). 

\og Si le Seigneur le veut \fg{}. La façon d'enlever l'incertitude de demain
 et d'amener la certitude, c'est de s'en remettre à Lui.
 Croyez que Dieu vous aime et qu'Il a un plan pour votre vie.
 Croyez qu'Il est avec vous indépendamment de ce qui arrive
 \ocadr et restez ouverts à l'idée d'être dirigés et guidés par Lui.

\dvrule

\dvprayer{
Père, nous Te remercions de ce que Ta main est au-dessus des circonstances
 qui entourent nos vies.
 Aide-nous, Seigneur, à Te prendre en considération dans tous nos plans. 
}{\DlNdJnp}


%%%%%%%%%%%%
% 14 mai
%%%%%%%%%%%%

\dvday{La Vie dans le Fils}

\dvquote{
J'ai vu tous les ouvrages qui se font sous le soleil ;
 voici que tout est vanité et poursuite du vent.
}{\bibleverse{Eccl}(1:14)}

\dvlettrine{S}{alomon a vécu} à la recherche du savoir, des plaisirs,
 de la richesse, du pouvoir, de la renommée
 \ocadr mais à la fin de sa vie, tout ce qui lui est resté
 n'était que frustration, vide et un cœur insatisfait. 

La vie sous le soleil (séparée de Dieu) est vide.
 La vie dans le Fils\NdT{l'auteur fait un jeu de mot en anglais
 en contrastant les deux mots anglais \emph{sun} (soleil)
 et \emph{son} (fils).}, au contraire est riche et satisfaisante
 parce que ce que vous faites pour le Seigneur n'est jamais fait en vain.
 Comme c'est à l'opposé des choses que l'on fait pour soi-même
 \ocadr les ambitions, les buts et les aspirations égoïstes!
 Peu de temps après que vous aurez quitté cette terre,
 ces réussites accomplies égoïstement auront disparu et seront oubliées. 

\dvbox{
Seule la vie vécue pour Jésus amène un bénéfice durable et éternel. 
}

À la fin de sa vie, l'apôtre Paul écrivait~: \punct{deux-points}
 \og J'ai combattu le bon combat, j'ai achevé la course, j'ai gardé la foi.
 Désormais la couronne de justice m'est réservée ;
 le Seigneur, le juste juge, me la donnera en ce Jour-là \fg{}
 (\bibleverse{IITim}(4:7-8)). 

Et voici ce témoignage~:
 \og Dieu nous a donné la vie éternelle, et cette vie est en son Fils.
 Celui qui a le Fils a la vie \fg{} (\bibleverse{IJn}(5:11-12)). 

Qu'il est glorieux d'arriver à la fin de vos jours en anticipant un héritage
 non souillé et incorruptible \ocadr un héritage qui vous attend
 dans le Royaume de Dieu.
 Car pour celui qui a vécu sa vie dans le Fils, le bout du chemin
 ce n'est pas le vide, le chagrin ou les regrets. C'est juste le commencement. 

\dvrule

\dvprayer{
Père, notre coupe déborde et nous ne pouvons pas retenir nos sentiments
 de joie et de bénédiction rencontrés dans la vie dans le Fils. 
 Merci pour Ta grâce envers nous. 
}{\DlNdJnp}


%%%%%%%%%%%%
% 15 mai
%%%%%%%%%%%%

\dvday{Qui Sait?}

\dvquote{
Qui donc sait ce qui est bon pour l'homme pendant la vie,
 pendant le nombre des jours de sa vaine existence,
 qu'il mène comme une ombre ?
}{\bibleverse{Eccl}(6:12)}

\dvlettrine{L}{a vie est courte,} la vie est incertaine et une vie vécue
 pour la chair ne satisfait pas. Cela, nous le savons.
 Mais au delà de ça, nous ne savons pas grand chose. 

Si nous avions le choix entre richesse et pauvreté,
 nous choisirions la richesse de manière écrasante.
 Cependant qu'est-ce-qui vaut mieux? La Bible contient
 beaucoup d'avertissements pour les riches.
 Nous ne savons pas comment la richesse pourrait affecter
 notre attitude par rapport à la vie et à Dieu. 

\dvbox{
Qu'est-ce qui est le meilleur pour nous?
 Nous n'en savons rien. Dieu seul le sait. 
}

Si le choix nous était donné, nous choisirions la santé
 plutôt que la maladie. Pourtant qu'est-ce qui vaut mieux?
 La maladie ne pourrait-elle pas me rapprocher de Dieu
 et m'amener à une plus riche relation avec Lui?
 Ou m'amènerait-elle à être en colère contre Dieu? 

Qu'est-ce qui vaut mieux, le succès ou l'échec?
 Encore une fois, si le choix nous était donné,
 le succès l'emporterait haut la main.
 Pourtant combien de personnes ne sont-elles pas venues à Christ
 en raison d'un échec dans leur vie?
 Parvenues au point du désespoir et de l'échec personnel,
 beaucoup de personnes sont attirées par la croix. 

Puisque nous ne pouvons pas savoir ce que le futur nous réserve,
 n'est-il pas sage de le remettre entre les mains de Celui qui Lui sait?
 Puisque nous ne savons pas ce qui est le meilleur pour nous,
 n'est-ce pas de la folie que de contester à Dieu Son choix des choses
 qu'Il laisse entrer dans nos vies? 

L'homme sage se soumet à Dieu et croit qu'Il va accomplir
 ce qui est le meilleur. 

Il l'a fait, pour vous aussi.

\dvrule

\dvprayer{
Père, faisant confiance à Ton infinie sagesse,
 nous plaçons notre futur entre Tes mains,
 sachant que Toi seul sait ce que demain nous réserve. 
}{\DlNdJnp}


%%%%%%%%%%%%
% 16 mai
%%%%%%%%%%%%

\dvday{Semez}

\dvquote{
Qui observe le vent ne sèmera point,
 qui fixe les regards sur les nuages ne moissonnera pas.
}{\bibleverse{Eccl}(11:4)}

\dvlettrine{T}{out ce que nous faisons} dans la vie
 comporte des risques et des difficultés.
 La plupart du temps, nous ne laissons pas ces risques potentiels
 nous empêcher de réaliser nos buts.
 Nous voyons ces risques comme des défis et nous continuons d'avancer.
 Sinon, nous limitons ce que nous pouvons accomplir pendant notre vie. 

Salomon évoque ceux qui laissent la présence du vent les empêcher
 de semer et la possibilité de pluie les empêcher de récolter.
 Si l'hésitation appréhensive peut vous empêcher de récolter
 une moisson dans le domaine physique, à combien plus forte raison
 cela est vrai dans le domaine spirituel! 

Quand elle est semée, la Parole de Dieu tombe dans différents types de sols.
 Tous ne sont pas réceptifs.
 Quelquefois, nous regardons certaines personnes et nous concluons
 qu'elles ont tellement dépassé les limites, que partager l'évangile \suggest{Évangile}
 avec elles ne servirait à rien. Mais c'est faux!
 Ne pas semer la Parole de Dieu pour des raisons réelles ou imaginées,
 c'est désobéir au clair commandement de Jésus de porter Son évangile \suggest{Évangile}
 partout dans le monde. 

Benjamin Franklin a déclaré~: \punct{deux-points, majuscule}
 \og Un homme qui est doué pour trouver des excuses est rarement doué
 pour autre chose. \fg{} Il n'y a pas d'excuses à la désobéissance. 

\dvbox{
Nous sommes appelés à semer la semence dans des endroits improbables
 en laissant à Dieu le soin de s'occuper des résultats.
 Nous ne sommes pas appelés à juger \emph{a priori} de la qualité du sol. 
}

Le psalmiste a écrit~: \punct{deux-points}
 \og Celui qui s'en va en pleurant, quand il porte la semence à répandre,
 s'en revient avec des cris de triomphe, quand il porte ses gerbes \fg{} \punct{Point en trop}
 (\bibleverse{Ps}(126:6)). Une moisson attend. Ignorez le vent et la pluie,
 et la logique qui vous mettrez sur la touche.
 Semez les semences confiées à votre bon soin.

\dvrule

\dvprayer{
Père, aide-nous à être fidèle pour implanter Ta Parole
 et Ton amour dans la vie de ceux qui nous entourent. 
}{\DlNdJ}





