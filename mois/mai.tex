\dvmonth{Mai}

%%%%%%%%%%%%
% 1er mai
%%%%%%%%%%%%

\dvday{Le Monde Merveilleux de Dieu}

\dvquote{
Que Tes œuvres sont en grand nombre, ô Éternel!
 Tu les as toutes faites avec sagesse.
 La terre est remplie de ce que Tu possèdes.
}{\bibleverse{Ps}(104:24)}

\dvlettrine{S}{ubmergé} par l'émotion devant la beauté créée par Dieu,
 le psalmiste utilise son langage le plus pittoresque pour essayer
 de décrire l'ouvrage de Dieu.

Il décrit le cycle de la pluie ou comment Dieu fait monter l'eau
 jusqu'aux sommets des montagnes d'où elle redescend dans les cours d'eau
 et comment les sources jaillissent alors dans les vallées pour que les animaux
 sauvages puissent y étancher leur soif, et comment bien que l'eau coule
 jusqu'à la mer, la mer ne déborde jamais.

En tout cela, le psalmiste voit la Sagesse de Dieu.
 Et en contemplant l'agencement spectaculaire et la beauté des choses que Dieu
 a créées, le psalmiste se met à adorer et à louer Dieu.

Il y a aujourd'hui des gens qui regardent la nature et voit le même agencement,
 la même beauté que le psalmiste voyait \ocadr{}mais au lieu d'adorer Dieu pour
 Sa création, ils adorent la création elle-même. Ils adorent la nature.
 Il est vraiment insensé de sentir une rose,
 de toucher la douceur de ses pétales, d'admirer la complexité de son agencement
 et la beauté de sa coloration et de conclure \og la rose est Dieu \fg{}.
 C'est totalement irrationnel. 

\dvbox{
La seule chose rationnelle à faire devant la beauté d'une rose,
 c'est d'en respirer le parfum et de dire~:
 \og voila une création de mon dieu. \fg{}
}

Et puis, comme le psalmiste, nous devrions louer le Dieu responsable
 de toute cette beauté \ocadr{}le Dieu des nuages, et du torrent de montagne,
 et de la mer\dots{} et de la rose. 

\dvrule

\dvprayer{
Père, nous T'adorons et Te louons pour la beauté de Ta création,
 dans laquelle nous nous émerveillons de Ta sagesse et de Ta bonté.
 Nous sommes stupéfaits devant l'œuvre de Tes mains.
 Puisses-Tu créer quelque chose de beau en chacun de nous.
}{\Amen}


%%%%%%%%%%%%
% 2 mai
%%%%%%%%%%%%

\dvday{Satisfaction de l'Âme Affamée }

\dvquote{
Que les hommes célèbrent l'Éternel pour Sa bienveillance et pour Ses merveilles
 en faveur des humains !
 Car Il a rassasié l'âme avide, a comblé de biens l'âme affamée.
}{\bibleverse{Ps}(107:8-9)}

\dvlettrine{V}{ous} est-il déjà arrivé de quitter la table d'un repas
 trop copieux en vous promettant~:
 \og Je ne mangerai plus jamais une autre bouchée de nourriture aussi longtemps
 que je vivrai \fg{} ?
 \`A ce moment-là vous le pensez vraiment sincèrement.
 Vous ne voulez vraiment plus d'une seule autre bouchée de nourriture.
 Mais arrive le soir\dots{} et vous vous retrouvez en train de mettre
 de la de crème chantilly sur une part de tarte qui restait.
 C'est parce que le corps exige d'être constamment nourri.

\dvbox{
La chair ne peut jamais être satisfaite.
}

Nourrissez votre chair, c'est-à-dire votre vieille nature,
 autant que vous le pouvez, elle en voudra toujours davantage.
 En fait si vous cédez à ses demandes dans un domaine,
 au lieu d'être satisfaite, votre vieille nature vous en redemandera
 toujours davantage au point où vous vous retrouvez esclaves.

Tout comme les hommes ont une faim et une soif physique,
 ils ont aussi une faim et soif spirituelles.
 Il y a problème dès que vous essayez de satisfaire un besoin spirituel
 par une expérience charnelle. Cela ne peut pas marcher.

Avez vous faim de Dieu? Soif de paix et de justice ?
 Dans \bibleverse{Jn}(6:35), Jésus a dit~:
 \og Moi, Je suis le pain de vie. Celui qui vient à Moi n'aura jamais faim,
 et celui qui croit en Moi n'aura jamais soif. \fg{}

Jésus est la réponse à toutes vos aspirations. 

\dvrule

\dvprayer{
Père, nous Te remercions que par Ton Fils, notre faim et notre soif
 ont été satisfaites. Pour ceux qui essayent encore de satisfaire
 leur besoin spirituel par des expériences physiques,
 nous demandons que Ton Saint Esprit les ramène à Jésus. 
}{C'est en Son Nom, que nous prions, \Amen}


%%%%%%%%%%%%
% 3 mai
%%%%%%%%%%%%

\dvday{Que Puis-je Donner à Dieu ?}

\dvquote{
Comment rendrai-je à l'Éternel tous ses bienfaits envers moi ?
}{\bibleverse{Ps}(116:12)}

\dvlettrine{S}{achant} que leur Grand-Maman aime les fleurs, nos petites-filles
 vont parfois cueillir pour elle des fleurs dans notre jardin.
 Elles ne font pas trop attention en les cueillant.
 D'habitude, elles ne gardent pas une tige assez longue.
 Le plus souvent, dans leur quête des fleurs parfaites, elles laissent derrière
 elles un sillage de plantes écrasées par leurs petits pieds.

Le jardin n'est pas le leur. C'est le nôtre. Elles vont donc dans notre jardin
 y cueillir nos fleurs pour les donner à Kay.
 Mais vous savez quoi ? Nous aimons ces cadeaux.

Je pense que Dieu reçoit nos cadeaux avec la même attitude.
 Il n'est pas une seule chose matérielle dont Il ait besoin.
 Et, parce que \og la Terre est au Seigneur, et tout ce qu'elle renferme \fg{}
 (\bibleverse{ICor}(10:26)), tout ce que nous pouvons Lui offrir matériellement
 Lui appartient, en fait, déjà. Quelquefois, quand nous traversons Son jardin
 à la recherche de fleurs à Lui offrir, nous y mettons un peu la pagaille.
 Cependant, Il reçoit ce que nous lui apportons avec une attitude
 pleine de grâce et d'amour.

Quand vous considérez tout ce que Dieu a fait pour vous
 et tout ce qu'Il a offert \ocadr le salut, la purification,
 l'espérance du ciel \fcadr{} cela vous donne envie de lui offrir
 quelque chose en retour.

\dvbox{
Que pourrions-nous bien donner à Dieu qui ait une quelconque valeur pour Lui?
}

Il n'a qu'une seule chose qu'Il désire vraiment de votre part.
 Une seule chose. Il veut votre c\oe{}ur. 

\dvrule

\dvprayer{
Père, quand nous pensons à tout ce que Tu nous a donné
 \ocadr le salut, la vie éternelle, notre pain quotidien \fcadr{}
 nous nous demandons ce que nous pourrions bien Te donner.
 Aussi Seigneur, désireux de t'offrir un sacrifice de remerciement,
 nous Te donnons nos c\oe{}urs et nos vies. 
}{}
