\jrnlmonth

%%%%%%%%%%%%%%
% 1er fevrier
%%%%%%%%%%%%%%

\jrnlday{Les exigences\\ du Seigneur}

\index{crainte|see{peur}}
\index{crainte!\sim~de Dieu|see{Dieu}}
\themeindex{Dieu!crainte de \sim}
\themeindex{commandement}
\themeindex{purification}
\themeindex{adoration}
\themeindex{epreuve@épreuve}
\themeindex{coeur@c\oe{}ur}
\themeindex{benediction@bénédiction}

\dvquote{
Maintenant, Israël, que demande de toi l'Éternel, ton Dieu, si ce n'est que
 tu craignes l'Éternel, ton Dieu, afin de marcher dans toutes Ses voies, d'aimer
 et de servir l'Éternel, ton Dieu, de tout ton c\oe{}ur et de toute ton âme;
 si ce n'est que tu observes les commandements de l'Éternel et Ses prescriptions
 que je te donne aujourd'hui,
 afin que tu sois heureux ?
}{\ibibleverse{Dt}(10:12-13)}

\lettrine{Q}{ue} demande de nous l'Éternel?
 Ce passage rend les exigences de Dieu très claires. \\[1ex]
Tout d'abord, Dieu voulait que les enfants d'Israël voient ce qui était dans
 leur c\oe{}urs. Face à l'épreuve, Lui obéiraient-ils ?
 Lui feraient-ils confiance ? L'adoreraient-t-ils? 

\dvbox{
Dieu se sert des épreuves pour nous montrer nos c\oe{}urs,
 afin de pouvoir nous purifier et nous faire ensuite entrer dans le pays
 de la bénédiction \ocadr{}physiquement et spirituellement. 
}

Deuxièmement, les Israélites devaient savoir que la vie est plus qu'une
 expérience physique. Dieu veut que nous vivions dans le domaine spirituel,
 en marchant selon l'Esprit et en étant guidés par l'Esprit.

Troisièmement, ils devaient savoir que Dieu punit les Siens, pour nous empêcher
 de faire des choses qui nous détruiraient.

Dieu se sert des épreuves de la même fa\c{c}on aujourd'hui.
 Il se sert des épreuves pour nous montrer nos c\oe{}urs,
 afin de pouvoir nous purifier et nous faire ensuite entrer dans le pays
 de la bénédiction. 

\dvrule

\dvprayer{
Père, puisse Ton Esprit Saint
 défier nos c\oe{}urs avec la vérité de Ta Parole.
 Aide-nous à renverser toutes les idoles et à mettre Christ sur le trône
 de nos vies. 
}{\Amen}

\suggest{ou Esprit-Saint?}

%%%%%%%%%%%%%%
% 2 fevrier
%%%%%%%%%%%%%%

\jrnlday{Le Prophète Promis}

\themeindex{prophete@prophète}
\themeindex{prophetie@prophétie}
\themeindex{Messie}
\themeindex{Bible}
\themeindex{Dieu!infinitude de \sim}
\index{infinitude de Dieu|see{Dieu}}
\themeindex{peche@péché}
\themeindex{injustice}
\themeindex{Dieu!pureté de \sim}
\index{purete@pureté!\sim~de Dieu|see{Dieu}}
\index{saintete@sainteté!\sim~de Dieu|see{Dieu}}
\themeindex{Dieu!saintete@sainteté de \sim}
\themeindex{mediateur@médiateur}
\themeindex{Dieu!promesses de \sim}
\themeindex{amour}
\themeindex{pardon}
\themeindex{esperance@espérance}
\themeindex{paix}

\dvquote{
Je leur susciterai du milieu de leurs frères un Prophète comme toi,
 je mettrai Mes paroles dans Sa bouche,
 et Il leur dira tout ce que Je Lui commanderai.
 Et si quelqu'un n'écoute pas Mes paroles qu'Il dira en Mon nom,
 c'est Moi qui lui en demanderai compte.
}{\ibibleverse{Dt}(18:18-19)}

\lettrine{J}{usqu'à} ce jour, les Juifs reconnaissent
 qu'il s'agit bien là d'une prophétie concernant le Messie.
 Tout comme Moïse était un médiateur qui a transmis la Parole de~Dieu~au peuple,
 le Prophète promis serait le médiateur qui transmettrait la Parole de Dieu
 au~peuple.
 Un médiateur est nécessaire en raison du fait que Dieu est infini
 et que l'homme est fini.
 L'homme ne peut pas comprendre ou appréhender Dieu qui est infini.

\dvbox{
Parce que je suis rempli de péché et d'injustice,
 je ne peux m'approcher de Dieu Qui est Lui absolument saint et pur.
 Aussi ai-je besoin d'un médiateur.
}

Cette prophétie du Deutéronome a été accomplie en Jésus-Christ.
 Paul nous dit dans \bibleverse{ITm}\ibiblephantom{ITm}(2:5)
 qu'il y a un seul Dieu
 et un seul Médiateur entre Dieu et les hommes,
 et que c'est l'homme Jésus-Christ. Il peut toucher Dieu parce qu'Il est Dieu.
 Il peut me toucher parce qu'Il s'est fait homme.
 Et maintenant par son intermédiaire, je peux toucher Dieu.

Dieu a tenu Sa promesse, Jésus-Christ est ce Prophète, ce Médiateur.
 Il est notre réconfort. Il est la Parole par laquelle nous avons la vie,
 la force, l'amour, le pardon, l'espérance et la paix. 

\dvrule

\dvprayer{
Père, merci d'avoir envoyé Ton Fils pour offrir le pardon qu'il nous faut.
 Puissions-nous marcher dans la lumière comme Jésus est dans la lumière,
 afin que nous puissions avoir la communion avec Toi. 
}{\EsN}


%%%%%%%%%%%%%%
% 3 fevrier
%%%%%%%%%%%%%%

\jrnlday{Mortifier la chair}

\themeindex{chair}
\themeindex{Dieu!crainte de \sim}
\themeindex{delivrance@délivrance}
\themeindex{heritage@héritage}
\themeindex{combat spirituel}
\themeindex{naissance!nouvelle \sim}
\themeindex{victoire}
\themeindex{Esprit-Saint}
\index{Dieu!Esprit de \sim|see{Esprit-Saint}}

\dvquote{
Souviens-toi de ce que te fit Amalec pendant la route,
 lors de votre sortie d'Égypte, comment il te rencontra pendant
 la route et coupa ton arrière-garde, tous ceux qui se traînaient les derniers,
 pendant que tu étais las et fatigué, et cela parce qu'il ne craignait pas Dieu.
 Lorsque l'Éternel, ton Dieu, en te délivrant de tous tes ennemis
 qui t'entourent, t'accordera du repos dans le pays que l'Éternel,
 ton Dieu, te donne en héritage pour en prendre possession,
 tu effaceras la mémoire d'Amalec de dessous les cieux\frcolon{} ne l'oublie pas.
}{\ibibleverse{Dt}(25:17-19)}

\lettrine{E}{n} typologie biblique, Amalec est un type
 \NdT{une préfiguration symbolique} de la chair.
 Tout comme Amalec cherchait à détruire Israël,
 la chair cherche à détruire le peuple de Dieu.
 Quand nous naissons de nouveau par l'Esprit de Dieu, l'ennemi,
 notre chair vient nous attaquer.
 La chair fait la guerre contre l'Esprit et l'Esprit contre la chair.

\dvbox{
Les attaques contre notre chair se concentrent toujours
 sur les domaines où nous sommes les plus vulnérables.
}

Dans \ibibleverse{Col}(3:5), Dieu nous dit de faire mourir notre chair.
 Nous devons la mettre à mort. Ce sont des termes cruels.
 Mais vous devez vous rappeler ce que la chair vous a fait,
 comment elle vous a exploités. Faites-la totalement disparaître!
 Détruisez-la complètement! C'est un combat à la vie, à la mort.
 Si vous cédez à la chair, elle vous détruira.

Peut-être êtes-vous las et fatigués de vous battre contre la chair.
 La bataille est féroce, mais Dieu veut vous délivrer.
 L'Esprit de Dieu peut vous donner la victoire sur votre chair.
 Demandez à Dieu Son aide. 

\dvrule

\dvprayer{
Aide-nous, Seigneur, à considérer la vieille vie comme morte,
 afin que nous puissions vivre une vie qui Te plaise et Te soit acceptable. 
}{\DlNdJ}


%%%%%%%%%%%%%%
% 4 fevrier
%%%%%%%%%%%%%%

\jrnlday{Servir le Seigneur\\ avec joie}

\themeindex{Dieu!service de \sim}
\index{temoin@témoin!\sim~de Dieu|see{Dieu}}
\themeindex{Dieu!temoin@témoin de \sim}
\themeindex{joie}
\themeindex{Satan}
\themeindex{etranger@étranger}

\dvquote{
Pour n'avoir pas servi l'Éternel, ton Dieu, avec joie et de bon c\oe{}ur,
 en ayant tout en abondance, tu serviras tes ennemis\dots{}
}{\ibibleverse{Dt}(28:47-48)}

\lettrine{Q}{uel} honneur que d'être choisis par Dieu
 pour être Son peuple spécial !
 Cependant, avec cet honneur, vient une formidable responsabilité.
 Nous devons être les témoins de Dieu pour le monde.
 Nous devons leur montrer la joie qui accompagne ceux qui suivent
 et servent le Seigneur, afin qu'ils puissent être aussi attirés par Lui
 et soumettre leurs vies à Sa seigneurie. \typo{seigneurie}

\dvbox{
Nous servons tous quelqu'un.
 Dans cette vie, vous allez soit servir Satan soit servir Dieu.
}

Ceux qui servent Satan se sentent chez eux dans ce monde.
 Ils sont à l'aise parce qu'ils sont complètement intégrés.
 Mais le résultat final d'une vie passée à servir Satan, c'est la mort.

Ceux qui servent Dieu sont des étrangers dans ce monde.
 C'est parce que plus vous vous approchez de Dieu,
 et plus vous allez vous sentir en décalage avec ce monde.
 Mais notre Maître est un Maître plein d'amour qui s'intéresse à vous
 et désire ce qu'il y a de mieux pour votre vie.
 Et Il va vous préparer émotionnellement, physiquement et spirituellement
 à faire n'importe quel travail auquel Il vous a appelés
 pour accomplir Son bon plaisir.

Servir le Seigneur est un privilège et une bénédiction.
 Si le travail que vous faites est ce à quoi Dieu vous a appelés,
 alors vous n'allez pas le trouver insupportable, intolérable ou pénible.
 Il a dit\frcolon{} \Og Mon joug est aisé et Mon fardeau léger. \Fg{}
 \ibiblephantom{Mt}(11:30)
 Au lieu de \c{c}a, vous allez vous trouver à servir avec joie. 

\dvrule

\dvprayer{
Aide-nous, Seigneur, à entreprendre avec joie le travail
 pour lequel Tu nous as créés.
 Puissions-nous découvrir la joie, le contentement
 et la satisfaction de Te servir. 
}{\DlNdJ}


%%%%%%%%%%%%%%
% 5 fevrier
%%%%%%%%%%%%%%

\jrnlday{Les secrets de Dieu}

\themeindex{revelation@révélation}
\themeindex{science}
\themeindex{Dieu!omniscience de \sim}
\themeindex{confiance}
\themeindex{loi!\sim{}s physiques}
\themeindex{loi!\sim{}s spirituelles}

\dvquote{
Les choses cachées sont à l'Éternel, notre Dieu ;\\
 les choses révélées sont à nous et à nos fils, à perpétuité,\\
 afin que nous mettions en pratique toutes les paroles de cette loi.
}{\ibibleverse{Dt}(29:28)}

\lettrine{D}{eux} domaines existent \ocadr celui des choses secrètes
 qui appartiennent à Dieu et celui des choses qu'Il a choisies de nous révéler.

La science a fait de nombreuses découvertes, mais il subsiste encore
 de vastes mystères et des secrets de l'univers que nous n'avons
 pas encore exploités.
 Dieu les connaît tous; Et quand Il choisit de nous les révéler,
 nous continuons d'apprendre.
 La même chose est vraie dans nos vies personnelles.

\dvbox{
Nous ne savons pas toujours pourquoi certaines choses nous arrivent,
 mais Dieu le sait.
}

Nous voulons des réponses. Nous voulons savoir pourquoi.
 Mais quelquefois la réponse de Dieu est simplement\frcolon{}
 \Og Fais-Moi confiance. \Fg{}

Tout comme Dieu a établi les lois qui gouvernent notre univers physique,
 telles que les lois de l'électricité, de l'aérodynamique et de la gravité,
 Il a aussi établi des lois spirituelles.
 Et tout comme la découverte de ces lois physiques a profité à nos vies,
 les lois spirituelles de Dieu \ocadr le secret de Son amour, de Sa fidélité,
 de Sa grâce et miséricorde \fcadr{}
 peuvent aussi être utilisées pour notre bénéfice et notre bénédiction.

Vous n'allez pas toujours comprendre ce que Dieu est en train de faire.
 Mais Il est fidèle. Il révèlera Ses secrets quand vous vous attendrez à Lui.
 Et un jour, vous Le remercierez de l'épreuve que vous traversez maintenant,
 parce que les secrets de Dieu auront été révélés grâce à elle. 

\dvrule

\dvprayer{
Père, nous Te sommes si reconnaissants de la révélation de Ton amour
 et de Ta loi.
 Aide-nous à garder Ta loi et à l'utiliser pour notre bénéfice,
 pour l'enrichissement et l'amélioration de nos vies. 
}{\DlNdJ}


%%%%%%%%%%%%%%
% 6 fevrier
%%%%%%%%%%%%%%

\jrnlday{Les bras éternels de Dieu}

\themeindex{Dieu!eternite@éternité de \sim}
\themeindex{eternite@éternité}
\index{eternite@éternité!\sim~de Dieu|see{Dieu}}
\themeindex{Bible}
\themeindex{peur}
\themeindex{confiance}

\dvquote{
Le Dieu d'éternité est un refuge,
 sous toi se trouvent des bras éternels\dots{}
}{\ibibleverse{Dt}(33:27)}

\lettrine{P}{arce que} le futur est inconnu et incertain,
 l'homme a tendance à en avoir peur. Mais ce court verset
 contient trois éléments qui ont le pouvoir de guérir de cette peur\frcolon{}

\begin{enumerate}[(1)]% arabic numbers within brackets
  \item Dieu est éternel. Avant que vous n'arriviez à une situation
   ou à une expérience, Dieu vous y a déjà précédés;
  \item Dieu est votre refuge. Un refuge est un abri,
   un endroit sûr où vous pouvez être à l'abri de la tempête.
   Et, Oh combien, ceux qui ont fait de Dieu leur abri, sont en sécurité !
   Les Écritures nous disent que le nom de l'Éternel, est une tour forte;
   le juste y court et s'y trouve hors d'atteinte (\ibibleverse{Pr}(18:10));
  \item Sous moi, se trouvent les bras éternels.
   Cela signifie que je ne peux pas tomber.
   Bien que je puisse penser que je vais tomber,
   la vérité c'est que Dieu me retient;
\end{enumerate}

\dvbox{
Peu importe ce que demain va amener,
 Dieu est prêt à nous soutenir et à nous protéger.
}

Quelqu'un dira peut-être\frcolon{}
 \Og Hé, J'ai vraiment touché le fond cette semaine. \Fg{}
 C'est formidable. Cela signifie que vous reposez maintenant
 directement sur Dieu, parce que quand vous arrivez au fond,
 sous vous se trouvent Ses bras éternels.

Nous n'avons pas à avoir peur.
 Le Dieu éternel nous a précédés et connaît toutes les choses
 qui nous attendent. Il est notre demeure,
 l'endroit où nous sommes absolument sains et saufs.
 Ses bras sont toujours en-dessous de nous, qui nous soutiennent fermement. 

\dvrule

\dvprayer{
Merci Père, de ce que Tu es toujours là quand nous avons besoin de Toi,
 et que Tes bras éternels sont toujours en-dessous de nous.
 Tiens-nous fermement aujourd'hui. Donne à ceux qui ont peur,
 la foi de croire que Tu ne vas pas les laisser tomber. 
}{\DlNdJ}


%%%%%%%%%%%%%%
% 7 fevrier
%%%%%%%%%%%%%%

\jrnlday{Dieu des cieux\\ et de la terre}

\themeindex{Dieu!infinitude de \sim}
\themeindex{foi}
\themeindex{revelation@révélation}
\themeindex{Bible}

\dvquote{
Nous l'avons appris, le c\oe{}ur nous a manqué,
 et chacun a perdu le souffle devant vous, car l'Éternel,
 votre Dieu, est Dieu dans les cieux, là-haut, et sur terre, ici-bas.
}{\ibibleverse{Jos}(2:11)}

\lettrine{R}{ahab} ne connaissait Dieu que par ouï-dire,
 et pourtant elle en avait une claire perception.
 \Og Il est Dieu dans les cieux là-haut, \Fg{} disait-elle.
 Mais parce que les cieux là-haut sont si vastes,
 des personnes pensent que Dieu semble lointain et impersonnel.
 Bien qu'elles croient en un Dieu qui serait une puissance ou une force,
 elles ne pensent pas qu'elles puissent le connaître.

Mais Rahab a alors continué en disant\frcolon{}
 \Og Il est le Dieu sur le terre ici-bas. \Fg{}
 Et c'est là où nous nous approchons de Lui
 et commen\c{c}ons à comprendre Qu'il n'est pas juste un Dieu
 d'une formidable puissance, mais qu'Il est aussi un Dieu du détail précis.

Quand on contemple les cieux, Dieu nous semble éloigné.
 Qui suis-je pour que Dieu s'intéresse à moi ?
 Pourtant le psalmiste écrit\frcolon{}
 \Og Que tes pensées, ô Dieu, me semblent impénétrables !
 Que la somme en est grande ! Si je les compte, elles sont plus nombreuses
 que les grains de sable \Fg{} (\ibibleverse{Ps}(139:17-18)).

\dvbox{
Si vous voulez faire une expérience spirituelle forte,
 allez à la plage et ramassez une poignée de sable.
 Chaque grain de sable représente les pensées de Dieu à votre égard.
}

Rahab est arrivée à croire en Dieu juste par ouï-dire,
 mais vous avez beaucoup plus que des ouï-dires.
 Vous disposez de la claire révélation de Dieu par Lui-même dans Sa Parole.

\dvrule

\dvprayer{
Père, nous savons que Tu n'es pas juste une vague forme
 d'énergie immatérielle perdue quelque part dans l'univers.
 Puissions-nous arriver à Te connaître comme un Dieu personnel plein d'amour. 
}{\DlNdJ}


%%%%%%%%%%%%%%
% 8 fevrier
%%%%%%%%%%%%%%

\jrnlday{L'incertitude\\ du lendemain}

\themeindex{incertitude}
\themeindex{sanctification}
\themeindex{desert@désert}
\themeindex{peur}
\themeindex{Dieu!omnipresence@omniprésence de \sim}
\index{omnipresence@omniprésence de Dieu|see{Dieu}}
\themeindex{delivrance@délivrance}
\themeindex{Dieu!promesses de \sim}

\dvquote{
Josué dit au peuple\frcolon{} Sanctifiez-vous,
 car demain l'Éternel accomplira des prodiges au milieu de vous.
}{\ibibleverse{Jos}(3:5)}

\lettrine{A}{vec} la traversée du désert derrière eux,
 les enfants d'Israël sont sur le point de commencer
 une expérience entièrement nouvelle avec le Seigneur.
 Quarante années d'errance leur avaient beaucoup appris
 sur eux-mêmes et sur Dieu.
 Mais maintenant, ils étaient confrontés à un futur incertain.

Un territoire inconnu et inexploré fait souvent peur.
 Peut-être vais-je commencer un nouveau travail
 \ocadr vais-je pouvoir être à la hauteur?
 Qu'est-ce que demain me réserve ?

Debout devant le peuple, Josué l'a rassuré avec ces mots réconfortants\frcolon{}
 \Og Sanctifiez-vous, car demain Dieu\dots{} \Fg{}
 Et c'est la chose que je dois aussi garder à l'esprit.
 \Og Demain Dieu. \Fg{}
 Voyez-vous, Dieu m'a déjà précédé. Dieu est déjà là.

Bien que le futur soit plein d'incertitudes, il y aussi quelques certitudes
 \ocadr dont l'une d'elles est hier.
 Le passé me donne l'assurance pour le futur.
 La délivrance passée de Dieu annonce la délivrance future de Dieu.

\dvbox{
Bien que le futur ne soit pas clair,
 vous pouvez toujours y faire face en faisant confiance à Dieu.
}

Dieu ne va pas vous abandonner. Il ne va pas vous décevoir.
 Le Dieu qui vous a fait traverser le désert va vous guider
 jusqu'à ce que vous ayez conquis le pays et pris possession des promesses. 

\dvrule

\dvprayer{
Père, merci pour l'espérance qui vient de Ta fidélité passée.
 Puissions-nous trouver la sécurité et la paix en Jésus
 qui a promis de ne jamais nous laisser ni nous abandonner. 
}{\EsN}


%%%%%%%%%%%%%%
% 9 fevrier
%%%%%%%%%%%%%%

\jrnlday{Dieu est pour nous}

\index{armee@armée de Dieu|see{Dieu}}
\themeindex{Dieu!armee@armée de \sim}
\themeindex{adoration}
\themeindex{combat spirituel}

\dvquote{
Josué [\dots{}] leva les yeux et regarda\frcolon{}
 voici qu'un homme se tenait en face de lui, son épée nue à la main.
 Josué marcha vers lui et lui dit\frcolon{}
 \Og Es-tu pour nous ou pour nos ennemis ? \Fg{}\\
 Il dit\frcolon{} \Og Non, mais je suis le chef de l'armée de l'Éternel,
 j'arrive maintenant. \Fg{}
 Josué tomba le visage contre terre et se prosterna.
}{\ibibleverse{Jos}(5:13-14)}

\lettrine{L}{a} charge de Josué est lourde et effrayante.
 Dieu l'a chargé de diriger le peuple.
 Alors qu'il se tient sur les hauteurs pour observer Jéricho en contrebas
 \ocadr et qu'il développe probablement une stratégie
 pour l'assaut des murailles \fcadr{}
 il voit un Homme\suggest{homme} avec une épée nue à la main.
 Josué Lui demande\frcolon{} \Og Es-tu pour nous ou pour nos ennemis ? \Fg{}

Quand il apprend que L'Homme\suggest{l'homme} vient d'auprès de Dieu,
 Josué se prosterne pour l'adorer.
 De par cette action, nous pouvons conclure que ce commandant en chef
 n'est autre que Jésus-Christ. Un ange n'aurait pas accepté d'être ainsi adoré.

Il m'a fallu du temps avant de me rendre compte que Dieu
 était pour moi plutôt que~pour mes adversaires.
 Je croyais que Dieu ne faisait qu'attendre
 que je fasse une erreur pour pouvoir me punir.
 Je pensais qu'Il allait me châtier
 parce que je n'étais pas toujours ce que j'aurais dû être.

\dvbox{
Quand vous faites une vraie rencontre avec Dieu,
 vous en sortez avec la prise de conscience que Dieu est pour vous.
}

Comme Paul l'a dit\frcolon{}
 \Og Si Dieu est pour nous, qui sera contre nous? \Fg{} (\ibibleverse{Rm}(8:31)).
 Avec Dieu de notre côté, comment pourrions-nous perdre? 

\dvrule

\dvprayer{
Père, nous Te remercions de nous assurer que Tu es de notre côté.
 Seigneur, en Ton nom, nous allons faire face à nos problèmes.\\
 En Ton nom, nous allons revendiquer le territoire que Tu nous a promis. 
}{\Amen}


%%%%%%%%%%%%%%
% 10 fevrier
%%%%%%%%%%%%%%

\jrnlday{Glorieuse victoire}

\themeindex{victoire}
\index{nature!vieille \sim|see{chair}}
\themeindex{Jesus@Jésus-Christ!oeuvre@\oe{}uvre de \sim}
\index{oeuvre@\oe{}uvre!\sim~de Jésus-Christ|see{Jésus-Christ}}
\index{oeuvre@\oe{}uvre!\sim~de l'Esprit|see{Esprit-Saint}}
\themeindex{Dieu!plan de \sim}
\themeindex{obeissance@obéissance}
\themeindex{chair}

\dvquote{
L'Éternel dit à Josué\frcolon{}
 Vois, j'ai livré entre tes mains Jéricho\dots{}
}{\ibibleverse{Jos}(6:2)}

\lettrine{S}{i} vous souhaitez la victoire sur les places fortes
 de la vieille nature, considérez la conquête du pays par Israël.
 Les principes de victoire s'y trouvent précisément décrits.
 Tout d'abord, notez que Dieu a parlé à Josué au passé.
 \Og Je l'ai fait. C'est terminé. \Fg{}
 De la même fa\c{c}on, votre victoire est une \oe{}uvre accomplie de Jésus-Christ.
 Le Seigneur vous a déjà accordé la victoire sur votre vie charnelle
 \ocadr tout ce qu'il vous suffit de faire est d'y croire.

Le deuxième principe est de suivre le plan de Dieu pour la victoire
 même quand nous ne le comprenons pas.
 Le plan de Dieu pour conquérir Jéricho était,
 pour le moins qu'on puisse en dire, intéressant\dots{}
 Quand Josué a réuni ses généraux pour leur expliquer
 la stratégie de bataille qui consistait à défiler autour de la cité,
 ces derniers ont probablement commencé à nourrir des doutes
 sur leur obligation de loyauté envers ce gars.

\dvbox{
Le plan de Dieu pour votre situation peut paraître bizarre.
 Et il se peut même que vous soyez réticent à l'essayer.
 Obéissez à ce que Dieu a mis sur votre c\oe{}ur.
}

\typo{paraître}

Le troisième principe, c'est que le peuple devait pousser
 un grand cri de victoire \ocadr bien que les murs soient toujours debout.
 Ceci constituait une expression de foi et d'anticipation.

Nous pouvons faire confiance à la Parole de Dieu. Il promet qu'Il va \oe{}uvrer.
 Louez-Le pour ces promesses de victoire.
 Dieu ne veut pas que vous continuiez dans la défaite,
 en restant dominé par votre chair et les places fortes de votre vie charnelle.
 Remettez-Lui votre vie et faites Lui confiance.
 Il fera s'écrouler les murailles. 

\dvrule

\dvprayer{
Père, nous Te remercions pour la victoire glorieuse
 dont nous pouvons faire l'expérience par la puissance de Jésus-Christ.
 Merci de nous avoir libérés de nos péchés passés. 
}{\Amen}


%%%%%%%%%%%%%%
% 11 fevrier
%%%%%%%%%%%%%%

\jrnlday{Mauvaises alliances}

\themeindex{alliance}
\themeindex{paix}
\themeindex{Bible}
\themeindex{confiance}
\themeindex{intelligence}
\themeindex{Dieu!voies de \sim}
\themeindex{decision@décision}

\dvquote{
On prit de leurs provisions sans avoir consulté l'Éternel.
 Josué fit la paix et conclut une alliance avec eux,
 en leur garantissant la vie,
 et les princes de la communauté leur en firent le serment.
}{\ibibleverse{Jos}(9:14-15)}

\lettrine{D}{ieu} avait spécifiquement dit aux Israélites
 qu'ils ne devaient conclure aucune alliance avec les habitants
 de la Terre Promise, mais qu'ils devaient les en chasser complètement.
 Dieu savait que si ces habitants restaient, leurs pratiques finiraient
 par détruire Son peuple.
 Les Gabaonites s'étant rendu compte qu'ils ne faisaient
 pas le poids contre les armées d'Israël,
 avaient concocté un plan pour donner l'impression
 qu'ils arrivaient d'un pays lointain
 dans le but de conclure une alliance avec Israël.

L'erreur de Josué consista à examiner les éléments de preuve
 sans demander l'avis du Seigneur.
 Dieu l'aurait mis en garde contre cette mauvaise alliance,
 mais Josué n'a pas recherché le conseil de Dieu.
 Il s'est contenté d'évaluer la situation au regard de ce qui était devant lui.
 Certaines choses paraissent si évidentes.
 Vous ne devriez pas avoir à déranger Dieu à propos de ces choses,
 n'est-ce pas? \typo{pas de tiret à n'est-ce pas}
 Il vous suffit d'aller de l'avant en utilisant votre bon sens.

Faux! Les Écritures nous disent\frcolon{}
 \Og Confie-toi en l'Éternel de tout ton c\oe{}ur,
 et ne t'appuie pas sur ton intelligence;
 \grammar{point-virgule \emph{ou} majuscule}reconnais-le dans toutes tes voies,
 et c'est lui qui aplanira tes sentiers \Fg{} (\ibibleverse{Pr}(3:5-6)).

\dvboxnocenter{
Nous pouvons éviter des erreurs tragiques\frcolon{}
\begin{enumerate}[(1)]% arabic numbers within brackets
 \item en permettant toujours à la Parole de Dieu de guider nos décisions;
 \item en ne prenant jamais une décision tant que nous n'avons pas d'abord
 cherché la direction du Seigneur par la prière.
\end{enumerate}
}

N'est-ce pas plutôt fou de faire confiance à mon propre jugement
 quand j'ai l'avantage de pouvoir aller à Dieu pour être conseillé? 

\dvrule

\dvprayer{
Seigneur, aide-nous à faire attention aux avertissements de Ta Parole.
 Puissions-nous Te rechercher pour que Tu nous aides
 à prendre toutes nos décisions. 
}{\Amen}


%%%%%%%%%%%%%%
% 12 fevrier
%%%%%%%%%%%%%%

\jrnlday{La puissance de Dieu}

\themeindex{Dieu!puissance de \sim}
\themeindex{miracle}
\themeindex{victoire}
\themeindex{Bible}

\dvquote{
Alors Josué parla à l'Éternel, le jour où l'Éternel livra les Amoréens
 aux Israélites, et il dit en présence d'Israël\frcolon{}
 \Og Soleil, tiens-toi immobile sur Gabaon, et toi, lune,
 sur la vallée d'Ayalôn. \Fg{}
 Et le soleil se tint immobile,
 et la lune s'arrêta, jusqu'à ce que la nation eût tiré
 vengeance de ses ennemis.
}{\ibibleverse{Jos}(10:12-13)}

\lettrine{L}{es} Israélites s'étaient bien battus
 et la victoire était proche \ocadr mais la nuit arrivait.
 Josué réalisa que quand la nuit serait tombée,
 les Amorites pourraient s'échapper. Ils avaient besoin de plus de temps.
 Aussi, devant toutes ses troupes, il commanda\frcolon{}
 \Og Soleil, tiens-toi immobile, et toi, lune, sur la vallée d'Ayalôn,
 ne bouge pas! \Fg{}
 Et le soleil se tint immobile et la lune garda sa position.

Les mêmes ressources qui étaient à la disposition de Josué
 \ocadr ressources qui ont permis l'existence de l'univers,
 les ressources qui ont mis la terre en mouvement sur son axe
 et qui l'ont mise en rotation autour du soleil \fcadr{}
 sont à votre disposition en tant qu'enfant de Dieu.

Nous faisons souvent l'erreur de penser que Dieu va tout faire
 et que nous n'avons rien à faire.
 Mais la Bible nous dit que la foi sans les \oe{}uvres est morte.

\dvbox{
Dieu veut que nous fassions de notre mieux
 et que nous confiions le reste à Dieu.
}

\grammar{subjonctif: confiions}

Quand nous atteignons nos limites, nous découvrons les ressources infinies
 qui sont à notre disposition pour accomplir et terminer le travail de Dieu.

\dvrule

\dvprayer{
Père, nous sommes tellement reconnaissants de Te servir\\
 \ocadr Toi qui est un Dieu illimité, infini,
 capable de faire infiniment au-delà de tout ce que nous demandons ou pensons.
 Seigneur, nous sommes faibles et nous avons besoin de Ta force.\\
 Prend le contrôle de nos vies, Seigneur, guide-nous et dirige-nous. 
}{\Amen}


%%%%%%%%%%%%%%
% 13 fevrier
%%%%%%%%%%%%%%

\jrnlday{La fidélité de Dieu}

\themeindex{Dieu!fidelite@fidélité de \sim}
\index{fidelite@fidélité!\sim~de Dieu|see{Dieu}}
\themeindex{repos}
\themeindex{Egypte@Égypte}
\themeindex{Dieu!Fils de \sim}
\index{Fils!\sim~de Dieu|see{Dieu}}
\index{naissance!\sim~de Jésus-Christ|see{nativité}}
\index{Jesus@Jésus-Christ!naissance de \sim|see{nativité}}
\themeindex{nativité}
\themeindex{vierge}
\themeindex{Jesus@Jésus-Christ!retour de \sim}

\dvquote{
C'est ainsi que l'Éternel donna à Israël tout le pays qu'Il avait juré
 de donner à leurs pères; ils en prirent possession et s'y établirent.
 L'Éternel leur accorda du repos tout alentour, comme Il l'avait juré
 à leurs pères; aucun de leurs ennemis ne put leur résister,
 et l'Éternel livra tous leurs ennemis entre leurs mains.
 De toutes les bonnes paroles que l'Éternel avait dites à la maison d'Israël,
 aucune ne resta sans effet\frcolon{} toutes s'accomplirent.
}{\ibibleverse{Jos}(21:43-45)}

\lettrine{Q}{uel} puissant témoignage est ainsi apporté
 à la fidélité de Dieu! \ocadr car Dieu s'est montré fidèle
 à Sa Parole alors même que son peuple ne l'était pas.
 Ils avaient brisé leur alliance avec Dieu à de multiples occasions.
 \`A un moment, ils essayaient même de persuader quelqu'un
 de les ramener en Égypte. Mais Dieu a tenu Sa Parole.

\dvbox{
Pas une seule des paroles de Dieu n'est restée sans effet.
}

Nous devons réaliser la certitude de la Parole de Dieu.
 Il nous a promis d'envoyer Son Fils qui devait naître
 d'une vierge et mourir pour les péchés de l'homme.
 Ce que Dieu a promis, Il l'a accompli.
 Il a promis de rassembler Israël de nouveau
 et de faire naître une nation en un jour.
 Le 14~mai~1948, Israël est devenue une nation. Dieu a tenu Sa Parole.

Et Il a promis de revenir et de nous prendre avec Lui.

Si Dieu a été fidèle jusqu'à ce moment, vous pouvez être certains
 qu'Il va continuer à être fidèle. Même si les cieux et la terre disparaissent,
 Sa Parole va demeurer éternellement.
 Vous pouvez parier tout ce que vous voulez là-dessus. 

\dvrule

\dvprayer{
Père, merci pour avoir fait ce que Tu as dit.
 Seigneur, aide-nous aujourd'hui à avoir
 l'assurance que nous pouvons faire confiance à Ta Parole. 
}{\Amen}


%%%%%%%%%%%%%%
% 14 fevrier
%%%%%%%%%%%%%%

\jrnlday{Sincèrement vôtre}

\themeindex{Dieu!service de \sim}
\themeindex{idole}
\themeindex{Egypte@Égypte}
\themeindex{Dieu!crainte de \sim}
\themeindex{esclave}
\themeindex{delivrance@délivrance}

\dvquote{
Maintenant donc, craignez le Seigneur, et servez-Le en sincérité et en vérité ;
 et ôtez les dieux que vos pères ont servis de l'autre côté du fleuve
 et en Égypte, et servez le Seigneur.
}{\ibibleverse{Jos}(24:14) \KJF}

\lettrine{L}{es} enfants d'Israël avaient commencé à servir d'autres dieux.
 Ils reconnaissaient toujours l'existence de Dieu, mais leur temps
 et leur énergie étaient absorbés par d'autres poursuites.
 Ils avaient relégué Dieu à une place secondaire.

Josué vit comment le c\oe{}ur des enfants de Dieu se détournaient de Lui.
 Il leur lan\c{c}a donc un défi. S'adressant au peuple réuni devant lui,
 il leur offrit un choix. Ils pouvaient continuer à être
 leurs propres serviteurs
 \ocadr en servant leur intellect et leurs plaisirs \fcadr{}
 ou ils pouvaient servir le Seigneur.

\dvbox{
Dieu n'est pas intéressé par nos faux-semblants,
 bien que ce soit ce que la plupart des gens Lui offrent.
 Nous pouvons servir Dieu avec sincérité et vérité
 quand nous nous souvenons de Sa bonté et de Sa fidélité à notre égard.
}

Dans le \bibleverse{Dt}, Moïse a demandé\frcolon{}
 \Og Qu'est-ce que le Seigneur attend de vous? \Fg{}
 Les deux premières exigences mentionnées
 étaient que nous craignions Dieu et que nous le servions de tout notre c\oe{}ur.
 Josué a donc rappelé aux gens les exigences de Dieu\frcolon{}
 craindre Dieu (Le traiter avec révérence),
 puis Le servir avec sincérité et vérité,
 c'est-à-dire Le servir de fa\c{c}on authentique.

Dieu avait amené les Israélites dans le pays,
 chassé leurs ennemis et les avait affranchis de l'esclavage.
 Parce que Dieu avait été si bon envers eux, ils auraient dû vouloir Le servir !

Ce qui est glorieux quand on choisit de servir le Seigneur,
 c'est que cela vous amène tout le reste.
 Quand vous choisissez de Lui donner la première place dans votre vie,
 Il vous bénit alors en vous donnant les désirs de votre c\oe{}ur. 

\dvrule

\dvprayer{
Père, nous Te sommes si reconnaissants pour les occasions
 que Tu nous a données.
 Apprends-nous à Te servir avec sincérité et vérité. 
}{\DlNdJ}


%%%%%%%%%%%%%%
% 15 fevrier
%%%%%%%%%%%%%%

\jrnlday{Le seul choix judicieux}

\themeindex{idole}
\themeindex{Dieu!service de \sim}
\themeindex{prosperite@prosperité}

\dvquote{
[\dots{}] choisissez aujourd'hui qui vous voulez servir\frcolon{}
 ou les dieux que servaient vos pères au-delà du fleuve,
 ou les dieux des Amoréens dans le pays desquels vous habitez.
 Moi et ma maison, nous servirons l'Éternel.
}{\ibibleverse{Jos}(24:15)}

\lettrine{D}{ans} les moments de prospérité,
 il est aisé de détacher le regard de Dieu.
 C'est précisément ce qui arrivait au gens d'Israël.
 Bien installés dans le pays, ils commen\c{c}aient à prospérer
 \ocadr et commen\c{c}aient à se tourner de nouveau vers les dieux
 de leurs ancêtres.
 Un choix les attendait \ocadr le même choix qui attend tout homme.

\dvbox{
Quelle que soit votre raison de vivre,
 quelle que soit la passion qui domine votre vie,
 quelle~que soit la raison qui vous tire du lit le matin et vous motive
 \ocadr voilà votre dieu.
}

Qui allez-vous servir? Qui servez-vous? Pour le découvrir,
 il vous suffit de terminer la phrase\frcolon{}
 \Og Pour moi, vivre c'est\dots{} \Fg{}
 Au bout du compte, quand vous atteindrez l'heure de la mort
 et que vous serez face à l'éternité, l'argent ne pourra pas vous sauver.
 Les plaisirs ne pourront pas vous sauver.
 L'intellect ne pourra pas vous sauver.
 Si vous avez fait de ces choses votre dieu, vous avez un problème.
 Jésus seul a le pouvoir de vous sauver
 et Jésus est donc le seul choix judicieux que vous puissiez faire.

Le fait est, que vous avez déjà fléchi le genou devant quelque chose.
 Si vous avez choisi de servir un faux dieu,
 choisissez aujourd'hui de servir le vrai Dieu vivant
 \ocadr le Dieu qui est fidèle, le Dieu qui est puissant,
 le Dieu qui est capable de délivrer Son peuple. 

\dvrule

\dvprayer{
Père, nous choisissons de Te servir, de Te couronner comme le Roi de nos vies,
 de fléchir le genou devant Ton trône, de toucher la pointe de Ton sceptre
 et de nous soumettre en tant que serviteurs obéissants
 par l'intermédiaire de Jésus-Christ. 
}{\Amen}


%%%%%%%%%%%%%%
% 16 fevrier
%%%%%%%%%%%%%%

\jrnlday{Une loi inéluctable}

\themeindex{peche@péché}
\themeindex{amour}
\themeindex{pardon}
\themeindex{consequence@conséquence}
\themeindex{Dieu!royaume de \sim}

\dvquote{
Adoni-Bézeq dit\frcolon{}
 Soixante-dix rois, ayant les pouces des mains et des pieds coupés,
 ramassaient les restes sous ma table ;
 Dieu me rend ce que j'ai fait. On l'emmena à Jérusalem, et il y mourut.
}{\ibibleverse{Jg}(1:7)}

\lettrine{A}{doni-Bézeq} était un roi puissant.
 Il avait conquis soixante-dix autres royaumes, et fait couper les pouces
 et les gros orteils de chacun des rois vaincus.
 Quand il fut finalement capturé par les hommes de Juda, que lui firent-ils?
 Ils lui coupèrent les pouces et les gros orteils.
 Et il dit\frcolon{} \Og Ce que j'ai fait subir aux autres, Dieu l'a requis de moi. \Fg{}

Dans le Nouveau Testament nous lisons\frcolon{}
 \Og Ne vous y trompez pas\frcolon{} on ne se moque pas de Dieu.
 Ce qu'un homme aura semé, il le moissonnera aussi \Fg{} (\ibibleverse{Ga}(6:7)).

\dvbox{
Nous sommes souvent contrariés à l'idée que quelqu'un
 est en train d'échapper aux conséquences de ses mauvaises actions.
 Mais tôt ou tard, leurs péchés vont les rattraper.
}

La loi de récolter ce que vous semez est une loi inéluctable.
 Mais c'est en même temps une belle loi.
 Si vous semez de l'amour, de la gentillesse et du pardon,
 vous allez récolter de l'amour, de la gentillesse et du pardon.
 Si vous semez de la miséricorde, vous allez récolter de la miséricorde.

Peut-être votre passé est-il sombre et lamentable,
 et vous avez semé beaucoup de mauvaises semences.
 Soyez réconfortés!
 Dieu a ouvert un chemin afin que vous n'ayez pas à récolter
 ce que vous avez semé.
 Il a pris la culpabilité de votre iniquité et l'a totalement placée
 sur Son Fils.
 Jésus-Christ a récolté les conséquences à votre place
 \ocadr afin que vous puissiez récolter la gloire de l'amour
 et du royaume éternel de Dieu. 

\dvrule

\dvprayer{
Aide-nous Seigneur à semer de bonnes semences dans les c\oe{}urs
 et les vies de ceux qui nous entourent;
 afin que nous puissions récolter, Seigneur, le beau fruit de cette récolte.
}{\DlNdJ}


%%%%%%%%%%%%%%
% 17 fevrier
%%%%%%%%%%%%%%

\jrnlday{Le péché de ne rien faire}

\themeindex{peche@péché}
\themeindex{Bible}
\themeindex{action}
\themeindex{commandement}

\dvquote{
\Og Maudissez Méroz \Fg{} dit l'Ange de l'Éternel,
 \Og Maudissez, maudissez ses habitants, car ils ne vinrent pas
 à l'aide de l'Éternel, à l'aide de l'Éternel, parmi les héros. \Fg{}
}{\ibibleverse{Jg}(5:23)}

\lettrine{L}{'ange} de l'Éternel pronon\c{c}a une sévère malédiction sur Méroz
 parce qu'ils n'avaient pas aidé en temps de crise et de besoin.
 Le jour de la bataille, ils n'avaient rien fait pour venir en aide au Seigneur.

Dieu a appelé son peuple à l'action. Vous avez été sauvés pour servir.
 \Og Pratiquez la parole,  nous dit Jacques,
 \punct{Pas de répétition des guillemets pour l'incise}
 et ne l'écoutez pas seulement,
 en vous abusant par de faux raisonnements \Fg{} (\ibibleverse{Jc}(1:22)).
 Trop souvent quand nous entendons la Parole de Dieu, nous disons\frcolon{}
 \Og Absolument d'accord! C'est tellement vrai!
 Oui, vraiment, nous devrions faire davantage pour Dieu! \Fg{}
 Puis\dots{} nous ne faisons rien.

\dvbox{
L'amour est démontré par nos actions.
}

Jésus a dit\frcolon{}
 \Og Celui qui a mes commandements et qui les garde,
 c'est celui qui M'aime \Fg{} (\ibibleverse{Jn}(14:21)).
 Il se peut que vous quittiez l'église le dimanche matin en disant\frcolon{}
 \Og Ouah, j'ai vraiment aimé ce sermon. \Fg{}
 Mais qu'allez-vous en faire? Car voyez-vous, s'il ne vous conduit pas
 à l'action, vous bâtissez sur le sable.
 Vous vous appuyez sur un faux sentiment de sécurité.
 Quand la tempête arrivera, votre maison va s'écrouler
 parce que vous n'avez rien fait \ocadr seulement écouté.

Je me demande combien d'entre nous appartiennent au clan de Méroz.
 Nous sommes si occupés à faire \Og notre truc \Fg{} que nous ne répondons pas
 à l'appel de Dieu. Il n'y a aucun signe dans nos vies montrant
 que nous faisons le travail de Dieu.
 Nous sommes au contraire engagés dans le péché de ne rien faire. 

\dvrule

\dvprayer{
Père, nous prions pour que Ton Esprit Saint
 nous pose le défi de l'action.
 Aide-nous à nous proposer comme volontaires de notre plein gré, Seigneur,
 pour obéir à Ta Parole et garder Tes commandements. 
}{\DlNdJ}

\suggest{ou Esprit-Saint?}


%%%%%%%%%%%%%%
% 18 fevrier
%%%%%%%%%%%%%%

\jrnlday{L'armée de Dieu}

\themeindex{Dieu!armee@armée de \sim}
\themeindex{peur}
\themeindex{courage}
\themeindex{combat spirituel}

\dvquote{
L'Éternel dit à Gédéon\frcolon{}
 C'est par les trois cents hommes qui ont lapé,
 que je vous sauverai et que je livrerai Madian entre tes mains. 
}{\ibibleverse{Jg}(7:7)}

\lettrine{D}{ieu} a réduit le nombre des hommes de Gédéon
 dans le but d'obtenir des hommes qu'Il pourrait utiliser pour Sa gloire.
 Finalement, des \numprint{32000}~hommes qui ont répondu à l'appel de Gédéon,
 il n'en est resté que~\numprint{300}.
 Dieu a disqualifié la majorité de deux fa\c{c}ons.
 Il a éliminé ceux qui avaient peur et ceux qui ne faisaient pas attention.

Les hommes qui ont peur s'enfuient pendant la bataille et leur peur peut
 se répandre par contagion.

\dvbox{
Dieu sait que la peur arrive quand on se concentre
 sur le problème ou sur la puissance de l'ennemi.
}

Deux tiers de l'armée de Gédéon sont retournés à la maison
 après le premier test, laissant \numprint{10000}~hommes.
 Dieu a alors dit\frcolon{}
 \Og Gédéon, les hommes avec toi sont encore trop nombreux.
 Emmène les au bord du torrent et observe la fa\c{c}on dont ils boivent.
 Tous ceux qui se mettent à genoux et plongent la tête dans l'eau pour boire
 sont disqualifiés. \Fg{}
 Seuls \numprint{300}~lapèrent l'eau en la portant à la bouche avec leurs mains
 tout en continuant de surveiller les alentours. Dieu avait trouvé Son armée.

Aujourd'hui, Dieu recherche des hommes qui sont courageux et alertes.
 Un courage qui a pour origine la foi va vous conduire
 à fixer les yeux sur Dieu. Il est aussi à la recherche d'hommes
 qui sont alertes et conscients de la guerre qui fait rage autour d'eux.
 Des hommes, qui même lorsqu'ils sont impliqués dans les nécessités de la vie
 \ocadr que ce soit boire de l'eau ou travailler pour vivre \fcadr{}
 sont toujours conscients des problèmes plus importants,
 du combat spirituel dans lequel nous sommes tous engagés. 

\dvrule

\dvprayer{
Père, aide-nous à progresser dans la foi sans souci,
 sachant que si Tu es pour nous, aucun homme ne peut s'opposer à nous.
 Puissions-nous nous habituer à accomplir Tes desseins et Ta volonté. 
}{\Amen}


%%%%%%%%%%%%%%
% 19 fevrier
%%%%%%%%%%%%%%

\jrnlday{Se détourner et servir}

\themeindex{idole}
\themeindex{delivrance@délivrance}
\themeindex{Dieu!service de \sim}
\themeindex{confession}
\themeindex{Dieu!fidelite@fidélité de \sim}
\themeindex{Dieu!justice de \sim}
\themeindex{Dieu!misericorde@miséricorde de \sim}
\index{misericorde@miséricorde!\sim~de Dieu|see{Dieu}}

\dvquote{
Les Israélites dirent à l'Éternel\frcolon{}
 \Og Nous avons péché ; traite-nous comme il Te plaira.
 Seulement, daigne nous délivrer aujourd'hui! \Fg{}
 Et ils ôtèrent les dieux étrangers du milieu d'eux
 et servirent l'Éternel qui fut touché des maux d'Israël.
}{\ibibleverse{Jg}(10:15-16)}

\lettrine{I}{sraël} avait abandonné Dieu.
 Une fois de plus, ils s'étaient remis à adorer d'autres dieux.
 Et une fois de plus, ils s'étaient retrouvés esclaves
 et se sont mis à implorer Dieu.

Dieu a répondu en leur disant\frcolon{}
 \Og Je vous ai secourus dans le passé,
 mais je ne vais plus vous délivrer. \Fg{}
 Dans \ibibleverse{Gn}(6:3), Dieu avait prévenu\frcolon{}
 \Og Mon esprit ne contestera pas toujours avec l'homme \Fg{}.
 Arrive un moment où Dieu déclare\frcolon{} \Og Ça suffit! \Fg{}
 Et aux enfants d'Israël, Il a dit\frcolon{}
 \Og Allez donc pleurer auprès des dieux que vous avez choisis. \Fg{}

Ils savaient que ces dieux ne pouvaient pas les aider.
 Il est intéressant de noter que lorsque les gens ont de graves problèmes,
 ils savent instinctivement que le Seul qui peut aider est le vrai Dieu vivant.
 Quand les ennuis arrivent, même les gens rebelles s'écrient\frcolon{}
 \Og Oh Dieu, Aide-moi ! \Fg{}

\dvbox{
Comme \ibibleverse{IJn}(1:9) nous le dit\frcolon{}
 \Og Si nous confessons nos péchés, Il est fidèle et juste
 pour nous pardonner nos péchés et nous purifier de toute injustice. \Fg{}
}

Ces gens ont fait une chose sage. Ils s'en sont remis à la miséricorde de Dieu.
 Et parce qu'ils ont agi de la sorte, Dieu les a encore délivrés
 de la main de leurs ennemis. Dieu est si miséricordieux!
 Et la même miséricorde est à notre disposition aujourd'hui.

Dieu ne nous doit rien. Mais Il a promis de montrer de la miséricorde
 à tous ceux qui se détournent de leurs mauvaises voies
 pour se tourner vers Lui. 

\dvrule

\dvprayer{
Seigneur, Merci de continuer à contester aujourd'hui avec les hommes
 et les femmes. Aide-nous à Te soumettre nos c\oe{}urs. 
}{\DlNdJ}


%%%%%%%%%%%%%%
% 20 fevrier
%%%%%%%%%%%%%%

\jrnlday{V\oe{}ux}

\themeindex{voeu@v\oe{}u}
\themeindex{peche@péché}
\themeindex{confession}
\themeindex{sauveur}
\themeindex{salut}
\themeindex{amour}
\themeindex{paix}

\dvquote{
Jephté fit un v\oe{}u à l'Éternel et dit\frcolon{}
 Si Tu livres totalement entre mes mains les Ammonites,
 quiconque sortira des portes de ma maison à ma rencontre,
 à mon heureux retour de chez les Ammonites, sera consacré à l'Éternel,
 et je l'offrirai en holocauste\dots{}
 J'ai trop ouvert la bouche devant l'Éternel,
 et je ne puis revenir en arrière.
}{\ibibleverse{Jg}(11:30-31,35)}

\lettrine{J}{ephté} avait prononcé un terrible v\oe{}u. \\[1ex]
Quelquefois, quand nous avons des ennuis, c'est ce que nous faisons.
 \Og Dieu, si seulement Tu me tires de ce mauvais pas,
 alors je ferai \c{c}a et \c{c}a. \Fg{}
 Quand une personne a fait un v\oe{}u imprudent,
 il vaut vraiment mieux qu'elle confesse le péché d'avoir fait un v\oe{}u imprudent,
 plutôt que d'insister et d'accomplir son v\oe{}u.
 Dans le cas de Jephté, accomplir son v\oe{}u était un plus grand péché
 que de rompre son v\oe{}u.

\dvbox{
Certains v\oe{}ux devraient simplement ne pas être respectés.
 Mais il est d'autres v\oe{}ux sur lesquels vous ne pouvez pas revenir.
}

J'ai ouvert ma bouche devant le Seigneur et confessé que Jésus-Christ
 est mon Sauveur\frcolon{} que je dépends uniquement de Lui pour mon salut;
 et qu'Il est le Seigneur de ma vie.

Et pour cette raison, il y a certaines choses que je ne peux pas faire.
 La première\frcolon{} je ne peux pas retourner à ma vieille vie de péché et d'égoïsme.
 Avec les apôtres, je dois déclarer\frcolon{}
 \Og Seigneur, Toi seul a les paroles de la vie éternelle.
 \`A qui pourrais-je aller ? \Fg{}
 La seconde\frcolon{} je ne peux pas nier l'amour et la paix que Jésus
 a amenés dans ma vie. 

\dvrule

\dvprayer{
Seigneur, si nous nous sommes écartés de ce premier amour,
 aide-nous à écouter Ton Saint Esprit et à encore revenir,
 jusqu'à ce que nous T'aimions avec tout ce qui est en nous. 
}{\Amen}


%%%%%%%%%%%%%%
% 21 fevrier
%%%%%%%%%%%%%%

\jrnlday{Le secret de la force}

\dvquote{
Dalila dit à Samson\frcolon{}
 Révèle-moi, je te prie, d'où vient ta grande force,
 et avec quoi il faudrait te lier pour te dompter. 
}{\ibibleverse{Jg}(16:6)}

\lettrine{L}{a force} de Samson était devenue légendaire.
 Quand l'Esprit de Dieu venait sur lui, il faisait des exploits
 qui montraient une force étonnante.
 Ses ennemis cherchaient à trouver le secret de sa force.
 Quand ils ont vu qu'il était tombé amoureux d'une Philistine, Dalila,
 ils ont approché la jeune fille en lui promettant\frcolon{}
 \Og Écoute, chacun de nous te donnera mille cent pièces d'argent,
 si tu nous trouves le secret de la force de cet homme. \Fg{}

Dalila a donc demandé à Samson\frcolon{}
 \Og Samson, dis-moi mon chéri, qu'est-ce qui te rend si fort?
 Quel est le secret de ta force? \Fg{}

Malheureusement, l'une des faiblesses de Samson était
 un sentiment d'invicibilité.
 Il semblait aimer flirter avec le danger,
 et ne cessait de s'aventurer en territoire ennemi.

Samson a finalement révélé son secret à Dalila.
 Il expliqua le v\oe{}u qu'il avait fait
 \ocadr un v\oe{}u de consécration à Dieu.
 Aucun rasoir n'avait touché sa tête.
 \Og Si mes cheveux étaient rasés,
 je serais faible comme n'importe quel homme. \Fg{}
 Mais la longue chevelure de Samson n'était qu'un symbole.
 Le secret de sa force, c'était son engagement envers Dieu.
 Tant qu'il était fidèle à cet engagement, il était réellement invincible.
 Quand l'engagement a été rompu, il est devenu faible
 comme n'importe quel autre homme.

\dvbox[0.85\textwidth]{
Le secret de votre force réside dans votre engagement envers Jésus-Christ.
}

Si vous vous engagez pleinement envers Jésus, vous serez invincible.
 Même les portes de l'enfer ne pourront prévaloir contre vous.
 Mais si vous rompez cet engagement envers Jésus-Christ,
 vous deviendrez faible comme n'importe quel autre homme. 

\dvrule

\dvprayer{
Seigneur, aide-nous à ne pas flirter avec le danger;\\
 mais puissions-nous nous consacrer pleinement et entièrement à Toi. 
}{\Amen}


%%%%%%%%%%%%%%
% 22 fevrier
%%%%%%%%%%%%%%

\jrnlday{Le rédempteur}

\dvquote{
Alors Booz dit aux anciens et à tout le peuple\frcolon{}
 \Og Vous êtes témoins aujourd'hui que j'ai acquis de la main de Noémi
 tout ce qui appartenait à Élimélek\dots{}
 et que je me suis également acquis pour femme Ruth la Moabite,
 femme de Mahlôn, pour maintenir le nom du défunt sur son héritage\dots{} \Fg{}
}{\ibibleverse{Rt}(4:9-10)}

\lettrine{S}{elon} la loi juive, si un homme marié mourait \grammar{mourait}
 avant d'avoir eu des enfants, le frère de cet homme était obligé
 d'épouser sa femme.
 On donnait au premier né de leurs enfants le nom du frère défunt
 afin que l'héritage puisse rester en Israël.

Booz \ocadr un parent d'Élimélek \fcadr{} a racheté le champ parce
 qu'il était amoureux de Ruth.
 Il a acheté le champ afin de pouvoir obtenir une épouse qu'il aimait.

\dvbox{
Jésus a dit\frcolon{}
 \Og Le royaume des cieux est semblable à un trésor caché dans un champ.
 L'homme qui l'a trouvé le cache de nouveau;
 et, dans sa joie, il va vendre tout ce qu'il a et achète ce champ \Fg{}
 (\ibibleverse{Mt}(13:44)).
}

Le champ dans cette parabole est le monde, et Jésus est Celui
 qui a tout donné pour acheter le monde. Qui est donc le trésor?
 Aussi incroyable que cela paraisse, le trésor c'est vous et moi
 \ocadr nous qui avons mis notre confiance en Lui comme étant notre Seigneur
 et Sauveur.
 Nous sommes l'épouse de Christ.
 Tout comme Booz était prêt à payer le champ pour obtenir l'épouse,
 Jésus aussi a volontairement payé le prix de la rédemption
 pour acheter le monde.
 Il l'a fait parce qu'Il vous aimait et voulait que vous Lui apparteniez. 

\dvrule

\dvprayer{
Merci, Seigneur, de ce que Tu nous accordes une valeur
 telle que Tu T'es volontairement donné pour nous racheter
 de la captivité et de l'esclavage du péché,
 afin que nous puissions devenir Ton épouse. 
}{\Amen}


%%%%%%%%%%%%%%
% 23 fevrier
%%%%%%%%%%%%%%

\jrnlday{Réponses retardées}

\dvquote{
Elle fit un v\oe{}u et dit\frcolon{}
 \Og Éternel des armées ! Si Ton regard s'arrête sur l'humiliation
 de Ta servante, si Tu Te souviens de moi et n'oublies pas Ta servante,
 et si Tu donnes un gar\c{c}on à Ta servante, je le donnerai à l'Éternel
 pour tous les jours de sa vie, et le rasoir ne passera pas sur sa tête. \Fg{}
}{\ibibleverse{IS}(1:11)}

\lettrine{A}{nne} désirait un fils afin de pouvoir être libérée
 des moqueries incessantes de Peninna, l'autre épouse de son mari.
 Jour après jour, elle suppliait Dieu d'être libérée des tourments
 imposés par cette femme.
 Mais Dieu voulait beaucoup plus. Il désirait un homme pour délivrer
 la nation entière de sa corruption morale.
 Aussi a-t-Il permis à cette agonie quotidienne de former et fa\c{c}onner Anne,
 et a différé la réponse à ses prières jusqu'à ce qu'Il l'amène
 à être en harmonie avec Ses désirs.
 Puis, par son fils Samuel, le réveil spirituel est arrivé.

Peut-être Dieu tarde-t-Il à répondre à certaines de vos prières.
 Peut-être avez-vous tant attendu que vous êtes désespérés.
 Si c'est le cas, il est temps de commencer à intercéder auprès du Seigneur.
 Ce faisant, il se pourrait bien que vous soyez stupéfiés par les changements
 que Dieu va apporter dans vos attitudes alors qu'Il vous forme
 et vous amène à cette harmonie avec Lui.

\dvbox[0.87\textwidth]{
Ne soyez pas découragés quand Dieu tarde à répondre à vos prières.
}

Généralement, Dieu veut faire quelque chose de beaucoup plus grand,
 mais vous ne l'avez pas encore compris.
 Dès que vous vous en rendrez compte, vous verrez la main de Dieu à l'\oe{}uvre. 

\dvrule

\dvprayer{
Père, permets-nous d'être en parfaite harmonie avec Ton but et Tes désirs.\\
 Accomplis en nous les changements nécessaires pour nous amener à cet état
 où Tu peux mettre en \oe{}uvre et accomplir Ta volonté. 
}{\DlNdJ}


%%%%%%%%%%%%%%
% 24 fevrier
%%%%%%%%%%%%%%

\jrnlday{Touchés par Dieu}

\dvquote{
Saül s'en alla aussi chez lui à Guibéa.
 Il fut accompagné par les hommes de valeur, dont Dieu avait touché le c\oe{}ur.
}{\ibibleverse{IS}(10:26)}

\lettrine{Q}{ue} signifie avoir votre c\oe{}ur touché par Dieu ?
 Cela signifie que Dieu est maintenant au centre de votre volonté
 et de votre être.
 Cela signifie que vous Lui avez totalement remis le contrôle de votre vie.

Comment Dieu touche-t-Il votre c\oe{}ur ?
 Il le fait par la subtile influence de Son Esprit Saint. \suggest{ou Esprit-Saint?}
 Le Saint-Esprit nous dirige vers la Parole de Dieu et par les Écritures,
 Dieu touche nos c\oe{}urs.
 Une fois que le Saint-Esprit commence à illuminer nos pensées,
 il est incroyable de voir comment la Bible devient merveilleuse,
 excitante, et glorieuse.

Quand Dieu touche votre c\oe{}ur, amertume, anxiété et peurs doivent disparaître.
 Elles ne peuvent pas rester dans un c\oe{}ur qui a été touché par Dieu.
 Les habitudes, les vices qui détruisaient votre vie se dissipent.
 Les choses que vous aimiez alors, vous les méprisez maintenant.

\dvbox{
Quand Dieu touche votre c\oe{}ur, il le remplit d'amour et de paix.
 Bien que les tempêtes puissent toujours se déchaîner,
 vous avez une confiance glorieuse en la capacité de Dieu
 à contrôler les circonstances de la vie.
}

Jésus S'est entouré d'un groupe d'hommes dont Il avait touché le c\oe{}ur
 \ocadr des hommes qui sont partis retourner le monde de fond en comble.
 Quand Dieu touche votre c\oe{}ur, vous recevez une toute nouvelle perspective.
 Vous cessez de vivre seulement pour aujourd'hui et vous commencez
 à vivre pour l'éternité. 

\dvrule

\dvprayer{
Merci, Père, pour Ton Saint-Esprit qui touche nos c\oe{}urs
 et nous pose des défis par l'intermédiaire de la Parole.
 Touche les c\oe{}urs de Ton peuple aujourd'hui. 
}{\Amen}


%%%%%%%%%%%%%%
% 25 fevrier
%%%%%%%%%%%%%%

\jrnlday{Pas de pitié}

\dvquote{
Ainsi parle l'Éternel des armées\frcolon{}
 \Og Je veux intervenir contre Amalec à cause de ce qu'il a fait à Israël,
 lorsqu'il s'est mis sur son chemin à sa sortie d'Égypte.\\
 Va maintenant, frappe Amalec, et vouez à l'interdit
 tout ce qui lui appartient. \Fg{}
}{\ibibleverse{IS}(15:2-3)}

\lettrine{L}{es Amalécites} étaient probablement le peuple le plus violent,
 le plus vil qui ait jamais existé à la surface de la terre.
 Dans l'intérêt de l'humanité, Dieu ordonna l'extermination de ces gens
 et choisit les enfants d'Israël pour être l'instrument de Son jugement.

Dans l'histoire d'Esther, nous entendons parler de Haman,
 l'homme qui avait comploté la destruction des Juifs.
 Haman était un Amalécite, un descendant du Roi Agag que Saül
 avait laissé vivre.
 Dieu avait dit à Saül de \Og détruire complètement \Fg{} les Amalécites,
 mais Saül n'avait pas obéi.
 Et pour cette raison, Amalec est revenu et a presque détruit tout Israël.

La Bible dit que la chair fait la guerre à l'Esprit et que l'Esprit
 fait la guerre à la chair.
 Comme Amalec cherchait à attaquer Israël à son point faible,
 votre chair sera attaquée à ce point faible.
 C'est une bataille pour la suprématie.

\dvbox{
Dieu a décidé que notre chair doit être mise à mort.
}

Bien que nous voulions la réformer, Dieu dit non.
 Son décret pour votre chair est la destruction complète.

Lequel des deux va dominer votre vie ?
 Cèderez-vous à l'Esprit ou à la chair ?
 Goûterez-vous à la communion avec Dieu ou bien aurez-vous
 de la pitié envers votre chair et deviendrez-vous ainsi
 un esclave de ses appétits, vous éloignant vous-mêmes de Dieu ?
 Choisissez avec sagesse. 

\dvrule

\dvprayer{
Père, fortifie-nous par Ton Esprit afin que nous puissions
 vivre selon l'Esprit, marcher selon l'Esprit
 et avoir les pensées de l'Esprit, qui est vie, paix et joie. 
}{\Amen}


%%%%%%%%%%%%%%
% 26 fevrier
%%%%%%%%%%%%%%

\jrnlday{Vaincre les géants}

\dvquote{
David dit au Philistin\frcolon{}
 \Og Tu marches contre moi avec l'épée, la lance et le javelot ;
 et moi je marche contre toi au nom de l'Éternel des armées,
 du Dieu des troupes d'Israël, que tu as mises au défi. \Fg{}
}{\ibibleverse{IS}(17:45)}

\lettrine{C}{haque jour}, ce géant dénommé Goliath arrivait
 sur le champ de bataille et mettait les troupes d'Israël
 au défi de lui trouver un champion pour l'affronter
 en un combat singulier qui déciderait de l'issue de la guerre.
 Se servant de ses propos, de son armure et de sa taille,
 Goliath cherchait à intimider Israël. \typo{Israël}

Et \c{c}a marchait ! L'armée de Saül répondait à cette confrontation
 avec le géant par la peur et le désarroi.
 Mais David est arrivé sur la scène et a exprimé
 sa grande confiance devant Goliath.

David ne pla\c{c}ait pas sa confiance en lui-même, mais dans le Seigneur.
 Il voyait que la capacité du Seigneur pour le délivrer était
 beaucoup plus forte que celle du géant pour le détruire.
 \Og Ce n'est pas vraiment une lutte entre le géant et moi.
 C'est une lutte entre le géant et le Seigneur.
 Et même s'il est vrai que je n'ai aucune chance de l'emporter contre le géant,
 lui n'en a aucune contre le Seigneur. \Fg{}
 David a vu le conflit comme une opportunité de glorifier Dieu.

\dvbox{
Nous avons besoin d'une juste perspective sur les géants dans nos vies.
}

Nous devons détacher nos regards de ces problèmes pour les porter,
 au contraire, vers le Seigneur.

Nous devons nous rappeler que Dieu est pour nous et qu'Il a mis
 toutes les ressources du ciel à notre disposition.
 Par la puissance du Seigneur, chacun des géants de votre vie peut tomber.
 Mais il faut Lui faire confiance. 

\dvrule

\dvprayer{
Père, nous Te remercions d'être plus grand qu'aucun des géants
 à qui nous pouvons être confrontés.
 Aide-nous à nous rappeler que la bataille ne se gagne pas avec des épées,
 des lances et la sagesse humaine, mais avec la puissance du Dieu éternel. 
}

\missing{Fin de prière?}


%%%%%%%%%%%%%%
% 27 fevrier
%%%%%%%%%%%%%%

\jrnlday{Un pas}

\dvquote{
David fit encore ce serment\frcolon{}
 \Og Ton père sait bien que j'ai obtenu ta faveur et il aura dit\frcolon{}
 Que Jonathan ne le sache pas ;\\
 cela lui ferait de la peine.
 Mais, aussi vrai que l'Éternel est vivant et que tu es vivant,
 il n'y a qu'un pas entre moi et la mort! \Fg{}
}{\ibibleverse{IS}(20:3)}

\lettrine{D}{avid} venait d'échapper à la mort.
 Saül avait essayé de le tuer, et David se rendait compte
 qu'il pouvait perdre la vie à n'importe quel moment.

Quand on l'a ainsi échappé belle, on prend conscience de combien
 on est proche de la mort.
 Peu importe que je sois en bonne santé ou que je sois fort,
 il n'y a qu'un pas entre moi et la mort.

Si la mort est si inévitable et si proche, comment devrais-je donc vivre ?
 Premièrement, je ne devrais pas vivre pour cette vie seulement.

\dvbox{
Je devrais vivre en étant constamment prêt pour l'éternité.
}

La Bible enseigne que le vrai moi est esprit, non pas corps.
 Le corps n'est qu'une tente. Mais un jour je vais quitter cette tente
 pour emménager dans une belle demeure.
 Si je vis dans la communion avec Dieu par Jésus-Christ,
 je n'aurais pas à faire ce pas entre la vie et la mort.
 D'après les paroles de Jésus, si je vis et crois en Lui, je ne mourrai jamais
 \ocadr mais je déménagerai.
 Je quitterai cette tente pour emménager dans ce bâtiment de Dieu.
 Il y a donc un pas qui me sépare de l'éternité.

J'attends ce déménagement avec impatience.
 Car voyez-vous, bien que je ne L'ai pas encore vu,
 je L'aime et je me réjouis d'une joie glorieuse,
 inexprimable dans l'anticipation de ce jour où je me tiendrai
 à côté de mon Seigneur, dans ce nouveau corps, dans Son royaume éternel. 

\dvrule

\dvprayer{
Merci Père, pour le don de la vie éternelle en Jésus-Christ. 
}{\Amen}


%%%%%%%%%%%%%%
% 28 fevrier
%%%%%%%%%%%%%%

\jrnlday{Un bon mécontentement}

\dvquote{
Tous ceux qui se trouvaient dans la détresse,
 qui avaient des créanciers ou qui étaient mécontents,
 se rassemblèrent auprès de lui, et il devint leur chef.
 Il y eut avec lui environ quatre cents hommes.
}{\ibibleverse{IS}(22:2)}

\lettrine{D}{ieu} a oint David pour être roi.
 Cependant, Saül est toujours sur le trône.
 Et parce que Saül désire s'accrocher à ce qui n'est plus à lui,
 et qu'il essaye de tuer David, David est devenu un fugitif.

Ceux qui sont venus rejoindre David sont une bande de marginaux
 laissés-pour-compte \ocadr candidats improbables pour devenir
 les hommes forts de David.
 Cependant, Dieu les a fait mûrir et s'est servi d'eux
 et de David pour établir le royaume d'Israël.

Il y a ici un parallèle intéressant à dresser.
 Dieu a ordonné Son Fils, Jésus-Christ, pour qu'Il règne comme Roi sur la terre;
 cependant, Satan est encore sur le trône.
 Et Satan fait de son mieux pour s'accrocher au royaume.
 Jésus assemble autour de Lui des hommes
 \ocadr candidats improbables, \c{c}a c'est sûr \fcadr
 mais des hommes par l'intermédiaire desquels Il entend établir Son royaume
 et faire arriver Son règne sur la terre.

\dvbox{
Dieu recherche des hommes et des femmes qui~ont le c\oe{}ur ouvert et disponible,
 ceux qui veulent Lui dire\frcolon{}
 \Og Me voici, Seigneur. Je ne suis pas satisfait de ma vie telle qu'elle est,
 Seigneur. Je veux Te la donner complètement. \Fg{}
}

Quand le mécontentement vous amène à un engagement total envers Jésus-Christ,
 c'est un bon mécontentement parce qu'il conduit au progrès. 

\dvrule

\dvprayer{
Père, nous Te remercions d'assembler autour de Toi ceux que Tu as choisis.
 Forme-nous et fa\c{c}onne-nous pour être ces personnes que Tu veux que nous soyons
 afin de pouvoir un jour régner avec Toi dans la gloire de Ton royaume. 
}{\DlNdJ}


%%%%%%%%%%%%%%
% 29 fevrier
%%%%%%%%%%%%%%

\jrnlday{L'insensé}

\dvquote{
Saül dit\frcolon{}
 \Og J'ai péché\dots{}
 J'ai agi comme un insensé, et j'ai commis une grande erreur. \Fg{}
}{\ibibleverse{IS}(26:21)}

\lettrine{J}{aloux} de la popularité de David, Saül cherchait à le tuer.
 Une nuit, David réussit à s'introduire furtivement dans le camp de Saül
 pendant qu'il dormait. Au lieu de le tuer, David dit à un de ses compagnons\frcolon{}
 \Og Cet homme a été oint par Dieu. Ne le touche pas.
 Si Dieu veut s'occuper de lui, c'est Son affaire. \Fg{}
 David se contenta donc de saisir sa cruche d'eau et sa lance.

Une fois assez éloigné et en sécurité, David interpella
 le garde du corps de Saül.
 Saül se réveilla et dit\frcolon{}
 \Og Est-ce bien ta voix, mon fils David ? \Fg{}
 David répondit\frcolon{}
 \Og Pourquoi me poursuis-tu ? Regarde, j'ai ta lance.
 Les hommes qui étaient avec moi voulaient te tuer,
 mais je ne les a pas laissés faire. \Fg{}
 C'est alors que Saul déclara\frcolon{}
 \Og J'ai agi comme un insensé, et j'ai commis une grande erreur. \Fg{}
 \ibiblephantom{IS}(26:17-21)

Saül avait beaucoup d'atouts naturels dans la vie,
 mais \c{c}a n'a pas garanti son succès.
 La vie de Saül nous enseigne qu'il est possible
 de gaspiller les bénédictions divines.
 Nous apprenons qu'un homme agit comme un insensé quand,
 comme Saül, il essaye de se dérober à l'appel de Dieu sur sa vie,
 ou qu'il s'attribue le mérite des victoires d'un autre,
 ou qu'il fait des promesses irréfléchies.
 Un homme agit en insensé quand il n'obéit pas complètement à Dieu,
 ou propose des excuses outrancières à son échec,
 ou devient jaloux d'amis loyaux consacrés à Dieu
 ou recherche la direction de la part d'être spirituels incertains.

\dvbox{
La véritable folie de la vie de Saül a été de ne pas arriver
 à abandonner totalement sa vie à Dieu.
}

Quand vous n'arrivez pas à abandonner, c'est qu'en réalité vous dites\frcolon{}
 \Og Je suis plus sage que Dieu. \Fg{}
 Et seul un fou pourrait penser cela. 

\dvrule

\dvprayer{
Père, nous prions pour que tu nous aides à nous abandonner complètement à Toi.
}{\DlNdJ}





