\dvmonth{Février}


% 1er fevrier

\dvday{Les Exigences du Seigneur}

\dvquote{
Maintenant, Israël, que demande de toi l'Éternel, ton Dieu, si ce n'est que
 tu craignes l'Éternel, ton Dieu, afin de marcher dans toutes Ses voies, d'aimer
 et de servir l'Éternel, ton Dieu, de tout ton cœur et de toute ton âme~;
 si ce n'est que tu observes les commandements de l'Éternel et Ses prescriptions
 que je te donne aujourd'hui,
 afin que tu sois heureux ?
}{\bibleverse{Dt}(10:12-13)}

\dvlettrine{Q}{ue} demande de nous l'Éternel?
 Ce passage rend les exigences de Dieu très claires.

Tout d'abord, Dieu voulait que les enfants d'Israël voient ce qui était dans
 leur cœurs. Face à l'épreuve, Lui obéiraient-ils ?
 Lui feraient-ils confiance ? L'adoreraient-t-ils? 

\dvbox{
Dieu se sert des épreuves pour nous montrer nos cœurs,
 afin de pouvoir nous purifier et nous faire ensuite entrer dans le pays
 de la bénédiction --- physiquement et spirituellement. 
}

Deuxièmement, les Israélites devaient savoir que la vie est plus qu'une
 expérience physique. Dieu veut que nous vivions dans le domaine spirituel,
 en marchant selon l'Esprit et en étant guidés par l'Esprit.

Troisièmement, ils devaient savoir que Dieu punit les Siens, pour nous empêcher
 de faire des choses qui nous détruiraient.

Dieu se sert des épreuves de la même façon aujourd'hui.
 Il se sert des épreuves pour nous montrer nos cœurs,
 afin de pouvoir nous purifier et nous faire ensuite entrer dans le pays
 de la bénédiction. 

\dvrule

\dvprayer{
Père, puisse Ton Esprit Saint défier nos cœurs avec la vérité de Ta Parole.
 Aide-nous à renverser toutes les idoles et à mettre Christ sur le trône
 de nos vies. 
}{\Amen}


% 2 fevrier

\dvday{Le Prophète Promis}

\dvquote{
Je leur susciterai du milieu de leurs frères un Prophète comme toi, je mettrai
 Mes paroles dans Sa bouche, et Il leur dira tout ce que Je Lui commanderai.
 Et si quelqu'un n'écoute pas Mes paroles qu'Il dira en Mon nom, c'est Moi
 qui lui en demanderai compte.
}{\bibleverse{Dt}(18:18-19)}


\dvlettrine{J}{usqu'à} ce jour, les Juifs reconnaissent qu'il s'agit bien là
 d'une prophétie concernant le Messie.
 Tout comme Moïse était un médiateur qui a transmis la Parole de Dieu au peuple,
 le Prophète promis serait le médiateur qui transmettrait la Parole de Dieu
 au peuple.
 Un médiateur est nécessaire en raison du fait que Dieu est infini et que
 l'homme est fini.
 L'homme ne peut pas comprendre ou appréhender Dieu qui est infini. 


\dvbox{
Parce que je suis rempli de péché et d'injustice,
 je ne peux m'approcher de Dieu qui est lui absolument saint et pur.
 Aussi ai-je besoin d'un médiateur. 
}


Cette prophétie du Deutéronome a été accomplie en Jésus-Christ.
 Paul nous dit dans \bibleverse{ITim} qu'il y a un seul Dieu
 et un seul Médiateur entre Dieu et les hommes,
 et que c'est l'homme Jésus-Christ.
 Il peut toucher Dieu parce qu'Il est Dieu.
 Il peut me toucher parce qu'Il s'est fait homme.
 Et maintenant par son intermédiaire, je peux toucher Dieu.

Dieu a tenu Sa promesse, Jésus-Christ est ce Prophète, ce Médiateur.
 Il est notre réconfort. Il est la Parole par laquelle nous avons la vie,
 la force, l'amour, le pardon, l'espérance et la paix. 

\dvrule

\dvprayer{
Père, merci d'avoir envoyé Ton Fils pour offrir le pardon qu'il nous faut.
 Puissions-nous marcher dans la lumière comme jésus est dans la lumière,
 afin que nous puissions avoir la communion avec Toi. 
}{En Son Nom, nous prions, \Amen}




