\jrnlmonth

%%%%%%%%%%%%%%%%
% 1er septembre 
% M 2012/09/09
%%%%%%%%%%%%%%%%

\jrnlday{Chants dans la nuit}

\themeindex{priere@prière}
\themeindex{louange}
\themeindex{chant}

\dvquote{
Vers le milieu de la nuit, Paul et Silas priaient et chantaient
 les louanges de Dieu, et les prisonniers les écoutaient.
}{\ibibleverse{Ac}(16:25)}

\lettrine{A}{lors} qu'ils se trouvaient à Philippes,
 Paul et Silas furent dépouillés de leurs vêtements,
 battus et mis aux fers au fin fond de la prison.
 Pouvez-vous imaginer une situation plus désastreuse et désespérée?

Que feriez-vous dans de telles circonstances? À quoi penseriez-vous?
 Tout ce que vous vouliez, c'était servir Dieu.
 Vous aviez ressenti l'appel à amener l'Évangile en Macédoine,
 et maintenant votre dos est en proie
 à des douleurs lancinantes;
 vos mains, vos pieds, votre cou sont entravés dans des chaînes.
 Votre futur est dans une incertitude absolute.
 Vous ne connaissez pas la gravité des chefs d'accusation
 contre vous et vous ne savez pas combien de temps
 on va vous garder en prison.

Dans cette situation très inconfortable, Paul et Silas
 ont fait ce qui leur paraissait comme la seule chose sensée\frcolon{}
 ils se sont mis à prier et entonner des chants de louange à Dieu.
 Bien que leurs corps soient enchaînés, leurs esprits étaient libres.
 Au lieu de gémir et de se plaindre, ils ont fait de leur sombre cachot
 une maison de louange. Se réjouir ainsi leur a remonté le moral.
 Plus Paul et Silas se concentraient sur le Seigneur,
 et plus leurs problèmes rapetissaient.

\dvbox{
Chanter détourne nos pensées de nous-mêmes
 et les concentre sur le Seigneur.
}

La prochaine fois que vous vous retrouvez enchaînés par des soucis,
 de la confusion ou de la douleur, levez les yeux.
 Détachez vos regards de votre problème et portez-les sur Dieu.
 Chantez à votre Créateur \ocadr et regardez votre sombre prison
 devenir une maison de louange.

\dvrule

\dvprayer{
Père, nous Te remercions pour les chants que Tu mets dans nos c\oe{}urs
 \ocadr chants de joie, de bénédiction et de réjouissance en Ta bonté.
}{\DlNdJ}


%%%%%%%%%%%%%%%%
% 2 septembre 
% M 2012/09/09
%%%%%%%%%%%%%%%%

\jrnlday{Le remède à la peur}

\themeindex{peur}
\themeindex{reconfort@réconfort}
\themeindex{Dieu!presence@présence de \sim}

\dvquote{
Le Seigneur dit à Paul en vision pendant la nuit\frcolon{}
 Sois sans crainte, mais parle et ne te tais pas, car Moi,
 Je suis avec toi, et personne ne mettra la main sur toi
 pour te faire du mal\frcolon{} parce que J'ai un peuple nombreux
 dans cette ville.
}{\ibibleverse{Ac}(18:9-10)}

\lettrine{P}{artout} où Paul prêchait Jésus-Christ,
 les gens se soulevaient contre lui.
 Il avait été battu, emprisonné et lapidé.
 À Thessalonique, il avait dû quitter furtivement la ville de nuit.
 Les habitants de Bérée l'avaient chassé de leur ville.
 Mais ces endroits étaient encore assez repectueux des lois.
 Corinthe par contre \ocadr la ville où Paul se trouvait alors \fcadr{} était
 une ville pleine de violence et d'immoralité.
 Paul était sûrement anxieux.
 Il se faisait sûrement du souci au sujet des Juifs qui détenaient
 son sort entre leurs mains. C'est dans cet état de peur que Jésus
 a parlé à Paul dans une vision.
 \Og Sois sans crainte [\dots{}] car Je suis avec toi. \Fg{}

\dvbox{
La réalisation de la présence de Dieu \ocadr c'est en cela
 que se trouve le remède contre la peur.
}

Paul a écrit aux Romains\frcolon{} \punct{deux-points}
 \Og Si Dieu est pour nous, qui sera contre nous? \Fg{} (\ibibleverse{Rm}(8:31)),
 David a dit\frcolon{} \punct{deux-points}
 \Og L'Éternel est pour moi, je ne crains rien\frcolon{}
 Que peuvent me faire des hommes \Fg{} (\ibibleverse{Ps}(118:6)).
 Le Seigneur a dit au prophète Ésaïe\frcolon{} \punct{deux-points}
 \Og Sois sans crainte, car Je suis avec toi; n'ouvre pas des yeux inquiets,
 car Je suis ton Dieu ; Je te fortifie, Je viens à ton secours,
 Je te soutiens de Ma droite victorieuse \Fg{} (\ibibleverse{Is}(41:10)).

Traversez-vous des moments d'incertitude? Vous inquiétez-vous pour l'avenir?
 Souvenez-vous de ces promesses. Souvenez-vous de Celui qui a dit\frcolon{} \punct{deux-points}
 \Og Je ne te délaisserai jamais, je ne t'abandonnerai jamais \Fg{}
 (\ibibleverse{He}(13:5)).
 Vous ne faites jamais face à rien tout seul. Votre Dieu est avec vous.

\dvrule

\dvprayer{
Père, nous Te remercions pour le réconfort de Ta présence
 avec nous au milieu des nuits obscures, au milieu des heures
 de découragement et dans les moments de peur.
}{\Amen}


%%%%%%%%%%%%%%%%
% 3 septembre 
% M 2012/09/09
%%%%%%%%%%%%%%%%

\jrnlday{Acceptation}

\themeindex{Dieu!volonté de \sim}
\themeindex{soumission}
\themeindex{obeissance@obéissance}
\themeindex{souffrance}
\themeindex{paix}

\dvquote{
Comme il ne se laissait pas persuader,
 nous n'avons plus insisté et nous avons dit\frcolon{}
 Que la volonté du Seigneur se fasse!
}{\ibibleverse{Ac}(21:14)}

\lettrine{P}{artout} où Paul allait, le Saint-Esprit l'avertissait
 que des souffrances, des liens et des emprisonnements l'attendaient
 à Jérusalem. Ayant entendu ces prophéties, les amis de Paul pleurèrent
 et le supplièrent de ne pas s'y rendre. Mais il leur répondit\frcolon{}
 \Og Moi, je suis prêt, non seulement à être lié,
 mais encore à mourir à Jérusalem pour le nom du Seigneur Jésus \Fg{}
 (\ibibleverse{Ac}(21:13)). Voyant sa détermination, ils dirent\frcolon{} \punct{deux-points}
 \Og Que la volonté du Seigneur soit faite. \Fg{}

\dvbox{
L'acceptation de la volonté de Dieu est le seul chemin qui mène
 à la paix véritable. Sans cette acceptation, il n'y a qu'agitation
 et luttes intérieures.
}

Quelquefois, les gens sont réticents à se soumettre à la volonté de Dieu,
 parce qu'ils ont peur qu'Il les force à faire quelque chose
 qu'ils ne veulent pas faire. Mais ce n'est pas ainsi que Dieu agit.
 Dieu révèle magnifiquement Sa volonté en alignant les désirs de notre
 c\oe{}ur avec les Siens.

Les plans de Dieu pour nos vies sont bien supérieurs à tout
 ce que nous pourrions jamais imaginer par nous-mêmes.
 Il voit au-delà de nos épreuves le bien éternel qu'Il va accomplir
 par ces épreuves. Pour Dieu, la fin justifie les moyens.
 Si quelques épreuves et souffrances présentes vont accomplir
 une bonne fin dans nos vies, alors Dieu va permettre
 ces souffrances momentanées.

Comme David le disait\frcolon{} \punct{deux-points}
 \Og Je prends plaisir à faire Ta volonté, mon Dieu! \Fg{}
 (\ibibleverse{Ps}(40:8)). Quand nous arrivons vraiment à connaître Dieu,
 Sa volonté devient notre volonté; ce qui Lui fait plaisir
 devient ce qui nous fait plaisir.

\dvrule

\dvprayer{
Père, nous avons tendance à rechercher le chemin plus facile,
 le chemin présentant le moins de souffrances.
 Mais Tu sais ce qui est le mieux pour nous.
 Nous T'offrons nos désirs pour que Tu les raffines
 et que Tu en fasses les Tiens.
}{\DlNdJ}


%%%%%%%%%%%%%%%%
% 4 septembre 
% M 2012/09/09
%%%%%%%%%%%%%%%%

\jrnlday{Connaître la volonté de~Dieu}

\themeindex{Dieu!volonté de \sim}

\dvquote{
Il dit\frcolon{} \Og Le Dieu de nos pères t'a destiné
 à connaître Sa volonté\dots{} \Fg{}
}{\ibibleverse{Ac}(22:14)}

\lettrine{À}{ Jérusalem}, Paul a raconté aux Juifs sa conversion
 sur la route de Damas, comment il est devenu momentanément aveugle
 en raison d'une brillante lumière, et comment il a recouvré la vue
 quand Ananias est venu lui imposer les mains.
 Ananias a dit alors à Paul que Dieu l'avait choisi afin qu'il découvre
 Sa Dieu.

Ceci s'applique aussi bien à vous et à moi.
 Parce qu'Il vous aime, Dieu vous a choisi afin que vous découvriez Sa volonté.

Savez-vous ce que Dieu a prévu et planifié pour vous?
 Ceci devrait avoir une grande importance pour vous,
 parce que tout ce que vous faites en dehors de la volonté de Dieu s'évaporera en fumée.

\dvbox{
Connaissez-vous la volonté de Dieu pour votre vie?
}

Nous pouvons devenir si absorbés à juste survivre dans notre existence
 quotidienne, que nous ne prenons pas l'éternité en considération.
 Mais vous n'avez qu'une vie à vivre, qui sera bientôt passée
 et seulement ce que vous faites pour Christ demeurera.

Comment pouvez-vous connaître la volonté de Dieu pour votre vie
 de fa\c{c}on spécifique?
 Vous découvrez le dessein spécifique et individuel
 de Dieu pour vous en étant \Og transformés par le renouvellement
 de votre intelligence, afin que vous discerniez quelle est la volonté de Dieu\frcolon{}
 ce qui est bon, agréable et parfait \Fg{} (\ibibleverse{Rm}(12:2)).
 Quand vous offrez votre corps à Dieu comme un instrument désireux
 d'être utilisé pour Son \oe{}uvre, quand vous désirez les plans de Dieu
 au-dessus des vôtres, alors votre vie va devenir une manifestation
 progressive de la volonté de Dieu.


\dvrule

\dvprayer{
Père, merci de nous avoir choisis pour passer l'éternité avec Toi.
 Aide-nous à voir nos vies avec l'éternité en vue et à toujours chercher
 à faire Ta volonté.
}{\Amen}


%%%%%%%%%%%%%%%%
% 5 septembre 
% M 2012/09/10
%%%%%%%%%%%%%%%%

\jrnlday{Dans la tempête}

\themeindex{epreuve@épreuve}
\themeindex{Dieu!desseins de \sim}

\dvquote{
Mais bientôt après, venant de l'île, un vent de tempête
 appelé Euraquilon se déchaîna.
}{\ibibleverse{Ac}(27:14)}

\lettrine{B}{ien} que ce ne soit pas du tout de sa faute,
 Paul s'est retrouvé en plein milieu d'une terrible tempête.
 Il avait même prévenu le capitaine du navire de ne pas appareiller.
 Mais le capitaine avait refusé de l'écouter.

Parfois, nous pensons à tort que, parce que nous servons le Seigneur,
 nous devrions bénéficier d'un temps au beau fixe pour toute la durée du voyage.
 À coup sûr, le Seigneur va apaiser les flots pour nous,
 et envoyer une légère brise gonfler nos voiles. Pas du tout!
 Jésus ne promets pas de vous épargner les tempêtes, par contre,
 Il a promis d'être avec vous dans la tempête.

\dvbox{
Les tempêtes servent à quelque chose \ocadr elles servent le dessein de Dieu.
}

Quand ces tempêtes éclatent, nous nous demandons si nous allons survivre.
 Paul s'est probablement posé la même question.
 Mais le Seigneur, présent à ses côtés pendant la tempête,
 l'a encouragé par Ses paroles.
 Il a dit à Paul qu'il allait survivre,
 parce que Dieu avait une mission pour lui.

Le but réel de la tempête n'allait être révélé à Paul que beaucoup plus tard.
 Et c'est souvent le cas dans nos vies. Quand nos mers se déchaînent,
 nous questionnons nos chances de survie.
 C'est dans ces moments-là que nous devons nous souvenir des paroles du Seigneur\frcolon{}
 \Og Sois sans crainte \Fg{} (\ibibleverse{Ac}(27:24)).
 En d'autres termes\frcolon{} \punct{deux-points}
 \Og Courage. Ce n'est pas la fin. J'ai un plan pour toi. \Fg{}

Il a vraiment un plan. Il ne vous a pas oubliés \ocadr en fait,
 Il est juste là, tout à côté de vous, et c'est Lui qui vous aide à surnager.
 Puis, quand la tempête sera passée et que le manteau nuageux s'ouvrira de nouveau,
 vous verrez la raison de la tempête.

\dvrule

\dvprayer{
Père, nous Te remercions de ce que Tu es avec nous dans la tempête.
 Car Tu as promis que Tu ne nous délaisserais jamais
 et que Tu ne nous abandonnerais jamais.
 Utilise ces tempêtes pour Tes desseins et pour Ta gloire.
}{\DlNdJ}


%%%%%%%%%%%%%%%%
% 6 septembre 
% M 2012/09/10
%%%%%%%%%%%%%%%%

\jrnlday{La bonne nouvelle venue~de~Dieu}

\index{Bonne Nouvelle|see{Évangile}}
\themeindex{Evangile@Évangile}
\themeindex{salut}
\themeindex{verite@vérité}
\themeindex{justice}
\themeindex{peche@péché}
\themeindex{oeuvres@\oe{}uvres}
\themeindex{foi}
\themeindex{loi!\sim~de Moïse}

\dvquote{
Car je n'ai pas honte de l'Évangile\frcolon{}
 c'est une puissance de Dieu pour le salut de quiconque croit,
 du Juif premièrement, puis du Grec.
}{\ibibleverse{Rm}(1:16)}

\lettrine{D}{ans} sa jeunesse, Paul était enfermé
 dans un système religieux où l'homme cherchait à être justifié\NdT{c'est-à-dire \Og déclaré juste, déclaré droit \Fg{}.}
 devant Dieu en respectant la Loi\NdT{essentiellement
 les Dix Commandements.}.
 Mais la Loi n'avait jamais été prévue pour rendre
 l'homme juste devant Dieu. La Loi était destinée à montrer
 à l'homme quel grand pécheur il était, et à faire que le monde entier
 se sache coupable devant Dieu. Aussi lorsque Paul est parvenu
 à la connaissance de la vérité \ocadr quand il a rencontré Jésus-Christ
 et trouvé le salut fondé sur Sa justice et non sur la justice
 de l'homme \fcadr{} Paul a allègrement rejeté les fausses notions
 de bonnes \oe{}uvres, et il est devenu impatient de faire part
 de cette bonne nouvelle à tout le monde.

\dvbox{
L'Évangile libère les hommes.
}

\Og C'est la puissance de Dieu pour le salut. \Fg{}
 Quelle joie que de voir des vies transformées par cette bonne nouvelle.
 Elle fait briller la lumière dans l'obscurité et brise les liens
 qui asservissent au péché.

Cette puissance, cette bonne nouvelle du salut, n'est pas réservée
 exclusivement aux Juifs mais elle est destinée à toute personne qui croit.
 C'est pour le monde entier.

Oh, quel message glorieux d'espoir et de salut
 nous avons re\c{c}u!  Puissions-nous ne pas garder cette nouvelle
 pour nous-mêmes; puissions-nous ne jamais avoir honte de la vérité
 que nous avons re\c{c}ue. Puissions-nous, comme Paul, être prêts à proclamer
 l'Évangile \suggest{Évangile} de Jésus-Christ au monde en grand besoin
 dans lequel nous vivons.

\dvrule

\dvprayer{
Père, nous Te remercions pour ce glorieux Évangile
 par lequel nous avons été nettoyés de nos péchés.
 Nous prions pour ceux qui ne connaissent pas encore
 cette merveilleuse bonne nouvelle de Ton amour
 et de Ton offre de pardon.
}{\DlNdJ}


%%%%%%%%%%%%%%%%
% 7 septembre 
% M 2012/09/10
%%%%%%%%%%%%%%%%

\jrnlday{Une affaire de~c\oe{}ur}

\themeindex{circoncision}
\themeindex{chair}
\themeindex{coeur@c\oe{}ur}
\themeindex{traditions}
\themeindex{bapteme@baptème}
\themeindex{purete@pureté}
\themeindex{loi!\sim~de Moïse}

\dvquote{
Le Juif, ce n'est pas celui qui en a les apparences ;
 et la circoncision, ce n'est pas celle qui est apparente dans la chair.
 Mais le Juif, c'est celui qui l'est intérieurement ;
 et la circoncision, c'est celle du c\oe{}ur, selon l'esprit
 et non selon la lettre. La louange de ce Juif ne vient pas des hommes,
 mais de Dieu.
}{\ibibleverse{Rm}(2:28-29)}

\lettrine{V}{ous} pouvez dire toutes les choses qu'il faut dire
 et pourtant être loin de Dieu. C'est parce que Dieu s'intéresse au c\oe{}ur.

Extérieurement, les Juifs respectaient la Loi\NdT{essentiellement, les Dix Commandements.}.
 Mais intérieurement, ils la violaient.
 Dieu a abordé cette contradiction par l'intermédiaire
 du prophète Ésaïe, qui leur a dit\frcolon{} \punct{deux-points}
 \Og Ainsi quand ce peuple s'approche, il me glorifie de la bouche
 et des lèvres ; mais son c\oe{}ur est éloigné de moi, et la crainte
 qu'il a de moi n'est qu'un commandement de tradition humaine \Fg{}
 (\ibibleverse{Is}(29:13)). \punct{Point}

\dvbox{
Si vous continuez a vivre pour la chair, aucun rite ne peut vous sauver.
}

Ce que Paul dit aux Juifs, c'est que la circoncision n'a aucune valeur
 si vous continuez à vivre pour la chair\NdT{la vieille nature pécheresse.}.

Ceci est vrai aussi dans l'Église. \suggest{Église}
 Certains font confiance au rite du baptême pour leur salut,
 plutôt qu'à une relation vivante avec Jésus.
 Mais le baptême n'est qu'un symbole qui représente l'enterrement
 de la vieille vie et la naissance à la nouvelle vie.

Vous pouvez même aller à l'église, vous pouvez chanter tous les chants,
 vous pouvez connaître la Parole et dire un \Og Amen \Fg{} de temps en temps.
 Mais rien de tout \c{c}a ne fait de vous un enfant de Dieu.
 Il voit votre c\oe{}ur. Il sait si Lui et vous vivez et appréciez
 la relation qui unit un parent à un enfant.

\dvrule

\dvprayer{
Père, aide-nous à avoir un c\oe{}ur pur et à
 T'aimer de tout notre c\oe{}ur.
}{\DlNdJ}


%%%%%%%%%%%%%%%%
% 8 septembre 
% M 2012/09/10
%%%%%%%%%%%%%%%%

\jrnlday{Les clés de la foi}

\themeindex{foi}
\themeindex{Dieu!promesses de \sim}

\dvquote{
Et, sans faiblir dans la foi, il considéra son corps presque mourant,
 puisqu'il avait près de cent ans, et le sein maternel de Sara
 déjà atteint par la mort.
 Mais face à la promesse de Dieu il ne douta point, par incrédulité,
 mais fortifié par la foi, il donna gloire à Dieu pleinement
 convaincu de ceci\frcolon{} ce que Dieu a promis, il a aussi la puissance
 de l'accomplir.
}{\ibibleverse{Rm}(4:19-21)}

\lettrine{C}{haque} fois que les rédacteurs de la Bible ont voulu
 donner un exemple classique d'un homme de foi, ils ont choisi Abraham.
 En fait, Abraham est même appelé \Og le père de la foi \Fg{}.

La première clé de la foi d'Abraham, c'est qu'il n'a pas considéré
 son âge ou celui de Sarah comme un obstacle pour Dieu.
 C'est-à-dire qu'il n'a pas restreint la capacité de Dieu
 à dépasser les limitations humaines.
 Par contre, c'est souvent la première chose que nous faisons.
 Souvent, nous projetons nos propres limitations sur Dieu.

Deuxièmement, il n'a douté d'aucune des promesses de Dieu.
 Voici une bonne devise\frcolon{} \Og Si Dieu l'a dit alors je le crois. \Fg{}
 La Bible est si riche de promesses que vous pouvez toujours trouver
 celle qui correspond exactement à votre besoin présent.
 Dieu est prêt à parer à toute éventualité et vous pouvez
 faire confiance à ce qu'Il dit.

Troisièmement, Abraham a rendu gloire à Dieu.
 Il a loué Dieu avant même de voir des preuves que Dieu
 allait tenir Sa promesse. Pourquoi? Parce que la foi croit
 que Dieu est capable d'accomplir quoi que ce soit qu'Il ait promis.

\dvbox{
La foi croit et se réjouit à l'avance dans l'attente.
}

Bien qu'il ait fallu un miracle pour cela, Dieu a bien tenu Sa parole
 envers Abraham. Et Il tiendra Sa parole envers vous.
 Un jour, vous verrez l'accomplissement de Ses promesses
 si seulement vous utilisez les clés de la foi.

\dvrule

\dvprayer{
Père, nous Te remercions car ce que Tu as promis,
 Tu es capable de l'accomplir. Aide-nous, Seigneur
 pour que nous puissions vraiment fonder nos vies sur Tes promesses.
}{\Amen}


%%%%%%%%%%%%%%%%
% 9 septembre 
% M 2012/09/10
%%%%%%%%%%%%%%%%

\jrnlday{Conscient de la grâce}

\themeindex{louange}
\themeindex{Dieu!grace@grâce de \sim}
\themeindex{Dieu!amour de \sim}
\index{amour!\sim~de Dieu|see{Dieu}}
\themeindex{peche@péché}
\themeindex{salut}
\themeindex{loi!\sim~de Moïse}

\dvquote{
Or, la loi est intervenue pour que la faute soit amplifiée;
 mais là où le péché s'est amplifié, la grâce a surabondé\dots{}
}{\ibibleverse{Rm}(5:20)}

\lettrine{P}{arfois}, quand dans ma vie je fais l'expérience
 d'un échec lamentable, je pense\frcolon{} \punct{deux-points, majuscule}
 \Og Quand vais-je enfin apprendre? J'ai encore tout gâché. \Fg{}
 Mais c'est souvent dans ces situations de dépit et de condamnation
 de moi-même que Dieu choisit de faire quelque chose d'exceptionnellement
 glorieux pour moi. Conscient de Sa grâce et de Sa miséricorde,
 je ne peux alors m'empêcher de répondre\frcolon{} \punct{deux-points}
 \Og Oh Seigneur, je T'aime! \Fg{}

\dvbox{
La vraie louange s'élève spontanément du c\oe{}ur
 pour reconnaître la bonté et la grâce de Dieu.
}

Enregistrez bien le fait étonnant que Dieu vous aime et a offert
 le pardon pour vos péchés. Méditez à nouveau sur la vérité
 que le salut ne vient pas de votre propre vertu,
 de votre propre capacité à respecter la Loi,
 ou en méritant la faveur de Dieu mais simplement en croyant en Son Fils,
 qui est venu porter votre culpabilité et vous amener dans la famille de Dieu.

En raison du sacrifice obéissant de Jésus, vous pouvez avoir la victoire
 sur votre chair. Grâce à Jésus, vous n'avez pas à redouter la mort.
 Vous avez été libérés de l'esclavage du péché. Vous avez été rachetés.
 Vous avez le Saint-Esprit \ocadr un Protecteur, un Guide, un Éducateur,
 un Consolateur. Vous avez le corps de Christ. Vous avez un Père.
 Vous avez la promesse d'une demeure éternelle.
 Et tout cela vient de Sa grâce.

À la lumière de tout cela, quelle est votre réponse aujourd'hui?
 N'avez-vous pas envie de Le louer?

\dvrule

\dvprayer{
Père, nous sommes émerveillés par Ton pardon, Ton amour,
 Ta miséricorde et Ta grâce. Nous Te remercions de ce que même
 quand nous échouons lamentablement, Tu es là pour nous relever,
 nous laver et nous aider à repartir. Nous T'aimons, Seigneur.
}{\DlNdJ}


%%%%%%%%%%%%%%%%
% 10 septembre 
% M 2012/09/10
%%%%%%%%%%%%%%%%

\jrnlday{Qui me délivrera?}

\themeindex{delivrance@délivrance}
\themeindex{chair}
\themeindex{combat spirituel}
\themeindex{coeur@c\oe{}ur}

\dvquote{
Je ne fais pas le bien que je veux,
 mais je pratique le mal que je ne veux pas.
 Malheureux que je suis !
 Qui me délivrera de ce corps de mort ?
}{\ibibleverse{Rm}(7:19,24)}

\lettrine{N}{ous} avons tous vécu le combat décrit par Paul dans ce passage.
 Dans nos c\oe{}urs nous savons ce qui est bien.
 Nous savons exactement ce que nous sommes censés faire
 et exactement ce que nous sommes censés ne pas faire.
 Et la plupart du temps nous faisons tous nos efforts pour faire
 ce qui est juste, mais c'est alors que nous découvrons qu'une loi perverse
 est à l'\oe{}uvre contre nous dans notre chair,
 et en fin de compte nous ne faisons pas les choses
 que nous savons devoir faire.
 Dans nos pensées, nous servons Dieu,
 mais notre chair continue de se montrer déficiente.

Expliquant ce dilemme, Paul déclare\frcolon{} \punct{deux-points}
 \Og Malheureux que je suis! Qui me délivrera? \Fg{}
 Et dans ce cri nous entrevoyons la réponse. La réponse n'est pas en nous.
 Nous ne pouvons pas y arriver. Dans le même chapitre, \typo{chapitre sans accent}
 Paul dit\frcolon{} \punct{deux-points}
 \Og En moi (c'est-à-dire dans ma chair) ne demeure aucune bonne chose \Fg{}
 (\ibiblevs{Rm}(7:18)).
 Nous devons chercher de l'aide en dehors de nous-mêmes.

Paul répond alors à sa propre question\frcolon{} \punct{deux-points}
 \Og Grâces soient rendues à Dieu par Jésus-Christ \Fg{}
 (\ibiblevs{Rm}(7:25)).

Reconnaissant la faiblesse de notre chair, Dieu a fourni la réponse
 pour nous \ocadr pour vous. Vous ne pouvez pas obéir par votre propre force.
 Mais vous pouvez faire toutes choses par Jésus-Christ et par la puissance
 du Saint-Esprit. Avec leur aide, vous pouvez être la personne que Dieu
 veut que vous soyez.

\dvrule

\dvprayer{
Père, nous reconnaissons notre besoin et nous reconnaissons notre incapacité.
 Merci de nous rendre fort par Ton Saint-Esprit.
 Puissions-nous demeurer en Toi afin de ne pas satisfaire
 les désirs de notre chair.
}{\DlNdJ}


%%%%%%%%%%%%%%%%
% 11 septembre 
% M 2012/12/08
%%%%%%%%%%%%%%%%

\jrnlday{Qui nous condamnera ?}

\themeindex{Jesus@Jésus!resurrection@résurrection de \sim}
\index{resurrection@résurrection!\sim~de Jésus|see{Jésus}}
\themeindex{Jesus@Jésus!intercession de \sim}
\themeindex{condamnation}
\themeindex{salut}
\themeindex{Trinite@Trinité}

\dvquote{
Qui les condamnera ?
 Le Christ-Jésus est celui qui est mort ;
 bien plus, il est ressuscité, il est à la droite de Dieu,
 et il intercède pour nous !
}{\ibibleverse{Rm}(8:34)}

\lettrine[ante=\Og]{Q}{ui} les condamnera? \Fg{}
 Répondant à sa propre question, Paul dit\frcolon{} \punct{deux-points}
 \Og Ce n'est pas Jésus-Christ; Il est mort pour vous.
 En fait, Il se tient à la droite du Père et intercède pour vous. \Fg{}
 Ce que Jésus fait est le contraire de la condamnation
 \ocadr Il intercède pour vous !

Quand Jésus a parlé à Nicodème, Il a dit\frcolon{} \punct{deux-points}
 \Og Dieu n'a pas envoyé Son Fils dans le monde pour juger le monde;
 mais pour que le monde soit sauvé par Lui. Celui qui croit en Lui
 n'est pas jugé \Fg{} (\ibibleverse{Jn}(3:17-18)).
 \ibibleverse{Rm}(8:1) le redit\frcolon{} \punct{deux-points}
 \Og Il n'y a donc maintenant aucune condamnation
 pour ceux qui sont en Christ-Jésus. \Fg{}

\dvbox{
Jésus ne vous condamne pas \ocadr Il intercède pour vous.
}

Satan aime condamner. Chaque fois que vous trébuchez,
 chaque fois que vous tombez, il est là pour pointer un doigt accusateur
 vers vous. Il veut que vous focalisiez vos pensées sur vos propres faiblesses
 et vos échecs plutôt que sur la puissance de Dieu.
 Le Saint-Esprit, au contraire, vient nous convaincre de péché.
 Il nous dit doucement\frcolon{} \punct{deux-points}
 \Og Attention, ce n'était pas bien. \Fg{}
 Mais ce n'est pas condamner, c'est convaincre.
 Il veut que nous fixions nos pensées sur Jésus,
 qui est prêt à écouter notre confession et à nous laver de nos péchés.

Dieu est pour vous. Jésus est assis à la droite du Père,
 intercédant pour vous. Et le Saint-Esprit demeure en vous
 pour vous donner la puissance et vous aider à surmonter vos faiblesses.
 Avec cette Trinité divine dans \suggest{"dans", plutôt que "sur"}
 votre équipe, comment pourriez-vous perdre?

\dvrule

\dvprayer{
Père, confrontés aux tentations et aux pressions du monde,
 puissions-nous découvrir que la puissance
 et la force de Ton Saint-Esprit nous suffisent.
}{\DlNdJ}


%%%%%%%%%%%%%%%%
% 12 septembre 
% M 2012/12/08
%%%%%%%%%%%%%%%%

\jrnlday{Seulement de l'argile}

\themeindex{Dieu!desseins de \sim}
\themeindex{paix}

\dvquote{
Le potier n'est-il pas maître de l'argile,
 pour faire avec la même pâte un vase destiné
 à l'honneur et un vase destiné au mépris ?
}{\ibibleverse{Rm}(9:21)}

\lettrine{L}{'argile} doit être malléable.
 La seule fa\c{c}on dont elle peut découvrir ce qu'elle est censée devenir
 est de s'abandonner aux doigts du potier.
 Cette pâte d'argile sur le tour du potier pourrait devenir
 un vase de grande beauté destiné à être placé dans un endroit
 bien en vue. Ou elle pourrait devenir un pot destiné à contenir
 des détritus. L'argile n'a aucune idée de ce que le potier
 a en tête et n'a pas à lui dire ce qu'il peut fa\c{c}onner.

Nous sommes ces pâtes d'argiles.
 Nous ne savons pas ce à quoi Dieu nous a destinés,
 ou comment Dieu a prévu de nous utiliser.
 Nous ne savons pas ce que nous sommes appelés à devenir.
 La seule fa\c{c}on dont nous puissions le découvrir,
 c'est de nous abandonner à Lui.

\dvbox{
L'argile ne peut que se laisser fa\c{c}onner.
}

Les mains du Potier sont puissantes \ocadr assez puissantes pour modeler
 et former une vie. Le tour tourne à une vitesse effrayante.
 Alors que nous commen\c{c}ons à tourner de plus en plus vite,
 des questions se posent\frcolon{} Qu'est-ce que Dieu fait?
 Pourquoi ai-je ce sentiment de ne plus rien contrôler?
 Que va-t-il advenir de moi? Mais si vous arrivez à taire ces questions
 et à faire confiance à ces mains, si vous pouvez accepter la pression
 et ignorer la peur, si vous pouvez vous souvenir de la bonté du Potier
 et de la pureté de Son c\oe{}ur, alors la foi va remplacer l'anxiété.
 La peur va se dissiper. La paix va vous entourer.

Vous allez même peut-être vous retrouver à apprécier ces tours de manège!

\dvrule

\dvprayer{
Père, alors que nous tournons à toute allure sur le tour,
 confus quant à ce que Tu es en train de faire,
 aide-nous à nous abandonner au contact de tes doigts.
 Puissions-nous ne pas résister sous peine d'être gâchés entre Tes mains.
 Fa\c{c}onne-nous en des vases qui T'amèneront gloire et honneur.
}{\DlNdJ}


%%%%%%%%%%%%%%%%
% 13 septembre 
% M 2012/12/08
%%%%%%%%%%%%%%%%

\jrnlday{Que Ses voies sont incompréhensibles}

\themeindex{Dieu!desseins de \sim}
\themeindex{Dieu!omniscience de \sim}

\dvquote{
Ô profondeur de la richesse, de la sagesse
 et de la connaissance de Dieu!
 Que ses jugements sont insondables
 et ses voies incompréhensibles!
}{\ibibleverse{Rm}(11:33)}

\lettrine{A}{près} avoir fait part aux Romains
 du plan glorieux de Dieu du rachat de l'homme,
 de comment Dieu fait miséricorde à tous,
 aussi bien aux Juifs qu'aux Païens,
 Paul ne peut que laisser éclater des louanges.
 Il est dans la plus complète admiration de la sagesse
 et de l'intelligence de Dieu manifestées
 dans Son plan du rachat de l'homme.

\dvbox{
Seul Dieu sait ce qui va arriver dès le commencement.
}

Le fait que \Og Ses voies sont incompréhensibles \Fg{}
 est la source de beaucoup de frustrations dans notre vie de chrétien. \typo{chrétien}
 Nous voulons comprendre les voies de Dieu. Nous voulons savoir ce qu'Il pense
 lorsqu'Il permet à certaines choses d'arriver.
 Mais comme Dieu l'a dit à Ésaïe\frcolon{} \punct{deux-points}
 \Og Mes pensées ne sont pas vos pensées, et vos voies ne sont pas Mes voies,
 \ocadr Oracle de l'Éternel. Autant les cieux sont élevés au-dessus
 de la terre, autant Mes voies sont élevées au-dessus de vos voies,
 et Mes pensées au-dessus de vos pensées \Fg{}
 (\ibibleverse{Is}(55:8-9)).

Dieu est en train de mettre en \oe{}uvre un plan dont nous ne sommes
 pas tenus au courant. Nous ne pouvons comprendre Sa volonté
 que lorsqu'elle est dévoilée, jour après jour.
 Bien qu'à certains moments nous ne comprenions pas la raison
 des vallées sombres par lesquelles Il nous fait passer,
 une fois que nous atteignons le sommet de la montagne
 et regardons en arrière, alors nous voyons tout le chemin
 qui nous y a conduit. Nous comprenons alors ce qu'Il faisait.
 Nous comprenons que nous n'aurions jamais pu atteindre
 les hauteurs si nous n'étions pas d'abord passés par les profondeurs.

\dvrule

\dvprayer{
Père, Tu es si riche en bonté, si incomparable en sagesse et en douceur!
 Nous voulons seulement Te dire que nous T'aimons.
 Seigneur, aide-nous à nous consacrer à Toi et à faire confiance à Ta volonté.
}{\DlNdJ}


%%%%%%%%%%%%%%%%
% 14 septembre 
% M 2012/12/13
%%%%%%%%%%%%%%%%

\jrnlday{Un service raisonnable}

\themeindex{sacrifice}
\themeindex{saintete@sainteté}
\index{temple!\sim~de l'Esprit-Saint|see{Esprit-Saint}}
\themeindex{Esprit-Saint!temple de l'\sim}
\themeindex{corps}
\themeindex{ame@âme}
\themeindex{redemption@rédemption}
\index{rachat|see{rédemption}}
\themeindex{Dieu!service de \sim}

\dvquote{
Je vous supplie donc, frères, par les miséricordes de Dieu,
 à présenter vos corps en un sacrifice vivant, saint,
 acceptable à Dieu, qui est votre service raisonnable.
}{\ibibleverse{Rm}(12:1) \KJF}

\lettrine{E}{n} écrivant aux chrétiens de l'église de Corinthe,
 Paul se fait un devoir de leur dire de faire attention à ce qu'ils
 font avec leurs corps.
 \Og Ne savez-vous pas ceci\frcolon{} votre corps est le temple du Saint-Esprit
 [\dots{}] vous n'êtes pas à vous-mêmes ?
 Car vous avez été rachetés à grand prix. Glorifiez donc Dieu
 dans votre corps et dans votre esprit qui appartiennent à Dieu \Fg{}
 (\ibibleverse{ICo}(6:19-20)).

Dieu a payé un prix immense pour vous racheter,
 et le moins que vous puissiez faire, c'est de vous présenter à Lui
 \ocadr c\oe{}ur et âme, pensées et corps \fcadr{} comme un sacrifice vivant.
 C'est votre service raisonnable. Après tout, vous devez votre propre
 existence à votre Créateur. Il est Celui qui a insufflé la vie en vous;
 Il est Celui qui vous soutient jour après jour.

\dvbox{
Vous appartenez à Dieu \ocadr c\oe{}ur et âme, pensées et corps.
}

Au nom du plaisir, les gens maltraitent leur corps par l'abus de drogues
 et d'alcool, et même par des perversions. \suggest{Phrase reformulée}
 C'est mauvais pour tout le monde,
 mais c'est absolument impensable pour le chrétien.
 Ne détruisez pas l'instrument de votre service. De toutes fa\c{c}ons,
 il ne vous appartient pas. Au contraire, sanctifiez-vous pour Lui.\grammar{tiret}

La plupart des gens utilisent leurs corps à la poursuite de plaisirs passagers
 \ocadr des choses qui n'ont ni le pouvoir de les racheter, ni aucune valeur éternelle
 \suggest{Phrase pas très claire, reformulée}
 quelle qu'elle soit. Voyons ce que nous pouvons faire pour Dieu qui va compter
 pour l'éternité. Présentez votre corps à Dieu comme un sacrifice vivant,
 saint et acceptable à Lui, ce qui est non seulement la moindre chose
 que vous puissiez faire, mais aussi la plus sage.

\dvrule

\dvprayer{
Père, puissions-nous consacrer notre temps, nos talents et nos énergies
 à faire ces choses qui vont durer pour toujours.
}{\Amen}


%%%%%%%%%%%%%%%%
% 15 septembre 
% M 2012/12/13
%%%%%%%%%%%%%%%%

\jrnlday{Réveillez-vous!}

\themeindex{vie chretienne@vie chrétienne}
\themeindex{reveil@réveil spirituel}
\themeindex{Jesus@Jésus!retour de \sim}
\themeindex{esperance@espérance}
\themeindex{sanctification}
\themeindex{sommeil}

\dvquote{
D'autant que vous savez en quel temps nous sommes\frcolon{}
 c'est l'heure de vous réveiller enfin du sommeil,
 car maintenant le salut est plus près de nous
 que lorsque nous avons cru.
}{\ibibleverse{Rm}(13:11)}

\lettrine{Q}{uelle} heure est-il?
 Paul répondait\frcolon{} \punct{deux-points}
 \Og Il est l'heure de vous réveiller enfin du sommeil. \Fg{}
 Paul reconnaissait que l'Église \suggest{Église} de son temps
 dormait face aux vrais problèmes de la vie.
 Combien plus encore en est-il ainsi de l'Église \suggest{Église} aujourd'hui!
 Nous devons être endormis car il ne semble pas y avoir une urgence
 dans nos c\oe{}urs pour les choses du Seigneur.

Nous avons assez dormi. Le monde autour de nous se retrouve sous la coupe de Satan,
 et c'est arrivé pendant que nous faisions la sieste.
 Nous avons dormi pendant la suppression de la prière
 de nos écoles publiques. Nous avons dormi pendant la suppression
 des lois qui nous protégeaient des promoteurs de la pornographie.
 Nous avons dormi pendant l'ouverture des portes de l'avortement à la demande.
 Nous avons dormi pendant la paganisation de notre pays.
 Il est temps de nous réveiller.

\dvbox{
Dépouillons les \oe{}uvres des ténèbres et revêtons-nous du Seigneur Jésus-Christ.
}

Le retour du Seigneur est proche.
 À la fin de chaque jour, nous pouvons vraiment dire que nous sommes
 un jour plus proche du retour de Jésus-Christ.
 Le salut est plus près de nous que lorsque nous avons cru.
 Oh, la bienheureuse espérance de Jésus-Christ qui revient bientôt
 pour nous délivrer de ce monde présent si mauvais!

Il est grand temps pour nous d'enlever les \oe{}uvres des ténèbres
 et de nous revêtir du Seigneur Jésus-Christ.
 Il est temps pour nous de vivre et de marcher en suivant l'Esprit.
 Puisse Dieu réveiller Son Église \suggest{Église}
 et nous mettre au défi par Sa Parole de vivre de fa\c{c}on juste et sainte.

\dvrule

\dvprayer{
Nous nous rendons compte, Seigneur, que nous avons dormi pendant
 que la situation du monde se dégradait autour de nous.
 Nous voyons que le monde est au bord de l'éternité.
 Réveille-nous, Seigneur. Aide-nous à toucher les perdus.
}{\DlNdJ}


%%%%%%%%%%%%%%%%
% 16 septembre 
% M 2012/12/14
%%%%%%%%%%%%%%%%

\jrnlday{Le Dieu de l'espérance}

\themeindex{esperance@espérance}
\themeindex{joie}
\themeindex{paix}
\themeindex{reconfort@réconfort}
\themeindex{Croix}
\themeindex{Jesus@Jésus!retour de \sim}

\dvquote{
Que le Dieu de l'espérance vous remplisse\linebreak
 de toute joie et de toute paix\dots{}
}{\ibibleverse{Rm}(15:13)}

\lettrine{D}{es} sociologues, cherchant à établir une corrélation
 entre l'espoir et les chances de survie,
 ont procédé à une expérience intéressante
 avec des rats de laboratoire.
 Un premier groupe de rats a été placé dans une cuve
 dans laquelle on les arrosait d'eau sans arrêt
 pour qu'ils ne puissent pas se retourner et se mettre à flotter.
 Ces rats ont survécu en moyenne dix-sept minutes
 avant de se laisser aller et de\suggest{de} se noyer.
 Les rats du second groupe ont été retirés de la cuve
 juste avant la dix-septième minute, juste avant qu'ils ne se noient.
 Quelques jours plus tard, ces rats ont été remis dans l'eau
 pour le même traitement.
 Mais cette fois, les rats ont survécu pendant\dots{} trente-six heures \footnote{126~fois plus de temps!}!
 La différence entre dix-sept minutes et trente-six heures de survie,
 c'est\dots{} l'espoir!

\dvbox{
Quand nous passons par des moments sombres ou terribles,
 c'est notre espoir dans le futur qui nous soutient.
}

Nous nous fortifions en réalisant que nous ne serons pas ici pour toujours,
 que Jésus va revenir bientôt pour nous délivrer de ce monde mauvais.
 Bien qu'il soit difficile d'être patient, nous attendons plein de confiance
 la joie de Le voir face à face, et la bénédiction de régner avec Lui
 quand Il établira Son glorieux royaume sur la terre.

Jésus a été envoyé par le Père pour une mission\frcolon{}
 Il est venu pour nous donner la paix, le réconfort et l'espérance.
 En allant à la Croix et en prenant nos péchés sur Lui-même,
 Il a rendu la paix avec Dieu possible.
 En marchant dans la chair d'un homme,
 Il nous réconforte par sa connaissance de notre condition.
 En assurant notre futur éternel, Il nous offre l'espérance du ciel.

\dvrule

\dvprayer{
Merci, Père, pour la paix qui dépasse l'entendement humain
 et qui garde nos c\oe{}urs et nos vies.
 Aide-nous à comprendre combien il est stupéfiant
 que Tu nous aies aimés au point de donner Ton Fils
 pour amener l'espérance, le réconfort et la paix.
}{\DlNdJ}


%%%%%%%%%%%%%%%%
% 17 septembre 
% M 2012/12/14
%%%%%%%%%%%%%%%%

\jrnlday{Le message de la Croix}

\themeindex{Croix}
\themeindex{amour}
\themeindex{Esprit-Saint!choses de l'\sim}
\themeindex{salut}
\themeindex{Dieu!puissance de \sim}
\themeindex{nature!homme \sim{}l}
\themeindex{Messie}

\dvquote{
Car la parole de la croix est folie pour ceux qui périssent ;
 mais pour nous qui sommes sauvés, elle est puissance de Dieu.
}{\ibibleverse{ICo}(1:18)}

\lettrine{P}{our} les Juifs, l'idée que leur Messie soit arrêté,
 battu et crucifié sur une croix était absolument inimaginable.
 Ils s'attendaient à ce que leur Messie, le descendant de David,
 règne sur le monde dans la justice et la paix.

Pour les Grecs, l'idée qu'un homme puisse mourir pour les péchés
 de tous les hommes était pure folie. Aucun des nombreux dieux
 qu'ils adoraient n'aurait jamais fait quelque chose démontrant tant d'amour.
 Leurs dieux étaient égoïstes. Aussi l'idée qu'un Dieu soit prêt
 à se sacrifier pour sauver Son peuple était ridicule.

\Og L'homme naturel ne re\c{c}oit pas les choses de l'Esprit de Dieu,
 car elles sont une folie pour lui \Fg{} (\ibibleverse{ICo}(2:14)).
 Bien que le monde considère la Croix comme étant de la folie, nous,
 qui sommes sauvés, la voyons comme le bel instrument de notre salut.
 Nous n'avons pas honte de la Croix.

\dvbox{
C'est la puissance de Dieu pour le salut.
}

Par la Croix, Dieu nous libère de l'esclavage au péché.
 Par la Croix, Il a conquis la mort et la tombe.
 Par la Croix, Il nous a adoptés.
 Par la Croix, Il a mis un terme à la guerre qui existait entre nous.
 Par la Croix, Il a garanti notre éternel futur.

Que le monde pense ce qu'il veut. Nous connaissons la vérité.
 Et oh, combien nous aimons cette \Og vieille Croix en bois mal équarri \Fg{}\NdT{titre d'un hymne anglo-saxon bien connu, écrit en~1912 par l'évangéliste et compositeur George Bennard (1873--1958).}!

\dvrule

\dvprayer{
Seigneur, nous Te remercions de nous avoir rendu la liberté\\
 \ocadr liberté de T'aimer, liberté de Te servir,
 liberté de vivre en communion avec Toi.\\
 Que le message de la Croix parle à nos c\oe{}urs aujourd'hui.
}{\DlNdJ}


%%%%%%%%%%%%%%%%
% 18 septembre 
% M 2012/12/14
%%%%%%%%%%%%%%%%

\jrnlday{L'homme naturel, l'homme spirituel}

\themeindex{nature!homme \sim{}l}
\themeindex{Esprit-Saint!choses de l'\sim}
\themeindex{naissance!nouvelle \sim}
\index{communion!\sim~avec Dieu|see{Dieu}}
\themeindex{Dieu!communion avec \sim}
\index{Parole de Dieu|see{Bible}}
\index{Dieu!Parole de \sim|see{Bible}}
\themeindex{Bible}

\dvquote{
Mais l'homme naturel ne re\c{c}oit pas les choses de l'Esprit de Dieu [\dots{}]
 parce que c'est spirituellement qu'on en juge.
 L'homme spirituel, au contraire, juge de tout \dots{}
}{\ibibleverse{ICo}(2:14-15)}

\lettrine{L}{'homme} naturel est un homme qui n'est pas né de nouveau.
 En fait, l'homme naturel est bien tel que son nom le laisse entendre
 \ocadr il est naturel tel le jour où il est né,
 sans concept ni compréhension de Dieu.

L'homme naturel ne comprend pas les choses de l'Esprit,
 parce que c'est spirituellement qu'on en juge.
 Tout comme un aveugle ne peut pas apprécier l'éclat d'un coucher de soleil,
 il manque à l'homme naturel les facultés qui permettent
 de comprendre et d'apprécier les choses spirituelles.

\dvbox{
Une toute nouvelle dimension s'ouvre à l'homme spirituel.
}

Qu'est-ce donc que l'homme spirituel?
 Il est l'homme dont l'esprit est né de nouveau
 et qui est donc conscient des choses de l'Esprit
 et comprend les choses de l'Esprit.

Et cependant Paul a dit\frcolon{} \punct{deux-points}
 \Og L'homme spirituel, au contraire, juge de tout,
 et il n'est lui-même jugé par personne \Fg{}
 (\ibibleverse{ICo}(2:15)).
 Avez-vous remarqué à quel point il est fréquent\suggest{à quel point il est fréquent}
 que les gens ne vous comprennent
 pas en tant que chrétiens? Vous êtes une énigme pour eux.
 \Og Pourquoi aimez-vous autant aller à l'église?
 Pourquoi ne buvez-vous pas? Que faites-vous pour vous amuser? \Fg{}

L'homme naturel ne peut pas comprendre la joie de vivre en communion avec Dieu,
 mais nous qui sommes nés de nouveau par l'Esprit connaissons
 fort bien la bénédiction de marcher avec Jésus
 \ocadr et la paix qui vient d'avoir été pardonnés et lavés.

\dvrule

\dvprayer{
Père, merci à Toi de nous avoir sauvés.
 Merci de ce que par Ton Saint-Esprit,
 nous pouvons discerner Ta vérité et comprendre les mystères de Ta Parole.
 Puisses-Tu ouvrir les yeux de ceux qui ne Te connaissent pas encore.
}{\DlNdJ}


%%%%%%%%%%%%%%%%
% 19 septembre 
% M 2012/12/14
%%%%%%%%%%%%%%%%

\jrnlday{Le chrétien charnel}

\themeindex{chair}
\themeindex{carnalite@carnalité}
\themeindex{nature!homme \sim{}l}
\themeindex{Bible}
\themeindex{amour}
\themeindex{vie chretienne@vie chrétienne}

\dvquote{Pour moi, frères, ce n'est pas comme à des hommes spirituels
 que j'ai pu vous parler, mais comme à des hommes charnels,
 comme des petits enfants en Christ.
}{\ibibleverse{ICo}(3:1)}

\lettrine{P}{aul} a déjà décrit les catégories de l'homme naturel
 et de l'homme spirituel, mais maintenant, il présente
 une troisième catégorie \ocadr l'homme charnel.
 Paul les appelle des \Og petits enfants en Christ \Fg{}.
 Le fait qu'ils sont \Og en Christ \Fg{} les identifie
 \grammar{identifie}
 comme des chrétiens, mais tragiquement, ils ne se sont pas développés
 spirituellement. Ce sont encore des bébés spirituels.
 Malheureusement, il y a tant de chrétiens dans cet état
 de développement arrêté, qu'ils ne constituent même plus une nouveauté.

Les chrétiens charnels\typo{charnels} ont re\c{c}u Jésus comme leur Sauveur
 mais ils n'ont pas renoncé à eux-mêmes, ni pris leur croix pour suivre Jésus.
 La chair continue de régner dans leur vie.
 Jésus est Sauveur, mais Il n'est pas le Seigneur de leurs vies.

\dvbox{
Abandonnez la carnalité.
}

Quel est le remède au christianisme charnel?
 La première étape, c'est de marcher dans l'amour.
 Si vous marchez dans l'amour, vous ne vous engagerez pas
 dans la voie de l'envie et des conflits.
 Si vous marchez dans l'amour, vous n'amènerez pas
 la division dans le corps de Christ.

Le deuxième remède au christianisme charnel,
 c'est de se plonger dans la Parole de Dieu.
 La vraie croissance spirituelle ne peut venir qu'en se se nourrissant
 de la Parole. Vous avez besoin d'un bon régime pour grandir,
 et la Parole de Dieu est le régime qui favorise la croissance
 spirituelle du croyant.
 Entrez dans la Parole et faites entrer la Parole en vous.

Abandonnez la carnalité. Grandissez, développez-vous
 par l'intermédiaire de la Parole. Laissez-la nourrir votre âme.
 Puis marchez dans l'amour et grandissez dans la
 \typo{missing la}
 grâce et la connaissance de notre Seigneur et Sauveur, Jésus-Christ.

\dvrule

\dvprayer{
Seigneur, nous désirons la maturité spirituelle.
 Nous voulons marcher dans l'amour.
 Puissions-nous arriver à une compréhension plus profonde,
 plus riche de Ta Parole.
}{\DlNdJ}

%%%%%%%%%%%%%%%%
% 20 septembre 
% M 2012/12/15
%%%%%%%%%%%%%%%%

\jrnlday{Fous à cause de Christ}

\themeindex{folie}
\themeindex{sacrifice}
\themeindex{souffrance}
\themeindex{persecution@persécution}

\dvquote{
Nous sommes fous à cause de Christ\dots{}
}{\ibibleverse{ICo}(4:10)}

\lettrine{L}{es} gens se comportent de fa\c{c}on ridicule
 pour toutes sortes de raisons \ocadr parce qu'ils ont bu,
 parce qu'ils veulent faire rire les autres,
 parce qu'ils veulent gagner de l'argent,
 ou simplement parce qu'ils veulent faire croire aux autres qu'ils sont malins.
 Et personne ne semble trouver \c{c}a bizarre.

Mais que quelqu'un déclare qu'il est fou à cause de Christ
 et aussitôt on commencera à le ridiculiser.
 Le monde considère que c'est folie que de tout abandonner pour Dieu.
 Quelques fois, nos propres familles se moquent de nous.
 Ces gens ne peuvent pas comprendre les sacrifices
 que nous sommes prêts à faire pour Jésus.

Paul a été battu, emprisonné, lapidé et finalement décapité
 pour la cause de Jésus-Christ. Le monde dirait\frcolon{} \typo{deux-points}
 \Og C'était un fou. \Fg{}
 Mais comme Paul l'a écrit, ses souffrances n'étaient pas
 \Og dignes d'être comparées à la gloire à venir
 qui sera révélée pour nous \Fg{}\linebreak (\ibibleverse{Rm}(8:18)).

\grammar{en héros}
\dvbox{
Comment se fait-il que si vous donnez votre vie au service de votre pays
 on vous traite en héros,
 mais que si vous donnez votre vie au service de Christ,
 on vous traite de fou?
}

Adolescent, j'ai fortement été inspiré par un homme
 qui avait l'habitude de dire\frcolon{} \typo{deux-points}
 \Og Tout le monde est un fou pour quelqu'un d'autre.
 Autant être un fou pour Christ. \Fg{}
 Si aimer Jésus de tout mon c\oe{}ur fait de moi un fou,
 alors je suis un fou. Si désirer servir Jésus avec tout ce que j'ai
 fait de moi un fou, alors je suis un fou.
 Si faire confiance à Jésus pour tout fait de moi un fou,
 alors je suis un fou.
 Mais je n'ai pas honte d'être un fou pour la cause de Christ.

\dvrule

\dvprayer{
Père, aide-nous à nous opposer à la marée du mal
 et à nous prononcer pour la cause de Christ,
 même au prix d'être considérés comme des fous.
}{\DlNdJ}



%%%%%%%%%%%%%%%%
% 21 septembre 
% M 2012/12/15
%%%%%%%%%%%%%%%%

\themeindex{purification}
\themeindex{Paque@Pâque}
\themeindex{levain}
\themeindex{pain}
\themeindex{peche@péché}
\themeindex{Egypte@Égypte}
\themeindex{agneau!\sim~pascal}
\themeindex{Dieu!Agneau de \sim}
\themeindex{mort}

\jrnlday{Christ, notre Pâque}

\dvquote{
Purifiez-vous du vieux levain,
 afin que vous soyez une pâte nouvelle,
 puisque vous êtes sans levain,
 car Christ, notre Pâque, a été immolé.
}{\ibibleverse{ICo}(5:7)}

\lettrine{Q}{uand} on ajoute du levain à du pain,
 des gaz sont libérés par la fermentation.
 C'est un processus de décomposition.
 De cette fa\c{c}on, le pain lève.
 Juste un peu de levain peut imprégner
 toute la boule de pâte à pain.

Le péché est comparable.
 Un peu de péché toléré dans l'église
 peut faire son chemin dans le corps entier.
 Et c'est précisément dans ce contexte que Paul
 s'adresse à l'église de Corinthe.
 Ses membres avaient toléré le mal
 dans leur communauté
 \ocadr du péché qui avait besoin d'être purgé.

\dvbox{
Juste un peu de péché toléré peut imprégner le corps entier,
 le pourrissant de l'intérieur.
}

Dans le cadre de la dernière plaie infligée à l'Égypte,
 Dieu a pris des dispositions pour Son propre peuple.
 Chacune des maisons protégées par le sang de l'agneau
 offert en sacrifice a échappé à la plaie.
 Mais dans chacune des maisons non protégées par le sang,
 le premier-né fut retrouvé mort le lendemain matin.

De la même fa\c{c}on, Dieu a pris des dispositions
 pour le pardon de nos péchés pour que nous n'ayons pas
 à mourir à cause d'eux.
 Cette disposition est en Jésus-Christ,
 notre Agneau offert en sacrifice.
 Grâce à Son sang versé,
 nous pouvons avoir nos péchés pardonnés.

Paul a déclaré que \Og Christ est notre Pâque. \Fg{}
 Ce n'est pas par accident que Jésus a été crucifié le jour de la Pâque.
 L'Agneau de Dieu a accompli ce qui était préfiguré
 dans la Pâque d'Israël, en purgeant le péché
 qui souillait nos vies, et en nous permettant d'échapper
 à une mort autrement inévitable.

\dvrule

\dvprayer{
Père, nous Te remercions d'avoir envoyé Ton Fils
 pour être notre Pâque.
 Nous recevons maintenant la purification qu'Il offre.
 Nous recevons le pardon de nos péchés.
}{\DlNdJ}



%%%%%%%%%%%%%%%%
% 22 septembre 
% M 2012/12/17
%%%%%%%%%%%%%%%%

\jrnlday{Limites à la liberté}

\themeindex{liberte@liberté}
\themeindex{carnalite@carnalité}
\themeindex{debauche@débauche}
\themeindex{idolatrie@idolâtrie}
\themeindex{adultere@adultère}
\themeindex{homosexualite@homosexualité}
\themeindex{vol}
\themeindex{cupidite@cupidité}
\themeindex{ivrognerie}
\themeindex{calomnie}
\themeindex{extorsion}
\themeindex{esclave}

\dvquote{
Tout m'est permis, mais tout n'est pas utile,
 tout m'est permis, mais je ne me laisserai
 pas asservir par quoi que ce soit.
}{\ibibleverse{ICo}(6:12)}

\lettrine{S}{ouvent}, \punct{virgule}
 des chrétiens demandent s'ils peuvent se livrer
 à certaines activités. Ce qu'ils demandent en réalité,
 c'est\frcolon{}\punct{deux-points}
 \Og Jusqu'où puis-je aller dans la participation
 aux choses du monde tout en restant chrétien? \Fg{}

Juste avant d'écrire \Og Tout m'est permis, \Fg{}
 Paul avait donné une liste assez spécifique
 (mais non exhaustive) des choses qui sont vraiment illégitimes\frcolon{}
 débauche, idolâtrie, adultère, homosexualité,
 vol, cupidité, ivrognerie, calomnie, extorsion
 (\ibibleverse{ICo}(6:9-10)).\punct{point}

\dvbox{
Dès que je tombe sous l'emprise de quelque chose,
 je ne suis plus libre.
}

Certains demandent\frcolon{}\punct{deux-points}
 \Og Puis-je boire une bière avec ma pizza?
 En tant que chrétien, puis-je fumer des cigarettes? \Fg{}
 Laissez-moi vous demander\frcolon{} \punct{deux-points, guillemets?}
 \Og Est-il possible de se laisser asservir en buvant de la bière?
 Est-il possible de devenir ``accro'' au tabac? \Fg{}
 Ce n'est peut-être pas un péché abominable,
 cela ne vous mènera peut-être pas en enfer,
 mais si cette chose peut vous amener sous son emprise,
 alors vous sacrifiez la glorieuse liberté que vous avez en Christ.
 Dès le moment où vous tombez sous cette emprise,
 vous êtes alors devenus esclaves.

N'essayons pas de discerner si quelque chose est bien ou mal.
 Quand nous considérons les problèmes de la vie,
 demandons-nous plutôt\frcolon{} \Og \punct{guillemets}
 Cela va t-il entraver ma vie avec Dieu?
 Pourrais-je voir Jésus participer à cette chose avec moi?
 Cette chose pourrait-elle avoir une telle emprise sur moi,
 qu'elle fasse de moi un esclave et me dérobe ma liberté?
 Cette chose m'édifie-t-elle en Christ?
 Me fait-elle ressembler davantage à Lui? \Fg{}

\dvrule

\dvprayer{
Père, montre-nous ces choses qui finissent par détruire plutôt que fortifier
 la présence de Christ en nous.
 Puissions-nous vraiment être le reflet de notre Seigneur
 Jésus-Christ pour le monde dans lequel nous vivons.
}{\ESN}



%%%%%%%%%%%%%%%%
% 23 septembre 
% M 2012/12/17
%%%%%%%%%%%%%%%%

\jrnlday{Vivre dans l'attente}

\themeindex{temps}
\themeindex{attente}
\themeindex{vie chretienne@vie chrétienne}
\themeindex{Jesus@Jésus!retour de \sim}
\themeindex{mission}
\themeindex{purification}

\dvquote{
Voici ce que je dis, frères\frcolon{}
 le temps est court\dots{}
}{\ibibleverse{ICo}(7:29)}

\lettrine{J}{e} suis convaincu
 que Dieu a voulu que chaque génération
 croie qu'elle sera celle qui verra le retour de Jésus.
 Pourquoi? Parce qu'Il veut que nous vivions
 dans l'attente de Son retour.

Quand nous croyons que Sa venue est imminente,
 nous ressentons l'urgence d'amener l'Évangile au monde.
 La prise de conscience qu'il ne reste plus beaucoup
 de temps nous motive à remplir la mission que Jésus nous a confiée. 

\dvbox{
Gardez tout contact de votre part avec le monde aussi léger que possible.
}

La conscience de l'imminence de Son retour nous donne aussi
 la juste perspective sur les choses du monde.
 Quand vous savez que le rideau qui va refermer cette vie
 pourrait tomber à n'importe quel moment,
 et que le rideau cachant l'éternité pourrait se lever en un clin d'\oe{}il,
 vous êtes plus à même de garder un contact léger avec les choses du monde.
 Vous êtes moins enclins à vous enraciner ici-bas.
 Vous avez moins de chances de devenir matérialistes.

Anticiper le retour de Jésus a un effet purificateur
 sur nos vies individuelles et sur l'Église\grammar{Église} dans son ensemble.
 Quand Il va revenir, nous ne voulons pas être engagés dans une activité
 qui serait contraire à Ses souhaits.
 Nous voulons être occupés à construire le royaume,
 occupés à utiliser ce qu'Il nous a donné pour Sa gloire.
 Nous ne voulons pas gaspiller notre temps à nous complaire
 dans les chagrins ou les plaisirs.
 Nous ne voulons pas gaspiller du temps à amasser des possessions.

Il ne reste plus beaucoup de temps et le monde est en train de passer.
 Donnez-vous, donnez votre temps, votre énergie et vos ressources
 à des choses qui sont éternelles.
 Vivez pour le royaume de Dieu et vivez pour toujours.

\dvrule

\dvprayer{
Seigneur, garde-nous de devenir si profondément impliqués
 dans des choses temporelles que nous n'ayons
 plus de temps pour les choses éternelles.
Donne-nous la sagesse de racheter nos jours.
}{\DlNdJ}



%%%%%%%%%%%%%%%%
% 24 septembre 
% M 2012/12/17
%%%%%%%%%%%%%%%%

\jrnlday{L'amour qui édifie}

\themeindex{amour}
\themeindex{connaissance}
\themeindex{orgueil}
\themeindex{edification@édification}

\dvquote{
Nous savons que tous, nous avons de la connaissance.
 La connaissance enorgueillit,
 mais l'amour édifie.
}{\ibibleverse{ICo}(8:1)}

\lettrine{U}{ne} division s'était développée dans l'église de Corinthe.
 Certains avaient de fortes convictions
 contre la consommation de viandes d'animaux
 qui avaient été sacrifiés à des idoles.
 D'autres se déclaraient libres de manger
 de telles viandes parce que \ocadr raisonnaient-ils \fcadr{}
 \Og Nous savons que les idoles ne sont rien. \Fg{}
 Ceux qui avaient cette liberté regardaient de haut
 ceux qui ne l'avaient pas, les considérant comme inférieurs.

Il y aura toujours des snobs intellectuels.
 Il y a toujours ceux qui se sentent intellectuellement
 supérieurs à d'autres et qui se moquent d'eux
 parce que ces derniers suivent \Og des superstitions d'ignorants \Fg{}.
 Mais le prophète Ésaïe a écrit\frcolon{} \punct{deux-points}
 \Og Ta sagesse et ta science t'ont trompé;
 tu as dit dans ton c\oe{}ur\frcolon{} \punct{deux-points, majuscule}
 \Og J'ai raison et aucune autre opinion n'est valable \Fg{}
 (paraphrase de \ibibleverse{Is}(47:10)).
 Paul a abordé ce problème dans l'église de Corinthe,
 en avertissant que \Og la connaissance enorgueillit. \Fg{}

\dvbox{
Il nous est demandé de vivre dans l'amour.
}

Si nous vivons vraiment dans l'amour,
 nous allons chercher à fortifier le frère plus faible.
 Je ne vais pas essayer de le convaincre qu'il a tort,
 ni faire étalage de ma liberté devant lui
 au risque de le faire tomber.
 Il est dangereux d'encourager quelqu'un
 à faire quelque chose qui viole sa conscience,
 ou de le convaincre de faire quelque chose
 qu'il considère comme mal.

\Og En péchant de la sorte contre les frères
 et en heurtant leur conscience faible,
 vous péchez contre Christ.
 C'est pourquoi, si un aliment fait tomber mon frère,
 jamais plus je ne mangerai de viande,
 afin de ne pas faire tomber mon frère \Fg{}
 (\ibibleverse{ICo}(8:12-13)).
 C'est \c{c}a, vivre dans l'amour.
 N'allez pas vous promener en faisant étalage
 de votre liberté pour ainsi détruire un frère plus faible.
 La connaissance enorgueillit, mais l'amour édifie.

\dvrule

\dvprayer{
Dieu, aide-nous à vivre dans l'amour et à nous édifier
 les uns dans les autres dans les choses spirituelles,
 sans toujours essayer de prouver que nous avons raison
 mais en reconnaissant que nous ne ressentons
 pas tous les mêmes libertés.
}{\DlPNdJ}


%%%%%%%%%%%%%%%%
% 25 septembre 
% M 2012/12/17
%%%%%%%%%%%%%%%%

\jrnlday{Le rocher}

\themeindex{eau vive}
\themeindex{miracle}
\themeindex{justification}
\themeindex{Dieu!soif de \sim}

\dvquote{
Car ils buvaient à un Rocher spirituel qui les suivait,
 et ce Rocher était le Christ\dots{}
 Or ce sont là des exemples pour nous\dots{}
}{\ibibleverse{ICo}(10:4,6)}

\lettrine{L}{e} Dieu qui donna de l'eau aux enfants d'Israël
 assoiffés en faisant frapper le rocher
 est le même Dieu qui nous a donné la vie
 en faisant frapper le corps de Son Fils.
 Le don miraculeux de l'eau dont ils ont fait l'expérience
 n'est qu'une préfiguration du don miraculeux
 dont nous avons fait l'expérience,
 celui d'être déclarés justes par Dieu.

\dvbox{
Rien ne peut satisfaire la profonde soif de l'homme
 pour Dieu en dehors d'une riche relation avec Dieu.
}

Paul déclare\frcolon{}\punct{deux-points}
 \Og Ce Rocher était Christ. \Fg{}
 Jésus parlait de l'eau de la vie qu'Il donnerait
 à ceux qui ont soif. Dans \bibleverse{Jn}(4:),
 Jésus a dit à la femme samaritaine\frcolon{}\punct{deux-points}
 \Og Quiconque boit de cette eau aura encore soif;
 mais celui qui boira de l'eau que je lui donnerai,
 n'aura jamais soif, et l'eau que je lui donnerai
 deviendra en lui une source d'eau qui jaillira
 jusque dans la vie éternelle \Fg{}
 (\ibibleverse{Jn}(4:13-14)).

Ceux qui essayent d'étancher leur soif
 avec toute autre chose vont réaliser qu'ils ont encore et toujours soif.
 Mais quand vous allez au Rocher qui est Christ,
 non seulement l'eau qu'Il vous donne satisfait,
 mais elle déborde de vous telle une fontaine d'eau vive.

Avez-vous soif aujourd'hui?
 Tout ce qu'il vous suffit de faire est de parler au Rocher.
 Il fera que l'eau de la vie jaillisse de votre être.
 Il va ouvrir les écluses et étancher votre soif.

\dvrule

\dvprayer{
Père, nous Te remercions pour Jésus-Christ,
 le Rocher des siècles qui a été frappé pour nous.
 Garde-nous d'aller à toute autre source quand nous avons soif.
 Aide-nous à T'amener notre soif,
 pour que nous puissions boire librement et pleinement
 et que nous trouvions la satisfaction à laquelle nous aspirons.
}{\DlNdJ}


%%%%%%%%%%%%%%%%
% 26 septembre 
% M 2012/12/18
%%%%%%%%%%%%%%%%

\jrnlday{Examinez-vous vous-mêmes}

\themeindex{communion!Saint Cène}
\index{cene@Cène!Sainte \sim|see{communion}}
\themeindex{vin}
\themeindex{pain}
\themeindex{peche@péché}
\themeindex{confession}

\dvquote{
Que chacun donc s'examine soi-même,
 et qu'ainsi il mange du pain et boive la coupe.
}{\ibibleverse{ICo}(11:28)}

\lettrine{P}{aul} venait juste de reprendre
 les gens de l'église de \typo{manque de} Corinthe
 pour la fa\c{c}on dont ils venaient à la Table du Seigneur
 et participaient à la communion.
 Il les a avertis qu'en le faisant à la légère,
 en manquant de respect, ils devenaient coupables
 du corps et du sang de Jésus.
 Et il leur a donc dit\frcolon{} \punct{deux-points}
 \Og Que chacun s'examine soi-même. \Fg{}

C'est un bon conseil pour nous tous. C'est difficile mais important.
 Vous ne pouvez pas corriger un problème tant que vous ne l'avez pas
 d'abord identifié et reconnu.
 \Og Si nous confessons nos péchés, Il est fidèle et juste
 pour nous pardonner nos péchés et nous purifier
 de toute injustice \Fg{} (\ibibleverse{IJn}(1:9)).

\dvbox{
Une fois que nous nous sommes examinés et que nous reconnaissons
 notre péché, nous pouvons alors l'amener à Jésus.
}

Peut-être y-a-t-il dans votre vie quelque chose dont Dieu
 veut que vous vous occupiez, un problème que vous avez essayé
 de Lui cacher.
 Que vous en soyez conscients ou non,
 ce péché vous empêche d'avoir une pleine communion avec Lui.
 Il veut corriger cela aujourd'hui.
 Si vous vous jugez vous-mêmes, vous ne serez pas jugés par Lui.
 Laissez le Saint Esprit sonder votre c\oe{}ur et révéler la vérité
 qu'Il connaît, puis assurez-vous de répondre et de réagir
 en confessant vos péchés et en recevant Son pardon.

Comme David le disait\frcolon{} \punct{deux-points}
 \Og Sonde-moi, ô Dieu, et connais mon c\oe{}ur!
 Éprouve-moi, et connais mes préoccupations!
 Regarde si je suis sur une mauvaise voie\dots{} \Fg{}
 (\ibibleverse{Ps}(139:23-24)).
 Ce qui compte, ce n'est pas ce que vous pensez de vous-mêmes,
 c'est\dots{} ce que Dieu pense de vous!

\dvrule

\dvprayer{
Père, merci pour Ton amour et Ta miséricorde envers-nous.
 Seigneur, aide-nous à reconnaître Ta vérité et à nous repentir.
}{\DlNdJ}



%%%%%%%%%%%%%%%%
% 27 septembre 
% M 2012/12/18
%%%%%%%%%%%%%%%%

\jrnlday{La plus grande des trois}

\themeindex{foi}
\themeindex{esperance@espérance}
\themeindex{amour}
\themeindex{Evangile@Évangile}

\dvquote{
Maintenant donc ces trois choses demeurent\frcolon{}
 la foi, l'espérance, l'amour;
 mais la plus grande, c'est l'amour.
}{\ibibleverse{ICo}(13:13)}

\lettrine[lhang=0.8]{V}{oici} les trois pieds sur lesquels
 repose la Foi Chrétienne\frcolon{}

La Foi \ocadr quand tout le reste s'effondre, la foi nous soutient \fcadr{}
 la foi que Dieu nous aime, la foi que Dieu est \Og aux commandes \Fg{}.

L'Espérance \ocadr l'espérance est l'anticipation d'une issue favorable.
 Alors que tout autour de nous semble morose et désespéré,
 comment se fait-il que les chrétiens puissent avoir une perspective
 joyeuse de l'avenir, de l'espoir pour le futur?
 C'est parce que nous disposons du \Og Guide de l'Utilisateur \Fg.
 Nous connaissons la fin de l'histoire.

\dvbox{
Le ciel et la terre passeront, mais la foi,
 l'espérance, et l'amour demeureront.
}

L'Amour \ocadr Paul a dit\frcolon{} \punct{deux-points}
 \Og La plus grande des trois, c'est l'amour. \Fg{}
 Le c\oe{}ur de l'Évangile chrétien, c'est l'amour.
 Nous sommes appelés à aimer Dieu de fa\c{c}on suprême
 et à aimer notre prochain comme nous nous aimons nous-mêmes.
 C'est l'essence du message chrétien.\typo{chrétien}
 Dieu veut que nous connaissions et que nous offrions
 l'amour \emph{agapé} \suggest{emph} décrit dans ce chapitre\frcolon{}
 \Og L'amour est patient, l'amour est serviable,
 il n'est pas envieux; l'amour ne se vante pas,
 il ne s'enfle pas d'orgueil, il ne fait rien de malhonnête,
 il ne cherche pas son intérêt, il ne s'irrite pas,
 il ne médite pas le mal, il ne se réjouit pas de l'injustice,
 mais il se réjouit de la vérité;
 il pardonne tout, il croit tout, il espère tout,
 il supporte tout. L'amour ne succombe jamais \Fg{}
 (\bibleverse{ICo}(13:4-8)). \punct{point}

Ce sont les trois pieds sur lesquels la Foi Chrétienne repose. \typo{capitales?}
 Même si tout le reste cesse d'exister, ces trois choses demeureront.
 En fait, le ciel et la terre passeront, mais ces choses resteront\frcolon{}
 la foi, l'espérance et l'amour.
 Mais la plus grande des trois, c'est l'amour.

\dvrule

\dvprayer{
Père, nous demandons la foi afin de croire en Toi.
 Nous demandons l'espérance afin d'attendre
 la réalisation de Tes promesses.
 Et nous demandons l'amour afin de pouvoir
 donner aux autres ce que Tu nous a donné.
}{\DlNdJ}


%%%%%%%%%%%%%%%%
% 28 septembre 
% M 2012/12/18
%%%%%%%%%%%%%%%%

\jrnlday{L'auteur de la paix}

\themeindex{paix}
\themeindex{desordre@désordre}
\themeindex{ordre}
\themeindex{confusion}
\themeindex{langues!don des \sim}
\index{don!\sim~des langues|see{langues}}
\themeindex{Esprit-Saint!dons de l'\sim}
\index{don!\sim{}s~de l'Esprit|see{Esprit-Saint}}
\themeindex{Bible}
\themeindex{amour}
\themeindex{pardon}

\dvquote{
Car Dieu n'est pas un Dieu de désordre, mais de paix,
 comme dans toutes les églises des saints.
}{\ibibleverse{ICo}(14:33)}

\lettrine{L}{es} choses étaient devenues incontrôlables
 à l'église de Corinthe.
 Les gens faisaient un mauvais usage du don des langues
 en parlant en langues sans interprétation.
 Ils s'interrompaient les uns les autres.
 Quelquefois ils parlaient en langues tous en même temps.
 Paul les a réprimandés pour la fa\c{c}on dont ils conduisaient
 leurs cultes dans le plus grand désordre en disant\frcolon{} \punct{deux-points}
 \Og Voyons, Dieu n'est pas l'auteur de la confusion! \Fg{}

Il vous suffit de regarder l'univers
 pour voir que Dieu est Dieu d'ordre.
 Et dans les descriptions que nous avons du ciel,
 nous pouvons voir que la louange s'y déroule d'une fa\c{c}on ordonnée.
 La louange ne devrait-elle donc pas aussi se dérouler
 d'une fa\c{c}on ordonnée dans l'église?

\dvbox{
Dieu est capable de faire ressortir l'ordre du chaos
 \ocadr même du chaos que nous créons dans nos vies personnelles.
}

Dieu n'est pas l'auteur de la confusion mais de la paix.
 La confusion, le désordre et les querelles remplissent nos vies
 quand nous insistons pour faire ce que nous voulons.
 Mais quand nous suivons les règles établies par Dieu pour nous dans Sa Parole
 \ocadr règles qui nous disent comment nous devons vivre \fcadr{}
 alors la paix de Dieu commence à remplir nos c\oe{}urs.
 Au lieu de vivre des vies désordonnées,
 nous pouvons alors vivre des vies paisibles,
 productives et remplies d'amour.

Le désordre peut nous rendre impatients et susceptibles.
 Mais quand la paix de Dieu commence à régner dans votre c\oe{}ur,
 Dieu est capable de vous aider à répondre aux autres avec amour,
 pardon, et dans la paix, et vous êtes ainsi capables de maintenir
 la paix avec ceux qui vous entourent.
 Le choix vous appartient \ocadr vous pouvez vivre dans la confusion et le chaos,
 ou bien vous pouvez vivre une vie caractérisée par la paix.

\dvrule

\dvprayer{
Père, nous Te remercions de ce que nous avons re\c{c}u Ta paix
 \ocadr une paix qui surpasse toute intelligence.
}{\DlNdJ}


%%%%%%%%%%%%%%%%
% 29 septembre 
% M 2012/12/20
%%%%%%%%%%%%%%%%

\jrnlday{Le Dieu de réconfort}

\themeindex{consolation}
\themeindex{tribulation}
\themeindex{souffrance}
\themeindex{Bible}
\themeindex{Dieu!amour de \sim}
\themeindex{confiance}
\themeindex{reconfort@réconfort}

\dvquote{
Béni soit le Dieu et Père de notre Seigneur Jésus-Christ,
 le Père compatissant et le Dieu de toute consolation,
 Lui qui nous console dans toutes nos afflictions,
 afin que, par la consolation que nous recevons nous-mêmes
 de la part de Dieu, nous puissions consoler
 ceux qui se trouvent dans toutes sortes d'afflictions!
}{\ibibleverse{IICo}(1:3-4)}

\lettrine{P}{aul} parle ouvertement des tribulations,
 des souffrances et des afflictions dont il a fait
 l'expérience en tant que disciple de Jésus-Christ.
 Il n'était pas à l'abri des ennuis.
 Les ennuis suivaient Paul où qu'il aille
 \ocadr tout cela parce qu'il proclamait Jésus.

\dvbox{
Quand les souffrances arrivent, nous avons un choix à faire.
}

Les \Og gens bien \Fg{} sont tout aussi sujets
 à la souffrance que les autres.
 Les Écritures l'affirment.
 Aussi quand les souffrances arrivent
 \ocadr et elles vont arriver \fcadr{}
 nous sommes placés devant un choix.
 Soit nous pouvons mépriser le caractère de Dieu
 en disant\frcolon{} \punct{deux-points}
 \Og Eh bien, si c'est comme \c{c}a que Dieu me traite,
 alors je ne crois plus qu'Il m'aime \Fg{},
 soit nous pouvons Lui faire confiance et nous attacher
 encore plus fort à Lui.
 Nous pouvons dire\frcolon{} \punct{deux-points}
 \Og Oh Dieu, je sais que Tu m'aimes vraiment.
 Je ne comprends pas ce qui se passe, Seigneur,
 mais je Te fais confiance. \Fg{}

Paul s'est rendu compte que Dieu pouvait utiliser
 ses souffrances pour lui donner de la sympathie
 pour ceux qui passaient par des expériences similaires
 et lui permettre ainsi de les servir avec compassion.
 La même chose est vraie pour nous.
 Quand nous passons par des épreuves ou des tragédies,
 nous comprenons ce que les autres ressentent
 quand ils doivent faire face à leurs propres pertes,
 et nous sommes mieux à même d'aller vers eux.

Les souffrances écrivent pour nous une histoire
 avec le Saint-Esprit.
 Nous apprenons qu'Il est Celui qui nous réconforte.
 Nous apprenons qu'Il est fidèle pour nous fortifier pour l'épreuve.
 Et nous, à notre tour, nous pouvons partager ce réconfort
 et cette force avec d'autres personnes qui en ont besoin.

\dvrule

\dvprayer{
Seigneur, aide-nous à comprendre la valeur d'une confiance
 totale à Ton égard.
 Puissions-nous accepter le ministère de Ton Saint-Esprit,
 et le partager avec les autres.
}{\DlNdJ}



%%%%%%%%%%%%%%%%
% 30 septembre 
% M 2012/12/20
%%%%%%%%%%%%%%%%

\jrnlday{Victoire en Christ}

\themeindex{victoire}
\themeindex{Dieu!enfant de \sim}
\themeindex{combat spirituel}
\themeindex{rebellion@rébellion}
\themeindex{chair}
\themeindex{Satan}
\index{diable|see{Satan}}
\themeindex{Bible}
\themeindex{renoncement}
\themeindex{Croix}
\themeindex{tentation}

\dvquote{
Grâces soient rendues à Dieu,
 qui nous fait toujours triompher en Christ,
 et qui par nous, répand en tout lieu l'odeur
 de sa connaissance!
}{\ibibleverse{IICo}(2:14)}

\lettrine{L}{e} mot \Og triomphe \Fg{}
 est un mot qui présuppose l'existence d'un conflit.
 En tant qu'enfants de Dieu,
 nous sommes engagés dans un conflit permanent.
 Il y a, fondamentalement, trois forces qui s'opposent
 à notre expérience chrétienne\frcolon{} le monde
 (le système du monde qui est en rébellion contre Dieu),
 notre chair (cette vieille nature déchue et corrompue
 avec les besoins biologiques qui contrôlent
 la vie d'une personne) et le diable.

\dvbox{
La victoire vient quand nous apprenons à demeurer en Christ.
}

Pourquoi est-il si difficile de vivre une vie de victoire en Christ?
 Eh bien, certains ne livrent pas un combat bien violent!
 Les Écritures disent\frcolon{} \punct{deux-points}
 \Og Résistez au diable et il fuira loin de vous \Fg{}
 (\ibibleverse{Jc}(4:7)).
 Mais à chaque fois que Satan se pointe, certains chrétiens rendent
 les armes sans vraiment avoir livré de combat.
 Ils cèdent sans aucune sorte de résistance.
 Jésus a dit que si vous voulez Le suivre vous devez renoncer à vous-mêmes,
 prendre votre croix et Le suivre.
 Mais la chair n'aime pas cela. Elle n'aime pas qu'on renonce à elle.
 Donc certains d'entre nous n'y renoncent pas du tout.

La victoire vient quand nous apprenons à faire confiance à Christ
 pour nous aider à gérer les tentations que l'ennemi peut nous lancer.
 La victoire vient quand nous Lui demandons la force
 que nous n'avons pas en nous-mêmes, une puissance au-delà
 de nos propres capacités.
 Comme Jésus nous l'a dit, nous devons demeurer en Lui
 et continuer à demeurer en Lui, car sans Lui, nous ne pouvons rien faire.
 Alors que nous demeurons en Lui et qu'Il demeure en nous,
 la présence résidente de Son Saint-Esprit
 devient le secret d'une victoire permanente.

\dvrule

\dvprayer{
Père, comme nous Te sommes reconnaissants pour la victoire
 que nous avons en Christ \ocadr pour la puissance
 que Tu nous a donnée pour vaincre le monde, la chair et le diable.
}{\DlNdJ}




