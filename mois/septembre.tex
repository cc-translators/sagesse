\dvmonth{Septembre}

%%%%%%%%%%%%%%%%
% 1er septembre 
%%%%%%%%%%%%%%%%

\dvday{Chants dans la Nuit}

\dvquote{
Vers le milieu de la nuit, Paul et Silas priaient et chantaient
 les louanges de Dieu, et les prisonniers les écoutaient.
}{\ibibleverse{Ac}(16:25)}

\lettrine{A}{lors} qu'ils se trouvaient à Philippes,
 Paul et Silas furent dépouillés de leurs vêtements,
 battus et mis aux fers au fin fond de la prison.
 Pouvez-vous imaginer une situation désespérée plus horrible?

Que feriez-vous dans de telles circonstances? À quoi penseriez-vous?
 Tout ce que vous vouliez, c'était servir Dieu.
 Vous aviez ressenti l'appel à amener l'Évangile en Macédoine,
 et maintenant votre dos est en proie
 à des spasmes lancinants; \grammar{lancinants} vos mains,
 vos pieds et votre cou sont enserrés dans des entraves.
 Votre futur est incertain. Vous ne connaissez pas la gravité
 des chefs d'accusation et vous ne savez pas combien de temps
 on va vous garder en prison.

Dans cette situation très désagréable, Paul et Silas
 ont fait ce qui leur semblait comme la seule chose sensée à faire~:
 ils ont prié et chanté des louanges à Dieu.
 Bien que leurs corps soient liés, leurs esprits étaient libres.
 Au lieu de gémir et de se plaindre, ils ont fait de leur sombre cachot
 une maison de louange. Se réjouir ainsi leur a remonté le moral.
 Plus Paul et Silas se concentraient sur le Seigneur,
 et plus leurs problèmes rapetissaient.

\dvbox{
Les chants détournent nos pensées de nous-mêmes
 et les concentrent sur le Seigneur.
}

La prochaine fois que vous vous retrouvez enchaînés par des soucis,
 de la confusion ou de la douleur, levez les yeux.
 Détachez vos regards de votre problème et portez-les sur Dieu.
 Chantez à votre Créateur \ocadr et regardez votre sombre prison
 devenir une maison de louange.

\dvrule

\dvprayer{
Père, nous Te remercions pour les chants que Tu mets dans nos c\oe{}urs
 \ocadr chants de joie, de bénédiction et de réjouissance en Ta bonté.
}{\DlNdJ}


%%%%%%%%%%%%%%%%
% 2 septembre 
%%%%%%%%%%%%%%%%

\dvday{Le Remède à la Peur}

\dvquote{
Le Seigneur dit à Paul en vision pendant la nuit~:
 Sois sans crainte, mais parle et ne te tais pas, car Moi,
 Je suis avec toi, et personne ne mettra la main sur toi
 pour te faire du mal~: parce que J'ai un peuple nombreux
 dans cette ville.
}{\ibibleverse{Ac}(18:9-10)}

\lettrine{P}{artout} où Paul prêchait Jésus-Christ,
 les gens se soulevaient contre lui. Il avait été battu, emprisonné et lapidé.
 À Thessalonique, il avait dû quitter furtivement la ville de nuit.
 Les habitants de Bérée l'avaient chassé de leur ville.
 Mais ces endroits étaient encore assez repectueux des lois.
 Corinthe par contre \ocadr la ville où Paul se trouve maintenant \fcadr{} est
 une ville violente et immorale. Paul était sûrement anxieux.
 Il se faisait sûrement du souci au sujet des Juifs qui détenaient
 son sort entre leurs mains. C'est dans cet état de peur que Jésus
 a parlé à Paul dans une vision.
 \og Sois sans crainte [\dots{}] car Je suis avec toi. \fg{}

\dvbox{
Là, se trouve le remède contre la peur
 \ocadr la réalisation de la présence de Dieu.
}

Paul a écrit aux Romains~: \punct{deux-points}
 \og Si Dieu est pour nous, qui sera contre nous? \fg{} (\ibibleverse{Rm}(8:31)),
 David a dit~: \punct{deux-points}
 \og L'Éternel est pour moi, je ne crains rien~:
 Que peuvent me faire des hommes \fg{} (\ibibleverse{Ps}(118:6)).
 Le Seigneur a dit au prophète Ésaïe~: \punct{deux-points}
 \og Sois sans crainte, car Je suis avec toi; n'ouvre pas des yeux inquiets,
 car Je suis ton Dieu ; Je te fortifie, Je viens à ton secours,
 Je te soutiens de Ma droite victorieuse \fg{} (\ibibleverse{Is}(41:10)).

Traversez-vous des moments d'incertitude? Vous inquiétez-vous pour l'avenir?
 Souvenez-vous de ces promesses. Souvenez-vous de Celui qui a dit~: \punct{deux-points}
 \og Je ne te délaisserai jamais, je ne t'abandonnerai jamais \fg{}
 (\ibibleverse{He}(13:5)).
 Vous ne faites jamais face à rien tout seul. Votre Dieu est avec vous.

\dvrule

\dvprayer{
Père, nous Te remercions pour le réconfort de Ta présence
 avec nous au milieu des nuits noires, au milieu de ces heures
 de découragement et dans les moments de peur.
}{\Amen}


%%%%%%%%%%%%%%%%
% 3 septembre 
%%%%%%%%%%%%%%%%

\dvday{Acceptation}

\dvquote{
Comme il ne se laissait pas persuader,
 nous n'avons plus insisté et nous avons dit~:
 Que la volonté du Seigneur se fasse!
}{\ibibleverse{Ac}(21:14)}

\lettrine{P}{artout} où Paul allait, le Saint-Esprit l'avertissait
 que des souffrances, des liens et des emprisonnements l'attendaient
 à Jérusalem. Ayant entendu ces prophéties, les amis de Paul pleurèrent
 et le supplièrent de ne pas s'y rendre. Mais il leur répondit~:
 \og Moi, je suis prêt, non seulement à être lié,
 mais encore à mourir à Jérusalem pour le nom du Seigneur Jésus \fg{}
 (\ibibleverse{Ac}(21:13)). Voyant sa détermination, ils dirent~: \punct{deux-points}
 \og Que la volonté du Seigneur soit faite. \fg{}

\dvbox{
L'acceptation de la volonté de Dieu est le seul chemin qui mène
 à la paix véritable. Sans cette acceptation, il n'y a qu'agitation
 et luttes intérieures.
}

Quelquefois, les gens sont réticents à se soumettre à la volonté de Dieu,
 parce qu'ils ont peur qu'Il les force à faire quelque chose
 qu'ils ne veulent pas faire. Mais ce n'est pas ainsi que Dieu agit.
 Dieu révèle magnifiquement Sa volonté en alignant les désirs de notre
 c\oe{}ur avec les Siens.

Les plans de Dieu pour nos vies sont bien supérieurs à tout
 ce que nous pourrions jamais imaginer par nous-mêmes.
 Il voit au-delà de nos épreuves le bien éternel qu'Il va accomplir
 par ces épreuves. Pour Dieu, la fin justifie les moyens.
 Si quelques épreuves et souffrances présentes vont accomplir
 une bonne fin dans nos vies, alors Dieu va permettre
 ces souffrances momentanées.

Comme David le disait~: \punct{deux-points}
 \og Je prends plaisir à faire Ta volonté, mon Dieu! \fg{}
 (\ibibleverse{Ps}(40:8)). Quand nous arrivons vraiment à connaître Dieu,
 Sa volonté devient notre volonté; Ce qui Lui fait plaisir
 devient ce qui nous fait plaisir.

\dvrule

\dvprayer{
Père, nous avons tendance à rechercher le chemin plus facile,
 le chemin présentant le moins de souffrances.
 Mais Tu sais ce qui est le mieux pour nous.
 Nous T'offrons nos désirs pour que Tu les raffines
 et que Tu en fasses les Tiens.
}{\DlNdJ}


%%%%%%%%%%%%%%%%
% 4 septembre 
%%%%%%%%%%%%%%%%

\dvday{Connaître la Volonté de~Dieu}

\dvquote{
Il dit~: \og Le Dieu de nos pères t'a destiné
 à connaître Sa volonté\dots{} \fg{}
}{\ibibleverse{Ac}(22:14)}

\lettrine{À}{ Jérusalem}, Paul a raconté aux Juifs sa conversion
 sur la route de Damas, comment il est devenu momentanément aveugle
 en raison d'une brillante lumière, et comment il a recouvré la vue
 quand Ananias est venu lui imposer les mains.
 Ananias a alors dit à Paul que Dieu l'avait choisi afin qu'il connaisse
 la volonté de Dieu.

Ceci s'applique aussi bien à vous et à moi.
 Parce qu'Il vous aime, Dieu vous a choisi afin que vous connaissiez Sa volonté.

Savez-vous ce que Dieu a prévu et planifié pour vous?
 Ceci devrait avoir une grande importance pour vous,
 parce que tout ce que vous faites en dehors de la volonté de Dieu disparaîtra.

\dvbox{
Connaissez-vous la volonté de Dieu pour votre vie?
}

Nous pouvons devenir si absorbés à juste survivre dans notre existence
 quotidienne, que nous ne prenons pas l'éternité en considération.
 Mais vous n'avez qu'une vie à vivre, qui sera bientôt passée
 et seulement ce que vous faites pour Christ durera.

Comment pouvez-vous connaître la volonté de Dieu pour votre vie
 de façon spécifique? Vous découvrez le dessein spécifique et individuel
 de Dieu pour vous en étant \og transformés par le renouvellement
 de votre intelligence, afin que vous discerniez quelle est la volonté de Dieu~:
 ce qui est bon, agréable et parfait \fg{} (\ibibleverse{Rm}(12:2)).
 Quand vous offrez votre corps à Dieu comme un instrument désireux
 d'être utilisé pour Son \oe{}uvre, quand vous désirez les plans de Dieu
 au-dessus des vôtres, alors votre vie va devenir une manifestation
 progressive de la volonté de Dieu.


\dvrule

\dvprayer{
Père, merci de nous avoir choisis pour passer l'éternité avec Toi.
 Aide-nous à voir nos vies avec l'éternité en vue et à toujours chercher
 à faire Ta volonté.
}{\Amen}


%%%%%%%%%%%%%%%%
% 5 septembre 
%%%%%%%%%%%%%%%%

\dvday{Dans la Tempête}

\dvquote{
Mais bientôt après, venant de l'île, un vent de tempête
 appelé Euraquilon se déchaîna.
}{\ibibleverse{Ac}(27:14)}

\lettrine{B}{ien} que ce ne soit pas du tout de sa faute,
 Paul s'est retrouvé au beau milieu d'une violente tempête.
 Il avait prévenu le capitaine du navire de ne pas appareiller.
 Mais le capitaine ne l'avait pas écouté.

Parfois, nous pensons à tort que, parce que nous servons le Seigneur,
 nous devrions bénéficier d'un temps au beau fixe pour tout le voyage.
 À coup sûr, le Seigneur va calmer la mer pour nous,
 et envoyer un vent léger gonfler nos voiles. Pas du tout!
 Jésus ne promets pas de vous épargner les tempêtes, par contre,
 Il a promis d'être avec vous dans la tempête.

\dvbox{
Les tempêtes servent à quelque chose \ocadr elles servent le dessein de Dieu.
}

Quand ces tempêtes éclatent, nous nous demandons si nous allons survivre.
 Paul s'est probablement demandé la même chose.
 Mais le Seigneur, présent à ses côtés pendant la tempête,
 a encouragé Paul par des paroles. Il a dit à Paul qu'il allait survivre,
 parce que Dieu avait une mission pour lui.

Le but réel de la tempête de Paul n'allait être révélé que beaucoup plus tard.
 Et c'est souvent le cas dans nos vies. Quand nos mers se déchaînent,
 nous questionnons la probabilité de notre survie.
 C'est alors que nous devons nous souvenir des paroles du Seigneur~:
 \og Sois sans crainte \fg{} (\ibibleverse{Ac}(27:24)).
 En d'autres termes~: \punct{deux-points}
 \og Courage. Ce n'est pas la fin. J'ai un plan pour toi. \fg{}

Il a vraiment un plan. Il ne vous a pas oubliés \ocadr en fait,
 Il est juste là, avec vous, et Il vous maintient à flots.
 Et quand la tempête est passée et que les nuages s'ouvrent de nouveau,
 vous verrez la raison de la tempête.

\dvrule

\dvprayer{
Père, nous Te remercions de ce que Tu es avec nous dans la tempête.
 Car Tu as promis que Tu ne nous délaisserais jamais
 et que Tu ne nous abandonnerais jamais.
 Utilise ces tempêtes pour Tes desseins et pour Ta gloire.
}{\DlNdJ}


%%%%%%%%%%%%%%%%
% 6 septembre 
%%%%%%%%%%%%%%%%

\dvday{La Bonne Nouvelle de~Dieu}

\dvquote{
Car je n'ai pas honte de l'Évangile~:
 c'est une puissance de Dieu pour le salut de quiconque croit,
 du Juif premièrement, puis du Grec.
}{\ibibleverse{Rm}(1:16)}

\lettrine{D}{ans} ses jeunes années, Paul était enfermé
 dans un système religieux où l'homme cherchait à être justifié
 \NdT{c'est-à-dire \og déclaré droit \fg{}.}
 devant Dieu en respectant la loi. \suggest{Loi}
 Mais la loi \suggest{Loi} n'avait jamais été prévue pour rendre
 l'homme juste devant Dieu. La loi était destinée à montrer
 à l'homme quel grand pécheur il était, et à faire que le monde entier
 se sache coupable devant Dieu. Aussi lorsque Paul est arrivé
 à connaître la vérité \ocadr quand il a rencontré Jésus-Christ
 et trouvé le salut fondé sur Sa justice et non sur la justice
 de l'homme \fcadr{} Paul a allègrement rejeté les fausses notions
 de bonnes \oe{}uvres, et il est devenu impatient de faire part
 de cette bonne nouvelle à tout le monde.

\dvbox{
L'Évangile libère les hommes.
}

\og C'est la puissance de Dieu pour le salut. \fg{}
 Quelle joie que de voir des vies transformées par cette bonne nouvelle.
 Elle fait briller la lumière dans l'obscurité et brise les liens
 qui asservissent au péché.

Cette puissance, cette bonne nouvelle du salut, n'est pas réservée
 exclusivement aux Juifs mais elle est destinée à toute personne qui croie.
 C'est pour le monde entier.

Oh, quel message glorieux d'espoir et de salut, \punct{Virgule pas nécessaire}
 nous avons reçu!  Puissions-nous ne pas garder cette nouvelle
 pour nous-mêmes; puissions-nous ne jamais avoir honte de la vérité
 que nous avons reçue. Puissions-nous, comme Paul, être prêts à proclamer
 l'Évangile \suggest{Évangile} de Jésus-Christ au monde en grand besoin
 dans lequel nous vivons.

\dvrule

\dvprayer{
Père, nous Te remercions pour ce glorieux Évangile
 par lequel nous avons été nettoyés de nos péchés.
 Nous prions pour ceux qui ne connaissent pas encore
 cette merveilleuse bonne nouvelle de Ton amour
 et de Ton offre de pardon.
}{\DlNdJ}


%%%%%%%%%%%%%%%%
% 7 septembre 
%%%%%%%%%%%%%%%%

\dvday{Une Affaire de~C\oe{}ur}

\dvquote{
Le Juif, ce n'est pas celui qui en a les apparences ;
 et la circoncision, ce n'est pas celle qui est apparente dans la chair.
 Mais le Juif, c'est celui qui l'est intérieurement ;
 et la circoncision, c'est celle du cœur, selon l'esprit
 et non selon la lettre. La louange de ce Juif ne vient pas des hommes,
 mais de Dieu.
}{\ibibleverse{Rm}(2:28-29)}

\lettrine{V}{ous} pouvez dire toutes les choses qu'il faut dire
 et pourtant être loin de Dieu. C'est parce que Dieu s'intéresse au c\oe{}ur.

Extérieurement, les Juifs respectaient la loi. Mais intérieurement,
 ils la violaient. Dieu a abordé cette contradiction par l'intermédiaire
 du prophète Ésaïe, qui leur a dit~: \punct{deux-points}
 \og Ainsi quand ce peuple s'approche, il me glorifie de la bouche
 et des lèvres ; mais son cœur est éloigné de moi, et la crainte
 qu'il a de moi n'est qu'un commandement de tradition humaine \fg{}
 (\ibibleverse{Is}(29:13)). \punct{Point}

\dvbox{
Si vous continuez a vivre pour la chair, aucun rite ne peut vous sauver.
}

Ce que Paul dit aux Juifs, c'est que la circoncision n'a aucune valeur
 si vous continuez à vivre pour la chair.

Ceci est vrai aussi dans l'Église. \suggest{Église}
 Certains font confiance au rite du baptême pour leur salut,
 plutôt qu'à une relation vivante avec Jésus.
 Mais le baptême n'est qu'un symbole qui représente la vieille vie
 enterrée et la nouvelle vie en train d'être relevée.

Vous pouvez même aller à l'église, vous pouvez chanter tous les chants,
 vous pouvez connaître la Parole et dire un \og Amen \fg{} de temps en temps.
 Mais rien de tout ça ne fait de vous un enfant de Dieu.
 Il voit votre c\oe{}ur. Il sait si vous deux vivez et appréciez
 la relation qui unit un parent à un enfant.

\dvrule

\dvprayer{
Père, aide-nous à avoir un c\oe{}ur pur et à \suggest{à T'aimer}
 T'aimer de tout notre c\oe{}ur.
}{\DlNdJ}



