\dvmonth{Août}

%%%%%%%%%%%%%%
% 1er aout
%%%%%%%%%%%%%%

\dvday{Foi Aveugle}

\dvquote{
Simon lui répondit~:
 Maître, nous avons travaillé toute la nuit sans rien prendre,
 mais, sur Ta parole, je jetterai les filets.
}{\ibibleverse{Lc}(5:5)}

\lettrine{J}{ésus} avait dit à Simon~:
 \og Avance en eau profonde, et jetez vos filets pour pêcher \fg{}
 (\ibibleverse{Lc}(5:4)).

Mais ils venaient de passer toute la nuit à pêcher.
 Et ces pêcheurs expérimentés n'avaient rien pris.
 Vous pouvez presque entendre le ton poli mais condescendant
 de la voix de Pierre. Il aurait pu tout aussi bien dire~: \punct{deux-points}
 \og Tu es un bon enseignant, mais c'est moi le pêcheur.
 Tu sais sans doute plein de choses sur Dieu, mais le poisson,
 ça, c'est mon affaire. \fg{}

\dvbox{
La foi, c'est obéir sans tenir compte de notre compréhension des choses.
}

Mais Pierre n'a pas répondu de cette façon.
 Au contraire, il a dit~: \punct{deux-points}
 \og Mais, sur Ta parole, je jetterai les filets. \fg{}
 Et parce qu'il a obéi aveuglément \ocadr non pas parce que la requête était
 logique mais parce qu'elle venait de Jésus \fcadr{}
 ils ont remonté un filet tellement plein de poissons,
 qu'il s'est pratiquement déchiré.

Nos meilleurs efforts peuvent mener à des filets vides.
 Mais l'obéissance peut amener le succès au-delà de nos rêves les plus fous.
 Il y a eu de nombreuses années dans mon ministère où j'ai jeté mes filets
 et n'ai rien ramené. Mais quand le Seigneur décida de remplir les filets,
 ils furent soudain remplis jusqu'à en déborder.
 Cela n'avait rien à voir avec mes propres capacités ou ma propre puissance.
 Cela s'est produit simplement parce que Dieu l'avait décidé.
 Et toute la gloire Lui en revient.

Ce que Dieu vous demande peut sembler contraire à votre logique.
 Mais si vous écoutez Ses instructions et agissez en obéissant,
 vous trouverez la différence entre l'échec et le succès.
 \og Mais, sur Ta parole, je le ferai. \fg{}
 Que ces mots soient vrais dans nos vies.

\dvrule

\dvprayer{
Père, fais de nous des enfants si obéisssants qu'au moindre chuchotement
 de Ta part, nous agissions. Apprends-nous à faire confiance
 à Ta voix plutôt qu'à la nôtre.
}{\Amen}


%%%%%%%%%%%%%%
% 2 aout
%%%%%%%%%%%%%%

\dvday{La Vie Équilibrée}

\dvquote{
Le Seigneur lui répondit~: Marthe, Marthe, tu t'inquiètes
 et tu t'agites pour beaucoup de choses.
 Or une seule chose est nécessaire. Marie a choisi la bonne part,
 qui ne lui sera pas ôtée.
}{\ibibleverse{Lc}(10:41-42)}

\lettrine{P}{réoccupée} par tous les petits détails du dîner,
 Marthe a passé la soirée, en s'irritant de son travail et en passant
 à côté de la tâche la plus importante \ocadr s'asseoir aux pieds de Jésus.
 Elle était tellement occupée à servir Jésus,
 qu'elle n'avait plus de temps à passer avec Jésus.

Servir le Seigneur est une chose merveilleuse et nécessaire.
 Cependant ce service devrait venir d'une position de joie et d'excitation.
 Nous ne devrions jamais nous retrouver à nous plaindre à Jésus
 des choses que nous faisons pour Lui.

\dvbox{
Si vous servez Jésus, vous devez d'abord L'adorer.
}

À la différence de Marthe, Marie avait choisi \og la bonne part \fg{}.
 La bonne part, c'était l'intimité avec Jésus.
 Pendant que Marthe se démenait dans la cuisine,
 Marie s'était assise à Ses pieds et buvait chacunes de Ses paroles.

Puisque le service et l'adoration sont tous les deux des ingrédients
 indispensables à la vie d'un croyant, nous parlons en fait
 du besoin d'équilibre. La vie bien équilibrée, c'est celle qui manifeste
 l'amour pour le Seigneur au travers du service pour le Seigneur,
 mais qui laisse aussi du temps pour l'intimité et l'adoration.
 On a besoin des deux. Votre vie est-elle équilibrée?
 Servez-vous par amour ou par obligation?

Si vous voulez Le servir, vous devez d'abord L'adorer.
 C'est quand nous L'adorons, assis à Ses pieds, que nous obtenons
 la force qu'il nous faut pour vivre.
 Quand nous L'adorons assis devant Lui,
 Il nous dirige alors dans nos activités.
 \typo{Majuscules dans le paragraphe}

\dvrule

\dvprayer{
Père, merci de ce que chaque fois que nous nous venons nous mettre
 à Tes pieds, Tu nous y attends. Enseigne-nous Seigneur.
 Dis-nous comment nous pouvons Te servir parce que nous voulons Te bénir.
}{\DlNdJ}


%%%%%%%%%%%%%%
% 3 aout
%%%%%%%%%%%%%%

\dvday{L'Abondance des Choses}

\dvquote{
Puis il leur dit~:
 Gardez-vous attentivement de toute cupidité ;
 car même dans l'abondance, la vie d'un homme ne dépend pas
 de ce qu'il possède.
}{\ibibleverse{Lc}(12:15)}

\lettrine{D}{eux frères} se disputaient au sujet de leur héritage.
 L'un d'eux pensait que l'autre essayait de le tromper,
 aussi avait-il demandé à Jésus d'intervenir.
 Mais Jésus refusa de se mêler d'une querelle portant
 sur des choses matérielles. Il choisit plutôt de les mettre en garde
 contre la cupidité. Après leur avoir fait part d'une parabole
 et souligné le soin que Dieu accorde aux corbeaux et aux lys des champs,
 Il leur dit, en conclusion~: \punct{deux-points}
 \og Cherchez plutôt son royaume ; et cela vous sera donné par surcroît \fg{}
 (\ibibleverse{Lc}(12:31)). 

\dvbox{
Quand vous mettez Dieu en premier, Il prend soin de tout le reste.
}

C'est ahurissant de voir combien le vie des gens est altérée
 et déséquilibrée quand ils ne vivent que pour les choses matérielles.
 Mais quand des personnes commencent à suivre les choses de l'Esprit,
 leur vie s'équilibre. De plus, ils acquièrent la paix,
 le contentement et une joie qu'ils n'avaient pas aupararavant.

Jésus a dit~: \punct{deux-points, majuscule}
 \og Votre Père a trouvé bon de vous donner le royaume \fg{}
 (\ibibleverse{Lc}(12:32)). Mais est-ce à cela que vous accordez
 de la valeur? Votre intérêt se porte-t-il sur les choses éternelles
 ou seulement sur les choses temporelles, matérielles
 \ocadr ces choses qui ne font que passer?

Vous avez peut-être beaucoup de succès. Mais peu importe combien
 vous pouvez amasser, votre vie ne consiste pas à courir après
 les appétits charnels. Elle ne consiste pas dans l'abondance
 de vos possessions mais dans une vraie relation avec Dieu
 quand votre esprit s'est ouvert à la vie avec Lui par la foi en Jésus.

\dvrule

\dvprayer{
Père, aide-nous à ne pas apprendre les leçons que nous devons connaître
 de la façon la plus difficile. Aide-nous à vivre notre vie
 avec des mains ouvertes, sans étreindre ce monde,
 mais en les tendant vers Ton royaume.
}{\Amen}


%%%%%%%%%%%%%%
% 4 aout
%%%%%%%%%%%%%%

\dvday{Le Chemin vers le Haut}

\dvquote{
En effet quiconque s'élève sera abaissé,
 et celui qui s'abaisse sera élevé.
}{\ibibleverse{Lc}(14:11)}

\lettrine{D}{ans la maison} du Pharisien chez qui Il avait été invité à dîner,
 Jésus observait les gens manœuvrer pour obtenir les places
 d'importance autour de la table. Finalement, Il leur expliqua pourquoi
 ça n'était pas sage.
 \og Que se passerait-il si une personne plus honorable que vous arrivait
 et que celui qui vous a invité doive vous dire~: \punct{deux-points}
 \og Désolé, mon ami, vous êtes assis au mauvais endroit.
 Pourquoi n'allez-vous pas vous asseoir au bout de la table? \fg{}
 Jésus ajouta alors~: \punct{deux-points}
 \og Ne vous mettez pas dans une situation aussi embarrassante.
 Il vaut mieux que vous preniez la dernière place, et que vous laissiez
 l'hôte venir vous dire~: \punct{deux-points}
 \og Pourquoi êtes-vous assis si loin?
 J'aimerais que vous preniez une meilleure place. \fg{}
 Jésus conclua en disant~:
 \og En effet quiconque s'élève sera abaissé,
 et celui qui s'abaisse sera élevé \fg{} (\bibleverse{Lc}(14:11)).

\dvbox{
Spirituellement parlant, le chemin vers le haut est celui qui va vers le bas,
 et le chemin vers le bas est celui qui va vers le haut.
}

Si vous vous abaissez, vous serez élevés.
 Prenez la dernière place et vous serez invités à venir plus haut.
 Mais si vous prenez la plus haute place,
 il y a de fortes chances qu'on vous demande d'en descendre.

Puisse Dieu nous aider à vivre nos vies comme des serviteurs
 \ocadr conscients des besoins des autres et désireux d'aller vers eux
 et de les servir de toutes les façons possibles.
 Puissions-nous suivre l'exemple de notre Seigneur,
 qui n'a pas pensé à Son propre bien-être mais a vécu
 dans l'humilité une vie au service de tous.

\dvrule

\dvprayer{
Seigneur, nous voyons comment Tu as laissé de côté Ta gloire
 pour venir sur la terre, vivre dans l'humilité et mourir pour nous.
 Nous voyons que Tu as été maintenant élevé à la plus haute place;
 que Tu es le Roi des rois et le Seigneur des seigneurs.
 Apprends-nous à suivre Ton exemple.
}{\DlNdJ}


%%%%%%%%%%%%%%
% 5 aout
%%%%%%%%%%%%%%

\dvday{L'homme le plus riche}

\dvquote{
Si donc vous n'avez pas été fidèles dans les richesses injustes,
 qui vous confiera le (bien) véritable ?
}{\ibibleverse{Lc}(16:11)}

\lettrine{J}{ésus} a parlé d'un serviteur qui avait commencé
 à dissiper les biens de son maître à son profit.
 Quand le maître eut vent de ces rumeurs, il demanda à son serviteur
 de lui rendre des comptes. Le serviteur se doutant qu'il allait bientôt
 se retrouver sans travail, a rapidement utilisé les ressources
 de son maître pour préparer le futur qu'il voyait arriver.

Imaginons qu'il vous reste cinquante ans pour apprécier les bénédictions
 dont Dieu vous a comblées. Si vous utilisez ces choses seulement
 pour vous-mêmes sans égards pour le royaume de Dieu et le futur éternel,
 vous allez le regretter. Bien que vous soyez aisés maintenant,
 vous allez passer l'éternité dans la misère.

Permettez-moi de vous assurer que l'homme le plus pauvre au paradis
 est bien plus riche que l'homme le plus riche en enfer.
 Dieu va un jour demander à chacun de nous de Lui rendre des comptes
 sur la façon dont nous aurons utilisé les choses qu'Il a mises
 \grammar{choses qu'il a mises}
 à notre disposition. Si vous êtes sages,
 vous les utiliserez pour le royaume de Dieu.

Le ciel et la terre renferment tous les deux des richesses.
 Les richesses de la terre se mesurent en comptes en banque,
 portefeuilles de valeurs, amas de possessions que les hommes passent
 leurs vies à accumuler. Ces richesses vont finir par brûler si les mites
 et la rouille ne les détruisent pas avant. Les richesses du ciel sont,
 par contre, éternelles. Aucun feu ne peut les toucher.
 Aucune rouille ni aucune mite ne peuvent les endommager.
 Aucun voleur ne peut les dérober. Quoi que vous ayez déposé dans le ciel
 vous y attendra, précisément là, pour le jour où vous passerez ces portes.

\dvbox{
Investissez vos richesses soigneusement.
}

\dvrule

\dvprayer{
Père, fais de nous de bons gérants de ce que Tu nous a confié.
 Donne-nous Ta perspective sur ce que sont les vraies richesses.
}{\DlNdJ}


%%%%%%%%%%%%%%
% 6 aout
%%%%%%%%%%%%%%

\dvday{Les Neuf}

\dvquote{
L'un d'eux, se voyant guéri, revint sur ses pas et glorifia Dieu à haute voix.
 Il tomba face contre terre aux pieds de Jésus et lui rendit grâces.
 C'était un Samaritain. Jésus prit la parole et dit~:
 Les dix n'ont-ils pas été purifiés ? Mais les neuf autres, où sont-ils ?
}{\ibibleverse{Lc}(17:15-17)}

\lettrine{S}{i vous étiez} en train de mourir sans plus aucun espoir,
 et que quelqu'un arrivait et vous guérissait, ne croyez-vous pas
 que le moins que vous vouliez faire serait de dire merci?
 Jésus a guéri dix lépreux. Mais un seul est revenu exprimer sa gratitude.

Il est facile de considérer cette histoire et de secouer la tête
 devant l'ingratitude des neuf. Mais ne faisons nous pas, parfois,
 partie des neuf? Combien de fois Dieu nous a-t-Il bénis,
 nous nous sommes saisis de la bénédiction et \suggest{(nous) sommes repartis}
 reparti en courant sans regarder en arrière ni vers le ciel?
 Combien de fois nous a-t-Il permis de l'échapper belle,
 et nous avons continué nottre chemin sans le remercier?

\dvbox{
Beaucoup de gens ont vite fait de blâmer Dieu pour tous les problemes,
 au lieu de Le remercier.
}

\og Qu'ils célèbrent l'Éternel pour sa bienveillance
 et pour ses merveilles en faveur des humains ! \fg{}
 (\ibibleverse{Ps}(107:8)).
 Dieu est si bon envers nous, si digne de notre louange.

Les neuf ont vite fait de blâmer Dieu pour les choses qui ne vont pas
 dans leur vie, mais pas de Le remercier pour les choses qui vont bien.

Les neuf sont comme le garçon qui tombe en glissant du toit
 et qui crie~: \punct{deux-points}
 \og Oh Seigneur, aide-moi! \fg{} Quand ses pantalons s'accrochent à un clou,
 il s'arrête brutalement dans un grand bruit de déchirure.
 Levant les yeux vers le ciel, il dit alors~: \punct{deux-points}
 \og C'est plus la peine, Seigneur, le clou m'a arrêté. \fg{}

Êtes-vous avec l'un ou avec les neuf?

\dvrule

\dvprayer{
Père, nous venons dans le nom de Jésus Te remercier aujourd'hui
 pour Ta bonté et pour Tes œuvres merveilleuses pour nous.
}{}


%%%%%%%%%%%%%%
% 7 aout
%%%%%%%%%%%%%%

\dvday{Veillez et Priez}

\dvquote{
Veillez donc, et priez sans cesse, afin que vous soyez estimés
 dignes d’échapper à toutes ces choses qui arriveront,
 et de vous tenir debout devant le Fils d’homme.
}{\ibibleverse{Lc}(21:36) \KJF}

\lettrine{D}{ieu} a été patient et endurant.
 Il a supporté énormément de mauvais traitements,
 mais le jour vient où Dieu va prendre Sa revanche.

Jésus venait juste de décrire pour Ses disciples la période
 de la grande tribulation, ce temps où Dieu va déchaîner les forces
 de la nature. Des signes cataclysmiques dans les cieux vont provoquer famine,
 épidémies et terribles tremblements de terre.
 Mais immédiatement après ces événements, Jésus ajouta~: \punct{deux-points}
 \og Alors on verra le Fils de l'homme venir sur une nuée avec beaucoup
 de puissance et de gloire \fg{} (\ibibleverse{Lc}(21:27)).

\dvbox{
La seule façon d'être trouvé digne d'échapper à la grande tribulation,
 c'est de recevoir Jésus-Christ et le pardon qu'Il offre.
}

Premièrement, que nous soyons jugés dignes d'échapper à la calamité
 qui arrive, et deuxièmement de pouvoir nous tenir debout
 devant le Fils de l'Homme.

Les enfants de Dieu n'ont pas été destinés à la colère mais se tiendront
 en glorieuse companie dans le ciel pour chanter que l'Agneau
 est digne de prendre le livre et d'en ouvrir les sceaux.

Jésus a bien averti au verset~\ibiblevs{Lc}(21:34)~: \punct{deux-points}
 \og Prenez donc garde à vous-mêmes, de peur que vos cœurs ne soient
 alourdis par l’excès, et l’ivrognerie et les soucis de cette vie ;
 et qu'ainsi ce jour-là ne vous surprenne à l’improviste \fg{}
 (\textit{King James Française}).
 Ne vous laissez pas distraire par la vie.
 Ne soyez pas pris au dépourvu par Son retour.

Veillez et priez \ocadr sans cesse.

\dvrule

\dvprayer{
Seigneur, notre espoir est bâti sur rien de moins que le sang de Jésus
 et Sa justice. Nous prions~: Délivre-nous de la colère à venir.
}{\DlNdJ}


%%%%%%%%%%%%%%
% 8 aout
%%%%%%%%%%%%%%

\dvday{Une Foi Infaillible}

\dvquote{
Simon, Simon, Satan vous a réclamés pour vous passer au crible comme le blé.
 Mais j'ai prié pour toi, afin que ta foi ne défaille pas, et toi,
 quand tu seras revenu à moi, affermis tes frères.
}{\ibibleverse{Lc}(22:31-32)}

\lettrine{J}{ésus} avait choisi Pierre pour être un responsable
 dans l'église \suggest{Église}
 et donc Satan désirait le détruire.
 Satan vise toujours les responsables dans l'église, \suggest{Église}
 parce qu'il sait que s'il peut détruire un responsable,
 il détruit beaucoup de monde.

Mais Jésus n'a pas prié pour que Satan laisse Pierre tranquille.
 Au lieu de ça, Il a prié pour que la foi de Pierre ne défaille pas.
 Nous aurions probablement prié~: \og Seigneur, évite à Pierre le tamis.
 Ne le laisse pas avoir de problèmes. \fg{}
 Mais cela aurait empêché Pierre de connaître la croissance qui vient
 par l'expérience de la difficulté.

\dvbox{
Il est glorieux de se rendre compte que Jésus lui-même
 est en train de prier pour vous.
}

Dans \ibibleverse{Jn}(17:20), Jésus vous inclue dans Ses prières.
 Il a dit~: \punct{deux-points}
 \og Ce n'est pas pour eux seulement que Je prie, mais encore pour ceux
 qui croiront en Moi par leur parole. \fg{}
 Si vous êtes parvenus à la foi en Jésus comme Fils de Dieu,
 alors cette prière était pour vous.

Tout comme Pierre, il est possible que nous perdions quelques batailles
 en chemin, mais la victoire \typo{victoire} finale appartient à Jésus
 qui prie pour nous pour que notre foi ne défaille pas.
 Et quand, par Sa force, nous serons tirés d'affaire,
 nous pourrons alors nous tourner vers ceux qui traverseront
 les mêmes problèmes et leur offrir compréhension et compassion.

\dvrule

\dvprayer{
Seigneur, nous Te remercions parce que, quand nous tombons,
 Tu nous relèves et nous prends dans Tes bras d'amour pour secouer
 la poussière, nous nettoyer et nous faire repartir sur le chemin.
}{\DlNdJ}


%%%%%%%%%%%%%%
% 9 aout
%%%%%%%%%%%%%%

\dvday{Détruisez ce temple}

\dvquote{
Jésus leur répondit~:
 Détruisez ce temple, et en trois jours je le relèverai.
}{\ibibleverse{Jn}(2:19)}

\lettrine{Q}{uand Jésus} est entré dans le temple et qu'Il a vu le commerce
 scandaleux qui y prenait place, Il est entré dans une telle colère
 qu'Il a confectionné un fouet et s'est mis à chasser les changeurs
 de monnaie, en renversant leurs tables et en leur adressant
 de vifs reproches. Bien sûr, cela a provoqué la colère des chefs religieux
 qui détenaient ces concessions et ils ont exigé de Jésus
 qu'Il leur montre des signes de l'autorité qui Lui permettait
 de se livrer à ces actes. Il leur a répondu~: \punct{deux-points}
 \og Détruisez ce temple et en trois jours, je le relèverai. \fg{}

Ils pensaient qu'Il faisait référence au temple d'Hérode,
 mais Jésus parlait de Son propre corps.
 Plus tard, Jésus allait dire de Sa vie~: \punct{deux-points}
 \og Personne ne me l'ôte, mais je la donne de moi-même ;
 j'ai le pouvoir de la donner et j'ai le pouvoir de la reprendre \fg{}
 (\ibibleverse{Jn}(10:18)).
 C'est une affirmation clé qui pourrait n'être que de la vantardise
 \ocadr une affirmation sur laquelle repose tout l'Évangile! \typo{Évangile}

\dvbox{
Si le corps crucifié de Jésus était resté dans le tombeau,
 il n'y aurait pas de foi chrétienne et pas d'Église.
}

L'Évangile \typo{Évangile} se fonde entièrement sur la résurrection
 de Jésus-Christ d'entre les morts. Il a fait exactement ce qu'Il a dit
 qu'Il ferait. Il est ressuscité d'entre les morts, a servi Ses disciples
 pendant quarante jours, puis Il est monté au ciel où Il est aujourd'hui
 à la droite du Père, d'où Il intercède pour nous. Un jour prochain,
 Il va revenir nous chercher pour que nous soyons avec Lui.

\dvrule

\dvprayer{
Père, nous Te remercions pour la résurrection.
 C'est le signe qui authentifie tout ce que Jésus a annoncé
 \ocadr que Tu es plein d'amour, de pardon, de grâce et de miséricorde.
 Merci de nous avoir donné cette vie riche et pleine
 quand nous marchons avec Toi.
}{\DlNdJ}


%%%%%%%%%%%%%%
% 10 aout
%%%%%%%%%%%%%%

\dvday{Eaux Vives}

\dvquote{
Jésus lui répondit~:
 \og Quiconque boit de cette eau aura encore soif ;
 mais celui qui boira de l'eau que je lui donnerai, n'aura jamais soif. \fg{}
}{\ibibleverse{Jn}(4:13-14)}

\lettrine{A}{u plus profond} de chaque homme, il y a une soif intense de Dieu.
 Comme le disait David~: \punct{deux-points}
 \og Mon âme a soif de toi, mon corps soupire après toi,
 dans une terre aride, desséchée, sans eau \fg{}
 (\ibibleverse{Ps}(63:1)). \punct{Point en trop}

L'homme essaye de satisfaire cette soif de différentes façons
 et par différentes expériences \ocadr la drogue, l'alcool ou,
 comme la femme samaritaine, par des relations amoureuses.
 Mais ces choses ne peuvent pas étancher la soif de Dieu.

\dvbox{
La soif en l'homme est une soif de Dieu.
}

Si vous buvez de l'eau venant des puits de la vie,
 vous allez encore avoir soif. Il serait sage d'écrire au-dessus
 de chacune de vos ambitions les mots suivants~:
 \og Vas-y, bois de cette eau. Mais tu vas encore avoir soif. \fg{}
 Écrivez-le \typo{tiret} au-dessus de chaque chose matérielle
 que vous rêver de posséder \ocadr la nouvelle voiture, la nouvelle maison,
 le nouveau bateau, ou n'importe quelle autre chose.
 Écrivez-le au-dessus de chacun des buts de votre vie.
 Accomplissez-les, mais à coup sûr, vous aurez encore soif.

Quand Jésus a parlé à la femme samaritaine de l'eau vive,
 il a parlé de la seule chose qui puisse satisfaire
 la soif profonde de l'âme \ocadr une relation avec Dieu.

Avez-vous soif aujourd'hui? Oubliez l'eau de ce monde.
 Buvez à grands traits de l'eau vive et trouvez la satisfaction
 à laquelle vous aspirez.

\dvrule

\dvprayer{
Père, nous Te remercions de ce que Tu as ouvert une voie pour satisfaire
 la soif profonde de nos âmes. Puissions-nous venir, puissions nous boire
 et trouver la satisfaction que nos âmes recherchent.
}{\DlNdJ}


%%%%%%%%%%%%%%
% 11 aout
%%%%%%%%%%%%%%

\dvday{Le Jésus des Écritures}

\dvquote{
Vous sondez les Écritures, parce que vous pensez avoir en elles
 la vie éternelle~: ce sont elles qui rendent témoignage de moi.
}{\ibibleverse{Jn}(5:39)}

\lettrine{T}{oute la Bible} parle de Jésus.
 En fait, Jésus est le point focal prédominant et central de l'Écriture
 sur lequel tout repose. Vous pouvez Le trouver dans chaque page.
 \og Voici~: je viens, \ocadr Dans le rouleau du livre,
 il est écrit à Mon sujet \fcadr{}
 Pour faire, ô Dieu, Ta volonté \fg{} (\ibibleverse{He}(10:7)).

Dieu a donné de nombreux passages dans l'Ancien Testament
 pour décrire la nature, le caractère du Messie qui devait arriver
 ainsi que les circonstances entourant cette venue, de façon à ce que
 lorsqu'Il viendrait, il n'y ait pas de question sur le fait qu'Il était
 le véritable Messie. Pour aider les gens à reconnaître le Sauveur,
 Dieu a donné plus de 300~prédictions et signes d'identification
 concernant Sa naissance, Son lieu de naissance, Son enfance,
 Son ministère, Son rejet, Sa mort et Sa résurrection.

\dvbox{
Vivre pour Jésus n'est pas le genre d'expérience
 qu'on vit une fois par semaine.
}

Les Juifs connaissaient les Écritures sur le bout des doigts.
 Ils les avaient étudiées diligemment et fidèlement.
 Mais Jésus a dit d'eux~: \punct{deux-points}
 \og Vous recherchez les Écritures mais vous ne venez pas à Moi
 pour pouvoir avoir la vie. \fg{}
 Connaître les Écritures n'est pas suffisant pour recevoir
 le don de la vie éternelle. Beaucoup de gens ont un faux sentiment
 de sécurité concernant leur salut, simplement parce qu'elles disent
 connaître Dieu, mais la vie éternelle ne vient pas de connaître la Bible
 \ocadr elle vient en recevant le Jésus des Écritures.

Vivre pour Jésus est une expérience faite jour après jour,
 heure après heure, minute après minute.
 C'est Lui abandonner votre vie et marcher en communion avec Lui.
 C'est tomber amoureux de Lui au point où Il devient le centre même
 et le point focal de votre vie.

\dvrule

\dvprayer{
Seigneur, nous voulons Te donner bien plus qu'un simple coup d'œil en passant.
 Nous voulons faire de Toi le centre même et la substance de nos vies.
}{\Amen}


%%%%%%%%%%%%%%
% 12 aout
%%%%%%%%%%%%%%

\dvday{Vous Avez Soif ?}

\dvquote{
Si quelqu'un a soif, qu'il vienne à moi et qu'il boive.
 Celui qui croit en moi, des fleuves d'eau vive couleront de son sein,
 comme dit l'Écriture.
}{\ibibleverse{Jn}(7:37-38)}

\lettrine{Q}{uand Jésus} a tenu ces propos,
 je crois que les disciples n'ont pas réellement compris
 ce qu'Il voulait dire. Mais quand Jean écrit cet évangile \suggest{Évangile}
 des années plus tard, il a l'avantage d'avoir du recul sur les évènements.
 Aussi au verset~\ibiblevs{Jn}(7:39), il ajoute son propre commentaire~:
 \og Il dit cela de l'Esprit qu'allaient recevoir ceux qui croiraient en lui;
 car l'Esprit n'était pas encore donné. \fg{}

\dvbox{
Dieu désire que nos vies soient comme un bassin rempli de l'Esprit de Dieu
 et que nous soyons un canal amenant l'Esprit de Dieu
 au monde assoiffé qui nous entoure.
}

Trop souvent, nous nous comportons comme des éponges;
 nous absorbons tout ce que nous pouvons mais nous n'avons jamais
 de trop-plein à offrir. Il ne reste rien pour ceux qui nous entourent.
 Dieu ne s'intéresse pas seulement à ce qu'Il peut faire en vous,
 mais aussi à ce qu'Il peut faire à travers vous.

Paul a écrit aux Galates~: \punct{deux-points}
 \og Le fruit (ou \og ce qui se dégage \fg{}) de l'Esprit,
 c'est l'amour \fg{} (\ibibleverse{Ga}(5:22)).
 Le monde a faim d'amour véritable, d'amour divin, d'amour \og agapé \fg{}.
 C'est ce que le monde a besoin de voir en nous. Quand votre vie déborde
 de l'Esprit de Dieu, c'est cet amour qui se dégage de vous.
 Il jaillit hors de vous comme un torrent d'eau vive,
 bénissant tout ceux qui vous entourent grâce à ce que Dieu
 a fait en vous et à ce qu'Il fait maintenant à travers vous.

\dvrule

\dvprayer{
Seigneur, puissions-nous recevoir la plénitude de Ton Saint-Esprit
 dans nos vies jusqu'à ce qu'Il s'écoule de nous comme un torrent d'eau vive,
 qui amène Ton amour et Ta lumière à ceux qui nous entourent.
}{\DlNdJ}


%%%%%%%%%%%%%%
% 13 aout
%%%%%%%%%%%%%%

\dvday{Vraiment Libres}

\dvquote{
Si vous demeurez dans ma parole, vous êtes vraiment mes disciples;
 vous connaîtrez la vérité et la vérité vous rendra libres.
}{\ibibleverse{Jn}(8:31-32)}

\lettrine{L}{es gens} aiment parler de liberté.
 Mais la vraie liberté n'est pas la liberté de faire tout ce que vous voulez
 \ocadr c'est la liberté de ne pas faire ce qui est mal.
 Quand Jésus vous libère, la liberté qu'Il donne est la vraie liberté.
 C'est la liberté de ne pas faire ces choses qui étaient destructrices
 pour vous et pour ceux qui vous entourent.

Certains de ceux qui sont esclaves du péché croient à tort qu'ils sont libres.
 Mais Jésus a dit~: \punct{deux-points, majuscule}
 \og Quiconque commet le péché est esclave du péché \fg{}
 (\ibibleverse{Jn}(8:34)). Qu'il s'agisse d'un péché évident à tous
 comme la drogue, l'abus d'alcool, ou d'un péché secret que personne
 ne connaît en dehors de vous-mêmes, le péché a le pouvoir de s'emparer
 de vous et de vous garder dans ses griffes;
 vous êtes impuissants à vous en dégager vous-mêmes.

\dvbox{
Dieu nous a donné le libre-arbitre.
}

Paul a dit~: \punct{deux-points}
 \og Tout m'est permis \fg{} (\ibibleverse{ICo}(6:12)).
 C'est très général. Mais il a continué en ajoutant~: \punct{deux-points, majuscule}
 \og Mais je ne me laisserai pas asservir par quoi que ce soit. \fg{}
 Oh oui, j'ai la liberté de faire quelque chose, mais si dans l'exercice
 de cette liberté, je suis amené à l'esclavage, alors je ne suis plus libre.
 J'ai fait l'exercice de ma liberté d'une façon telle qu'elle m'a conduit
 à me retrouver esclave.

Peut-être vous trouvez vous liés par quelque chose aujourd'hui.
 Vous avez le sentiment que vous ne pouvez pas être libérés.
 Mais quand vous arriverez à connaître la vérité
 \ocadr Jésus-Christ \fcadr{} alors la vérité vous rendra libres.

\dvrule

\dvprayer{
Père, merci à Toi pour la merveilleuse liberté que nous avons en Jésus-Christ.
 Nous sommes libres de vivre comme Tu voudrais que nous vivions,
 libres de marcher en communion avec Toi dans la puissance de l'Esprit.
}{\Amen}

\grammar{Tu voudrais}


%%%%%%%%%%%%%%
% 14 aout
%%%%%%%%%%%%%%

\dvday{Juge ou Médecin ?}

\dvquote{
Ses disciples lui demandèrent~: Rabbi, qui a péché,
 lui ou ses parents, pour qu'il soit né aveugle ?
}{\ibibleverse{Jn}(9:2)}

\lettrine{Q}{uand un accident se produit,} deux types de véhicules
 d'urgence arrivent sur les lieux. Les premiers qui arrivent
 sont généralement les policiers. Ils s'efforcent de déterminer
 qui est en faute et, si nécessaire, ils dressent un procès-verbal
 au coupable. Les secouristes arrivent ensuite.
 Ils ne se \typo{Manque \og se \fg{}}
 soucient pas de savoir qui est en tort
 \ocadr ils veulent seulement soulager la douleur et la souffrance.

Les disciples passaient à côté d'un homme, aveugle de naissance.
 Tels des policiers, ils ont tout de suite voulu déterminer
 qui était responsable de l'état de cette victime.
 Jésus a déclaré que ce n'était pas que lui ou ses parents aient été en tort,
 mais c'était afin que les œuvres de Dieu soient manifestées
 en lui quand Jésus allait le guérir.

\dvbox{
Dieu nous appelle à être des secouristes plutôt que des policiers.
}

Comment vous comportez-vous face aux tragédies humaines?
 Venez-vous comme un policier ou comme un secouriste?
 Jésus a dit que Dieu ne l'avait pas envoyé dans le monde
 pour condamner le monde, mais \typo{\og que \fg{} en trop}
 pour que le monde soit sauvé par Lui.

Souvent la vie d'une personne est brisée parce que cette personne
 récolte les conséquences de sa propre rébellion contre Dieu.
 Agitez-vous sous son nez un index critique en disant~: \punct{deux-points}
 \og Si vous n'aviez pas fait ça ou ça, vous n'en seriez pas là \fg{} ?
 Sortez-vous votre \og code pénal \fg{} pour lui mettre un PV
 pour violation de la loi? Ou bien venez-vous comme un secouriste
 qui s'efforce de panser ses blessures?

Il ne nous appartient pas de trouver les raisons de la souffrance,
 mais de chercher à réparer les dégâts déjà faits, tout comme Jésus l'a fait.

\dvrule

\dvprayer{
Père, puissions-nous être Tes témoins, faisant Ton œuvre
 dans ce monde qui souffre.
}{\DlNdJ}


%%%%%%%%%%%%%%
% 15 aout
%%%%%%%%%%%%%%

\dvday{Prier, Raisonner, S'engager}

\dvquote{
Maintenant mon âme est troublée. Et que dirai-je ? [\dots]
 Père, sauve-moi de cette heure ? [\dots]
 Mais c'est pour cela que je suis venu jusqu'à cette heure.
 Père, glorifie ton nom !
}{\ibibleverse{Jn}(12:27-28)}

\lettrine{J}{ésus} savait que la croix était la volonté de Dieu.
 Cependant, devant l'épreuve épouvantable qui L'attendait,
 Son cœur était troublé. En attendant là dans le jardin,
 Jésus a fait trois choses qui Lui ont permis d'obéir.

Premièrement, Il a prié. Quel exemple merveilleux pour nous,
 car nous aussi allons être confrontés à des moments d'incertitude,
 voire de peur, des moments où nous aurons à crier vers Celui
 qui s'occupe de tous les détails de nos vies.

\dvbox{
Quand vous êtes troublés par des circonstances que vous ne comprenez pas,
 suivez l'exemple de Jésus.
}

Deuxièmement, Il a raisonné. Il s'est convaincu en y réfléchissant
 bien que la terrible épreuve qui L'attendait allait accomplir
 les desseins éternels de Dieu.
 \og C'est pour cela que je suis venu jusqu'à cette heure. \fg{}

Et puis, troisièmement, Il s'est engagé à obéir.
 \og Père, glorifie ton nom. \fg{}
 Autrement dit~: \punct{deux-points}
 \og Père, Je m'engage à T'amener la gloire, quoi qu'il M'en coûte. \fg{}

Suivez l'exemple de Jésus. Premièrement, priez.
 La prière change les choses. Quelques fois, \suggest{Quelquefois}
 ce qui a la plus besoin de changer, c'est notre attitude.
 La prière nous donne aussi la force de supporter une épreuve
 et la capacité de l'accepter. Ensuite \ocadr raisonnez. \suggest{virgule}
 Rendez-vous compte que Dieu vous aime de façon suprême et qu'Il accomplit
 Son plan éternel dans votre vie.
 Son plan peut amener un inconfort temporaire, mais il va aussi amener
 un bien éternel. Et troisièmement, engagez-vous.
 \og Que Ta volonté soit faite, Seigneur.
 Utilise ma vie pour glorifier Ton nom. \fg{}

\dvrule

\dvprayer{
Père, aide-nous à suivre l'exemple de Jésus quand nous faisons face
 à des situations incertaines ou effrayantes.
 Rappelle-nous de prier, de considérer le problème avec notre raison
 et de nous engager à Te glorifier.
}{\DlNdJ}


%%%%%%%%%%%%%%
% 16 aout
%%%%%%%%%%%%%%

\dvday{Promesses, Promesses}

\dvquote{
Seigneur, lui dit Pierre,
 pourquoi ne puis-je pas te suivre maintenant ?
 Je donnerai ma vie pour toi.
}{\ibibleverse{Jn}(13:37)}

\typo{Guillemets en trop}

\lettrine{A}{lors} même que Pierre faisait cette belle
 et sincère déclaration de son attachement,
 Jésus savait qu'avant que le soleil ne se lève le lendemain matin,
 Pierre l'aurait renié par trois fois.

Il est si facile de s'engager verbalement.
 Mais quand arrive le moment de vérité où il faut tenir nos promesses,
 nous ne leur sommes pas toujours fidèles.

\dvbox{
Les paroles ne coûtent pas cher.
}

Nous faisons des promesses à Dieu quand nous essayons de conclure
 un marché avec Lui, pour qu'Il nous accorde nos désirs.
 \og Seigneur, si Tu fais ça pour moi, alors je ferais ceci pour Toi. \fg{}
 Ou bien,  nous formulons des v\oe{}ux juste après un échec.
 Nous affirmons en nous relevant~: \punct{deux-points}
 \og Seigneur, je ne referai plus jamais ça. \fg{}
 Nos paroles montrent que nous plaçons de nouveau notre confiance
 dans la chair, mais tant que nous mettons notre confiance en nous-mêmes,
 nous nous prédisposons à la rechute.

Dieu seul sait si nous tiendrons nos promesses ou non.
 Comme le psalmiste le disait~: \punct{deux-points}
 \og Éternel ! Tu me sondes et Tu me connais; Tu sais quand je m'assieds
 et quand je me lève; Tu comprends de loin ma pensée ;
 Tu sais quand je marche et quand je me couche,
 et Tu pénètres toutes mes voies \fg{} (\ibibleverse{Ps}(139:1-3)).

Ne promettez rien à Dieu qui s'appuie sur votre chair,
 car votre chair va vous trahir.
 Ne promettez de faire que les choses que Jésus vous suggère de faire,
 et comptez sur Lui pour la force d'obéir.

\dvrule

\dvprayer{
Père, que chaque prière reflète notre complète dépendance à Toi
 et à notre foi indéfectible en Ta souveraineté et en Ta sagesse.
}{\DlNdJ}


%%%%%%%%%%%%%%
% 17 aout
%%%%%%%%%%%%%%

\dvday{Fruit Naturel}

\dvquote{
Moi, je suis la vigne ; vous êtes les branches.
 Celui qui demeure en Moi, comme Moi en lui, porte beaucoup de fruit,
 car sans moi, vous ne pouvez rien faire.
}{\ibibleverse{Jn}(15:5)}

\lettrine{L}{es gens} aimeraient produire du fruit
 sans l'intermédiaire de la vigne. Ils pensent pouvoir extraire
 en forçant un fruit ou deux en menant de bonnes vies ou en devenant
 de meilleurs personnes. Mais le fait est, qu'en dehors de Jésus,
 nous ne pouvons rien faire. C'est seulement en demeurant en Lui
 que le Saint-Esprit commence à former du fruit en nous \ocadr naturellement.

Quel fruit l'Esprit produit-Il? Le fruit de l'Esprit est l'amour
 \og agapé \fg{} \ocadr amour profond, fervent \fcadr{} amour qui est patient
 et serviable; amour qui n'est pas envieux, qui ne se vante pas,
 qui ne fait rien de malhonnête; amour qui n'a pas l'esprit de clan,
 amour qui ne cherche pas son intérêt. C'est un amour extraordinaire.
 C'est un amour qui ne s'irrite pas, un amour qui ne médite pas le mal,
 un amour qui supporte tous les fardeaux, un amour qui croit tout,
 un amour qui ne succombe jamais.

\dvbox{
Il vous est impossible de produire, d'imiter ou de faire apparaître
 comme par magie l'amour de l'Esprit Saint.
}

Si l'Esprit de Dieu demeure en vous, le résultat naturel sera alors
 le fruit de l'amour \og agapé \fg{}.
 \suggest{\emph{agapé} plutôt que \og agapé \fg}
 Nous sommes impatients de voir ce fruit.
 Nous aimerions planter un pommier aujourd'hui et manger les pommes demain.
 Mais les fruits ne poussent pas du jour au lendemain.
 Ne soyez pas impatients avec Dieu et avec le Saint-Esprit pendant
 qu'Il développe du fruit dans votre vie.
 Il se développera en Son temps
 \suggest{Si Son se rapporte au fruit, faut-il le capitaliser?}
 \ocadr et quel beau jour ce sera quand vous porterez du fruit.

\dvrule

\dvprayer{
Merci, Père pour la présence de Ton Saint-Esprit qui demeure en nous.
 Seigneur, apprends-nous à être patients alors que le fruit de Ton Esprit
 commence à se développer dans nos vies.
}{\Amen}


%%%%%%%%%%%%%%
% 18 aout
%%%%%%%%%%%%%%

\dvday{Mis à Part}

\dvquote{
Sanctifie-les par la vérité~: Ta parole est la vérité.
}{\ibibleverse{Jn}(17:17)}

\lettrine{L}{e mot} \og sanctifier \fg{} signifie \og mettre à part \fg{}.
 Dans le culte du temple, les récipients étaient mis à part pour être
 exclusivement utilisés dans l'adoration de Dieu.
 Ils ne devaient être utilisés pour aucun autre usage.
 Et c'est ce que la sanctification signifie pour nous
 \ocadr que nous puissions être mis à part du monde et libérés
 des influences du monde, que nous puissions être engagés et consacrés à Lui,
 que nous puissions être la proprieté exclusive de Dieu.

\dvbox{
Nos vies sont quotidiennement bombardées par le péché.
}

Chaque fois que vous allumez la télévision ou la radio,
 chaque fois vous entrez dans un centre commercial ou dans un supermarché,
 des choses séduisantes et tentantes essayent de vous attirer
 vers les choses que Dieu a déclarées être impures.
 Nous devons recevoir la Parole quotidiennement pour contrebalancer
 ces influences du monde. À la fin de chaque jour,
 laissez la Parole de Dieu vous purifier de toutes les ordures
 auxquelles vous avez été exposés.

Comme Jésus l'a dit dans \ibibleverse{Jn}(15:3)~: \punct{deux-points}
 \og Vous êtes déjà purs, à cause de la parole que je vous ai annoncée. \fg{}
 C'est Sa Parole qui enlève la saleté à laquelle nous sommes exposés chaque jour.

Parce que nous aimons Dieu, nous devrions vouloir réaliser Ses désirs.
 S'Il désire que nous soyons gardés du monde, mis à part,
 alors nous devrions le vouloir aussi. Nous devrions être appliqués
 pour prendre notre position contre le mal.
 Et si Dieu dit que la Parole est ce qui nous nettoie,
 alors nous devrions aller à la Parole chaque jour.

\dvrule

\dvprayer{
Père, nous Te remercions de ce que Tu nous as aimés,
 enlevés du monde et de ce que Tu nous as donné une citoyenneté céleste.
 Mets nous à part, Seigneur, par Ta Parole.
}{\DlNdJ}


%%%%%%%%%%%%%%
% 19 aout
%%%%%%%%%%%%%%

\dvday{Tout est accompli}

\dvquote{
Quand il eut pris le vinaigre, Jésus dit~: Tout est accompli.
 Puis il baissa la tête et rendit l'esprit.
}{\ibibleverse{Jn}(19:30)}

\lettrine{Q}{uand} Jésus s'est écrié~: \punct{deux-points}
 \og Tout est accompli \fg{}, ça n'était pas un cri de défaite
 mais un cri de victoire. Par la croix, Jésus a vaincu la puissance
 de Satan à asservir l'homme, à détruire l'homme et à séparer l'homme de Dieu.
 Par la croix, Jésus a ouvert un chemin pour que l'homme
 s'approche de Dieu et vive de nouveau en communion avec Dieu.

Le pouvoir de vivre la vie que Dieu veut que nous vivions nous a été donné.
 Le pouvoir d'être comme Lui \ocadr d'être restauré à l'image de Dieu \fcadr{}
 nous a été donné. C'est le désir de Dieu et Son but pour votre vie.

\dvbox{
L'emprise puissante que le péché détenait autrefois sur vous,
 sur moi, a été brisée.
}

Dieu veut restaurer ce qui a été perdu dans le jardin d'Eden.
 À cette fin, l'Esprit de Dieu œuvre dans nos vies jour après jour,
 nous conformant, nous moulant, nous re-façonnant selon l'intention
 première de Dieu, pour que nous puissions vivre en communion avec Dieu
 et refléter Son amour, Sa grâce, Sa bonté et Sa miséricorde
 dans le monde obscur dans lequel nous vivons.
 Les barrières qui nous retenaient loin de Dieu ont été enlevées.

Le travail est terminé. Il est accompli. Jésus a conquis le péché,
 la mort, l'enfer et la tombe. Il a conquis Satan. En conséquence,
 nous pouvons faire l'expérience d'une vie grandement bénie avec et pour Dieu.

\dvrule

\dvprayer{
Père, puissions-nous profiter pleinement du temps que Tu nous a donné
 dans cette vie pour Te connaître, Te servir, et T'aimer.
}{\DlNdJ}


%%%%%%%%%%%%%%
% 20 aout
%%%%%%%%%%%%%%

\dvday{La Résurrection}

\dvquote{
Cependant, Marie se tenait dehors, près du tombeau, et pleurait.
 Comme elle pleurait, elle se baissa pour regarder dans le tombeau.
}{\ibibleverse{Jn}(20:11)}

\lettrine{L}{'Écriture} capture le moment même où le chagrin
 s'est transformé en espoir. Marie, voyant la pierre roulée
 de devant le tombeau, se baisse et regarde à l'intérieur.
 Ses pleurs vont sans doute momentanément se transformer en confusion,
 mais ce dont elle ne s'est pas encore rendu compte,
 c'est que c'est le jour où Dieu allait donner \og de la splendeur
 au lieu de cendre, une huile de joie au lieu du deuil,
 un vêtement de louange au lieu d'un esprit abattu \fg{}
 (\ibibleverse{Is}(61:3)).

\dvbox{
Avant la fin du jour, la vérité de la résurrection va être confirmée.
}

Jésus va apparaître à Marie et aux disciples, et ils vont découvrir
 que Ses paroles étaient vraies \ocadr qu'Il est vraiment le Fils de Dieu,
 qu'Il est l'Agneau de Dieu qui s'est offert en rançon pour le péché,
 qu'Il est tout ce qu'Il a proclamé être, qu'Il est la résurrection et la vie,
 le chemin et la vérité.

En ce nouveau jour, le premier jour de la semaine,
 quelque chose de nouveau est né. Un pont a été bâti sur le fossé
 entre l'homme et Dieu, et une nouvelle relation a été rendue possible.
 Parce qu'Il vit, nous pouvons vivre. Parce qu'Il a conquis le tombeau,
 nous n'avons plus à avoir peur de la mort. Parce qu'Il a vaincu le péché,
 nous pouvons être libérés des griffes du péché.

Là, devant ce tombeau vide, Marie a fait l'expérience de la vérité
 des mots du psalmiste~:
 \og Le soir arrivent les pleurs, et le matin la jubilation \fg{}
 (\ibibleverse{Ps}(30:6)).

\dvrule

\dvprayer{
Père, nous Te remercions pour le tombeau vide et pour l'espoir glorieux
 qu'il représente. Nous Te remercions de ce que nous n'avons pas
 à avoir peur de la mort, parce que Jésus a conquis la tombe. 
}{\Amen}


%%%%%%%%%%%%%%
% 21 aout
%%%%%%%%%%%%%%

\dvday{M'aimes-tu plus que ceux-ci?}

\dvquote{
Après qu'ils eurent mangé, Jésus dit à Simon Pierre~:
 \og Simon, fils de Jonas m'aimes-tu plus que ceux-ci ? \fg{}
}{\ibibleverse{Jn}(21:15)}

\lettrine{A}{près} une vaine nuit de pêche,
 un homme sur le rivage dit aux disciples de jeter leurs filets
 de l'autre côté du bateau. Immédiatement les filets furent remplis;
 à un point tel qu'ils ne pouvaient pas les remonter.
 Se rendant compte que l'homme était Jésus, Pierre plongea et nagea vers Lui.
 Les autres disciples le suivirent dans le petit bateau
 tirant le filet rempli de poissons, et quand ils arrivèrent,
 ils découvrirent que Jésus avait déjà préparé le petit déjeuner.

Assis autour du feu, Jésus demanda à Pierre~: \punct{deux-points}
 \og M'aimes-tu plus que ceux-ci? \fg{}

\dvbox{
Donnez au Seigneur la toute première place dans votre vie.
}

Peut-être Jésus regardait-il ces poissons qui frétillaient dans le filet,
 probablement la prise plus importante jamais ramenée par Pierre.
 Jésus voulait donc peut-être dire~: \punct{deux-points}
 \og M'aimes-tu plus que le summum du succès dans ta carrière? \fg{}

Si Jésus vous regardait droit dans les yeux en vous demandant~:
 \og M'aimes-tu plus que ceux-ci? \fg{} \suggest{Fin de phrase manquante?}
 Quelles sont les choses dans votre vie qui se disputent votre attention
 et éloignent votre amour de Lui? Vos buts, votre carrière, une relation,
 des plaisirs, la télévision? Quelle serait votre réponse?

Dieu désire impatiemment votre amour. Il veut être le tout premier
 dans votre vie. Puisse Dieu nous aider à pouvoir répondre~: \punct{deux-points}
 \og Seigneur, Tu sais toutes choses; Tu sais que je T'aime suprêmement. \fg{}

\dvrule

\dvprayer{
Père, tant de choses se disputent notre attention.
 Aide-nous à Te mettre au-dessus de tout le reste.
 Puisse notre amour pour Toi surpasser tous les autres amours dans notre vie.
}{\DlNdJ}


%%%%%%%%%%%%%%
% 22 aout
%%%%%%%%%%%%%%

\dvday{Puissance}

\dvquote{
Mais vous recevrez une puissance, celle du Saint-Esprit survenant sur vous,
 et vous serez mes témoins à Jérusalem, dans toute la Judée,
 dans la Samarie et jusqu'aux extrémités de la terre.
}{\ibibleverse{Ac}(1:8)}

\lettrine{D}{ans} cette conversation, juste avant Son ascension,
 Jésus déclare à Ses disciples qu'ils vont bientôt se lancer
 dans une tâche apparemment impossible~:
 ils doivent aller dans le monde entier prêcher la bonne nouvelle
 de Dieu à chaque créature.

Ils ne pouvaient pas accomplir cette tâche par leurs propres forces.
 Mais Jésus leur a promis qu'ils recevraient une puissance.
 Le mot grec est \emph{dunamis}, d'où vient notre mot \og dynamique \fg{}.
 Ils allaient recevoir une puissance dynamique, qui leur permettrait
 d'être les témoins que Dieu voulaient qu'ils soient.

\dvbox{
La puissance de Dieu est la même aujourd'hui que ce qu'elle était hier.
}

La puissance de Dieu affranchit toujours les hommes.
 Elle amène toujours l'espoir dans un monde complètement désespéré
 sans cela. Aujourd'hui, nous sommes les témoins que Dieu envoie.
 Le monde est toujours sans espoir et toujours hostile. Mais par nous,
 l'amour du Saint-Esprit conquiert la haine, les querelles et l'amertume.
 La puissance du Saint-Esprit nous fait briller
 comme des lumières dans l'obscurité.

Que votre vie soit un témoignage pour Dieu.
 Je prie pour que les autres voient briller Christ en vous
 et pour qu'ils soient attirés par votre témoignage.
 Qu'ils fassent l'expérience de l'amour de Jésus au travers de vos paroles
 et de vos actes, parce que vous marchez dans la dynamique de l'Esprit.

\dvrule

\dvprayer{
Dieu, aide-nous à recevoir de Toi la puissance impartie par Ton Esprit,
 pour que nous puissions accomplir la tâche que Tu as mis devant nous.
}{\DlPNdJ}


%%%%%%%%%%%%%%
% 23 aout
%%%%%%%%%%%%%%

\dvday{Aucun autre Nom}

\dvquote{
Le salut ne se trouve en aucun autre ;
 car il n'y a sous le ciel aucun autre nom donné parmi les hommes,
 par lequel nous devions être sauvés.
}{\ibibleverse{Ac}(4:12)}

\lettrine{C}{ette affirmation} par Pierre
 \ocadr prononcée après qu'il eut guéri un infirme au nom de Jésus \fcadr{}
 a d'abord été prononcée par Jésus Lui-même.
 \og Je suis le chemin, la vérité, la vie, a-t-Il dit.
 Nul ne vient au Père que par Moi \fg{} \punct{Ponctuation dans les guillemets}
 (\ibibleverse{Jn}(14:6)).
 Aujourd'hui, nous entendons dire~: \punct{deux-points}
 \og Tous les chemins mènent à Dieu, \fg{} \punct{virgule}
 mais cela revient à contredire Jésus. Il a dit~: \punct{deux-points}
 \og Je suis la porte des brebis. Tous ceux qui sont venus avant moi
 sont des voleurs et des brigands \fg{} (\ibibleverse{Jn}(10:7-8)).

Ces affirmations irritent beaucoup de monde.
 Ces personnes s'énervent si vous déclarez qu'Il est le seul chemin.
 Ils vous traitent de bigots ou de gens bornés.
 Les gens aimeraient croire que toutes les routes mènent au ciel.
 Ils aimeraient penser qu'ils peuvent vivre comme bon leur semble
 en respectant leurs propres règles.
 Mais Dieu a établi les règles pour l'humanité.
 Et la Parole de Dieu déclare qu'il n'y a qu'une seule façon
 par laquelle un homme peut-être sauvé.

\dvbox{
Vous ne pouvez pas être sauvé en étant bon, religieux ou sincère.
}

Vous ne pouvez pas être sauvé en respectant la Loi.
 La Bible dit~: \punct{deux-points}
 \og L'homme n'est pas justifié par les œuvres de la Loi \fg{}
 (\ibibleverse{Ga}(2:16)).

Si la puissance du nom de Jésus peut faire marcher un infirme,
 alors la puissance du nom de Jésus peut aussi laver une personne
 de ses péchés. Aucun autre nom sous les cieux n'a une telle puissance
 \ocadr aucun si ce n'est celui de Jésus-Christ.

\dvrule

\dvprayer{
Père, merci à Toi pour Jésus. Son nom est si doux et nous amène
 tant de réconfort et d'espoir. Merci à Toi d'avoir offert le salut
 par Son nom précieux.
}{\DlNdJ}

