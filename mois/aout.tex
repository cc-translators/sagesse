\dvmonth{Août}

%%%%%%%%%%%%%%
% 1er aout
%%%%%%%%%%%%%%

\dvday{Foi Aveugle}

\dvquote{
Simon lui répondit~:
 Maître, nous avons travaillé toute la nuit sans rien prendre,
 mais, sur Ta parole, je jetterai les filets.
}{\ibibleverse{Lc}(5:5)}

\lettrine{J}{ésus avait dit à Simon~:}
 \og Avance en eau profonde, et jetez vos filets pour pêcher \fg{}
 (\ibibleverse{Lc}(5:4)).

Mais ils venaient de passer toute la nuit à pêcher.
 Et ces pêcheurs expérimentés n'avaient rien pris.
 Vous pouvez presque entendre le ton poli mais condescendant
 de la voix de Pierre. Il aurait pu tout aussi bien dire~: \punct{deux-points}
 \og Tu es un bon enseignant, mais c'est moi le pêcheur.
 Tu sais sans doute plein de choses sur Dieu, mais le poisson,
 ça, c'est mon affaire. \fg{}

\dvbox{
La foi, c'est obéir sans tenir compte de notre compréhension des choses.
}

Mais Pierre n'a pas répondu de cette façon.
 Au contraire, il a dit~: \punct{deux-points}
 \og Mais, sur Ta parole, je jetterai les filets. \fg{}
 Et parce qu'il a obéi aveuglément \ocadr non pas parce que la requête était
 logique mais parce qu'elle venait de Jésus \fcadr{}
 ils ont remonté un filet tellement plein de poissons,
 qu'il s'est pratiquement déchiré.

Nos meilleurs efforts peuvent mener à des filets vides.
 Mais l'obéissance peut amener le succès au-delà de nos rêves les plus fous.
 Il y a eu de nombreuses années dans mon ministère où j'ai jeté mes filets
 et n'ai rien ramené. Mais quand le Seigneur décida de remplir les filets,
 ils furent soudain remplis jusqu'à en déborder.
 Cela n'avait rien à voir avec mes propres capacités ou ma propre puissance.
 Cela s'est produit simplement parce que Dieu l'avait décidé.
 Et toute la gloire Lui en revient.

Ce que Dieu vous demande peut sembler contraire à votre logique.
 Mais si vous écoutez Ses instructions et agissez en obéissant,
 vous trouverez la différence entre l'échec et le succès.
 \og Mais, sur Ta parole, je le ferai. \fg{}
 Que ces mots soient vrais dans nos vies.

\dvrule

\dvprayer{
Père, fais de nous des enfants si obéisssants qu'au moindre chuchotement
 de Ta part, nous agissions. Apprends-nous à faire confiance
 à Ta voix plutôt qu'à la nôtre.
}{\Amen}


%%%%%%%%%%%%%%
% 2 aout
%%%%%%%%%%%%%%

\dvday{La Vie Équilibrée}

\dvquote{
Le Seigneur lui répondit~: Marthe, Marthe, tu t'inquiètes
 et tu t'agites pour beaucoup de choses.
 Or une seule chose est nécessaire. Marie a choisi la bonne part,
 qui ne lui sera pas ôtée.
}{\ibibleverse{Lc}(10:41-42)}

\lettrine{P}{réoccupée par tous les petits détails du dîner,}
 Marthe a passé la soirée, en s'irritant de son travail et en passant
 à côté de la tâche la plus importante \ocadr s'asseoir aux pieds de Jésus.
 Elle était tellement occupée à servir Jésus,
 qu'elle n'avait plus de temps à passer avec Jésus.

Servir le Seigneur est une chose merveilleuse et nécessaire.
 Cependant ce service devrait venir d'une position de joie et d'excitation.
 Nous ne devrions jamais nous retrouver à nous plaindre à Jésus
 des choses que nous faisons pour Lui.

\dvbox{
Si vous servez Jésus, vous devez d'abord L'adorer.
}

À la différence de Marthe, Marie avait choisi \og la bonne part \fg{}.
 La bonne part, c'était l'intimité avec Jésus.
 Pendant que Marthe se démenait dans la cuisine,
 Marie s'était assise à Ses pieds et buvait chacunes de Ses paroles.

Puisque le service et l'adoration sont tous les deux des ingrédients
 indispensables à la vie d'un croyant, nous parlons en fait
 du besoin d'équilibre. La vie bien équilibrée, c'est celle qui manifeste
 l'amour pour le Seigneur au travers du service pour le Seigneur,
 mais qui laisse aussi du temps pour l'intimité et l'adoration.
 On a besoin des deux. Votre vie est-elle équilibrée?
 Servez-vous par amour ou par obligation?

Si vous voulez Le servir, vous devez d'abord L'adorer.
 C'est quand nous L'adorons, assis à Ses pieds, que nous obtenons
 la force qu'il nous faut pour vivre.
 Quand nous L'adorons assis devant Lui,
 Il nous dirige alors dans nos activités.
 \typo{Majuscules dans le paragraphe}

\dvrule

\dvprayer{
Père, merci de ce que chaque fois que nous nous venons nous mettre
 à Tes pieds, Tu nous y attends. Enseigne-nous Seigneur.
 Dis-nous comment nous pouvons Te servir parce que nous voulons Te bénir.
}{\DlNdJ}


%%%%%%%%%%%%%%
% 3 aout
%%%%%%%%%%%%%%

\dvday{L'Abondance des Choses}

\dvquote{
Puis il leur dit~:
 Gardez-vous attentivement de toute cupidité ;
 car même dans l'abondance, la vie d'un homme ne dépend pas
 de ce qu'il possède.
}{\ibibleverse{Lc}(12:15)}

\lettrine{D}{eux frères se disputaient au sujet de leur héritage.}
 L'un d'eux pensait que l'autre essayait de le tromper,
 aussi avait-il demandé à Jésus d'intervenir.
 Mais Jésus refusa de se mêler d'une querelle portant
 sur des choses matérielles. Il choisit plutôt de les mettre en garde
 contre la cupidité. Après leur avoir fait part d'une parabole
 et souligné le soin que Dieu accorde aux corbeaux et aux lys des champs,
 Il leur dit, en conclusion~: \punct{deux-points}
 \og Cherchez plutôt son royaume ; et cela vous sera donné par surcroît \fg{}
 (\ibibleverse{Lc}(12:31)). 

\dvbox{
Quand vous mettez Dieu en premier, Il prend soin de tout le reste.
}

C'est ahurissant de voir combien le vie des gens est altérée
 et déséquilibrée quand ils ne vivent que pour les choses matérielles.
 Mais quand des personnes commencent à suivre les choses de l'Esprit,
 leur vie s'équilibre. De plus, ils acquièrent la paix,
 le contentement et une joie qu'ils n'avaient pas aupararavant.

Jésus a dit~: \punct{deux-points, majuscule}
 \og Votre Père a trouvé bon de vous donner le royaume \fg{}
 (\ibibleverse{Lc}(12:32)). Mais est-ce à cela que vous accordez
 de la valeur? Votre intérêt se porte-t-il sur les choses éternelles
 ou seulement sur les choses temporelles, matérielles
 \ocadr ces choses qui ne font que passer?

Vous avez peut-être beaucoup de succès. Mais peu importe combien
 vous pouvez amasser, votre vie ne consiste pas à courir après
 les appétits charnels. Elle ne consiste pas dans l'abondance
 de vos possessions mais dans une vraie relation avec Dieu
 quand votre esprit s'est ouvert à la vie avec Lui par la foi en Jésus.

\dvrule

\dvprayer{
Père, aide-nous à ne pas apprendre les leçons que nous devons connaître
 de la façon la plus difficile. Aide-nous à vivre notre vie
 avec des mains ouvertes, sans étreindre ce monde,
 mais en les tendant vers Ton royaume.
}{\Amen}


%%%%%%%%%%%%%%
% 4 aout
%%%%%%%%%%%%%%

\dvday{Le Chemin vers le Haut}

\dvquote{
En effet quiconque s'élève sera abaissé,
 et celui qui s'abaisse sera élevé.
}{\ibibleverse{Lc}(14:11)}

\lettrine{D}{ans la maison du Pharisien} chez qui Il avait été invité à dîner,
 Jésus observait les gens manœuvrer pour obtenir les places
 d'importance autour de la table. Finalement, Il leur expliqua pourquoi
 ça n'était pas sage.
 \og Que se passerait-il si une personne plus honorable que vous arrivait
 et que celui qui vous a invité doive vous dire~: \punct{deux-points}
 \og Désolé, mon ami, vous êtes assis au mauvais endroit.
 Pourquoi n'allez-vous pas vous asseoir au bout de la table? \fg{}
 Jésus ajouta alors~: \punct{deux-points}
 \og Ne vous mettez pas dans une situation aussi embarrassante.
 Il vaut mieux que vous preniez la dernière place, et que vous laissiez
 l'hôte venir vous dire~: \punct{deux-points}
 \og Pourquoi êtes-vous assis si loin?
 J'aimerais que vous preniez une meilleure place. \fg{}
 Jésus conclua en disant~:
 \og En effet quiconque s'élève sera abaissé,
 et celui qui s'abaisse sera élevé \fg{} (\bibleverse{Lc}(14:11)).

\dvbox{
Spirituellement parlant, le chemin vers le haut est celui qui va vers le bas,
 et le chemin vers le bas est celui qui va vers le haut.
}

Si vous vous abaissez, vous serez élevés.
 Prenez la dernière place et vous serez invités à venir plus haut.
 Mais si vous prenez la plus haute place,
 il y a de fortes chances qu'on vous demande d'en descendre.

Puisse Dieu nous aider à vivre nos vies comme des serviteurs
 \ocadr conscients des besoins des autres et désireux d'aller vers eux
 et de les servir de toutes les façons possibles.
 Puissions-nous suivre l'exemple de notre Seigneur,
 qui n'a pas pensé à Son propre bien-être mais a vécu
 dans l'humilité une vie au service de tous.

\dvrule

\dvprayer{
Seigneur, nous voyons comment Tu as laissé de côté Ta gloire
 pour venir sur la terre, vivre dans l'humilité et mourir pour nous.
 Nous voyons que Tu as été maintenant élevé à la plus haute place;
 que Tu es le Roi des rois et le Seigneur des seigneurs.
 Apprends-nous à suivre Ton exemple.
}{\DlNdJ}


%%%%%%%%%%%%%%
% 5 aout
%%%%%%%%%%%%%%

\dvday{L'homme le plus riche}

\dvquote{
Si donc vous n'avez pas été fidèles dans les richesses injustes,
 qui vous confiera le (bien) véritable ?
}{\ibibleverse{Lc}(16:11)}

\lettrine{J}{ésus a parlé d'un serviteur} qui avait commencé
 à dissiper les biens de son maître à son profit.
 Quand le maître eut vent de ces rumeurs, il demanda à son serviteur
 de lui rendre des comptes. Le serviteur se doutant qu'il allait bientôt
 se retrouver sans travail, a rapidement utilisé les ressources
 de son maître pour préparer le futur qu'il voyait arriver.

Imaginons qu'il vous reste cinquante ans pour apprécier les bénédictions
 dont Dieu vous a comblées. Si vous utilisez ces choses seulement
 pour vous-mêmes sans égards pour le royaume de Dieu et le futur éternel,
 vous allez le regretter. Bien que vous soyez aisés maintenant,
 vous allez passer l'éternité dans la misère.

Permettez-moi de vous assurer que l'homme le plus pauvre au paradis
 est bien plus riche que l'homme le plus riche en enfer.
 Dieu va un jour demander à chacun de nous de Lui rendre des comptes
 sur la façon dont nous aurons utilisé les choses qu'Il a mises
 \grammar{choses qu'il a mises}
 à notre disposition. Si vous êtes sages,
 vous les utiliserez pour le royaume de Dieu.

Le ciel et la terre renferment tous les deux des richesses.
 Les richesses de la terre se mesurent en comptes en banque,
 portefeuilles de valeurs, amas de possessions que les hommes passent
 leurs vies à accumuler. Ces richesses vont finir par brûler si les mites
 et la rouille ne les détruisent pas avant. Les richesses du ciel sont,
 par contre, éternelles. Aucun feu ne peut les toucher.
 Aucune rouille ni aucune mite ne peuvent les endommager.
 Aucun voleur ne peut les dérober. Quoi que vous ayez déposé dans le ciel
 vous y attendra, précisément là, pour le jour où vous passerez ces portes.

\dvbox{
Investissez vos richesses soigneusement.
}

\dvrule

\dvprayer{
Père, fais de nous de bons gérants de ce que Tu nous a confié.
 Donne-nous Ta perspective sur ce que sont les vraies richesses.
}{\DlNdJ}


%%%%%%%%%%%%%%
% 6 aout
%%%%%%%%%%%%%%

\dvday{Les Neuf}

\dvquote{
L'un d'eux, se voyant guéri, revint sur ses pas et glorifia Dieu à haute voix.
 Il tomba face contre terre aux pieds de Jésus et lui rendit grâces.
 C'était un Samaritain. Jésus prit la parole et dit~:
 Les dix n'ont-ils pas été purifiés ? Mais les neuf autres, où sont-ils ?
}{\ibibleverse{Lc}(17:15-17)}

\lettrine{S}{i vous étiez en train de mourir} sans plus aucun espoir,
 et que quelqu'un arrivait et vous guérissait, ne croyez-vous pas
 que le moins que vous vouliez faire serait de dire merci?
 Jésus a guéri dix lépreux. Mais un seul est revenu exprimer sa gratitude.

Il est facile de considérer cette histoire et de secouer la tête
 devant l'ingratitude des neuf. Mais ne faisons nous pas, parfois,
 partie des neuf? Combien de fois Dieu nous a-t-Il bénis,
 nous nous sommes saisis de la bénédiction et \suggest{(nous) sommes repartis}
 reparti en courant sans regarder en arrière ni vers le ciel?
 Combien de fois nous a-t-Il permis de l'échapper belle,
 et nous avons continué nottre chemin sans le remercier?

\dvbox{
Beaucoup de gens ont vite fait de blâmer Dieu pour tous les problemes,
 au lieu de Le remercier.
}

\og Qu'ils célèbrent l'Éternel pour sa bienveillance
 et pour ses merveilles en faveur des humains ! \fg{}
 (\ibibleverse{Ps}(107:8)).
 Dieu est si bon envers nous, si digne de notre louange.

Les neuf ont vite fait de blâmer Dieu pour les choses qui ne vont pas
 dans leur vie, mais pas de Le remercier pour les choses qui vont bien.

Les neuf sont comme le garçon qui tombe en glissant du toit
 et qui crie~: \punct{deux-points}
 \og Oh Seigneur, aide-moi! \fg{} Quand ses pantalons s'accrochent à un clou,
 il s'arrête brutalement dans un grand bruit de déchirure.
 Levant les yeux vers le ciel, il dit alors~: \punct{deux-points}
 \og C'est plus la peine, Seigneur, le clou m'a arrêté. \fg{}

Êtes-vous avec l'un ou avec les neuf?

\dvrule

\dvprayer{
Père, nous venons dans le nom de Jésus Te remercier aujourd'hui
 pour Ta bonté et pour Tes œuvres merveilleuses pour nous.
}{}


