\dvmonth{Octobre}

%%%%%%%%%%%%%%%
% 1er octobre
%%%%%%%%%%%%%%%

\dvday{Notre Capacité}

\dvquote{
Non que nous soyons par nous-mêmes capables de concevoir quelque chose
 comme venant de nous-mêmes, mais notre capacité vient de Dieu.
}{\ibibleverse{IICo}(3:5)}

\dvlettrine{P}{aul venait de dire aux Corinthiens~:} \punct{deux-points}
 \og Vous êtes une lettre vivante, connue et lue de tous les hommes. \fg{}
 Votre vie est un exemple pour le monde de ce qu'est Dieu \ocadr de Son amour,
 Sa bonté et Sa grâce.
 Les gens se forment une impression de Dieu par ce qu'ils voient en vous.
 Comme Paul, quand vous considérez cette responsabilité, \suggest{virgule}
 vous pouvez vous demander~: \punct{deux-points}
 \og Qui est capable de faire ces choses? \fg{}
 Mais \typo{Mais} comme Paul le déclare~: \punct{deux-points}
 \og Notre capacité vient de Dieu. \fg{}

Peu importent vos ressources ou vos forces,
 vous allez être confrontés à des situations qui dépasseront vos capacités.

\dvbox{
Reconnaître votre incapacité vous conduit à vous appuyer
 sur la capacité infinie de Dieu.
}

Les gens font souvent le vœu de changer, seulement pour découvrir
 qu'ils n'ont pas en eux-mêmes la capacité de tenir ces promesses.
 Mais quand Dieu est votre capacité, il n'y a pas de limite à la puissance
 qui est à votre disposition.
 Sa grâce est sans mesure; Son amour sans limites;
 Sa puissance n'a pas de bornes connues des hommes.
 Car de Ses richesses infinies en Jésus, Il donne et donne, et donne encore.

Avec l'aide de Dieu et la force de Dieu, vous pouvez surmonter
 toute adversité qui vient dans votre vie. Reconnaissez vos limitations.
 Faites un pas de foi sachant que vous êtes faibles, qu'Il est fort,
 sachant que vous pouvez tout faire par Christ qui vous fortifie.

\dvrule

\dvprayer{
Seigneur, nous Te remercions de ce que Tu donnes de la force aux faibles.
 Nous reconnaissons que Tu nous a donné la grande responsabilité
 d'être Tes témoins dans le monde, et nous reconnaissons que nous n'avons pas
 la capacité d'accomplir cette tâche \ocadr mais que Toi, Tu l'as.
 Aide-nous à nous attendre à Toi, Seigneur.
}{\Amen}

