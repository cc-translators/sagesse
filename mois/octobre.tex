\dvmonth{Octobre}

%%%%%%%%%%%%%%%
% 1er octobre
%%%%%%%%%%%%%%%

\dvday{Notre Capacité}

\dvquote{
Non que nous soyons par nous-mêmes capables de concevoir quelque chose
 comme venant de nous-mêmes, mais notre capacité vient de Dieu.
}{\ibibleverse{IICo}(3:5)}

\dvlettrine{P}{aul venait de dire aux Corinthiens~:} \punct{deux-points}
 \og Vous êtes une lettre vivante, connue et lue de tous les hommes. \fg{}
 Votre vie est un exemple pour le monde de ce qu'est Dieu \ocadr de Son amour,
 Sa bonté et Sa grâce.
 Les gens se forment une impression de Dieu par ce qu'ils voient en vous.
 Comme Paul, quand vous considérez cette responsabilité, \suggest{virgule}
 vous pouvez vous demander~: \punct{deux-points}
 \og Qui est capable de faire ces choses? \fg{}
 Mais \typo{Mais} comme Paul le déclare~: \punct{deux-points}
 \og Notre capacité vient de Dieu. \fg{}

Peu importent vos ressources ou vos forces,
 vous allez être confrontés à des situations qui dépasseront vos capacités.

\dvbox{
Reconnaître votre incapacité vous conduit à vous appuyer
 sur la capacité infinie de Dieu.
}

Les gens font souvent le vœu de changer, seulement pour découvrir
 qu'ils n'ont pas en eux-mêmes la capacité de tenir ces promesses.
 Mais quand Dieu est votre capacité, il n'y a pas de limite à la puissance
 qui est à votre disposition.
 Sa grâce est sans mesure; Son amour sans limites;
 Sa puissance n'a pas de bornes connues des hommes.
 Car de Ses richesses infinies en Jésus, Il donne et donne, et donne encore.

Avec l'aide de Dieu et la force de Dieu, vous pouvez surmonter
 toute adversité qui vient dans votre vie. Reconnaissez vos limitations.
 Faites un pas de foi sachant que vous êtes faibles, qu'Il est fort,
 sachant que vous pouvez tout faire par Christ qui vous fortifie.

\dvrule

\dvprayer{
Seigneur, nous Te remercions de ce que Tu donnes de la force aux faibles.
 Nous reconnaissons que Tu nous a donné la grande responsabilité
 d'être Tes témoins dans le monde, et nous reconnaissons que nous n'avons pas
 la capacité d'accomplir cette tâche \ocadr mais que Toi, Tu l'as.
 Aide-nous à nous attendre à Toi, Seigneur.
}{\Amen}


%%%%%%%%%%%%%%%
% 2 octobre
%%%%%%%%%%%%%%%

\dvday{Voir l\ap{}Invisible}

\suggest{L majuscule?}

\dvquote{
Aussi nous regardons, non point aux choses visibles,
 mais à celles qui sont invisibles;
 car les choses visibles sont momentanées,
 et les invisibles sont éternelles.
}{\ibibleverse{IICo}(4:18)}

\dvlettrine{L}{e monde matériel}
 \ocadr le monde que nous voyons et touchons \fcadr{} a trois dimensions.
 Mais le monde spirituel a un nombre incalculable de dimensions auxquelles
 nous ne sommes pas habitués. Le monde spirituel est un monde invisible.
 Alors que le monde matériel est engagé dans un processus
 continuel de désintégration, le monde spirituel ne connaît pas
 de désintégration. Il est éternel.

\dvbox{
Quand nous laissons notre vision se tourner vers la terre,
 nous perdons notre perspective éternelle.
}

Que voyons-nous dans ce monde matériel à trois dimensions?
 Une grande complexité, des difficultés et des afflictions.
 Dieu ne nous promet pas le ciel dans ce monde maudit par le péché.
 Mais quand par la foi nous voyons la dimension spirituelle
 \ocadr la puissance de Dieu, la main de Dieu, l'amour de Dieu \fcadr{}
 nous trouvons la force d'endurer. Ayant porté les yeux sur Lui,
 ayant observé Sa puissance majestueuse, nous cessons de redouter
 les montagnes devant nous, parce que nous avons confiance
 qu'Il va nous nous en sortir.

Il est tragique de voir tant de gens échanger le monde
 qu'ils pourraient \grammar{pluriel}
 avoir pour le monde qu'ils connaissent.
 Ils sacrifient sans réfléchir le royaume éternel de Dieu
 pour quelques plaisirs temporels, quelques sensations, ou quelques possessions
 \ocadr plaisirs, sensations ou possessions qui vont finalement disparaître.
 Quand nous avons une perspective mondaine \suggest{wordly = matérialiste},
 nous perdons de vue l'éternité. Nous ne pouvons pas nous permettre de perdre
 cette perspective éternelle; c'est la clé de l'endurance.
 Au-delà de nos problèmes, nous devons nous tourner vers le Dieu éternel
 qui est notre refuge et notre force, et contempler les récompenses
 éternelles qui attendent ceux qui vivent une vie qui plaît à Dieu.

\dvrule

\dvprayer{
Père, le monde matériel est si séduisant et si attrayant par moments !
 Corrige notre perspective, Seigneur.
 Aide-nous à nous consacrer aux choses qui sont éternelles.
}{\DlNdJ}


%%%%%%%%%%%%%%%
% 3 octobre
%%%%%%%%%%%%%%%

\dvday{Perfectionner la Sainteté}

\dvquote{
\dots{} Je vous recevrai, et je serai pour vous un Père,
 et vous serez mes fils et mes filles\dots{}
 Ayant donc ces promesses, chers bien-aimés,
 nettoyons-nous nous-mêmes de toute souillure de la chair et de l'esprit,
 perfectionnant la sainteté dans la crainte de Dieu.
}{\ibibleverse{IICo}(6:17-18,7:1) \KJF}

\dvlettrine{À}{ la lumière} des grandes promesses que Dieu nous donne
 dans ces versets, quelle devrait être notre réponse? 

D'abord, nous devons nous nettoyer nous-mêmes de toute la saleté
 qui s'attache à nous quand nous essayons de satisfaire la chair.
 C'est la saleté qui est implantée dans nos pensées
 \ocadr les idées, les attitudes et les désirs qui n'ont rien à faire
 dans les pensées d'un enfant de Dieu. Comment faites-vous cela?
 Vous remplacez les mauvaises choses par de bonnes choses.
 Vous saturez vos pensées avec la Parole de Dieu.
 Comme le psalmiste le disait~: \punct{deux-points}
 \og Par quel moyen un jeune homme purifiera-t-il son chemin?
 En y prenant garde selon Ta Parole \fg{} (\ibibleverse{Ps}(119:9)).
 \punct{Point manquant}

\dvbox{
Quelle extraordinaire vérité~: vous êtes un enfant de Dieu.
}

Ensuite, nous devons nous séparer du monde incrédule qui nous entoure.
 Dieu veut que Son peuple soit saint. Le Seigneur dit~: \punct{deux-points}
 \og Sortez du milieu d’eux, et soyez séparés.
 Ne touchez pas à ce qui est impur \fg{} (\ibibleverse{IICo}(6:17)).
 En raison du grand amour que Dieu nous a montré,
 et des mesures extrêmes qu'Il a prises pour nous adopter
 et nous appeler Ses enfants, notre réponse devrait être la sainteté.
 Puisque Dieu désire cela de Ses enfants, nous devons décider de vivre
 des vies séparées de ceux qui nous entourent et marcher dans la pureté
 devant Dieu.

Par Jésus, Dieu vous a reçus et appelés les Siens.
 N'est-ce-pas là une motivation suffisante pour vouloir Lui plaire?

\dvrule

\dvprayer{
Dieu, aide-moi à être pur dans mes pensées, dans mon cœur
 et dans mon intelligence. Rends-moi pur comme Toi, Tu es pur.
}{\DlNdJ}


%%%%%%%%%%%%%%%
% 4 octobre
%%%%%%%%%%%%%%%

\dvday{Les Richesses de Sa Grâce}

\dvquote{
Car vous connaissez la grâce de notre Seigneur Jésus-Christ
 qui pour vous s'est fait pauvre de riche qu'Il était,
 afin que par Sa pauvreté vous soyez enrichis.
}{\ibibleverse{IICo}(8:9)}

\dvlettrine{Q}{ui pourra jamais estimer les richesses de Dieu?}
 Dans le \ibibleverse{Ps}(50:10), Il nous dit~: \punct{deux-points}
 \og Tous les animaux de la forêt sont à Moi,
 toutes les bêtes des montagnes par milliers. \fg{}
 Mais c'est seulement pour nous donner une image que nous pouvons comprendre.
 La vérité, c'est que tout l'univers et tout ce qu'il renferme
 Lui appartiennent. Et si ça ne suffisait pas, Il est capable au seul son
 de Sa voix, de faire apparaître un milliard d'autres univers
 qui seront créés par Sa Parole.

Jésus a prié~: \punct{deux-points}
 \og Père, glorifie-Moi auprès de Toi-même de la gloire
 que j'avais auprès de Toi, avant que le monde fût \fg{}
 (\ibibleverse{Jn}(17:5)).
 Toutes les richesses de Dieu sont partagées avec Son Fils, et pourtant,
 pour vous, Il s'est fait pauvre. Il a quitté le lieu de gloire,
 où Il était exalté et adoré, et Il S'est laissé naître dans une étable,
 naître de parents pauvres, naître dans l'obscurité.
 Il est venu sachant, qu'adulte, Il allait errer sur la terre,
 qu'Il ne possèderait rien alors que techniquement tout Lui appartenait.

\dvbox{
Les richesses que Jésus désire partager avec vous sont des richesses
 éternelles qui ne peuvent pas être mesurées avec des choses temporelles
 comme l'or ou l'argent.
}

Pierre décrit ces richesses comme \og un héritage qui ne peut ni se corrompre,
 ni se souiller, ni se flétrir et qui vous est réservé dans les cieux,
 à vous qui êtes gardés en la puissance de Dieu, par la foi \fg{}
 (\ibibleverse{IP}(1:4-5)). \punct{Point manquant}

Quelques richesses nous attendent dans les cieux.
 D'autres sont disponibles dès aujourd'hui~:
 joie abondante, miséricorde infinie, la paix de Christ,
 justice, espérance et amour.

\dvrule

\dvprayer{
Merci, Père, pour les merveilleuses richesses dont nous faisons l'expérience
 chaque jour en marchant dans l'amour avec Toi.
}{\DlNdJ}


%%%%%%%%%%%%%%%
% 5 octobre
%%%%%%%%%%%%%%%

\dvday{Le Combat Spirituel}

\dvquote{
Si nous marchons dans la chair, nous ne combattons pas
 selon la chair.
}{\ibibleverse{IICo}(10:3)}

\dvlettrine{I}{l est possible que,}
 les fois où nous nous sentons perturbés, contrariés, découragés,
 démoralisés ou déprimés sans raison apparente, il faille blâmer
 une force spirituelle.
 Que nous en soyons conscients ou pas, nous sommes tous engagés
 dans un combat spirituel. Et dans le combat spirituel,
 il est important que nous utilisions les bonnes armes.

Une des meilleures armes pour le combat spirituel, c'est la Parole de Dieu.
 \og Car la Parole de Dieu est vivante, agissante, plus acérée
 qu'aucune épée à deux tranchants \fg{} (\ibibleverse{He}(4:12)).
 Chaque fois que Satan l'a tenté, Jésus a répondu avec la Parole de Dieu.
 La Parole de Dieu devient une arme puissante dans votre vie
 contre les tentations placées devant vous par l'ennemi.

\dvbox{
Sachant combien la bataille serait féroce,
 Dieu a pris soin de nous équiper pour que nous nous battions efficacement.
}

La prière est une autre arme importante, mais étrangement,
 les gens lui réservent le rang de dernier recours.
 Nous avons la puissance de la prière. Utilisons ces armes puissantes
 pour résister en déjouant les ruses du diable qui cherche à nous détruire
 et à nous capturer.
 Puissions-nous utiliser ces armes pour proclamer la victoire
 que Jésus a déjà remportée à la croix!

\dvrule

\dvprayer{
Père, pardonne-nous de dépendre de nos propres ressources.
 Puissions-nous commencer à nous battre dans ce combat avec les avantages
 que Tu nous a donnés dans les choses de l'Esprit!
}{\DlNdJ}


%%%%%%%%%%%%%%%
% 6 octobre
%%%%%%%%%%%%%%%

\dvday{La Grâce Toute-Suffisante}

\dvquote{
Et Il m'a dit~:
 \og Ma Grâce te suffit, car Ma puissance s'accomplit dans la faiblesse. \fg{}
 Je me glorifierai donc bien plus volontiers de mes faiblesses,
 afin que la puissance de Christ repose sur moi.
}{\ibibleverse{IICo}(12:9)}

\dvlettrine{P}{aul avait ce qu'il appelait une \og écharde dans la chair \fg{}.}
 À trois reprises, il a demandé au Seigneur de l'enlever,
 et ce passage de l'Écriture est la réponse de Dieu.
 \og Ma grâce te suffit. \fg{}

La vie est remplie de chagrin, de douleur, de déceptions et de deuil,
 et Dieu ne nous promet pas l'immunité vis-à-vis de ces choses.
 La différence entre le chrétien et le non-chrétien,
 c'est que pour l'enfant de Dieu, ces expériences sont passées
 par le filtre de Dieu. Elles n'arrivent que dans la mesure
 où Dieu les laisse arriver.

\dvbox{
L'affliction résulte souvent en notre plus grande croissance spirituelle.
}

La souffrance développe une profonde relation avec Dieu qui semble
 ne pas pouvoir se développer en dehors de l'affliction.
 Paul a appris à louer Dieu par cette écharde dans sa chair.
 Le jour est même arrivé où Paul a vraiment commencé à se réjouir
 et à remercier Dieu pour cette écharde dans sa chair.
 Et il a écrit~: \punct{deux-points}
 \og J'estime qu'il n'y a pas de commune mesure entre les souffrances
 du temps présent et la gloire à venir qui sera révélée pour nous. \fg{}
 Cela a complètement changé son attitude. Il n'a plus vu l'écharde
 comme une malédiction mais comme une bénédiction.

Il est tellement important que nous arrivions à connaître
 et à faire l'expérience de cette grâce toute-suffisante,
 parce qu'elle nous guidera dans les moments les plus sombres,
 et nous aidera à garder le sourire dans nos moments de déception.
 C'est la grâce qui va nous soutenir quand tout le reste échoue.
 C'est la grâce qui transforme votre croix en couronne.

\dvrule

\dvprayer{
Père, nous Te remercions pour Ta grâce toute-suffisante qui nous soutient.
 Puissions-nous nous reposer dans cette grâce quand nous en avons
 le plus besoin.
}{\DlNdJ}


%%%%%%%%%%%%%%%
% 7 octobre
%%%%%%%%%%%%%%%

\dvday{Grâce et Paix}

\dvquote{
Que la grâce et la paix vous soient donnés de la part de Dieu,
 notre Père, et du Seigneur Jésus-Christ.
}{\ibibleverse{Ga}(1:3)}

\dvlettrine{P}{aul a pris la salutation grecque} \emph{charis}
 qui veut dire \og grâce \fg{}, et l'a combinée avec la salutation
 hébraïque \emph{shalom} qui veut dire \og paix \fg{}~: \punct{deux-points}
 \og Grâce et paix. \fg{}
 Cette salutation est utilisée dix-sept fois dans le Nouveau Testament,
 toujours dans cet ordre~: \punct{deux-points} \og grâce et paix. \fg{}
 Je pense que ce n'est pas par hasard, parce que j'ai découvert
 dans ma propre vie que je n'ai pas connu la paix de Dieu
 avant d'avoir compris la grâce de Dieu.

J'ai grandi en apprenant que je devais gagner les bénédictions de Dieu.
 Si je travaillais dur et que je respectais mes engagements,
 Dieu allait me bénir. Mais si j'échouais, Il ne le ferait pas.
 J'essayais de mériter quelque chose qui ne peut pas se mériter.
 Le mot \og grâce \fg{} signifie \og faveur imméritée, indûe,
 non gagnée. \fg{} Si nous pouvions la gagner,
 ce ne serait plus de la grâce, mais un paiement.

\dvbox{
Les bénédictions de Dieu ne dépendent pas de notre fidélité
 ou de notre dur labeur mais sont fondées sur Sa nature d'amour.
}

Maintenant que j'ai une meilleure compréhension de la grâce,
 j'ai appris à attendre que Dieu me bénisse même si je suis pleinement
 conscient que je ne le mérite pas.

Paul nous dit que Dieu a manifesté Sa grâce envers nous en envoyant Son Fils
 qui S'est offert Lui-même pour nos péchés.
 \og Nous étions tous errants comme des brebis, chacun suivait sa propre voie;
 et l'Éternel a fait retomber sur Lui la faute de nous tous \fg{}
 (\ibibleverse{Is}(53:6)).
 Lorsque nous étions encore pécheurs, Christ est mort
 \ocadr pas pour les gens biens, pas pour les gens vertueux,
 mais pour les gens mauvais.

C'est ça, \punct{virgule} la grâce.

\dvrule

\dvprayer{
Merci, Seigneur, pour la merveilleuse paix et la compréhension des choses
 où Tu gardes nos cœurs et nos pensée en raison de Ta grâce.
}{\DlNdJ}

