\dvmonth{Octobre}

%%%%%%%%%%%%%%%
% 1er octobre
%%%%%%%%%%%%%%%

\dvday{Notre Capacité}

\dvquote{
Non que nous soyons par nous-mêmes capables de concevoir quelque chose
 comme venant de nous-mêmes, mais notre capacité vient de Dieu.
}{\ibibleverse{IICo}(3:5)}

\dvlettrine{P}{aul venait de dire aux Corinthiens~:} \punct{deux-points}
 \og Vous êtes une lettre vivante, connue et lue de tous les hommes. \fg{}
 Votre vie est un exemple pour le monde de ce qu'est Dieu \ocadr de Son amour,
 Sa bonté et Sa grâce.
 Les gens se forment une impression de Dieu par ce qu'ils voient en vous.
 Comme Paul, quand vous considérez cette responsabilité, \suggest{virgule}
 vous pouvez vous demander~: \punct{deux-points}
 \og Qui est capable de faire ces choses? \fg{}
 Mais \typo{Mais} comme Paul le déclare~: \punct{deux-points}
 \og Notre capacité vient de Dieu. \fg{}

Peu importent vos ressources ou vos forces,
 vous allez être confrontés à des situations qui dépasseront vos capacités.

\dvbox{
Reconnaître votre incapacité vous conduit à vous appuyer
 sur la capacité infinie de Dieu.
}

Les gens font souvent le vœu de changer, seulement pour découvrir
 qu'ils n'ont pas en eux-mêmes la capacité de tenir ces promesses.
 Mais quand Dieu est votre capacité, il n'y a pas de limite à la puissance
 qui est à votre disposition.
 Sa grâce est sans mesure; Son amour sans limites;
 Sa puissance n'a pas de bornes connues des hommes.
 Car de Ses richesses infinies en Jésus, Il donne et donne, et donne encore.

Avec l'aide de Dieu et la force de Dieu, vous pouvez surmonter
 toute adversité qui vient dans votre vie. Reconnaissez vos limitations.
 Faites un pas de foi sachant que vous êtes faibles, qu'Il est fort,
 sachant que vous pouvez tout faire par Christ qui vous fortifie.

\dvrule

\dvprayer{
Seigneur, nous Te remercions de ce que Tu donnes de la force aux faibles.
 Nous reconnaissons que Tu nous a donné la grande responsabilité
 d'être Tes témoins dans le monde, et nous reconnaissons que nous n'avons pas
 la capacité d'accomplir cette tâche \ocadr mais que Toi, Tu l'as.
 Aide-nous à nous attendre à Toi, Seigneur.
}{\Amen}


%%%%%%%%%%%%%%%
% 2 octobre
%%%%%%%%%%%%%%%

\dvday{Voir l\ap{}Invisible}

\suggest{L majuscule?}

\dvquote{
Aussi nous regardons, non point aux choses visibles,
 mais à celles qui sont invisibles;
 car les choses visibles sont momentanées,
 et les invisibles sont éternelles.
}{\ibibleverse{IICo}(4:18)}

\dvlettrine{L}{e monde matériel}
 \ocadr le monde que nous voyons et touchons \fcadr{} a trois dimensions.
 Mais le monde spirituel a un nombre incalculable de dimensions auxquelles
 nous ne sommes pas habitués. Le monde spirituel est un monde invisible.
 Alors que le monde matériel est engagé dans un processus
 continuel de désintégration, le monde spirituel ne connaît pas
 de désintégration. Il est éternel.

\dvbox{
Quand nous laissons notre vision se tourner vers la terre,
 nous perdons notre perspective éternelle.
}

Que voyons-nous dans ce monde matériel à trois dimensions?
 Une grande complexité, des difficultés et des afflictions.
 Dieu ne nous promet pas le ciel dans ce monde maudit par le péché.
 Mais quand par la foi nous voyons la dimension spirituelle
 \ocadr la puissance de Dieu, la main de Dieu, l'amour de Dieu \fcadr{}
 nous trouvons la force d'endurer. Ayant porté les yeux sur Lui,
 ayant observé Sa puissance majestueuse, nous cessons de redouter
 les montagnes devant nous, parce que nous avons confiance
 qu'Il va nous nous en sortir.

Il est tragique de voir tant de gens échanger le monde
 qu'ils pourraient \grammar{pluriel}
 avoir pour le monde qu'ils connaissent.
 Ils sacrifient sans réfléchir le royaume éternel de Dieu
 pour quelques plaisirs temporels, quelques sensations, ou quelques possessions
 \ocadr plaisirs, sensations ou possessions qui vont finalement disparaître.
 Quand nous avons une perspective mondaine \suggest{wordly = matérialiste},
 nous perdons de vue l'éternité. Nous ne pouvons pas nous permettre de perdre
 cette perspective éternelle; c'est la clé de l'endurance.
 Au-delà de nos problèmes, nous devons nous tourner vers le Dieu éternel
 qui est notre refuge et notre force, et contempler les récompenses
 éternelles qui attendent ceux qui vivent une vie qui plaît à Dieu.

\dvrule

\dvprayer{
Père, le monde matériel est si séduisant et si attrayant par moments !
 Corrige notre perspective, Seigneur.
 Aide-nous à nous consacrer aux choses qui sont éternelles.
}{\DlNdJ}


%%%%%%%%%%%%%%%
% 3 octobre
%%%%%%%%%%%%%%%

\dvday{Perfectionner la Sainteté}

\dvquote{
\dots{} Je vous recevrai, et je serai pour vous un Père,
 et vous serez mes fils et mes filles\dots{}
 Ayant donc ces promesses, chers bien-aimés,
 nettoyons-nous nous-mêmes de toute souillure de la chair et de l'esprit,
 perfectionnant la sainteté dans la crainte de Dieu.
}{\ibibleverse{IICo}(6:17-18,7:1) \KJF}

\dvlettrine{À}{ la lumière} des grandes promesses que Dieu nous donne
 dans ces versets, quelle devrait être notre réponse? 

D'abord, nous devons nous nettoyer nous-mêmes de toute la saleté
 qui s'attache à nous quand nous essayons de satisfaire la chair.
 C'est la saleté qui est implantée dans nos pensées
 \ocadr les idées, les attitudes et les désirs qui n'ont rien à faire
 dans les pensées d'un enfant de Dieu. Comment faites-vous cela?
 Vous remplacez les mauvaises choses par de bonnes choses.
 Vous saturez vos pensées avec la Parole de Dieu.
 Comme le psalmiste le disait~: \punct{deux-points}
 \og Par quel moyen un jeune homme purifiera-t-il son chemin?
 En y prenant garde selon Ta Parole \fg{} (\ibibleverse{Ps}(119:9)).
 \punct{Point manquant}

\dvbox{
Quelle extraordinaire vérité~: vous êtes un enfant de Dieu.
}

Ensuite, nous devons nous séparer du monde incrédule qui nous entoure.
 Dieu veut que Son peuple soit saint. Le Seigneur dit~: \punct{deux-points}
 \og Sortez du milieu d’eux, et soyez séparés.
 Ne touchez pas à ce qui est impur \fg{} (\ibibleverse{IICo}(6:17)).
 En raison du grand amour que Dieu nous a montré,
 et des mesures extrêmes qu'Il a prises pour nous adopter
 et nous appeler Ses enfants, notre réponse devrait être la sainteté.
 Puisque Dieu désire cela de Ses enfants, nous devons décider de vivre
 des vies séparées de ceux qui nous entourent et marcher dans la pureté
 devant Dieu.

Par Jésus, Dieu vous a reçus et appelés les Siens.
 N'est-ce-pas là une motivation suffisante pour vouloir Lui plaire?

\dvrule

\dvprayer{
Dieu, aide-moi à être pur dans mes pensées, dans mon cœur
 et dans mon intelligence. Rends-moi pur comme Toi, Tu es pur.
}{\DlNdJ}






