\dvmonth{Octobre}

%%%%%%%%%%%%%%%
% 1er octobre
%%%%%%%%%%%%%%%

\dvday{Notre Capacité}

\dvquote{
Non que nous soyons par nous-mêmes capables de concevoir quelque chose
 comme venant de nous-mêmes, mais notre capacité vient de Dieu.
}{\ibibleverse{IICo}(3:5)}

\dvlettrine{P}{aul venait de dire aux Corinthiens~:} \punct{deux-points}
 \og Vous êtes une lettre vivante, connue et lue de tous les hommes. \fg{}
 Votre vie est un exemple pour le monde de ce qu'est Dieu \ocadr de Son amour,
 Sa bonté et Sa grâce.
 Les gens se forment une impression de Dieu par ce qu'ils voient en vous.
 Comme Paul, quand vous considérez cette responsabilité, \suggest{virgule}
 vous pouvez vous demander~: \punct{deux-points}
 \og Qui est capable de faire ces choses? \fg{}
 Mais \typo{Mais} comme Paul le déclare~: \punct{deux-points}
 \og Notre capacité vient de Dieu. \fg{}

Peu importent vos ressources ou vos forces,
 vous allez être confrontés à des situations qui dépasseront vos capacités.

\dvbox{
Reconnaître votre incapacité vous conduit à vous appuyer
 sur la capacité infinie de Dieu.
}

Les gens font souvent le vœu de changer, seulement pour découvrir
 qu'ils n'ont pas en eux-mêmes la capacité de tenir ces promesses.
 Mais quand Dieu est votre capacité, il n'y a pas de limite à la puissance
 qui est à votre disposition.
 Sa grâce est sans mesure; Son amour sans limites;
 Sa puissance n'a pas de bornes connues des hommes.
 Car de Ses richesses infinies en Jésus, Il donne et donne, et donne encore.

Avec l'aide de Dieu et la force de Dieu, vous pouvez surmonter
 toute adversité qui vient dans votre vie. Reconnaissez vos limitations.
 Faites un pas de foi sachant que vous êtes faibles, qu'Il est fort,
 sachant que vous pouvez tout faire par Christ qui vous fortifie.

\dvrule

\dvprayer{
Seigneur, nous Te remercions de ce que Tu donnes de la force aux faibles.
 Nous reconnaissons que Tu nous a donné la grande responsabilité
 d'être Tes témoins dans le monde, et nous reconnaissons que nous n'avons pas
 la capacité d'accomplir cette tâche \ocadr mais que Toi, Tu l'as.
 Aide-nous à nous attendre à Toi, Seigneur.
}{\Amen}


%%%%%%%%%%%%%%%
% 2 octobre
%%%%%%%%%%%%%%%

\dvday{Voir l\ap{}Invisible}

\suggest{L majuscule?}

\dvquote{
Aussi nous regardons, non point aux choses visibles,
 mais à celles qui sont invisibles;
 car les choses visibles sont momentanées,
 et les invisibles sont éternelles.
}{\ibibleverse{IICo}(4:18)}

\dvlettrine{L}{e monde matériel}
 \ocadr le monde que nous voyons et touchons \fcadr{} a trois dimensions.
 Mais le monde spirituel a un nombre incalculable de dimensions auxquelles
 nous ne sommes pas habitués. Le monde spirituel est un monde invisible.
 Alors que le monde matériel est engagé dans un processus
 continuel de désintégration, le monde spirituel ne connaît pas
 de désintégration. Il est éternel.

\dvbox{
Quand nous laissons notre vision se tourner vers la terre,
 nous perdons notre perspective éternelle.
}

Que voyons-nous dans ce monde matériel à trois dimensions?
 Une grande complexité, des difficultés et des afflictions.
 Dieu ne nous promet pas le ciel dans ce monde maudit par le péché.
 Mais quand par la foi nous voyons la dimension spirituelle
 \ocadr la puissance de Dieu, la main de Dieu, l'amour de Dieu \fcadr{}
 nous trouvons la force d'endurer. Ayant porté les yeux sur Lui,
 ayant observé Sa puissance majestueuse, nous cessons de redouter
 les montagnes devant nous, parce que nous avons confiance
 qu'Il va nous nous en sortir.

Il est tragique de voir tant de gens échanger le monde
 qu'ils pourraient \grammar{pluriel}
 avoir pour le monde qu'ils connaissent.
 Ils sacrifient sans réfléchir le royaume éternel de Dieu
 pour quelques plaisirs temporels, quelques sensations, ou quelques possessions
 \ocadr plaisirs, sensations ou possessions qui vont finalement disparaître.
 Quand nous avons une perspective mondaine \suggest{wordly = matérialiste},
 nous perdons de vue l'éternité. Nous ne pouvons pas nous permettre de perdre
 cette perspective éternelle; c'est la clé de l'endurance.
 Au-delà de nos problèmes, nous devons nous tourner vers le Dieu éternel
 qui est notre refuge et notre force, et contempler les récompenses
 éternelles qui attendent ceux qui vivent une vie qui plaît à Dieu.

\dvrule

\dvprayer{
Père, le monde matériel est si séduisant et si attrayant par moments !
 Corrige notre perspective, Seigneur.
 Aide-nous à nous consacrer aux choses qui sont éternelles.
}{\DlNdJ}


%%%%%%%%%%%%%%%
% 3 octobre
%%%%%%%%%%%%%%%

\dvday{Perfectionner la Sainteté}

\dvquote{
\dots{} Je vous recevrai, et je serai pour vous un Père,
 et vous serez mes fils et mes filles\dots{}
 Ayant donc ces promesses, chers bien-aimés,
 nettoyons-nous nous-mêmes de toute souillure de la chair et de l'esprit,
 perfectionnant la sainteté dans la crainte de Dieu.
}{\ibibleverse{IICo}(6:17-18)(7:1) \KJF}

\dvlettrine{À}{ la lumière} des grandes promesses que Dieu nous donne
 dans ces versets, quelle devrait être notre réponse? 

D'abord, nous devons nous nettoyer nous-mêmes de toute la saleté
 qui s'attache à nous quand nous essayons de satisfaire la chair.
 C'est la saleté qui est implantée dans nos pensées
 \ocadr les idées, les attitudes et les désirs qui n'ont rien à faire
 dans les pensées d'un enfant de Dieu. Comment faites-vous cela?
 Vous remplacez les mauvaises choses par de bonnes choses.
 Vous saturez vos pensées avec la Parole de Dieu.
 Comme le psalmiste le disait~: \punct{deux-points}
 \og Par quel moyen un jeune homme purifiera-t-il son chemin?
 En y prenant garde selon Ta Parole \fg{} (\ibibleverse{Ps}(119:9)).
 \punct{Point manquant}

\dvbox{
Quelle extraordinaire vérité~: vous êtes un enfant de Dieu.
}

Ensuite, nous devons nous séparer du monde incrédule qui nous entoure.
 Dieu veut que Son peuple soit saint. Le Seigneur dit~: \punct{deux-points}
 \og Sortez du milieu d’eux, et soyez séparés.
 Ne touchez pas à ce qui est impur \fg{} (\ibibleverse{IICo}(6:17)).
 En raison du grand amour que Dieu nous a montré,
 et des mesures extrêmes qu'Il a prises pour nous adopter
 et nous appeler Ses enfants, notre réponse devrait être la sainteté.
 Puisque Dieu désire cela de Ses enfants, nous devons décider de vivre
 des vies séparées de ceux qui nous entourent et marcher dans la pureté
 devant Dieu.

Par Jésus, Dieu vous a reçus et appelés les Siens.
 N'est-ce-pas là une motivation suffisante pour vouloir Lui plaire?

\dvrule

\dvprayer{
Dieu, aide-moi à être pur dans mes pensées, dans mon cœur
 et dans mon intelligence. Rends-moi pur comme Toi, Tu es pur.
}{\DlNdJ}


%%%%%%%%%%%%%%%
% 4 octobre
%%%%%%%%%%%%%%%

\dvday{Les Richesses de Sa Grâce}

\dvquote{
Car vous connaissez la grâce de notre Seigneur Jésus-Christ
 qui pour vous s'est fait pauvre de riche qu'Il était,
 afin que par Sa pauvreté vous soyez enrichis.
}{\ibibleverse{IICo}(8:9)}

\dvlettrine{Q}{ui pourra jamais estimer les richesses de Dieu?}
 Dans le \ibibleverse{Ps}(50:10), Il nous dit~: \punct{deux-points}
 \og Tous les animaux de la forêt sont à Moi,
 toutes les bêtes des montagnes par milliers. \fg{}
 Mais c'est seulement pour nous donner une image que nous pouvons comprendre.
 La vérité, c'est que tout l'univers et tout ce qu'il renferme
 Lui appartiennent. Et si ça ne suffisait pas, Il est capable au seul son
 de Sa voix, de faire apparaître un milliard d'autres univers
 qui seront créés par Sa Parole.

Jésus a prié~: \punct{deux-points}
 \og Père, glorifie-Moi auprès de Toi-même de la gloire
 que j'avais auprès de Toi, avant que le monde fût \fg{}
 (\ibibleverse{Jn}(17:5)).
 Toutes les richesses de Dieu sont partagées avec Son Fils, et pourtant,
 pour vous, Il s'est fait pauvre. Il a quitté le lieu de gloire,
 où Il était exalté et adoré, et Il S'est laissé naître dans une étable,
 naître de parents pauvres, naître dans l'obscurité.
 Il est venu sachant, qu'adulte, Il allait errer sur la terre,
 qu'Il ne possèderait rien alors que techniquement tout Lui appartenait.

\dvbox{
Les richesses que Jésus désire partager avec vous sont des richesses
 éternelles qui ne peuvent pas être mesurées avec des choses temporelles
 comme l'or ou l'argent.
}

Pierre décrit ces richesses comme \og un héritage qui ne peut ni se corrompre,
 ni se souiller, ni se flétrir et qui vous est réservé dans les cieux,
 à vous qui êtes gardés en la puissance de Dieu, par la foi \fg{}
 (\ibibleverse{IP}(1:4-5)). \punct{Point manquant}

Quelques richesses nous attendent dans les cieux.
 D'autres sont disponibles dès aujourd'hui~:
 joie abondante, miséricorde infinie, la paix de Christ,
 justice, espérance et amour.

\dvrule

\dvprayer{
Merci, Père, pour les merveilleuses richesses dont nous faisons l'expérience
 chaque jour en marchant dans l'amour avec Toi.
}{\DlNdJ}


%%%%%%%%%%%%%%%
% 5 octobre
%%%%%%%%%%%%%%%

\dvday{Le Combat Spirituel}

\dvquote{
Si nous marchons dans la chair, nous ne combattons pas
 selon la chair.
}{\ibibleverse{IICo}(10:3)}

\dvlettrine{I}{l est possible que,}
 les fois où nous nous sentons perturbés, contrariés, découragés,
 démoralisés ou déprimés sans raison apparente, il faille blâmer
 une force spirituelle.
 Que nous en soyons conscients ou pas, nous sommes tous engagés
 dans un combat spirituel. Et dans le combat spirituel,
 il est important que nous utilisions les bonnes armes.

Une des meilleures armes pour le combat spirituel, c'est la Parole de Dieu.
 \og Car la Parole de Dieu est vivante, agissante, plus acérée
 qu'aucune épée à deux tranchants \fg{} (\ibibleverse{He}(4:12)).
 Chaque fois que Satan l'a tenté, Jésus a répondu avec la Parole de Dieu.
 La Parole de Dieu devient une arme puissante dans votre vie
 contre les tentations placées devant vous par l'ennemi.

\dvbox{
Sachant combien la bataille serait féroce,
 Dieu a pris soin de nous équiper pour que nous nous battions efficacement.
}

La prière est une autre arme importante, mais étrangement,
 les gens lui réservent le rang de dernier recours.
 Nous avons la puissance de la prière. Utilisons ces armes puissantes
 pour résister en déjouant les ruses du diable qui cherche à nous détruire
 et à nous capturer.
 Puissions-nous utiliser ces armes pour proclamer la victoire
 que Jésus a déjà remportée à la croix!

\dvrule

\dvprayer{
Père, pardonne-nous de dépendre de nos propres ressources.
 Puissions-nous commencer à nous battre dans ce combat avec les avantages
 que Tu nous a donnés dans les choses de l'Esprit!
}{\DlNdJ}


%%%%%%%%%%%%%%%
% 6 octobre
%%%%%%%%%%%%%%%

\dvday{La Grâce Toute-Suffisante}

\dvquote{
Et Il m'a dit~:
 \og Ma Grâce te suffit, car Ma puissance s'accomplit dans la faiblesse. \fg{}
 Je me glorifierai donc bien plus volontiers de mes faiblesses,
 afin que la puissance de Christ repose sur moi.
}{\ibibleverse{IICo}(12:9)}

\dvlettrine{P}{aul avait ce qu'il appelait une \og écharde dans la chair \fg{}.}
 À trois reprises, il a demandé au Seigneur de l'enlever,
 et ce passage de l'Écriture est la réponse de Dieu.
 \og Ma grâce te suffit. \fg{}

La vie est remplie de chagrin, de douleur, de déceptions et de deuil,
 et Dieu ne nous promet pas l'immunité vis-à-vis de ces choses.
 La différence entre le chrétien et le non-chrétien,
 c'est que pour l'enfant de Dieu, ces expériences sont passées
 par le filtre de Dieu. Elles n'arrivent que dans la mesure
 où Dieu les laisse arriver.

\dvbox{
L'affliction résulte souvent en notre plus grande croissance spirituelle.
}

La souffrance développe une profonde relation avec Dieu qui semble
 ne pas pouvoir se développer en dehors de l'affliction.
 Paul a appris à louer Dieu par cette écharde dans sa chair.
 Le jour est même arrivé où Paul a vraiment commencé à se réjouir
 et à remercier Dieu pour cette écharde dans sa chair.
 Et il a écrit~: \punct{deux-points}
 \og J'estime qu'il n'y a pas de commune mesure entre les souffrances
 du temps présent et la gloire à venir qui sera révélée pour nous. \fg{}
 Cela a complètement changé son attitude. Il n'a plus vu l'écharde
 comme une malédiction mais comme une bénédiction.

Il est tellement important que nous arrivions à connaître
 et à faire l'expérience de cette grâce toute-suffisante,
 parce qu'elle nous guidera dans les moments les plus sombres,
 et nous aidera à garder le sourire dans nos moments de déception.
 C'est la grâce qui va nous soutenir quand tout le reste échoue.
 C'est la grâce qui transforme votre croix en couronne.

\dvrule

\dvprayer{
Père, nous Te remercions pour Ta grâce toute-suffisante qui nous soutient.
 Puissions-nous nous reposer dans cette grâce quand nous en avons
 le plus besoin.
}{\DlNdJ}


%%%%%%%%%%%%%%%
% 7 octobre
%%%%%%%%%%%%%%%

\dvday{Grâce et Paix}

\dvquote{
Que la grâce et la paix vous soient donnés de la part de Dieu,
 notre Père, et du Seigneur Jésus-Christ.
}{\ibibleverse{Ga}(1:3)}

\dvlettrine{P}{aul a pris la salutation grecque} \emph{charis}
 qui veut dire \og grâce \fg{}, et l'a combinée avec la salutation
 hébraïque \emph{shalom} qui veut dire \og paix \fg{}~: \punct{deux-points}
 \og Grâce et paix. \fg{}
 Cette salutation est utilisée dix-sept fois dans le Nouveau Testament,
 toujours dans cet ordre~: \punct{deux-points} \og grâce et paix. \fg{}
 Je pense que ce n'est pas par hasard, parce que j'ai découvert
 dans ma propre vie que je n'ai pas connu la paix de Dieu
 avant d'avoir compris la grâce de Dieu.

J'ai grandi en apprenant que je devais gagner les bénédictions de Dieu.
 Si je travaillais dur et que je respectais mes engagements,
 Dieu allait me bénir. Mais si j'échouais, Il ne le ferait pas.
 J'essayais de mériter quelque chose qui ne peut pas se mériter.
 Le mot \og grâce \fg{} signifie \og faveur imméritée, indûe,
 non gagnée. \fg{} Si nous pouvions la gagner,
 ce ne serait plus de la grâce, mais un paiement.

\dvbox{
Les bénédictions de Dieu ne dépendent pas de notre fidélité
 ou de notre dur labeur mais sont fondées sur Sa nature d'amour.
}

Maintenant que j'ai une meilleure compréhension de la grâce,
 j'ai appris à attendre que Dieu me bénisse même si je suis pleinement
 conscient que je ne le mérite pas.

Paul nous dit que Dieu a manifesté Sa grâce envers nous en envoyant Son Fils
 qui S'est offert Lui-même pour nos péchés.
 \og Nous étions tous errants comme des brebis, chacun suivait sa propre voie;
 et l'Éternel a fait retomber sur Lui la faute de nous tous \fg{}
 (\ibibleverse{Is}(53:6)).
 Lorsque nous étions encore pécheurs, Christ est mort
 \ocadr pas pour les gens biens, pas pour les gens vertueux,
 mais pour les gens mauvais.

C'est ça, \punct{virgule} la grâce.

\dvrule

\dvprayer{
Merci, Seigneur, pour la merveilleuse paix et la compréhension des choses
 où Tu gardes nos cœurs et nos pensée en raison de Ta grâce.
}{\DlNdJ}


%%%%%%%%%%%%%%%
% 8 octobre
%%%%%%%%%%%%%%%

\dvday{Merveilleuse Délivrance}

\dvquote{
Je suis crucifié avec Christ, et ce n'est plus moi qui vis,
 c'est Christ, qui vit en moi ; ma vie présente dans la chair,
 je (la) vis dans la foi au Fils de Dieu, qui m'a aimé
 et qui s'est livré lui-même pour moi.
}{\ibibleverse{Ga}(2:20)}

\dvlettrine{D}{ieu veut nous donner la victoire sur le péché.}
 Il sait que nous sommes tentés à tout bout de champ,
 et Il veut nous libérer à la fois de la tentation et des conséquences
 du péché. Sa solution, c'est que nous mettions à mort la vie de la chair.
 Paul a écrit~: \punct{deux-points}
 \og Nous savons que notre vieille nature a été crucifiée avec Lui,
 afin que ce corps de péché soit réduit à l'impuissance
 et que nous ne soyons plus esclaves du péché \fg{} (\ibibleverse{Rm}(6:6)).
 \og Que le péché ne règne donc pas dans votre corps mortel,
 et n'obéissez pas à ses convoitises. Ne livrez pas vos membres au péché,
 comme armes pour l'injustice\dots{}
 Le péché ne dominera pas sur vous \fg{} (\ibibleverse{Rm}(6:12-14)).

\dvbox{
La vieille nature pécheresse n'a plus à nous gouverner.
}

Quand la tentation surgit, nous pouvons considérer cette vieille nature
 comme étant crucifiée avec Christ. Quand nous chutons, nous pouvons
 tout de suite ramener ce péché à la croix et dire~: \punct{deux-points, majuscule}
 \og J'ai été crucifié \ocadr ce péché ne peut pas me dominer;
 je ne vais pas être gouverné par ça. \fg{}
 Et progressivement, lentement, le Seigneur va commencer à nous délivrer
 de ce péché jusqu'à ce qu'il perde son attraction sur nous.

Qu'il est merveilleux de voir Dieu faire ce que nous ne pouvons pas faire
 pour nous mêmes \ocadr nous délivrer.

\dvrule

\dvprayer{
Père, nous Te remercions pour la victoire que nous avons en Jésus-Christ
 sur la puissance que le péché avait autrefois sur nous.
 Aide-nous Seigneur, à saisir ces vérités et à vivre dans la justice
 et la paix.
}{\DlNdJ}


%%%%%%%%%%%%%%%
% 9 octobre
%%%%%%%%%%%%%%%

\dvday{La Malédiction du Péché}

\dvquote{
Christ nous a rachetés de la malédiction de la loi,
 étant devenu malédiction pour nous \ocadr car il est écrit~:
 \og Maudit soit quiconque est pendu au bois. \fg{}
}{\ibibleverse{Ga}(3:13)}

\dvlettrine{L}{a loi que Dieu avait donnée aux enfants d'Israël}
 était destinée à leur bénéfice, à leur bénédiction.
 Mais la loi est dure. La loi porte en elle une malédiction,
 qui dit que ceux qui n'obéissent pas à la loi doivent mourir.
 Comme Jacques l'a fait remarquer~: \punct{deux-points}
 \og Quiconque observe toute la loi, mais pèche contre un seul commandement,
 devient coupable envers tous \fg{} (\ibibleverse{Jc}(2:10)).
 Si vous essayez d'être justes devant Dieu en gardant la loi
 mais que vous en violez un seul point,
 alors vous tombez sous le coup de la malédiction de la loi.

Ainsi donc, nous sommes coupables. Nous méritons la mort.
 Cependant Jésus a pris nos péchés et Il est mort à notre place.
 Non seulement cela signifie que nous n'avons à souffrir de la malédiction
 de la loi, mais cela signifie aussi que nous sommes bénis.
 Son sacrifice nous a amené les bénédictions d'Abraham.
 Car nous lisons~: \punct{deux-points}
 \og afin que, pour les païens, la bénédiction d'Abraham se trouve
 en Jésus-Christ et que, par la foi, nous recevions la promesse
 de l'Esprit \fg{} (\ibibleverse{Ga}(3:14)).

\dvbox{
Par sa mort sur l'arbre, Jésus nous a amené la liberté, la délivrance,
 le pardon, la justice et l'espérance.
}

Oh, les bénédictions de Dieu qui sont arrivées jusqu'à moi
 par Jésus-Christ! Comme Il est bon, et comme nous sommes aimés,
 pour qu'Il ait accepté de devenir malédiction pour nous,
 pour qu'Il endure autant de honte, autant de douleur,
 afin que nous soyons libérés de la malédiction de la loi!

\dvrule

\dvprayer{
Père, comme nous Te sommes reconnaissants d'avoir envoyé Ton Fils
 pour nous racheter de la malédiction de la loi.
 Comme nous Te sommes reconnaissants pour Ton amour et Ta miséricorde.
}{\DlNdJ}


%%%%%%%%%%%%%%%
% 10 octobre
%%%%%%%%%%%%%%%

\dvday{Abba, Père}

\dvquote{
Et parce que vous êtes des fils, Dieu a envoyé dans nos cœurs
 l'Esprit de son Fils, qui crie~: \og Abba ! Père ! \fg{}
}{\ibibleverse{Ga}(4:6)}

\punct{Point en trop}

\dvlettrine{J}{e ne suis pas un fils de Dieu par la naissance naturelle;}
 je suis un fils de Dieu par la nouvelle naissance, en étant né de nouveau
 de Son Esprit.
 \og Car tous ceux qui sont conduits par l'Esprit de Dieu sont fils de Dieu.
 Et vous n'avez pas reçu un esprit de servitude, pour être encore
 dans la crainte, mais vous avez reçu un Esprit d'adoption, par lequel
 nous crions~: \og Abba ! Père ! \fg{} (\ibibleverse{Rm}(8:14-15)).
 \punct{On peut utiliser des guillemets français emboîtés, et on ne les ferme qu'une fois}

\emph{Abba}, est le mot hébreu pour \og père \fg{},
 mais il s'utilise davantage comme notre mot français \og papa \fg{}.
 Ce mot a une dimension attendrissante et intime.

\dvbox{
Nous devons croire que Dieu est \emph{notre} Abba, notre papa.
}

Certains affirment que parce que Dieu est si grandiose et impressionnant,
 nous ne devrions même pas prononcer Son nom. D'autres, au contraire,
 deviennent trop familiers, trop désinvoltes avec Dieu.
 Ils parlent de Lui comme du \og Bon Père Noël à l'étage. \fg{}

Il nous faut un équilibre entre ces deux positions.
 Il nous faut savoir que Dieu est notre Père, qu'Il est notre Abba.
 Il nous faut comprendre que nous pouvons avoir ce genre de merveilleuse
 intimité avec Lui. Mais il est aussi très important que nous ayons
 la crainte révérencieuse et le respect les plus profonds pour Lui,
 que nous n'arrivions jamais au point où nous penserions à Lui
 de façon désinvolte.

Puissions-nous en tant que Ses enfants, invoquer notre Abba,
 Père et nous rendre compte des bénéfices de cet héritage glorieux
 que nous avons en Jésus-Christ, des bénédictions qui sont à nous
 parce que nous sommes enfants de Dieu. 

\dvrule

\dvprayer{
Père, nous Te remercions de Ton plan qui fait que par Jésus,
 nous qui étions des étrangers pouvons maintenant être appelés Tes enfants.
 Merci, Abba, Père, pour les richesses qui sont à nous en Lui et par Lui.
}{}

\missing{Fin de prière manquante?}


%%%%%%%%%%%%%%%
% 11 octobre
%%%%%%%%%%%%%%%

\dvday{Œuvre ou Fruit ?}

\dvquote{
Mais le fruit de l'Esprit est~: amour\dots{}
}{\ibibleverse{Ga}(5:22)}

\dvlettrine{P}{aul parlait des œuvres de la chair,}
 puis au verset 22, il commence~: \punct{deux-points}
 \og Mais le fruit de l'Esprit. \fg{}
 Le mot \og mais \fg{} est une conjonction d'opposition.
 Elle relie deux idées opposées.
 L'opposition est entre les œuvres de la chair et le fruit de l'Esprit.

À chaque fois que vous parlez d'œuvres, vous parlez d'un effort charnel.
 Nous tombons tous dans ce piège de temps à autres. Nous promettons à Dieu
 de mieux faire la prochaine fois. Pour une raison ou pour une autre,
 en dépit de la sincérité de mon cœur, je ne peux pas tenir mes promesses.
 Si je m'efforce de plaire à Dieu avec mes œuvres, je me retrouve déjà
 dans une situation impossible parce que \punct{virgule en trop}
 \og nul ne sera justifié par les œuvres de la loi \fg{}
 (\ibibleverse{Ga}(2:16)). 

\dvbox{
Si vous avez l'amour \og agapé \fg{} de l'esprit, votre vie va porter du fruit.
}

\suggest{agapé en italique}

À chaque fois que vous parlez de fruit, vous parlez d'une relation.
 \og De même que le sarment ne peut de lui-même porter du fruit,
 s'il ne demeure sur le cep, a dit Jésus, \punct{Pas de répétition des guillemets}
 de même vous non plus, si vous ne demeurez en Moi \fg{}
 (\ibibleverse{Jn}(15:4)).
 Si vous avez cette juste relation avec Dieu par l'intermédiaire de Jésus-Christ,
 la conséquence naturelle en sera du fruit. Je ne peux pas porter
 le genre de fruit que Dieu désire, sauf si je demeure en Jésus.

Le fruit de l'Esprit est l'amour \og agapé \fg{}. Vous ne pouvez pas être
 dans une bonne relation avec Dieu sans que cet amour ne sorte de vous
 \ocadr cela arrive naturellement quand vous demeurez en Lui.

Le fruit de l'Esprit sort-il de votre vie? Demeurez en Lui et laissez Sa parole
 demeurer en vous. Accrochez-vous, et l'amour de Dieu va commencer
 à se développer et à se parfaire dans votre vie.

\dvrule

\dvprayer{
Père, aide-nous à laisser cette marque positive de Ton amour sur les autres.
}{\DlNdJ}


%%%%%%%%%%%%%%%
% 12 octobre
%%%%%%%%%%%%%%%

\dvday{Ma Gloire, La Croix}

\dvquote{
Quant à moi, certes non !
 je ne me glorifierai de rien d'autre que de la croix de notre Seigneur
 Jésus-Christ\dots{}
}{\ibibleverse{Ga}(6:14)}

\dvlettrine{S}{i vous possédez quelque chose de valeur,}
 Dieu vous l'a donnée. Quelquefois, les gens gaspillent les dons que Dieu
 leur a confiés, en les utilisant pour leur propre gloire au lieu
 de le faire pour la Sienne. Et souvent les gens se glorifient
 de leur talent, comme s'ils se l'étaient eux-mêmes donné.
 Mais vous ne pouvez pas vous glorifier de quelque chose qu'on vous a donné. 

\dvbox{
Vantez-vous et glorifiez-vous de la croix.
 Elle dit combien Jésus nous aime, vous et moi.
}

Pourquoi nous est-il dit de nous glorifier de la croix de Jésus-Christ ?
 Tout d'abord, nous nous vantons de la croix parce qu'elle nous parle
 de l'étendue de l'amour de Dieu pour nous. Elle nous rappelle jusqu'où Dieu
 était prêt à aller \ocadr et jusqu'où il est effectivement allé \fcadr{}
 afin de nous aider, nous sauver et nous bénir.
 Nous nous glorifions de la croix parce que c'est sur la croix que Jésus
 a conquis Satan et nous a libérés du péché pour que nous puissions
 profiter de la communion avec Dieu. C'est par l'intermédiaire de la croix
 que nous avons la victoire sur la chair et sur la tombe.

Ainsi nous nous glorifions de la croix de Jésus-Christ
 parce qu'elle nous a amené la délivrance, une vie abondante et l'espoir
 de la vie éternelle au ciel.

Que Dieu nous garde de nous glorifier en rien d'autre que la croix de Jésus !
 Il n'existe pas d'autre nom par lequel nous soyons sauvés;
 pas d'autre nom qui nous ait amené une si grande liberté
 et une si grande victoire.

\dvrule

\dvprayer{
Père, nous Te remercions de ce que Tu nous a tant aimés au point d'envoyer
 Ton Fils qui nous a rachetés de l'esclavage de la corruption
 afin que nous puissions Te connaître et vivre en communion éternelle
 avec Toi.
}{\DlNdJ}


%%%%%%%%%%%%%%%
% 13 octobre
%%%%%%%%%%%%%%%

\dvday{Connaître Dieu}

\dvquote{
Que le Dieu de notre Seigneur Jésus-Christ, le Père de gloire,
 vous donne un esprit de sagesse et de révélation
 qui vous le fasse connaître\dots{}
}{\ibibleverse{Ep}(1:17)}

\dvlettrine{I}{l est impossible à l'homme naturel} de comprendre
 les choses profondes de Dieu sans l'intervention de Son Esprit.
 \og Mais l'homme naturel ne reçoit pas les choses de l'Esprit de Dieu,
 car elles sont une folie pour lui, et il ne peut les connaître,
 parce que c'est spirituellement qu'on en juge \fg{}
 (\ibibleverse{ICo}(2:14)).
 Sans le Saint-Esprit, nous ne pouvons rien connaître de Dieu,
 si ce n'est qu'Il existe.

Nous savons que Dieu existe, en partie, parce que Sa création
 témoigne de Lui. \og Les cieux racontent la gloire de Dieu,
 et l'étendue céleste annonce l'œuvre de ses mains.
 Le jour en donne instruction au jour, la nuit en donne connaissance
 à la nuit \fg{} (\ibibleverse{Ps}(19:2-3)).
 Il est si clairement révélé dans la nature, que les hommes sont
 sans excuses quand il s'agit de savoir que Dieu existe.

Mais nous pouvons faire mieux que savoir seulement que Dieu existe;
 nous pouvons connaître Son caractère en considérant Jésus.
 Il est venu pour révéler Dieu à l'homme.
 \og Dieu, à la fin des temps, nous a parlé par son Fils \fg{}
 (\ibibleverse{He}(1:1-2)).
 Jésus a dit~: \punct{deux-points}
 \og Celui qui m'a vu a vu le Père \fg{} (\ibibleverse{Jn}(14:9)).

Nous savons des choses sur la miséricorde, la grâce, la compassion,
 la bonté et l'amour de Dieu parce que nous voyons ces choses en Jésus.
 Il est Dieu incarné.

Au travers de la nature, au travers de la Parole et au travers de Jésus,
 nous pouvons parvenir à une connaissance salvatrice de Dieu le Père.

\dvbox{
Alors que nous commençons à marcher avec Jésus,
 le Saint-Esprit nous enseigne les vérités profondes de Dieu.
}

Comme nous sommes bénis de connaître Dieu!

\dvrule

\dvprayer{
Père, merci de T'être revélé Toi-même à nous, afin que nous puissions
 connaître la joie de vivre en communion avec Toi et avoir l'espérance
 de la vie éternelle.
}{\DlNdJ}


%%%%%%%%%%%%%%%
% 14 octobre
%%%%%%%%%%%%%%%

\dvday{Son Ouvrage}

\dvquote{
Nous sommes son ouvrage, nous avons été créés en Christ-Jésus
 pour des œuvres bonnes que Dieu a préparées d'avance,
 afin que nous les pratiquions.
}{\ibibleverse{Ep}(2:10)}

\dvlettrine{D}{ieu a votre futur déjà tout tracé}
 \ocadr chaque seconde de chaque moment, chaque heure de chaque jour.
 Il sait exactement ce qu'Il veut que vous fassiez dans ce monde
 pour Sa gloire, et Il sait exactement ce qu'Il doit accomplir
 dans votre vie aujourd'hui pour vous préparer aux tâches de demain.
 Quel réconfort que de savoir qu'avant même que nous n'aspirions
 notre première bouffée d'air, Dieu avait déjà un plan pour nous
 et connaissait déjà chaque mesure qu'Il prendrait pour accomplir
 Son plan souverain pour nous.

\dvbox{
Dieu œuvre en vous parce que vous êtes Son ouvrage.
 Puissions-nous nous laisser faire sous Sa main.
}

Beaucoup des choses que Dieu fait, ou permet, ne semblent pas très agréables
 sur le moment. En fait, quelquefois, ces choses sont carrément
 désagréables. Parce que nous ne sommes pas capables d'avoir une vue
 d'ensemble \ocadr une vue des choses terminées \fcadr{} nous ne comprenons
 pas comment ces leçons désagréables contribuent à la solution du puzzle.
 Quelquefois les décisions de Dieu nous troublent ou nous inquiètent.
 Rien ne semble bien; rien ne semble approprié à la situation.
 Mais en fin de comptes, nous verrons que chacun des éléments
 était nécessaire pour créer la vie que Dieu voulait bâtir en nous.

Ces périodes de préparation ne sont pas faciles,
 mais elles sont nécessaires. Dieu œuvre en vous parce que vous êtes
 Son ouvrage. Puissions-nous nous laisser faire sous Sa main
 afin qu'Il puisse nous modeler et nous façonner selon Ses desseins.
 Puissions-nous utiliser chaque bouffée d'air pour Lui apporter
 la gloire et l'honneur.

\dvrule

\dvprayer{
Père, merci à Toi pour le travail que Tu as fait et que Tu fais
 dans nos vies aujourd'hui. Nous voulons être des vases qui Te glorifient.
 Que Ta volonté parfaite soit accomplie avec nous, Seigneur.
 Forme-nous pour Ton usage.
}{\DlNdJ}


%%%%%%%%%%%%%%%
% 15 octobre
%%%%%%%%%%%%%%%

\dvday{Il est Capable}

\dvquote{
Maintenant à celui qui est capable de faire excessivement plus que tout
 ce que nous demandons ou pensons, selon la puissance qui agit en nous\dots{}
}{\ibibleverse{Ep}(3:20) \KJF}

\dvlettrine{N}{ous avons tendance à mesurer les obstacles}
 qui se trouvent devant nous en fonction de notre propre capacité
 à les escalader. Et ce serait raisonnable si nous étions ceux qui avons
 à gravir la montagne. Mais c'est la mauvaise façon de mesurer
 si Dieu est Celui qui va s'occuper de cette montagne pour nous.
 La difficulté d'une tâche doit être mesurée en fonction de la capacité
 de l'agent qui est chargé d'accomplir la tâche.
 Si c'est Dieu qui s'affronte à la montagne, alors le mot
 de \suggest{le mot \og difficulté \fg{}} \og difficulté \fg{}
 n'a pas sa place dans l'équation.
 Comme Il l'a dit~: \punct{deux-points}
 \og Voici, je suis le \Seigneur, le Dieu de toute chair ;
 y a-t-il quelque chose trop difficile pour Moi? \fg{}
 (\ibibleverse{Jr}(32:27)). 

\dvbox{
Nous devons apprendre à regarder les situations non pas en fonction
 de notre force mais en fonction de la force de Dieu.
}

Quand le médecin nous dis~: \punct{deux-points}
 \og Nous sommes désolés, il n'y a plus d'espoir.
 Nous avons fait tout ce que nous pouvions faire \fg{},
 \punct{virgule après guillemet}
 nous sombrons dans le désespoir. Pourquoi?
 Parce que les hommes ont fait tout ce qui était en leur pouvoir.
 Nous arrivons à la fin de cette phrase et nous en concluons
 qu'il n'y a plus d'espoir, sans jamais prendre Dieu en compte.
 Mais dès \typo{dès} le moment où vous faites entrer Dieu
 dans l'équation du problème, alors le désespoir s'enfuit.
 Quand vous vous rappelez de Dieu, l'espoir revient. \suggest{l'espoir renaît}

Nous devons nous souvenir que Dieu est capable de nous délivrer
 de la fournaise brûlante et de la fosse aux lions.
 il est capable de vous délivrer du péché et de la mort
 \ocadr et de tout problème présent qui vous cause de l'angoisse.

Êtes-vous confrontés à une montagne aujourd'hui? Demandez de l'aide à Dieu.
 Demandez à Celui qui est capable, de se montrer puissant en votre place
 \ocadr et puis, croyez vraiment qu'Il va le faire.

\dvrule

\dvprayer{
Père, merci à Toi de nous aimer. Aide-nous à détacher nos yeux
 des circonstances et à les fixer sur Tes grandes capacités.
}{\Amen} 


%%%%%%%%%%%%%%%
% 16 octobre
%%%%%%%%%%%%%%%

\dvday{Édifiés dans l\ap{}Amour}

\dvquote{
Mais, en disant la vérité avec amour, nous croîtrons à tous égards
 en Celui qui est le chef, Christ De Lui, le corps tout entier
 bien ordonné et cohérent, grâce à toutes les jointures
 qui le soutiennent fortement, tire son accroissement dans la mesure
 qui convient à chaque partie, et s'édifie lui-même dans l'amour.
}{\ibibleverse{Ep}(4:15-16)}

\dvlettrine{A}{lors qu'il sortait de l'église un jour,}
 un petit garçon se touna vers son Papa \suggest{papa} en lui disant~: \punct{deux-points}
 \og On dirait que Dieu est encore en colère après nous. \fg{}
 C'est un triste commentaire sur une triste vérité.
 Bien que l'église ait été appelée à annoncer la vérité
 d'une façon empreinte d'amour, afin que le corps de Christ grandisse
 et mûrisse en Son image, la vérité n'est pas toujours annoncée avec amour.
 Quelquefois \suggest{virgule} on fait part aux autres du message
 qu'Il nous a confié dans une attitude de colère ou de frustration. 

\dvbox{
L'amour doit couvrir tout ce que nous faisons. 
}

Si nous servons les autres dans l'amour, si nous les édifions,
 les encourageons, les aidons à trouver leur place de service
 pour le Seigneur, les gens vont être aimés vers la maturité
 \ocadr et le corps de Christ grandira puissamment.

Chacun de nous a un appel spécial dans le corps de Christ.
 Si nous ne prenons pas cette place, nous laissons un vide.
 Le corps est affaibli, et l'Église \suggest{Église}
 ne peut pas être tout ce qu'elle pourrait être.
 Puisse Dieu vous aider à trouver votre service spécial,
 et puisse-t-Il vous utiliser pour nourrir les autres
 \ocadr les fortifier, les encourager et les aimer\dots{}
 pour Lui et pour Sa gloire.

\dvrule

\dvprayer{
Père, nous voulons devenir tout ce que Tu voudrait que nous soyons.
 Puissions-nous montrer Ton amour et être édifiés dans cet amour
 alors même que nous nous encourageons et nous fortifions
 les uns les autres par l'Esprit.
}{\Amen}


%%%%%%%%%%%%%%%
% 17 octobre
%%%%%%%%%%%%%%%

\dvday{La Conduite du Croyant}

\dvquote{
Veillez donc avec soin sur votre conduite, non comme des fous,
 mais comme des sages\dots{}
}{\ibibleverse{Ep}(4:15-16)}

\dvlettrine{L}{e fou tue le temps.}
 Il gâche les moments précieux que Dieu lui a donnés,
 vivant sa vie en faisant peu ou pas cas de ce qu'il fait
 avec le temps qui lui a été confié.
 Mais le sage se conduit avec circonspection. Il rachète le temps.

Le sage veut connaître la volonté de Dieu pour Sa vie,
 car c'est la connaissance la plus importante à laquelle vous puissiez accéder.
 Le fou progresse dans la vie à l'aveuglette en trébuchant,
 sans jamais savoir pourquoi il existe et sans jamais s'en soucier vraiment.
 Il vit seulement pour le moment présent sans aucun égard pour l'éternité.

\dvbox{
Se conduire avec circonspection, c'est connaître la volonté du Seigneur.
}

Se conduire avec circonspection c'est être rempli de l'Esprit.
 Le fou obscurcit ses pensées avec le vin, il s'abrutit et s'insensibilise
 vis-à-vis des choses qui importent. Le sage au contraire est rempli
 de l'Esprit, ce qui aiguise ses pensées sur les choses de l'Esprit.
 Son désir est de se conduire selon la volonté du Seigneur, \punct{virgule}
 de voir sa vie suivre avec précision le plan et le dessein
 pour lesquels Dieu l'a créé.

La volonté de Dieu pour nous est de rendre grâce en toutes circonstances
 et pour toutes choses (\ibibleverse{ITh}(5:18)).
 Mais je ne peux seulement rendre grâce à Dieu pour toutes choses
 que si je comprends et que je me rends compte que toutes choses
 concourent au bien.

Connaître Sa volonté et être rempli de Son Esprit nous aide \suggest{nous aident?}
 à nous conduire de la façon dont Il nous appelle à nous conduire
 \ocadr avec circonspection, sagement, en rendant grâce en toutes circonstances
 et pour toutes choses, tirant le plus grand avantage possible
 de chaque opportunité \fcadr{} toujours pour Sa gloire.

\dvrule

\dvprayer{
Père, aide-nous à considérer ce que Tu as fait et qui Tu es,
 et à Te remercier continuellement pour toutes choses au travers de Ton Fils,
 Jésus-Christ.
{\Amen}


%%%%%%%%%%%%%%%
% 18 octobre
%%%%%%%%%%%%%%%

\dvday{La Puissance de la Prière}

\dvquote{
Priez en tout temps par l'Esprit, avec toutes sortes de prières
 et de supplications. Veillez-y avec une entière persévérance.
 Priez pour tous les saints\dots{}
}{\ibibleverse{Ep}(6:18)}

\dvlettrine{N}{ous connaissons l'armure spirituelle} que Paul décrit
 au sixième chapitre de la Lettre aux Éphésiens avec, notamment,
 la cuirasse de la justice, le bouclier de la foi, le casque du salut,
 l'épée de l'Esprit, la prière. Sans cette armure,
 vous ne seriez pas capables de vous tenir debout.
 Le combat spirituel requiert des armes spirituelles.

Je vous suggère que, de toutes les armes à votre disposition,
 la prière est la plus vitale. Cest sans doute le cas car Satan
 fait tout ce qu'il peut pour nous empêcher de prier.
 Il ne veut pas que vous invitiez Dieu dans la bataille.
 Il aimerait que ça reste juste entre vous deux.

\dvbox{
La prière est l'arme décisive.
}

Dans le jardin, Jésus a prié pendant que Pierre a dormi.
 Quand Jésus l'a réveillé, Il lui a dit~: \punct{deux-points}
 \og Pierre, tu dors? \dots{} Veille et prie, afin de ne pas entrer
 en tentation \fg{} (\ibibleverse{Mc}(14:37-38)).
 Quand Jésus a fait face au conflit réel, Il a été victorieux
 parce qu'Il avait gagné la victoire dans la prière.
 Pierre a aussi fait face à un conflit cette nuit-là, mais il a été battu.
 Peut-être que s'il avait prié au lieu de dormir,
 il aurait été aussi victorieux. Je me demande combien de fois
 nous sommes vaincus parce que nous avons omis de prier.

Par la prière, nous prenons de la puissance à l'ennemi.
 Quand nous prions, Satan doit céder devant le nom de Jésus-Christ.

Puisse Dieu nous aider à utiliser les armes qu'Il nous a données
 \ocadr toutes les armes. Et puisse-t-Il nous inciter à prier,
 à nous attendre à Lui en persévérant jusqu'à ce que nous voyions
 la volonté de Dieu accomplie sur cette terre.

\dvrule

\dvprayer{
Père, nous Te remercions pour l'arme puissante de la prière par laquelle
 nous pouvons faire tomber les forteresses détenues par Satan dans la vies
 de ceux qui nous entourent. Apprends-nous à prier, Seigneur.
}{\DlNdJ}


%%%%%%%%%%%%%%%
% 19 octobre
%%%%%%%%%%%%%%%

\dvday{Pas de Peur Devant la Mort}

\dvquote{
Car pour moi, Christ est ma vie et la mort m'est un gain.
}{\ibibleverse{Ph}(1:21)}

\dvlettrine{L}{a Bible nous dit}
 qu'il est réservé aux hommes de mourir une seule fois.
 Cela signifie que vous avez un rendez-vous avec Dieu
 \ocadr un rendez-vous inévitable. Que ressentez-vous devant cela?
 Avez-vous peur de ce rendez-vous? L'homme a une peur naturelle de la mort,
 mais c'est parce que nous ne sommes pas sûrs de ce à quoi
 il faut nous attendre. Nous avons peur de l'inconnu.

\dvbox{
Grâce à Jésus, l'enfant de Dieu n'a jamais à craindre la mort.
}

La vérité, c'est que si les incroyants connaissaient la réalité
 de ce qui les attend après la mort, ils seraient plus terrifiés
 qu'ils ne le sont maintenant. Mais si les croyants connaissaient
 la réalité de ce qui les attend après la mort, ils perdraient toute peur.
 Paul, à qui il avait été donné d'entrevoir un petit aperçu du ciel
 (\ibibleverse{IICo}(12:2-7)), disait que le choix entre la vie et la mort
 était difficile, parce qu'il savait que tant qu'il demeurait dans son corps,
 il pouvait glorifier Dieu, mais qu'il préférait être absent de son corps
 afin d'être présent avec le Seigneur (\ibibleverse{IICo}(5:8)).

Grâce à Jésus, l'enfant de Dieu n'a jamais à craindre la mort.
 Il est venu pour nous sauver de nos péchés et pour enlever l'aiguillon
 de la mort. L'aiguillon étant enlevé, nous n'avons plus à craindre la mort.
 Et ainsi nous avons une situation gagnant-gagnant.

\og Car pour moi, Christ est ma vie et la mort m'est un gain. \fg{}
 Si je vis, je vis ma vie pour le Seigneur et dans la communion
 avec le Seigneur. Je vis pour faire Sa volonté.
 J'ai l'occasion d'amasser davantage de trésors dans les cieux.
 Et quand je mourrai, j'irai vivre en Sa présence,
 dans la plénitude de la joie éternelle.

\dvrule

\dvprayer{
Père, comme nous sommes reconnaissants que le futur n'est pas inconnu
 ou incertain pour nous, mais qu'il est assuré en Christ Jésus.
}{\DlNdJ}

\typo{futur}


%%%%%%%%%%%%%%%
% 20 octobre
%%%%%%%%%%%%%%%

\dvday{Ne Vous Inquiétez de Rien}

\dvquote{
Ne vous inquiétez de rien ; mais, en toutes choses, par la prière
 et la supplication, avec des actions de grâces, faites connaître à Dieu
 vos demandes. Et la paix de Dieu, qui surpasse toute intelligence,
 gardera vos cœurs et vos pensées en Christ-Jésus.
}{\ibibleverse{Ph}(4:6-7)}

\dvlettrine{L}{a Bible ne donne pas simplement un commandement}
 en vous abandonnant à vous-mêmes pour voir ce que vous allez comprendre
 et ce que vous allez en faire. Elle vous indique toujours les étapes
 à suivre pour y obéir. Quand Paul nous a dit de ne pas nous faire de soucis
 \ocadr de nous \og inquiéter de rien \fg{} \fcadr{}
 il nous a aussi donné l'antidote aux soucis.
 \og Mais, en toutes choses, par la prière et la supplication,
 avec des actions de grâces, faites connaître à Dieu vos demandes. \fg{}
 En d'autres termes, prenez les choses qui vous inquiètent
 et laissez les devenir les sujets de votre vie de prière.
 Plutôt que de vous inquiéter \typo{que de vous inquiéter}
 de ces choses, priez à leur sujet.

\dvbox{
Transformez vos soucis en prières.
}

Paul fait une distinction entre prière et supplication.
 La prière est communion avec Dieu. C'est parler avec Lui, L'adorer
 et simplement L'aimer pour qui Il est. La supplication, \punct{virgule}
 c'est présenter une requête.
 C'est ce qui prend place quand la conversation entre vous et Dieu se réduit
 à votre besoin présent. Paul nous instruit sur comment \suggest{la manière de}
 faire connaître nos demandes à Dieu, mais nous devons prendre les choses
 dans le bon ordre, en commençant d'abord par l'adoration aimante et la prière.
 La prière devrait toujours précéder nos requêtes.

Quand nous traitons de cette façon nos sujets de préoccupation
 \ocadr en venant à Dieu par la prière, la supplication
 et les actions de grâce \fcadr{}
 nous obtenons, au final, la paix qui dépasse toute intelligence.
 Bien que les problèmes extérieurs n'aient pas encore été réglés
 (et aillent même peut-être en empirant), vous découvrez que, malgré tout,
 vous ressentez une paix incroyable.

\dvrule

\dvprayer{
Père, merci à Toi pour qui Tu es et pour tout ce que Tu fais pour nous.
 Bénis-nous de Ta paix alors que nous Te recherchons.
}{\DlNdJ}


