\dvmonth{Avril}

%%%%%%%%%%%%
% 1er avril
%%%%%%%%%%%%

\dvday{Le secret du succès}

\dvquote{
Le Roi Ozias s'appliqua à rechercher Dieu du vivant de Zacharie,
 qui avait l'intelligence des visions de Dieu;
 et tant qu'il rechercha l'Éternel, Dieu lui donna du succès.
}{\ibibleverse{IICh}(26:5)}

\lettrine{L}{e secret} d'une vie réussie est dans la recherche du Seigneur.
 Jésus a dit \Og Cherchez premièrement le royaume de Dieu et sa justice,
 et tout cela vous sera donné par-dessus \Fg{} (\ibibleverse{Mt}(6:33)).
 Autrement dit, prenez soin de votre relation avec Dieu et Il prendra soin
 de tout le reste. 

Le Roi Ozias se focalisait sur sa relation avec Dieu,
 et \Og il recherchait Dieu du vivant de Zacharie qui avait l'intelligence
 des visions de Dieu \Fg.
 Ceci tendrait à indiquer que Zacharie lui faisait office de modèle spirituel.
 Et en fait, tant que Zacharie a vécu, Ozias a recherché le Seigneur.
 Et tant qu'il a recherché le Seigneur, Dieu l'a fait prospérer.

\dvbox{
Dès l'instant où je cesse de rechercher Dieu,
 je suis aussi faible que les autres hommes.
}

Néanmoins, la prospérité présente souvent des dangers.
 Quelquefois, vous vous mettez à croire que votre succès est dû à votre propre
 talent et à votre génie, et vous laissez Dieu en dehors de votre vie.
 Quand Ozias a considéré la puissance et la richesse de son royaume,
 son c\oe{}ur s'est enorgueilli.
 Nous lisons au verset \ibiblevs{IICh}(26:16)~:
 \Og Mais sa puissance le rendit orgueilleux,
 ce qui causa sa perte \Fg{} \BFC.

Je dois toujours me rappeler que le secret de ma force réside
 dans ma relation avec Dieu. 

\dvrule

\dvprayer{
Seigneur, aide-nous à ne pas Te combattre.
 Aide-nous à être sensible et ouvert à Ton Esprit. 
}{\DlNdJ}


%%%%%%%%%%%%
% 2 avril
%%%%%%%%%%%%

\dvday{Dernier appel}

\dvquote{
Les coureurs allèrent avec les lettres du roi et de ses ministres
 dans tout Israël et Juda.
 D'après l'ordre du roi ils dirent~:
 \Og Fils d'Israël, revenez à l'Éternel,
 le Dieu d'Abraham, d'Isaac et d'Israël, afin qu'Il revienne à vous,
 reste échappé de la main des rois d'Assyrie. \Fg{}
}{\ibibleverse{IICh}(30:6)}


\lettrine{Q}{uand} Ézéchias devint roi, il chercha à ramener le peuple
 à un renouveau spirituel, à une reconnaissance que Jéhovah était le Dieu
 qui régnait sur la nation.
 Il envoya donc une annonce dans tout le pays pour inviter le peuple
 à revenir au Seigneur et à célébrer la fête de la Pâque.

Les Assyriens avaient conquis le royaume du nord et emmené des captifs,
 mais beaucoup de gens avaient réussi à ne pas se faire capturer.
 L'invitation d'Ézéchias alla vers ceux qui restaient.
 Ils avaient abandonné Dieu, mais Celui-ci les rappelait.
 Il se trouve que ce fut en fait le dernier appel de Dieu en direction d'Israël.
 Le royaume du nord s'est moqué de l'invitation et en l'espace de trois ans,
 ils avaient disparus. 

\dvbox{
Il est effrayant de se rendre compte que si un homme persiste
 dans une voie contre laquelle Dieu l'a mis en garde,
 le jour viendra où Dieu lui adressera un appel final. 
}

Juda a toutefois répondu à l'invitation. Sous le règne d'Ézéchias,
 Juda est redevenu puissant et prospère parce que les gens s'étaient
 tournés vers le Seigneur d'un seul c\oe{}ur.

Notre Dieu est plein de miséricorde et de grâce.
 Si vous vous tournez vers Lui, Il vous libérera de la captivité
 de l'ennemi et de la puissance des ténèbres qui vous retient
 dans son étreinte. 

\dvrule

\dvprayer{
Père, réveille-nous pour que nous comprenions le sérieux de l'époque
 et de l'âge dans lesquels nous vivons.
 Puissions-nous conclure d'un seul c\oe{}ur une alliance par laquelle nous
 nous engageons à Te servir et à T'adorer,
 et à nous soumettre à la seigneurie de Jésus-Christ. 
}{\Amen}

\typo{seigneurie}



%%%%%%%%%%%%
% 3 avril
%%%%%%%%%%%%

\dvday{Attention recentrée}

\dvquote{
Fortifiez-vous et prenez courage! Soyez sans crainte et sans effroi
 devant le roi d'Assyrie et devant toute la multitude qui est avec lui;
 car avec vous il y a plus qu'avec lui~: avec lui il y a un bras de chair,
 et avec nous l'Éternel, notre Dieu, qui nous aidera
 et qui soutiendra nos combats.
 Le peuple s'appuya sur les paroles d'Ézéchias, roi de Juda.
}{\ibibleverse{IICh}(32:7-8)}


\lettrine{L}{es Assyriens} se mirent en route pour venir attaquer Jérusalem.
 Ézéchias fit preuve de sagesse en faisant des préparatifs.
 Il ordonna au peuple de renforcer les fortifications de la cité,
 de boucher les sources et les puits qui se trouvaient en dehors des murs
 et de stocker des lances et des boucliers.
 Mais à cause de la réputation de brutalité des Assyriens,
 les gens étaient découragés et terrifiés.
 Ézéchias les réunit donc pour les encourager.

Au lieu de simplement leur dire de ne pas avoir peur,
 il leur donna une raison de ne pas avoir peur.
 Il écarta leur attention de l'ennemi et la recentra sur le Seigneur.
 Il leur rappela que Dieu était de leur côté,
 prêt à se charger de leurs batailles.

\dvbox{
Dieu va mener nos combats.
}

Quand des situations impossibles surviennent,
 nous devons nous rappeler que Dieu peut faire infiniment au-delà
 de tout ce que nous demandons ou pensons.
 Ces batailles ne sont pas contre nous \ocadr{}elles sont contre Dieu.
 Et personne ne peut infliger de défaite à notre Dieu.

Si un problème vous cause du souci voire du stress,
 c'est que vous n'êtes pas encore arrivés à faire réellement
 confiance à Dieu.
 Dès l'instant où vous remettez toute la situation entre les mains de Dieu,
 les soucis s'évanouiront; la peur et l'anxiété disparaîtront.
 Reposez-vous en Dieu. Tout repose entre Ses mains. 

\dvrule

\dvprayer{
Père, nous sommes si reconnaissants d'entrer dans la bataille en vainqueurs.
 Aide-nous, Seigneur, à fixer les yeux sur Toi afin de pouvoir
 être forts et courageux. 
}{\DlNdJ}


%%%%%%%%%%%%
% 4 avril
%%%%%%%%%%%%

\dvday{Pas de remède}

\dvquote{
L'Éternel, le Dieu de leurs pères, leur avait envoyé de bonne heure
 des avertissements par l'intermédiaire de Ses messagers,
 car Il voulait épargner Son peuple et Sa propre demeure.
 Mais ils se moquaient des messagers de Dieu,
 ils méprisaient Ses paroles et se raillaient de Ses prophètes,
 jusqu'à ce que la fureur de l'Éternel contre Son peuple
 monte et soit sans remède.
}{\ibibleverse{IICh}(36:15-16)}


\lettrine{C}{'est} une chose terrible que d'entendre Dieu déclarer
 qu'il n'y a plus de traitement, plus de remède.
 Tel était le cas tragique de Juda.
 Ils avaient adoré d'autres dieux, avaient refusé d'écouter
 la voix du Seigneur, et s'étaient détournés de Dieu
 \ocadr{}jusqu'à ce que finalement, Il se détourne d'eux
 et les laisse se faire battre par leurs ennemis. 

\dvbox{
Si vous refusez le remède de Dieu, il n'y a pas d'autre traitement.
}

Dans Sa patience et Sa compassion, Dieu avait envoyé beaucoup de prophètes
 pour mettre Juda en garde.
 Jérémie était l'un de ses prophètes.
 Quand il annon\c{c}a le message que Dieu avait résolu de livrer Juda
 aux mains des Babyloniens,
 et qu'il vaudrait mieux pour eux qu'ils se soumettent aux Babyloniens,
 le Roi Sédécias l'emprisonna pour haute trahison.
 Le remède de Dieu avait été refusé, suscitant ainsi la colère de Dieu.

Chaque fois qu'il y a un problème, Dieu a toujours un remède.
 Pour le problème du péché, Dieu a prescrit un remède, un traitement.
 Il se trouve dans le sacrifice de Jésus-Christ, l'Agneau de Dieu.
 La Bible nous dit que le sang de Christ nous lave de tous nos péchés.
 Mais soyez prévenus~: si vous refusez le remède de Dieu,
 il n'y a pas d'autre traitement. 

\dvrule

\dvprayer{
Père, nous sommes reconnaissants de ce que Tu as été si patient avec nous,
 nous donnant maintes et maintes opportunités de nous détourner du monde
 pour vivre en Te suivant.
 Seigneur, aide-nous à vivre devant Toi d'une fa\c{c}on qui Te plaise. 
}{\DlNdJ}



%%%%%%%%%%%%
% 5 avril
%%%%%%%%%%%%

\dvday{Entretenus par le roi}

\dvquote{
Or, comme nous mangeons le sel du palais
 \NdT{expression équivalente à
 \Og nous sommes entretenus par le palais \Fg{}.}
 et qu'il ne nous paraît pas convenable de voir mépriser le roi,
 nous envoyons au roi ces informations.
}{\ibibleverse{Esd}(4:14)}


\lettrine{L}{es voisins} d'Israël ont envoyé cette lettre à Artaxerxès
 pendant la reconstruction du temple.
 Quand leurs offres d'aide ont été rejetées,
 ils ont cherché à entraver le travail.
 Ils ont embauché des hommes de loi pour faire échouer le projet
 et ils ont adressé un document au roi de Perse.
 Ils disaient~: \Og Nous croyons que ces gens vont vous déshonorer.
 Nous ne pouvions pas simplement voir ceci arriver sans intervenir;
 nous avons donc estimé nécessaire d'informer le roi. \Fg{}

Ce verset s'applique aujourd'hui à nous. Car nous avons également un Roi,
 et nous sommes entretenus par Lui.

Certains disent que Jésus est simplement une béquille pour les gens faibles.
 Ils ont raison. Je m'appuie sur Lui tout le temps.
 S'Il n'était pas pour moi une béquille,
 cela fait longtemps que je me serais effondré.
 Il ne m'a jamais abandonné ni laissé tomber.
 Vous n'avez pas à vous faire de soucis lorsque vous êtes entretenus par le Roi.

Mais le monde qui m'entoure déshonore mon Roi.
 Chaque fois que quelqu'un utilise le nom de mon Dieu en vain,
 mon c\oe{}ur frémit d'indignation. 

\dvbox{
Je n'aime pas voir mon roi déshonoré. 
}

Tout comme les voisins d'Israël ont envoyé une lettre informant
 le roi de ce qui se passait, nous devons, nous aussi, informer le nôtre.
 Nous devons nous mettre à genoux devant Dieu pour Lui parler de la condition
 de notre nation et des choses qui se passent \ocadr{}et prier pour Lui demander
 qu'Il commence à agir. 

\dvrule

\dvprayer{
Père, aide-nous aujourd'hui à amener gloire et honneur à Toi, notre Roi.
 Donne-nous l'audace de prendre la défense de ce qui est juste, saint et pur. 
}{\DlNdJ}


%%%%%%%%%%%%
% 6 avril
%%%%%%%%%%%%

\dvday{À court de mots}

\dvquote{
Maintenant, que dirons-nous après cela, ô notre Dieu ?
 Car nous avons abandonné Tes commandements.
}{\ibibleverse{Esd}(9:10)}


\lettrine{C}{inquante-sept} ans seulement après que Dieu
 \ocadr{}par l'intermédiaire de Cyrus\fcadr{} les ait libérés de leur captivité
 pour qu'ils puissent retourner à Jérusalem,
 reconstruire le temple et se remettre à L'adorer,
 les gens d'Israël étaient déjà retournés aux choses précises
 pour lesquelles Dieu les avait jugés plus tôt.
 Quand Esdras apprit ces nouvelles déconcertantes, son c\oe{}ur se serra.
 Le peuple ne s'était pas séparé des peuples qui l'entouraient,
 mais s'était mis, au contraire, à les suivre dans leurs abominations.

Esdras ne pouvait pas croire que le peuple puisse oublier Dieu si rapidement.
 Choqué, et à court de mots, il pria~: \Og Seigneur, je n'ai aucune excuse.
 Que pourrais-je dire? \Fg{}

Bien qu'Esdras ne puisse plus trouver de mots pour parler au Seigneur,
 il avait un bon nombre de mots à dire au peuple \ocadr{}des mots de mise en garde
 et de correction, par lesquels il appelait à une réforme très stricte
 et radicale.

Il appelait à une séparation totale du peuple de Dieu de ces choses du monde
 qui risquaient de polluer, diluer ou souiller leur témoignage.
 En tant que chrétiens, nous devons faire de même. 

\dvbox{
Nous devons prendre quelques décisions radicales concernant
 notre séparation du monde et de sa pollution. 
}


L'ennemi a ouvertement déclaré la guerre contre Dieu.
 Allons-nous rester à ne rien faire et laisser blasphémer notre Dieu?
 Ou allons-nous nous séparer du monde, nous lever et prendre parti pour Jésus? 

\dvrule

\dvprayer{
Par notre silence, Seigneur, nous avons encouragé l'ennemi.
 Aide-nous à tenir ferme par Ta force dans cette guerre totale
 que l'ennemi a déclaré contre Toi.
 Seigneur, sois notre force, notre forteresse, sois notre aide. 
}{\DlNdJ}


%%%%%%%%%%%%
% 7 avril
%%%%%%%%%%%%

\dvday{Prière pressante}

\dvquote{
Le roi me dit~: \Og Au fait, que demandes-tu donc ? \Fg{}
Je priai le Dieu des cieux.
}{\ibibleverse{Ne}(2:4)}

\lettrine{D}{epuis} plusieurs mois, Néhémie avait prié
 pour une opportunité d'être utile à la reconstruction de Jérusalem.

Quand le roi remarqua que Néhémie n'était pas dans sa bonne humeur habituelle,
 il lui demanda~: \Og Pourquoi es-tu aussi triste? \Fg{}
 Néhémie décrivit la condition de Jérusalem; les murs sont en ruine,
 les portes ont été brûlées et les gens sont démoralisés.
 \Og Comment pourrais-je être heureux après avoir re\c{c}u de telles nouvelles? \Fg{}

Le roi lui demanda donc~: \Og Qu'attends-tu de moi? \Fg{}
 C'était l'occasion que Néhémie avait demandée.
 Conscient de l'importance capitale du moment,
 il offrit une courte prière au Dieu du ciel.
 Le roi ne remarqua sans doute pas la légère pause, mais Dieu la nota.

\dvbox{
Ce n'est pas la durée de nos prières qui les rend efficaces
 \ocadr{}c'est la relation que nous avons avec Dieu qui compte.
}

\Og La prière agissante du juste a une grande efficacité \Fg{}
 (\ibibleverse{Jc}(5:16)).
 La prière de Néhémie a dû être courte et muette, mais elle a été efficace.
 Dieu a entendu et a répondu.
 Quel réconfort de se rendre compte qu'Il entend nos prières faites en silence.

Si vous n'avez pas de relation avec Dieu, il n'y a qu'une seule prière
 qui l'intéresse, à savoir~:
 \Og Ô Dieu, aie pitié de moi, qui suis un pécheur. \Fg{}
 Mais une fois que vous avez prié cette prière, alors en tant qu'enfant de Dieu,
 vous pouvez offrir à Dieu ces prières rapides,
 chaque fois que vous êtes dans le besoin.
 Si cette juste relation existe entre vous et Lui,
 ces petites prières peuvent marcher de fa\c{c}on formidable. 

\dvrule

\dvprayer{
Père, nous Te remercions de ce que Tu réponds aux prières
 \ocadr{}même à ces prières rapides qui s'élèvent spontanément
 par cause de faiblesse ou d'agitation. 
}{\DlNdJ}


%%%%%%%%%%%%
% 8 avril
%%%%%%%%%%%%

\dvday{Quand l'opposition apparaît}

\dvquote{
Lorsque Sanballat apprit que nous rebâtissions la muraille,
 il fut en colère et très mécontent. Il se moqua des Juifs.
}{\ibibleverse{Ne}(3:33-34)}

\lettrine{À}{ peine} Néhémie était-il retourné à Jérusalem
 et avait-il commencé à rebâtir la muraille que de l'opposition
 s'éleva contre lui.
 Tentant de stopper le travail, Sanballat et Tobiya ridiculisèrent Néhémie.
 Mais la réponse de Néhémie à leurs moqueries fut la prière.

Voyant que la moquerie ne marchait pas, leur prochaine man\oe{}uvre
 fut de préparer des attaques secrètes contre ceux qui construisaient
 la muraille.
 Néhémie donna donc aux ouvriers l'ordre de se protéger d'eux en instaurant
 une surveillance continue.

La première mesure est toujours la prière \ocadr{}et puis nous passons à l'action.
 La prière ne doit jamais remplacer la mise en place de mesures pratiques.
 Vous pourriez dire~:
 \Og Bien, j'ai prié et j'ai simplement la foi de croire que le Seigneur
 va régler la situation. \Fg{}
 Mais la Bible nous dit que la foi sans les \oe{}uvres est morte.

\dvbox{
Pour que la foi soit valide, il faut qu'elle conduise à l'action.
}

Paul a prévenu Timothée que ceux qui vivent pieusement en Christ Jésus
 subiront des persécutions.
 Chaque fois que vous prenez position pour Dieu,
 ou que vous essayez de faire le travail de Dieu,
 de l'opposition va s'élever et essayer de vous mettre en échec.
 Quand cela arrive, nous devons prier et demander à Dieu de nous fixer
 un plan d'action, puis nous devons prendre des mesures pour rebâtir
 nos murs et installer des sentinelles pour les surveiller.

C'est maintenant le moment de se réveiller et de se préparer. 


\dvrule

\dvprayer{
Père, nous nous rendons compte que les murailles ont été démolies.
 Aide-nous à avoir le courage de tenir bon contre le ridicule et les moqueries.
 Et, Seigneur, puissions-nous résoudre en nos c\oe{}urs de rebâtir
 ces murailles de défense. 
}{\DlNdJ}


%%%%%%%%%%%%
% 9 avril
%%%%%%%%%%%%

\dvday{La nature de Dieu}

\dvquote{
Mais Toi, Tu es un Dieu qui pardonne, qui est compatissant et qui fait grâce,
 lent à la colère et riche en bienveillance,
 et tu ne les as pas abandonnés.
}{\ibibleverse{Ne}(9:17)}

\lettrine{D}{ans} ce court verset, Néhémie décrit la nature de Dieu
 en faisant la liste de quelques-uns de ses attributs.
 Le premier qu'il mentionne est la volonté de Dieu de pardonner.
 Bien que Son peuple Lui ait été complètement infidèle,
 Dieu est resté complètement fidèle à Son peuple.
 Nous sommes tous privés de Sa gloire,
 mais le sang de Jésus est capable de nous purifier de tout péché.

\dvbox{
Tout comme Dieu était prêt a pardonné aux israélites,
 Il est prêt à vous pardonner \ocadr{}quoi que vous ayez fait.
}

Dieu est plein de grâce.
 Il accorde Ses bénédictions à ceux qui ne les méritent pas.

Dieu est plein de miséricorde.
 Comme David (qui a vraiment eu besoin de la miséricorde de Dieu)
 le faisait remarquer, Dieu ne nous rétribue pas selon nos péchés.
 Non, au contraire, Il offre Sa miséricorde à ceux qui le craignent
 \ocadr{}une miséricorde qui est aussi haute que les cieux sont hauts
 au-dessus de la Terre.

Dieu est lent à la colère. Quelquefois, \c{c}a me dérange un peu.
 Je ressemble beaucoup à Jacques et Jean qui voulaient appeler
 le feu du ciel à tomber et à consumer les samaritains qui insultaient Jésus.
 Mais Dieu est lent à la colère. Il est endurant et patient.

Dieu est bienveillant. Il a été bienveillant envers Israël.
 Il a été bienveillant envers notre pays,
 et il a certainement été bienveillant envers moi.

Dieu voit nos échecs, nos fautes, nos faiblesses.
 Et pourtant, Il se tient toujours prêt à nous pardonner.
 Quand nous L'invoquons, Il nous pardonne,
 Il nous lave et nous renouvelle. C'est simplement Sa nature.

\dvrule

\dvprayer{
Père, nous Te remercions pour Ta miséricorde.
 Puissions-nous abandonner nos mauvaises voies
 et recevoir Ton pardon et Ta purification.
}{\Amen}


%%%%%%%%%%%%
% 10 avril
%%%%%%%%%%%%

\dvday{Le Haman dans nos vies}

\dvquote{
Haman vit que Mardochée ne s'inclinait ni ne se prosternait en son honneur,
 et Haman fut rempli de fureur.
 Il considéra avec dédain l'idée de porter la main sur le seul Mardochée;
 on lui avait signalé, en effet, le peuple auquel appartenait Mardochée.
 Haman entreprit d'exterminer de tout le royaume d'Assuérus tous les Juifs,
 le peuple de Mardochée.
}{\ibibleverse{Est}(3:5-6)}


\lettrine{H}{aman} était un Agaguite, un descendant des Amalécites.
 Dieu avait ordonné à Saül de totalement exterminer les Amalécites,
 mais il avait désobéi et laissé le roi vivre.
 Le résultat fut que les Amalécites continuèrent de faire la guerre à Israël.

En typologie\NdT{\emph{typologie biblique}~:
 rapprochement entre une personne ou un évènement de l'Ancien Testament,
 le \Og type \Fg{} ou \Og préfiguration \Fg{},
 et de leur \Og antitype \Fg{} ou \Og accomplissement \Fg{},
 personne ou évènement du Nouveau Testament.}
 biblique, Haman est un type de la chair,
 quelque chose que Dieu hait.
 Tout comme les Amalécites faisaient constamment la guerre à Israël,
 la chair fait aussi constamment la guerre à l'Esprit.
 Comme les Amalécites attaquaient Israël à son point le plus faible,
 la chair va vous attaquer à votre point le plus faible.
 De même que les Amalécites essayaient d'empêcher Israël
 d'entrer en Terre Promise, la chair va essayer de vous empêcher
 d'être victorieux et de goûter aux promesses de la vie dans l'Esprit.

Dieu jura qu'Il serait en guerre avec Amalec de génération en génération,
 mais Il a aussi promis qu'un jour Amalec périrait et
 qu'il serait complètement oublié.

\dvbox{
Un jour, la chair sera éliminée à jamais.
}

Nous avons tous un Haman dans nos vies
 \ocadr{}ce problème particulier de la chair
 qui n'a pas encore été réglé.
 Considérons le comme mort, crucifié avec Christ,
 afin d'avoir Sa complète victoire et de marcher dans l'Esprit.

\dvrule

\dvprayer{
Père, Merci d'avoir offert la victoire sur la vieille nature
 par l'intermédiaire de Jésus-Christ.
 Aide-nous à continuer d'avancer dans l'Esprit
 afin que nous puissions connaître
 Ta puissance sur la chair.
}{\Amen}



%%%%%%%%%%%%
% 11 avril
%%%%%%%%%%%%

\dvday{Vaincre la peur}

\dvquote{
Va rassembler tous les Juifs qui se trouvent à Suse.\\
 Jeûnez à mon intention, sans manger ni boire pendant trois jours,
 vingt-quatre heures sur vingt-quatre.\\
 Moi aussi je jeûnerai de même avec mes jeunes servantes.\\
 Dans ces conditions, j'irai chez le roi malgré la loi.\\
 Si c'est pour ma perte, je périrai !}{\ibibleverse{Est}(4:16)}

\lettrine{À}{ la} demande d'Haman, le roi de Perse décréta qu'à un jour fixé,
 tous les juifs du royaume seraient mis à mort.
 Personne ne savait que la Reine Esther était juive.
 Son cousin l'encouragea à se présenter à son mari et à plaider pour son peuple,
 mais faire cela allait mettre sa vie en danger.

Les desseins de Dieu vont toujours être accomplis,
 indépendamment de ce que nous faisons ou ne faisons pas.
 Cependant, Il nous permet d'être des instruments par lesquels Il les accomplit.
 Esther choisit d'être cet instrument.
 Elle reconnut que Dieu l'avait conduite jusqu'à ce moment très spécial,
 et elle surmonta sa peur.
 \Og Même si cela doit me coûter la vie, \Fg{} dit-elle \Og je le ferai.
 Si je dois périr, je périrai. \Fg{}

Jésus nous a dit de ne pas nous inquiéter du lendemain,
 car le lendemain s'inquiétera de lui-même.
 Et pourtant, beaucoup de chrétiens s'inquiètent.
 Ils s'inquiètent parce qu'ils ne se sont pas encore
 abandonnés à la volonté de Dieu.

\dvbox[0.92\textwidth]{
L'abandon à Dieu libère de la peur et de l'anxiété.
}

Le jour où Dieu quittera le trône, nous aurons tous de gros ennuis.
 Mais, tant que Dieu est sur le trône, nous n'avons rien à craindre.
 Après tout, si nous pouvons Lui faire confiance pour notre destinée éternelle,
 ne pourrions-nous pas Lui faire confiance pour~demain?

\dvrule

\dvprayer{
Seigneur, libère Ton peuple des inquiétudes pour le futur.\\
 Aide-nous à croire que Tu vas accomplir Ton dessein éternel
 quand nous Te remettons notre sort,
 par l'intermédiaire de Jésus-Christ, notre Seigneur.
}{\Amen}



%%%%%%%%%%%%
% 12 avril
%%%%%%%%%%%%

\dvday{Pourquoi suis-je né ?}

\dvquote{
Pourquoi ne suis-je pas mort dès les entrailles de ma mère?\\
 Pourquoi n'ai-je pas expiré au sortir de son ventre?
}{\ibibleverse{Jb}(3:11)}

\lettrine{E}{n seulement} l'espace de quelques instants,
 Job perdit ses enfants, sa santé et toutes ses possessions terrestres.
 Il se retrouva comme une âme nue, réduite à l'expression la plus simple
 de l'existence, sans aucun support ni soutien.

\dvbox{
Quelles questions pose un homme qui a été dépouillé de tout ?
}

Job voulait savoir pourquoi il était né. Il n'a pas maudit Dieu,
 mais il a maudit le jour de sa naissance.
 \Og Pourquoi suis-je donc né? Quel est le sens de ma vie? \Fg{}

Si vous souscrivez à la théorie de l'évolution, la réponse est évidente.
 Votre vie n'a aucun sens, car votre existence est le fruit d'un accident.
 Votre existence est le résultat de l'occurrence fortuite
 de circonstances accidentelles survenues pendant des milliards d'années.

La Bible, toutefois, enseigne que la vie existe après la mort.
 Cet endroit-ci est simplement un endroit où Dieu me prépare
 afin que je puisse vivre avec Lui pour toujours.
 Les épreuves, les difficultés, les tribulations,
 les déceptions sont toutes destinées à me montrer
 combien les choses terrestres sont temporaires
 et à m'apprendre à vivre pour ce qui est éternel et non pour le présent.

Pourquoi suis-je né? Pourquoi êtes-vous nés?
 Dieu nous a créés pour Le connaître et Lui faire confiance,
 afin de pouvoir vivre avec Lui dans la gloire de Son royaume pour~toujours. 

\dvrule

\dvprayer{
Père, apprends-nous à Te faire confiance,
 sachant que nos vies sont entre Tes mains.\\
 Nous savons que l'ennemi ne peut pas faire plus
 que ce que Tu lui laisses faire.\\
 Quand nous trébuchons ou que nous échouons à une épreuve,
 c'est seulement pour que nous puissions voir
 combien nous sommes faibles
 afin de Te faire davantage confiance. 
}{\DlNdJ}



%%%%%%%%%%%%
% 13 avril
%%%%%%%%%%%%

\dvday{Qu'est-ce que l\ap{}homme?}

\dvquote{
Qu'est-ce que l'homme, pour que tu en fasses tant de cas,
 pour que tu le prennes tellement à c\oe{}ur?
}{\ibibleverse{Jb}(7:17)}

\lettrine{N}{ous} ne sommes tous qu'un infime grain de poussière
 sur un infime grain de poussière appelé la Terre,
 qui gravite autour du Soleil dans un petit coin de la galaxie
 de la Voie Lactée.
 Cependant les Écritures disent que les pensées de Dieu pour nous
 sont si nombreuses, que si nous pouvions les compter
 leur nombre dépasserait le nombre des grains de sable de la mer.

Que Dieu pense à nous est stupéfiant. Encore plus extraordinaire
 est le fait qu'Il nous exalte \ocadr{}au-dessus des plantes,
 au-dessus des animaux, et même au-dessus des anges.
 Job voulait savoir pourquoi.
 \Og Qu'est-ce que l'homme, pour que Tu l'exaltes? \Fg{}

Ce sont les voies de Dieu.

\dvbox{
Dieu exalte la personne qui Lui voue sa vie.
}

Dieu rend une personne plus grande qu'elle ne pourrait jamais l'être
 en dehors de Son intervention.
 Dieu agit ainsi parce qu'Il nous aime.
 Il nous aime tant qu'Il a donné Son Fils unique,
 qui est devenu ce que nous étions afin qu'Il puisse
 nous faire devenir ce qu'Il est.

Et ainsi, Dieu nous développe et nous forme.
 Il permet les épreuves et les déconvenues parce qu'Il sait
 que c'est la seule fa\c{c}on de nous préparer à l'éternité.
 Comme Job, qui passa par une période de mise à l'épreuve
 où il perdit tout ce qu'il avait, nous passons quelquefois
 par des moments de perte et de déconvenue.
 Mais chaque perte fait partie du plan éternel de Dieu
 qui nous prépare ainsi à être avec Lui pour toujours. 

\dvrule

\dvprayer{
Père, merci pour Ta fidélité à nous fa\c{c}onner pour faire de nous
 la personne que Tu veux que nous soyons et enlever tout ce qui pourrait
 nous polluer ou nous détruire.
 Combien nous sommes reconnaissants que Ton c\oe{}ur se soit fixé sur nous.
 Puissions-nous marcher dans cet amour. 
}{\Amen}


%%%%%%%%%%%%
% 14 avril
%%%%%%%%%%%%

\dvday{Cri du c\oe{}ur pour un médiateur}

\dvquote{
Il n'est pas un homme comme moi, pour que je lui réponde,
 pour que nous allions ensemble en justice.
 Il n'y a pas entre nous d'arbitre,
 qui pose sa main sur nous deux.
}{\ibibleverse{Jb}(9:32-33)}

\lettrine[ante=\Og]{S}{i seulement} tu te mettais en règle avec Dieu \Fg{},
 conseillaient les amis de Job, \Og tout le reste irait bien. \Fg{}
 Mais Job ne savait pas comment faire.
 Il voyait la grandeur de Dieu comparée à sa propre petitesse,
 et se rendait compte que l'écart entre le Dieu infini
 et sa créature limitée est bien trop grand pour que l'homme
 puisse le franchir lui-même.
 Reconnaissant ce dilemme, Job réclama un médiateur.

Dans le Nouveau Testament, nous lisons \Og Il y a un seul Dieu,
 et aussi un seul Médiateur entre Dieu, et les hommes,
 le Christ-Jésus homme \Fg{} (\ibibleverse{ITm}(2:5)).

\dvbox{
Jésus est le pont entre Dieu et l'homme.
}

Parce que Jésus et le Père sont un (\ibibleverse{Jn}(10:30)),
 Il est capable de toucher Dieu.
 Parce que Jésus \Og s'est fait chair et a habité parmi nous, \Fg{}
 (\ibibleverse{Jn}(1:14)), Il nous touche aussi.
 Il comprend nos faiblesses, nos peurs, nos tentations.
 Et ainsi, Jésus est capable de nous amener à Dieu.

Ce pont qui a amené l'homme à Dieu n'a pas été créé par l'homme.
 À la différence des autres religions, dans le Christianisme,
 ce n'est pas l'homme qui essaye d'atteindre Dieu
 \ocadr{}c'est Dieu qui s'abaisse pour atteindre l'homme.
 Si vous voulez trouver le vrai~Dieu éternel et vivant,
 vous ne pouvez le toucher que si vous Le laissez vous toucher
 par l'intermédiaire de Son Fils, Jésus. 

\dvrule

\dvprayer{
Père, nous sommes si reconnaissants de ce que Tu as offert un moyen
 par lequel nous pouvons être justifiés, non pas par nos \oe{}uvres de justice,
 mais simplement en croyant et en faisant confiance à notre Médiateur
 Jésus-Christ,
 Celui qui intercède pour nous. 
}{\Amen}


%%%%%%%%%%%%
% 15 avril
%%%%%%%%%%%%

\dvday{Rien + rien = rien}

\dvquote{
Qu'il ne croie pas au néant! Il se tromperait.
 Car le néant lui sera donné en échange.
}{\ibibleverse{Jb}(15:31)}

\lettrine{T}{entant} d'expliquer pourquoi Dieu avait dépouillé Job
 de toutes ses possessions, Éliphaz supposa que Job avait mis sa confiance
 dans ces possessions de fa\c{c}on déraisonnable.
 Bien que Job ne l'ait pas fait, Éliphaz avait raison de s'en prendre
 à la folie de ceux à qui l'on donne un sentiment de sécurité trompeur
 en leur richesse. 

La Bible nous met continuellement en garde contre la tromperie.
 Satan a trompé Ève en la faisant douter de la Parole de Dieu
 et en lui faisant, au contraire, mettre sa confiance dans les vaines
 promesses de Satan.
 Aujourd'hui encore, Satan essaye de nous tromper en nous faisant penser
 que les lois de Dieu ne sont pas justes ou qu'elles ne s'appliquent
 pas à nos vies. 

\dvbox{
Satan a trompé beaucoup de gens en les amenant à croire au néant. 
}

C'est un fait avéré~: Tout ce que nous semons, nous le récoltons.
 Nos pensées sont un champ fertile où nous plantons des semences chaque jour.
 Si nous semons pour la chair, nous ne pouvons pas espérer récolter
 une moisson spirituelle.
 Comme le disent les informaticiens~:
 \Og Ordures en entrée; Ordures en sortie. \Fg{}
 Ou pour parler comme en mathématiques~: rien plus rien égale rien. 

Certains se confient aux richesses pour obtenir sécurité et satisfaction.
 Certains se confient à des activités et des rituels religieux.
 Certains se confient dans le néant de leur propre justice.
 Mais quand vous vous confiez au néant, votre seule rétribution sera le néant.
 Ne soyez pas faussement conduits à croire que vous pouvez gagner
 votre entrée dans la grâce de Dieu.
 Mettez votre espérance en Jésus \ocadr{}le chemin, la vérité et la vie.

\dvrule

\dvprayer{
Seigneur, puissions-nous commencer à planter Ta Parole
 dans le sol fertile de nos pensées et semer pour l'Esprit
 afin de pouvoir récolter la vie éternelle en Jésus-Christ.
}

\missing{Fin de prière?}


%%%%%%%%%%%%
% 16 avril
%%%%%%%%%%%%

\dvday{Le triomphe de la foi}

\dvquote{
Mais je sais que mon Rédempteur est vivant,
 et qu'Il se lèvera le dernier sur la terre,\\
 après que ma peau aura été détruite;\\
 moi-même en personne, je contemplerai Dieu.\\
 C'est Lui que moi je contemplerai, que mes yeux verront,
 et non quelqu'un d'autre;\\
 mon c\oe{}ur languit au-dedans de moi.
}{\ibibleverse{Jb}(19:25-27)}

\lettrine{I}{l y avait} beaucoup de choses que Job ne savait pas~:
 il ne savait pas pourquoi il avait tout perdu, ni pourquoi
 il se trouvait dans cet état misérable,
 ni pourquoi il devait endurer tant de souffrances.
 Mais Job savait une chose~: \Og Mon Rédempteur est vivant. \Fg{}
 Et c'est à cela qu'il s'est accroché. 

Chacun reconnaît que Job a supporté ses afflictions avec patience.
 Le secret de la patience, c'est la foi. C'est en raison de ce qu'il croyait
 que Job a été capable d'endurer patiemment toutes les souffrances
 qu'il ne comprenait pas. 

\dvbox{
N'abandonnez jamais ce que vous savez
 à cause de quelque chose que vous ne savez pas. 
}

Qu'est-ce que Job croyait qui faisait qu'il avait une si grande patience?
 Premièrement, Job croyait que Dieu contrôlait toutes les circonstances
 de sa vie. Le Seigneur donne, et le Seigneur reprend. 

Deuxièmement, il comprenait que Dieu était pour lui.
 Bien que sa femme se soit retournée contre lui,
 bien que ses amis l'aient condamné et accusé de crimes odieux,
 il savait que Dieu était de son côté. 

Dieu est aussi de votre côté.
 Quand les choses sont confuses ou difficiles à comprendre, rappelez-vous le.

\dvrule

\dvprayer{
Père, nous sommes reconnaissants pour le triomphe de la foi
 qui nous amène à une glorieuse victoire.\\
 Apprends-nous à Te faire confiance comme Job l'a fait
 et à endurer la nuit obscure en attendant l'aube du nouveau jour,
 le jour de Ton glorieux royaume. 
}{\DlNdJ}


%%%%%%%%%%%%
% 17 avril
%%%%%%%%%%%%

\dvday{La quête de Dieu}

\dvquote{
Ah ! comme j'aimerais savoir où trouver Dieu !
 Je me rendrais alors jusqu'à Sa résidence !
}{\ibibleverse{Jb}(23:3)}

\lettrine{É}{liphaz} a dit à Job que si seulement il pouvait trouver Dieu,
 tout s'arrangerait pour lui. Mais Job ne savait pas où trouver Dieu.

Certains affirment que parce qu'ils ne peuvent pas voir Dieu,
 ils ne peuvent pas croire en Lui. Mais pouvez-vous voir le vent ?
 Pouvez-vous voir l'électricité ?
 Vous n'avez pas à voir quelque chose pour en ressentir les effets.
 Mettez votre doigt dans une prise électrique et la décharge
 que vous allez ressentir aura très vite fait de faire de vous
 un \Og croyant \Fg{} en l'électricité.

Bien que je n'ai jamais vu Dieu, j'ai cependant ressenti Sa présence
 et Son influence dans ma vie.
 J'ai entendu le son doux et subtil de Sa petite voix
 et j'ai vu Sa puissance à l'\oe{}uvre dans ma vie
 et dans la vie de personnes de mon entourage.

Êtes-vous à la recherche de Dieu ? Jésus a affirmé~:
 \Og Le Père et moi, nous sommes un \Fg{} (\ibibleverse{Jn}(10:30)).
 Avez-vous soif d'une relation réelle et pleine de sens avec Dieu ?
 Jésus s'est écrié vers la foule~:
 \Og Si quelqu'un a soif, qu'il vienne à Moi et qu'il boive \Fg{}
 (\ibibleverse{Jn}(7:37)).

Si votre esprit est à la recherche de Dieu, si vous avez soif
 d'une relation réelle et pleine de sens avec Lui,
 alors Jésus est la fin de votre quête.

\dvbox{
Quand vous venez à Jésus, votre quête est terminée,
 car vous avez alors découvert Dieu.
}

\dvrule

\dvprayer{
Père, merci de ce que nous avons pu découvrir Ton amour.
 Nous ressentons Ta présence, nous entendons Ta voix,
 et nous voyons les effets de Ta main à l'\oe{}uvre dans le monde.
}{\DlNdJ}


%%%%%%%%%%%%
% 18 avril
%%%%%%%%%%%%

\dvday{Rencontrer Dieu}

\dvquote{
Je ne savais de toi que ce qu'on m'avait dit, mais maintenant,
 c'est de mes yeux que je t'ai vu.\\
 C'est pourquoi je retire ce que j'affirmais,
 je reconnais avoir eu tort et m'humilie en m'asseyant
 dans la poussière et dans la cendre. 
}{\ibibleverse{Jb}(42:5-6)}


\lettrine{J}{usqu'à} ce moment de l'histoire,
 Job s'est défendu auprès de ses amis.
 Quand ils ont suggéré qu'il y avait peut-être quelque chose
 de pas très clair dans son passé, Job l'a farouchement contesté.
 \Og J'ai été honnête et juste. J'ai contribué aux besoins de ma communauté
 et j'ai maintenu mon intégrité. \Fg{}
 Quand il s'examinait à la lumière des autres, Job s'en sortait plutôt bien.

Nous aimons bien nous comparer aux autres.
 Quand nous faisons des autres la norme en matière de justice,
 la comparaison avec eux tourne souvent à notre avantage.
 \Og Je ne suis pas parfait, mais je suis bien mieux que tel autre. \Fg{}
 Mais, au bout du compte, nous ne serons pas jugés en prenant
 les autres pour norme.
 La norme par rapport à laquelle nous serons jugés est Jésus-Christ.

Quand Job a tourné son regard vers Dieu, sa vision de lui-même a changé.
 Il a dit alors~: \Og Je reconnais avoir eu tort et je m'humilie en m'asseyant
 dans la poussière et dans la cendre. \Fg{}
 Lui qui avait maintenu son intégrité, s'est soudain vu à la lumière
 de la sainteté et de la pureté de Dieu.
 Voyant sa propre impureté, il s'est repenti.

\dvbox{
Vous vous vantez peut-être de vos réussites et succès.
 Mais notez bien ceci~: quand vous voyez un homme orgueilleux,
 vous voyez un homme qui n'a pas encore rencontré Dieu.
}

Parce que quand vous voyez vraiment Dieu,
 il n'y a plus de place pour l'orgueil,
 même si vous êtes un ministre du gouvernement,
 un président de la République ou le Pape.
 Une fois qu'une personne voit vraiment Dieu,
 il en résulte l'humilité et une profonde contrition.
 Il en résulte un c\oe{}ur repentant,
 un c\oe{}ur qui s'écrie~:
 \Og Dieu, aie pitié de moi qui ne suis qu'un pécheur. \Fg{}

\dvrule

\dvprayer{
Père, nos c\oe{}urs ont soif de Te voir pour avoir une juste vision 
de nous-mêmes et pour que nous puissions être changés.
Seigneur rend nous semblables à Toi.
}{\Amen}


%%%%%%%%%%%%
% 19 avril
%%%%%%%%%%%%

\dvday{La Croix}

\dvquote{
Mon Dieu, Mon Dieu, pourquoi m'as-Tu abandonné?
}{\ibibleverse{Ps}(22:1)}

\lettrine{L}{e \ibibleverse{Ps}(22:)} est une démonstration
 de la prescience de Dieu, c'est-à-dire de sa connaissance anticipée
 des évènements futurs.
 Dans ce passage, Il nous annonce à l'avance Son plan de rachat
 de l'homme où Il laisse Son Fils mourir de l'infamante mort de la crucifixion.

Les Évangiles nous disent qu'après que Jésus a
 \grammar{après que + indicatif}
 été cloué à la croix
 près de trois heures \ocadr{}autour de midi\fcadr{} le ciel s'est obscurci soudainement.
 Et de l'obscurité est monté un cri de la part de l'Homme de la croix du milieu~:
 \Og Mon Dieu, Mon Dieu, pourquoi M'as-tu abandonné ? \Fg{}.
 Le cri de Jésus nous ramène au \ibibleverse{Ps}(22:).
 Mais il nous pose et nous impose la question~:
 Pourquoi Dieu abandonnerait-Il Son Fils à ce moment critique ?

\dvbox{
Dieu est absolue sainteté.
 En conséquence, Il lui est impossible d'avoir
 la moindre association avec le péché.
}

Au moment où Dieu a mis mon péché \ocadr{}et le péché du monde entier\fcadr{}
 sur Jésus-Christ, la conséquence inévitable s'est produite~:
 Jésus a été séparé de Dieu. Il a été abandonné de Son Père
 pour un temps afin que nous ne soyons pas abandonnés de Dieu pour l'éternité.

Rien ne peux amener dans votre vie une paix et une confiance plus grandes
 que de reconnaître la seigneurie \typo{seigneurie}
 de Jésus, de vous soumettre à Son autorité,
 et de placer votre vie entre Ses mains.
 Non seulement cette décision vous amène la paix et la confiance,
 elle vous amène aussi la force et l'endurance.

\dvrule

\dvprayer{
Père, nous sommes tellement reconnaissants pour la Croix et l'espoir
 qu'elle nous amène.\\
 Nous Te remercions de ce que Tu aies bien voulu abandonner
 Ton Fils pour un temps afin que Tu n'aies jamais à nous abandonner.\\
 Puissions-nous nous rendre à Lui pour être gouvernés par Lui
 et conduits par Ton Esprit Saint.
}{\Amen}

\suggest{ou Esprit-Saint?}


%%%%%%%%%%%%
% 20 avril
%%%%%%%%%%%%

\dvday{Le Seigneur\\ est mon berger}

\dvquote{
Le Seigneur est mon berger\dots{}
}{\ibibleverse{Ps}(23:1)}

\lettrine{D}{avid} comprend la profondeur de ces mots
 d'une fa\c{c}on que peu d'entre nous peuvent pleinement saisir.
 Étant lui-même berger, il connaissait la relation entre le berger
 et ses moutons.
 Il savait que c'était le travail du berger que de protéger
 des prédateurs les moutons sans défense et de les conduire
 jusqu'à la nourriture et l'eau.
 Il savait qu'à moins que le berger ne les empêche d'errer,
 les moutons pouvaient facilement s'égarer.

Au verset \ibiblevs{Ps}(23:4) de ce psaume, David déclare~:
 \Og ta houlette et ton bâton, voilà mon réconfort. \Fg{}
 La houlette était la baguette que le berger utilisait
 pour frapper le mouton sur le flanc, tandis que le bâton
 était un bâton de marche avec un bout recourbé
 que l'on utilisait pour tirer le mouton vers l'arrière.

\dvbox[0.94\textwidth]{
Les moutons peuvent être très têtus\dots{} nous aussi !
}

Quand un mouton a décidé de suivre son propre chemin,
 le berger va prendre sa houlette et le frapper
 sur le flanc pour le ramener dans le troupeau.

Quelquefois, il est nécessaire que Dieu nous donne aussi un bon coup.
 La Bible nous dit~:
 \Og Ne prends pas à la légère la correction du Seigneur\dots{}
 car le Seigneur corrige celui qu'Il aime \Fg{} (\ibibleverse{He}(12:5-6)).

Si vous êtes un enfant de Dieu et que vous agissez mal,
 Il ne va pas vous laisser vous en tirer comme \c{c}a.
 Tout autre personne s'en sortira sans se faire prendre, mais vous,
 vous vous ferez prendre à tous les coups, parce que le Seigneur
 corrige tous ceux qu'Il reconnaît comme Ses fils.
 Sa houlette sera là pour vous garder sur le droit chemin.
 Et Son bâton est là pour vous tirer hors de danger. 

\dvrule

\dvprayer{
Père, merci de nous protéger, de nous guider et de pourvoir à nos besoins.\\
 Merci de nous aimer assez pour nous corriger quand nous en avons besoin. 
}{\DlNdJ}


%%%%%%%%%%%%
% 21 avril
%%%%%%%%%%%%

\dvday{La lourde main du Seigneur}

\dvquote{
Tant que je me suis tu, mes os se consumaient, je gémissais toute la journée;
 car nuit et jour ta main pesait sur moi, ma vigueur
 n'était plus que sécheresse, comme celle de l'été.
}{\ibibleverse{Ps}(32:3-4)}

\lettrine{D}{avid} était coupable d'une relation adultère
 qui a finalement conduit au meurtre.
 Bien qu'il ait tenté de dissimuler sa culpabilité,
 son secret le rongeait.
 Il ne pouvait pas chasser de son esprit ce qu'il avait fait. 

Quand la lourde main de Dieu s'appesantit sur votre vie,
 vous ne pouvez pas trouver le repos.
 Jour et nuit, votre conscience vous tourmente. 

\dvbox{
Le péché est agréable pendant un moment, mais ce moment est bien éphémère.
 Un moment de~plaisir peut coûter à la personne des années de chagrin,
 de douleur, d'abattement. 
}

Peut-être faites-vous l'expérience de la culpabilité décrite par David.
 Votre péché vous tourmente.
 Vous en voyez les conséquences et cela vous ronge de l'intérieur.
 Jamais vous n'auriez pensé que les choses aillent aussi loin.
 Mais l'attraction a été la plus forte et vous vous sentez misérables.
 La main de Dieu s'appesantit sur vous. 

J'ai de bonnes nouvelles pour vous. Dès l'instant où David a dit~:
 \Og Je reconnais mon péché, je le confesse \Fg{} (\ibibleverse{Ps}(32:5)),
 Dieu l'a pardonné.
 Et vous pouvez être pardonnés tout aussi rapidement.
 La première lettre de Jean nous dit~:
 \Og Si nous confessons nos péchés, Il est fidèle et juste
 pour nous pardonner nos péchés et nous purifier de toute injustice \Fg{}
 (\ibibleverse{IJn}(1:9)).

Le pardon amène la délivrance, délivrance de la culpabilité,
 de la honte et de la lourde main de Dieu. 

\dvrule

\dvprayer{
Seigneur, puissions-nous faire l'expérience de la joie glorieuse du pardon,
 l'expérience de pouvoir marcher libérés de la culpabilité du péché
 parce que nous avons été lavés et purifiés par Ton amour
 par l'intermédiaire de Ton Fils Jésus-Christ. 
}{\Amen}


%%%%%%%%%%%%
% 22 avril
%%%%%%%%%%%%

\dvday{Ne vous inquiétez pas}

\dvquote{
Ne t'irrite pas\dots{}
}{\ibibleverse{Ps}(37:1)}

\lettrine{D}{ans} notre vie ici-bas, nous sommes enclins
 à nous faire du souci pour des tas de choses.
 Mais David nous donne le remède contre les soucis dans
 les versets \ibiblevs{Ps}(37:5-7) de ce psaume~:
 \Og Remets ton sort à l'Éternel, confie-toi en Lui,
 et c'est Lui qui agira\dots{}
 Garde le silence devant l'Éternel, et attends-toi à Lui. \Fg{}

Si nous concentrons notre attention sur un problème, il semble grandir.
 Il grandit au point de nous submerger et nous perdons de vue Le Seigneur.
 Mais quand nous nous concentrons sur Le Seigneur
 \ocadr sur Son amour, Sa grâce, Son pouvoir, Sa gloire, Sa puissance \fcadr{}
 alors ce sont nos problèmes qui rétrécissent au point
 où nous ne les voyons pratiquement plus. 

\dvbox{
Nous devons faire attention à ce sur quoi nous fixons notre attention. 
}

David nous dit que lorsque nous remettons nos problèmes à Dieu,
 deux choses se produisent~: Il agit, c'est-à-dire qu'Il accomplit
 Sa volonté nous concernant, et nous trouvons le repos auquel nous aspirons.

Le psalmiste déclare que la personne qui se confie au Seigneur
 n'aura pas peur au jour des mauvaises nouvelles.
 Cette semaine vous avez peut-être entendu des mauvaises nouvelles
 qui vous créent du souci. Rappelez-vous~: remettez cela à Dieu,
 reposez-vous en Lui, et attendez patiemment.
 Laissez la chose entre Ses mains.
 Les soucis ne vont pas régler vos problèmes,
 mais Lui faire confiance amène la paix et le repos. 

Puisse le Seigneur vous bénir et vous amener à cet état de foi,
 de confiance, de dépendance et de repos. 

\dvrule

\dvprayer{
Seigneur, apprends-nous à T'amener toutes choses.\\
 Quand les soucis menacent de nous submerger,\\
 aide-nous à remettre nos problèmes entre Tes mains et à nous reposer
 dans le fait que Tu es souverain et puissant.\\
 Puissions-nous fixer nos c\oe{}urs et nos vies sur Toi en ce jour. 
}{\Amen}


%%%%%%%%%%%%
% 23 avril
%%%%%%%%%%%%

\dvday{Le remède à la dépression}

\dvquote{
Pourquoi t'abats-tu, mon âme,
 et gémis-tu sur moi ? 
}{\ibibleverse{Ps}(43:5)}

\lettrine{E}{n langage moderne,} on pourrait traduire ce verset par~:
 \Og Pourquoi es-tu si déprimé ?
 Pourquoi es-tu autant rempli d'anxiétés ? \Fg{}

La société nous place dans des situations de stress nombreuses et difficiles.
 Nous devons, par moment, nous débattre avec nos finances,
 nos relations humaines, notre travail.
 Le futur semble incertain. Quand nous ne sommes plus capables
 de supporter ces pressions, le résultat\dots{}
 c'est l'anxiété ou la dépression.
 Ces sentiments de mécontentement ou de désespoir sont amplifiés
 quand nous ne pouvons plus voir d'issue à nos circonstances
 \ocadr quand nous nous sentons piégés dans un sombre labyrinthe
 dont nous ne pouvons plus trouver la sortie. 

L'âme de David était abattue. Il était perturbé à cause des nations impies
 qui l'entouraient et l'opprimaient.
 Il était perturbé à cause des hommes fourbes qui avaient été malhonnêtes
 dans leurs agissements avec lui. 

Mais au milieu de cette dépression, David se rappelle de la solution~:
 \Og Attends-toi à Dieu! \Fg{} (\ibibleverse{Ps}(43:5)).

\dvbox{
Vous n'allez pas trouver la solution à vos problèmes auprès des autres
 ou en vous-mêmes.
}

\typo{vous-mêmes}

Éloignez vos pensées du problème et dirigez les vers Dieu;
 éloignez-les de vos propres faiblesses pour les diriger vers Sa force.
 Vous devez vous souvenir que Dieu vous aime et qu'Il contrôle
 les circonstances qui entourent votre vie. 

\dvrule

\dvprayer{
Père, merci de t'intéresser à ces choses qui nous ont mis
 sous une telle pression et qui ont créé l'anxiété,
 le découragement et la dépression.\\
 Nous te demandons de prendre ces situations en main
 pour commencer à accomplir Ton plan. 
}{\DlNdJ}


%%%%%%%%%%%%
% 24 avril
%%%%%%%%%%%%

\dvday{La joie du salut}

\dvquote{
Rends-moi la joie de Ton salut\dots{}
}{\ibibleverse{Ps}(51:12)}

\lettrine{L}{a joie} est une des marques de la vie chrétienne.
 Mais la joie ne doit pas être confondue avec le bonheur.
 Le bonheur est une émotion qui dépend des circonstances
 extérieures de ma vie, ce qui signifie que je peux être très heureux
 à un instant donné, et très malheureux l'instant d'après. 

En revanche, la joie est une émotion suscitée par des circonstances
 intérieures \ocadr ma relation avec Dieu \fcadr{} et ainsi la joie
 est une expérience plus constante.
 Grâce à ma relation avec Dieu, je peux ressentir de la joie
 même au sein de circonstances difficiles ou tragiques. 

Quand David a péché \ocadr et c'étaient des péchés de la pire espèce~:
 adultère conduisant au meurtre \fcadr{} il a perdu la joie de son salut,
 car le péché rompt la communion avec Dieu.
 David ne pouvait plus ressentir la présence de Dieu. 

\dvbox{
Le péché entraîne le mal-être. 
}

Même si une personne est sauvée, son péché va lui voler sa joie.
 L'Esprit Saint \suggest{ou Esprit-Saint?}
 va vous convaincre de péché et cela va entraîner
 un sentiment d'abattement jusqu'à ce que vous arriviez au point de confession
 et de repentance.
 Quand cela arrive \ocadr quand nous laissons tomber notre péché
 et que nous retournons à Dieu \fcadr Il restaure la joie de notre salut
 et nous ramène dans la communion avec Lui. 

Peut-être qu'aujourd'hui, vous éprouvez de la culpabilité.
 Peut-être que comme David, le péché a volé la joie de votre salut.
 La solution est simple~: invoquez Dieu qui est miséricordieux
 et prêt à vous laver de votre péché et à restaurer la joie de votre salut.

\dvrule

\dvprayer{
Père, nous Te sommes tellement reconnaissants pour Ta fidélité
 et Ta volonté de pardonner tous ceux qui invoquent Ton nom.\\
 Qu'il est bon de marcher avec Toi,
 d'avoir conscience de Ta proche présence!\\
 Nous nous réjouissons aujourd'hui dans la joie de notre salut. 
}{\Amen}


%%%%%%%%%%%%
% 25 avril
%%%%%%%%%%%%

\dvday{Remets ton fardeau à l'Éternel}

\dvquote{
Remets ton fardeau à l'Éternel, et Il te soutiendra;
 Il ne laissera jamais chanceler le juste\dots{}
}{\ibibleverse{Ps}(55:23)}

\lettrine{O}{n pense} que David a écrit ce psaume quand son fils,
 Absalom, s'est rebellé contre lui.
 Au même moment, son plus proche ami, Ahitophel,
 s'est aussi retourné contre lui. Le c\oe{}ur brisé et la peur au ventre,
 David a d'abord cherché à fuir.
 Mais il s'est alors repris en se disant~:
 \Og Remets ton fardeau à l'Éternel, et Il te soutiendra. \Fg{}

\dvbox{
La paix survient quand nous remettons à Dieu les problèmes
 qui nous dépassent et que nous cessons d'essayer
 d'en contrôler les résultats. 
}

David a fait preuve de sagesse en remettant son problème à Dieu.
 Le prophète Ésaïe a dit~:
 \Og \`A celui qui est ferme dans ses dispositions, Tu assures la paix,
 la paix, parce qu'il se confie en Toi \Fg{} (\ibibleverse{Is}(26:3)).
 Si, par contre, vous essayez de manipuler une situation pour obtenir
 un résultat escompté, vous allez connaître l'angoisse. 

Nous trouvons la force de \Og lâcher prise \Fg{},
 quand nous nous rappelons que Dieu est un Père qui nous aime
 à la folie et prend grand soin des Siens.
 Il ne permettra à rien d'arriver dans votre vie qui puisse vous détruire.
 Son dessein est de faire grandir votre relation avec Lui au point
 où vous serez capable de Lui faire confiance quand des déceptions
 ou des malheurs inattendus arriveront. 

Déchargez-vous sur Lui de tous vos soucis et laissez-les là.
 Faites confiance au Seigneur.
 Il prendra soin de tous les fardeaux que vous portez. 

\dvrule

\dvprayer{
Père, nous Te remercions qu'au milieu de toute notre confusion
 \ocadr les pressions de la vie, les déceptions dans nos relations,
 les malheurs douloureux \fcadr{} Tu es là pour nous aider.\\
 Apprends-nous à déposer tous nos soucis à Tes pieds.
}{\DlNdJ}


%%%%%%%%%%%%
% 26 avril
%%%%%%%%%%%%

\dvday{Accablé}

\dvquote{
Du bout de la terre je crierai à Toi, quand mon c\oe{}ur est accablé;
 conduis-moi sur le roc qui est plus haut que moi.
}{\ibibleverse{Ps}(61:2) \KJF{}}

\lettrine{D}{avid} a eu sa part de gloire,
 mais il a aussi eu sa part de problèmes.
 Quand David n'était encore qu'un adolescent,
 le Roi Saül est devenu jaloux de lui et a tout fait pour le tuer.
 Plus tard encore, David a été confronté à des problèmes dans son mariage
 et avec ses enfants. Les psaumes sont le texte des prières prononcées
 par David dans les moments d'accablement de sa vie.
 Et c'est parce qu'il a eu autant de problèmes,
 que nous avons autant de psaumes.

\dvbox{
Les problèmes de David l'ont toujours amené à la prière.
}

David savait que quand il atteignait les limites de ses capacités,
 il pouvait invoquer Celui dont la puissance et les capacités sont sans limites.
 Nous aussi, nous pouvons invoquer le même Rocher \ocadr Jésus-Christ,
 notre rocher de délivrance, notre rocher de défense, notre rocher de salut,
 notre rocher de puissance.

Notez-bien que nous devons être conduits au Rocher.
 Notre part est d'invoquer; la part de l'Esprit Saint \suggest{ou Esprit-Saint?}
 est de nous conduire à Jésus.

Si vous n'en pouvez plus aujourd'hui, si vos circonstances dépassent
 ce que vous pouvez supporter, laissez-moi vous encourager à invoquer Jésus
 \ocadr le Rocher qui est plus haut que vous et plus haut
 que tous vos problèmes.
 Il va vous délivrer. Il va vous placer dans une position de puissance
 et de victoire. 

\dvrule

\dvprayer{
Père, beaucoup de personnes sont aujourd'hui confrontées à des problèmes
 et des situations qui dépassent leur compréhension et leurs capacités.\\
 Je prie que Ton Esprit les conduise aujourd'hui à Ton Fils,
 et qu'elles puissent trouver en Jésus leur force, leur délivrance,
 leur santé et leur paix. 
}{\DlNdJ}


%%%%%%%%%%%%
% 27 avril
%%%%%%%%%%%%

\dvday{Dieu est bon}

\dvquote{
Dieu est vraiment bon\dots{}
}{\ibibleverse{Ps}(73:1)}

\lettrine{T}{ôt ou tard,} nous allons tous nous retrouver confrontés
 à quelque chose que nous ne pouvons pas comprendre.
 Quand cela nous arrive \ocadr quand le monde autour de nous semble s'effondrer
 et que nous sommes déconcertés et complètement dépassés \fcadr{}
 il est important de se rappeler cette vérité fondatrice~: Dieu est bon.

Trop souvent, au sein de notre douleur, nous pensons l'inverse.
 Nous voyons autour de nous, des gens qui sont en bonne santé,
 qui sont forts et ne font pas l'expérience de la douleur
 et nous nous plaignons~:
 \Og J'ai consacré ma vie à Dieu et j'ai toujours essayé de bien faire.
 Pourquoi est-ce que je souffre ainsi?
 Pourquoi les méchants ne souffrent-ils pas? \Fg{}
 Satan \ocadr qui nous attaque toujours quand nous sommes faibles \fcadr{}
 vient alors tout de suite se joindre à notre remise en question de Dieu.
 Il essaye d'ébranler notre foi en la bonté et la puissance de Dieu
 en murmurant~:
 \Og Si Dieu est si puissant et si bon, alors pourquoi permet-Il
 qu'une telle chose t'arrive? \Fg{}

\dvbox{
La fa\c{c}on de réduire au silence les pleurnicheries de la chair
 et les murmures de l'ennemi, c'est de vous souvenir du grand amour de Dieu.
}

La lecture de ce psaume semble indiquer que le psalmiste traversait
 une certaine expérience douloureuse ou subissait une maladie physique.
 Comme cela nous arrive par moments, il se sentait faible.
 Mais il connaissait la vérité concernant Dieu.
 Vers la fin du psaume, il déclare~:
 \Og Ma chair et mon c\oe{}ur peuvent défaillir,
 mais Dieu sera toujours le rocher de mon c\oe{}ur et ma part. \Fg{}
 (\ibibleverse{Ps}(73:26)).

Souvenez-vous que Dieu contrôle votre vie.
 Souvenez-vous qu'Il fait concourir toutes choses à votre bien. 

\dvrule

\dvprayer{
Père, puissions-nous recevoir Ta vérité et puissions-nous marcher
 à la lumière de Ta Parole qui nous amène à la connaissance de Jésus-Christ,
 notre Seigneur.\\
 Apprend-nous à rejeter les mensonges de Satan et à nous attacher à Ta vérité. 
}{\Amen}


%%%%%%%%%%%%
% 28 avril
%%%%%%%%%%%%

\dvday{Unis mon c\oe{}ur}

\dvquote{
Enseigne-moi Ton chemin, ô Seigneur, je marcherai dans Ta vérité;\\
 unis mon c\oe{}ur pour craindre Ton nom.
}{\ibibleverse{Ps}(86:11) \KJF}

\lettrine{Q}{uand} on a demandé à Jésus~:
 \Og Quel est le plus grand de tous les commandements ? \Fg{},
 Il a répondu~: 
 \Og Tu aimeras le Seigneur, ton Dieu, de tout ton c\oe{}ur,
 de toute ton âme et de toute ta pensée \Fg{} (\ibibleverse{Mt}(22:37)). 

\`A quoi vous êtes-vous consacrés cette semaine passée?
 Quelles qu'elles soient, les activités qui ont consommé votre temps,
 vos pensées, et votre énergie sont les choses auxquelles
 vous êtes consacrés. Et si vous êtes consacrés à quelque chose
 d'autre que Dieu, vous avez un c\oe{}ur divisé. 

\dvbox{
Une consécration totale, un amour total
 \ocadr voilà ce que Dieu demande.
}

Jésus a dit à l'église d'Éphèse~:
 \Og J'ai contre toi que tu as abandonné ton premier amour \Fg{}
 (\ibibleverse{Ap}(2:4)).
 Leurs c\oe{}urs étaient divisés. Jésus leur a donné le remède quand
 Il a déclaré dans le verset suivant~:
 \Og Souviens-toi donc d'où tu es tombé. \Fg{}
 Souvenez-vous des jours où vous avez découvert pour la première fois
 le glorieux amour de Dieu en Jésus-Christ.
 Jésus a continué en disant~: \Og Repens-toi! \Fg{} Faites demi-tour!
 Demandez à Dieu de vous pardonner pour la froideur que vous avez laissée
 s'installer dans la relation. Puis Jésus a dit~:
 \Og et pratique tes premières \oe{}uvres. \Fg{}
 Retournez aux choses que vous faisiez quand vous étiez tellement épris de Lui. 

Rappelez-vous, Repentez-vous, Répétez. 

Et quand vous revenez dans une pleine communion avec Lui,
 suivez l'exemple de David~:
 \Og Je te célébrerai de tout mon c\oe{}ur, Seigneur, mon Dieu ! \Fg{}
 (\ibibleverse{Ps}(86:12)).

Consécration totale, absolue. Dieu ne mérite rien de moins. 

\dvrule

\dvprayer{
Père, sonde nos c\oe{}urs et montre-nous ces domaines d'attraction
 qui nous ont peut-être détournés de la consécration totale envers Toi qui,
 un jour, était la nôtre.\\
 Renouvelle le feu et la passion en nous.
}{\DlNdJ}


%%%%%%%%%%%%
% 29 avril
%%%%%%%%%%%%

\dvday{Le Dieu patient}

\dvquote{
Celui qui a formé l'\oe{}il ne verrait-Il pas? \dots{}\\
 Ne réprimanderait-Il pas, Lui qui enseigne la connaissance aux humains?
}{\ibibleverse{Ps}(94:9-10)}

\lettrine{Q}{uelquefois} les gens s'imaginent que Dieu ne voit pas
 ou ne sait pas quand ils pèchent.
 Cependant ce n'est pas l'inaptitude qui l'empêche de juger.
 Ce n'est pas non plus un manque d'intérêt.
 Et ce n'est pas dû à de la faiblesse.

\dvbox{
C'est l'amour qui fait que Dieu est si patient et si endurant
 avec les méchants, leur donnant maintes et maintes occasions
 de faire demi-tour, de se repentir, de changer.
}

Mais les méchants méprennent Sa patience pour de la faiblesse.
 Ils blasphèment contre Lui~:
 \Og Dieu sait-Il ? A-t-Il vu ? Entend-Il vraiment ? \Fg{}
 Ils supposent que la réponse est~: \Og Non \Fg{},
 parce que pensent-ils, sinon et à coup sûr, Il serait intervenu.

Comme nous sommes ignorants \ocadr et ce, délibérément \fcadr{}
 de l'histoire de l'humanité!
 Dieu a permis aux gens de Sodome d'aller très loin dans leur rébellion,
 si loin, que les homosexuels ont paradé dans les rues et sont devenus
 physiquement agressifs.
 Dieu a permis aux gens de l'époque de Noé d'aller si loin qu'ils ont
 rejeté toute retenue morale et que chaque homme faisait ce que bon
 lui semblait.
 Mais Dieu a finalement jugé cette époque et Il a détruit ces gens.

Combien de temps Dieu peut-Il laisser notre nation Lui tourner le dos?
 Jusqu'où Dieu va-t-Il nous laisser aller?
 Il a été beaucoup plus patient que je ne l'aurais jamais imaginé.

Mais ne méprenez pas la patience de Dieu pour de la faiblesse,
 de la tolérance ou de l'approbation.
 Car le jour du juste jugement de Dieu va venir.

\dvrule

\dvprayer{
Père, comme nous avons désespérément besoin de Toi.\\
 Notre monde est pervers et rebelle.
 Nous avons besoin que Tu interviennes et que Tu empêches
 les hommes de se détruire.\\
 Que Ton Règne vienne et que Ta volonté soit faite
 sur la terre comme au ciel. 
}{\Amen}

 \typo{désespérément}


%%%%%%%%%%%%
% 30 avril
%%%%%%%%%%%%

\dvday{Écoutez}

\dvquote{
Aujourd'hui, si vous entendez sa voix,\\
 N'endurcissez pas votre c\oe{}ur\dots{}
}{\ibibleverse{Ps}(95:7-8)}

\lettrine{J}{e suis tombé} dans le péché suite à un faux-pas
 plus d'une fois, mais à chaque fois \ocadr je dois le confesser \fcadr{}
 Dieu m'a parlé et m'a averti avant que je ne chute.
 Le problème, c'est que je n'ai pas tenu compte de ces avertissements.
 Je n'ai pas écouté la voix de Dieu. 

Quelquefois, quand la voix de Dieu se fait entendre et qu'Il nous donne
 un avertissement sur quelque chose que nous sommes sur le point de faire,
 nous disons~:
 \Og Merci Seigneur, mais je n'ai pas besoin de Ton aide dans cette situation.
 Je sais ce que je fais. \Fg{}
 La vérité, c'est que nous ne savons pas ce que nous faisons.
 Et nous avons vraiment besoin de Son aide. 

S'adressant aux sept églises des chapitres \ibiblechvs{Ap}(2:)
 et \ibiblechvs{Ap}(3:) de l'Apocalypse,
 Jésus a adressé le même avertissement à chacune d'entre elles~:
 \Og Que celui qui a des oreilles écoute
 ce que l'Esprit dit aux Églises. \Fg{}

\dvbox{
L'opposé de l'obéissance c'est l'endurcissement. 
}

Si vous ne tenez pas compte de la voix de Dieu,
 vous allez finir par vous endurcir et ne plus pouvoir entendre Dieu.

Quelquefois, nous endurcissons nos c\oe{}urs en réponse à la voix de Dieu
 parce que nous ne comprenons pas les circonstances douloureuses ou difficiles
 dans lesquelles nous nous trouvons.
 Mais il y a une autre option. Au lieu de nous aigrir,
 au lieu de passer des nuits sans sommeil à nous faire du souci
 pour nos problèmes, nous pouvons nous rappeler que dans notre douleur Dieu
 agit pour nous purifier et nous conformer à l'image de Jésus.
 Nous pouvons lui remettre ces problèmes et croire qu'Il va mettre
 en \oe{}uvre Son plan parfait. 

Un choix plaît à Dieu, celui de L'écouter et de Lui faire confiance. 

Mais ne méprenez pas la patience de Dieu pour de la faiblesse,
 de la tolérance ou de l'approbation.
 Car le jour du juste jugement de Dieu va venir.

\dvrule

\dvprayer{
Père, aide nous à être toujours attentifs à Ta voix.\\
 Quand les difficultés surviennent,
 rappelle-nous que Tu contrôles toutes choses,
 et que Tu es capable d'agir dans notre situation pour accomplir Ta volonté.\\
 Nous voulons te plaire, Père.
}{\DlNdJ}


