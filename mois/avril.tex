\dvmonth{Avril}

%%%%%%%%%%%%
% 1er avril
%%%%%%%%%%%%

\dvday{Le Secret du Succès}

\dvquote{
Le Roi Ozias s'appliqua à rechercher Dieu du vivant de Zacharie,
 qui avait l'intelligence des visions de Dieu~;
 et tant qu'il rechercha l'Éternel, Dieu lui donna du succès.
}{\bibleverse{IICh}(26:5)}

\dvlettrine{L}{e secret} d'une vie réussie est dans la recherche du Seigneur.
 Jésus a dit \og Cherchez premièrement le royaume de Dieu et sa justice,
 et tout cela vous sera donné par-dessus \fg{} (\bibleverse{Matt}(6:33)).
 Autrement dit, prenez soin de votre relation avec Dieu et Il prendra soin
 de tout le reste. 

Le Roi Ozias se focalisait sur sa relation avec Dieu,
 et \og il recherchait Dieu du vivant de Zacharie qui avait l'intelligence
 des visions de Dieu \fg.
 Ceci tendrait à indiquer que Zacharie lui faisait office de modèle spirituel.
 Et en fait, tant que Zacharie a vécu, Ozias a recherché le Seigneur.
 Et tant qu'il a recherché le Seigneur, Dieu l'a fait prospérer.

\dvbox{
Dès l'instant où je cesse de rechercher Dieu,
 je suis aussi faible que les autres hommes.
}

Néanmoins, la prospérité présente souvent des dangers.
 Quelquefois, vous vous mettez à croire que votre succès est dû à votre propre
 talent et à votre génie, et vous laissez Dieu en dehors de votre vie.
 Quand Ozias a considéré la puissance et la richesse de son royaume,
 son coeur s'est enorgueilli.
 Nous lisons au verset 16 : \og Mais sa puissance le rendit orgueilleux,
 ce qui causa sa perte \fg{} (Version de \emph{la Bible en Français Courant}).

Je dois toujours me rappeler que le secret de ma force réside
 dans ma relation avec Dieu. 

\dvrule

\dvprayer{
Seigneur, aide-nous à ne pas Te combattre.
 Aide-nous à être sensible et ouvert à Ton Esprit. 
}{Dans le Nom de Jésus, Amen.}


%%%%%%%%%%%%
% 2 avril
%%%%%%%%%%%%

\dvday{Dernier Appel}

\dvquote{
Les coureurs allèrent avec les lettres du roi et de ses ministres
 dans tout Israël et Juda.
 D'après l'ordre du roi ils dirent~:
 \og Fils d'Israël, revenez à l'Éternel,
 le Dieu d'Abraham, d'Isaac et d'Israël, afin qu'Il revienne à vous,
 reste échappé de la main des rois d'Assyrie. \fg{}
}{\bibleverse{IICh}(30:6)}


\dvlettrine{Q}{uand} Ézéchias devint roi, il chercha à ramener le peuple
 à un renouveau spirituel, à une reconnaissance que Jéhovah était le Dieu
 qui régnait sur la nation.
 Il envoya donc une annonce dans tout le pays pour inviter le peuple
 à revenir au Seigneur et à célébrer la fête de la Pâque.

Les Assyriens avaient conquis le royaume du nord et emmené des captifs,
 mais beaucoup de gens avaient réussi à ne pas se faire capturer.
 L'invitation d'Ézéchias alla vers ceux qui restaient.
 Ils avaient abandonné Dieu, mais Celui-ci les rappelait.
 Il se trouve que ce fut en fait le dernier appel de Dieu en direction d'Israël.
 Le royaume du nord s'est moqué de l'invitation et en l'espace de trois ans,
 ils avaient disparus. 

\dvbox{
Il est effrayant de se rendre compte que si un homme persiste
 dans une voie contre laquelle Dieu l'a mis en garde,
 le jour viendra où Dieu lui adressera un appel final. 
}

Juda a toutefois répondu à l'invitation. Sous le règne d'Ézéchias,
 Juda est redevenu puissant et prospère parce que les gens s'étaient
 tournés vers le Seigneur d'un seul coeur.

Notre Dieu est plein de miséricorde et de grâce.
 Si vous vous tournez vers Lui, Il vous libèrera de la captivité
 de l'ennemi et de la puissance des ténèbres qui vous retient
 dans son étreinte. 

\dvrule

\dvprayer{
Père, réveille-nous pour que nous comprenions le sérieux de l'époque
 et de l'âge dans lesquels nous vivons.
 Puissions-nous conclure d'un seul cœur une alliance par laquelle nous
 nous engageons à Te servir et à T'adorer,
 et à nous soumettre à la Seigneurie de Jésus-Christ. 
}{Amen.}



%%%%%%%%%%%%
% 3 avril
%%%%%%%%%%%%

\dvday{Attention Recentrée}

\dvquote{
Fortifiez-vous et prenez courage! Soyez sans crainte et sans effroi
 devant le roi d'Assyrie et devant toute la multitude qui est avec lui~;
 car avec vous il y a plus qu'avec lui~: avec lui il y a un bras de chair,
 et avec nous l'Éternel, notre Dieu, qui nous aidera
 et qui soutiendra nos combats.
 Le peuple s'appuya sur les paroles d'Ézéchias, roi de Juda.
}{\bibleverse{IICh}(32:7-8)}


\dvlettrine{L}{es Assyriens} se mirent en route pour venir attaquer Jérusalem.
 Ézéchias fit preuve de sagesse en faisant des préparatifs.
 Il ordonna au peuple de renforcer les fortifications de la cité,
 de boucher les sources et les puits qui se trouvaient en dehors des murs
 et de stocker des lances et des boucliers.
 Mais à cause de la réputation de brutalité des Assyriens,
 les gens étaient découragés et terrifiés.
 Ézéchias les réunit donc pour les encourager.

Au lieu de simplement leur dire de ne pas avoir peur,
 il leur donna une raison de ne pas avoir peur.
 Il écarta leur attention de l'ennemi et la recentra sur le Seigneur.
 Il leur rappela que Dieu était de leur côté,
 prêt à se charger de leurs batailles.

\dvbox{
Dieu va mener nos combats.
}

Quand des situations impossibles surviennent,
 nous devons nous rappeler que Dieu peut faire infiniment au-delà
 de tout ce que nous demandons ou pensons.
 Ces batailles ne sont pas contre nous --~elles sont contre Dieu.
 Et personne ne peut infliger de défaite à notre Dieu.

Si un problème vous cause du souci voire du stress,
 c'est que vous n'êtes pas encore arrivés à faire réellement
 confiance à Dieu.
 Dès l'instant où vous remettez toute la situation entre les mains de Dieu,
 les soucis s'évanouiront~; la peur et l'anxiété disparaîtront.
 Reposez-vous en Dieu. Tout repose entre Ses mains. 

\dvrule

\dvprayer{
Père, nous sommes si reconnaissants d'entrer dans la bataille en vainqueurs.
 Aide-nous, Seigneur, à fixer les yeux sur Toi afin de pouvoir
 être forts et courageux. 
}{Dans le Nom de Jésus, Amen.}


%%%%%%%%%%%%
% 4 avril
%%%%%%%%%%%%

\dvday{Pas de Remède}

\dvquote{
L'Éternel, le Dieu de leurs pères, leur avait envoyé de bonne heure
 des avertissements par l'intermédiaire de Ses messagers,
 car Il voulait épargner Son peuple et Sa propre demeure.
 Mais ils se moquaient des messagers de Dieu,
 ils méprisaient Ses paroles et se raillaient de Ses prophètes,
 jusqu'à ce que la fureur de l'Éternel contre Son peuple
 monte et soit sans remède.
}{\bibleverse{IICh}(36:15-16)}


\dvlettrine{C}{'est} une chose terrible que d'entendre Dieu déclarer
 qu'il n'y a plus de traitement, plus de remède.
 Tel était le cas tragique de Juda.
 Ils avaient adoré d'autres dieux, avaient refusé d'écouter
 la voix du Seigneur, et s'étaient détounés de Dieu
 --~jusqu'à ce que finalement, Il se détourne d'eux
 et les laisse se faire battre par leurs ennemis. 

\dvbox{
Si vous refusez le remède de Dieu, il n'y a pas d'autre traitement.
}

Dans Sa patience et Sa compassion, Dieu avait envoyé beaucoup de prophètes
 pour mettre Juda en garde.
 Jérémie était l'un de ses prophètes.
 Quand il annonça le message que Dieu avait résolu de livrer Juda
 aux mains des Babyloniens,
 et qu'il vaudrait mieux pour eux qu'ils se soumettent aux Babyloniens,
 le Roi Sédécias l'emprisonna pour haute trahison.
 Le remède de Dieu avait été refusé, suscitant ainsi la colère de Dieu.

Chaque fois qu'il y a un problème, Dieu a toujours un remède.
 Pour le problème du péché, Dieu a prescrit un remède, un traitement.
 Il se trouve dans le sacrifice de Jésus-Christ, l'Agneau de Dieu.
 La Bible nous dit que le sang de Christ nous lave de tous nos péchés.
 Mais soyez prévenus~: si vous refusez le remède de Dieu,
 il n'y a pas d'autre traitement. 

\dvrule

\dvprayer{
Père, nous sommes reconnaissants de ce que Tu as été si patient avec nous,
 nous donnant maintes et maintes opportunités de nous détourner du monde
 pour vivre en Te suivant.
 Seigneur, aide-nous à vivre devant Toi d'une façon qui Te plaise. 
}{Dans le Nom de Jésus, Amen.}



%%%%%%%%%%%%
% 5 avril
%%%%%%%%%%%%

\dvday{Entretenus par le Roi}

\dvquote{
Or, comme nous mangeons le sel du palais
 \NdT{expression équivalente à
 \og nous sommes entretenus par le palais \fg{}.}
 et qu'il ne nous paraît pas convenable de voir mépriser le roi,
 nous envoyons au roi ces informations.
}{\bibleverse{Esd}(4:14)}


\dvlettrine{L}{es voisins} d'Israël ont envoyé cette lettre à Artaxerxès
 pendant la reconstruction du temple.
 Quand leurs offres d'aide ont été rejetées,
 ils ont cherché à entraver le travail.
 Ils ont embauché des hommes de loi pour faire échouer le projet
 et ils ont adressé un document au roi de Perse.
 Ils disaient~: \og Nous croyons que ces gens vont vous déshonorer.
 Nous ne pouvions pas simplement voir ceci arriver sans intervenir~;
 nous avons donc estimé nécessaire d'informer le roi. \fg{}

Ce verset s'applique aujourd'hui à nous. Car nous avons également un Roi,
 et nous sommes entretenus par Lui.

Certains disent que Jésus est simplement une béquille pour les gens faibles.
 Ils ont raison. Je m'appuie sur Lui tout le temps.
 S'Il n'était pas pour moi une béquille,
 cela fait longtemps que je me serais effondré.
 Il ne m'a jamais abandonné ni laissé tomber.
 Vous n'avez pas à vous faire de soucis lorsque vous êtes entretenus par le Roi.

Mais le monde qui m'entoure déshonore mon Roi.
 Chaque fois que quelqu'un utilise le nom de mon Dieu en vain,
 mon cœur frémit d'indignation. 

\dvbox{
Je n'aime pas voir mon roi déshonoré. 
}

Tout comme les voisins d'Israël ont envoyé une lettre informant
 le roi de ce qui se passait, nous devons, nous aussi, informer le nôtre.
 Nous devons nous mettre à genoux devant Dieu pour Lui parler de la condition
 de notre nation et des choses qui se passent --~et prier pour Lui demander
 qu'Il commence à agir. 

\dvrule

\dvprayer{
Père, aide-nous aujourd'hui à amener gloire et honneur à Toi, notre Roi.
 Donne-nous l'audace de prendre la défense de ce qui est juste, saint et pur. 
}{Dans le Nom de Jésus, Amen.}


%%%%%%%%%%%%
% 6 avril
%%%%%%%%%%%%

\dvday{À Court de Mots}

\dvquote{
Maintenant, que dirons-nous après cela, ô notre Dieu ?
 Car nous avons abandonné Tes commandements.
}{\bibleverse{Esd}(9:10)}


\dvlettrine{C}{inquante-sept} ans seulement après que Dieu
 --~par l'intermédiaire de Cyrus~-- les ait libérés de leur captivité
 pour qu'ils puissent retourner à Jérusalem,
 reconstruire le temple et se remettre à L'adorer,
 les gens d'Israël étaient déjà retournés aux choses précises
 pour lesquelles Dieu les avait jugés plus tôt.
 Quand Esdras apprit ces nouvelles déconcertantes, son coeur se serra.
 Le peuple ne s'était pas séparé des peuples qui l'entouraient,
 mais s'était mis, au contraire, à les suivre dans leurs abominations.

Esdras ne pouvait pas croire que le peuple puisse oublier Dieu si rapidement.
 Choqué, et à court de mots, il pria~: \og Seigneur, je n'ai aucune excuse.
 Que pourrais-je dire? \fg{}

Bien qu'Esdras ne puisse plus trouver de mots pour parler au Seigneur,
 il avait un bon nombre de mots à dire au peuple --~des mots de mise en garde
 et de correction, par lesquels il appelait à une réforme très stricte
 et radicale.

Il appelait à une séparation totale du peuple de Dieu de ces choses du monde
 qui risquaient de polluer, diluer ou souiller leur témoignage.
 En tant que chrétiens, nous devons faire de même. 

\dvbox{
Nous devons prendre quelques décisions radicales concernant
 notre séparation du monde et de sa pollution. 
}


L'ennemi a ouvertement déclaré la guerre contre Dieu.
 Allons-nous rester à ne rien faire et laisser blasphémer notre Dieu?
 Ou allons-nous nous séparer du monde, nous lever et prendre parti pour Jésus? 

\dvrule

\dvprayer{
Par notre silence, Seigneur, nous avons encouragé l'ennemi.
 Aide-nous à tenir ferme par Ta force dans cette guerre totale
 que l'ennemi a déclaré contre Toi.
 Seigneur, sois notre force, notre forteresse, sois notre aide. 
}{Dans le Nom de Jésus, Amen.}


%%%%%%%%%%%%
% 7 avril
%%%%%%%%%%%%

\dvday{Prière Pressante}

\dvquote{
Le roi me dit~: \og Au fait, que demandes-tu donc ? \fg{}
Je priai le Dieu des cieux.
}{\bibleverse{Neh}(2:4)}

\dvlettrine{D}{epuis} plusieurs mois, Néhémie avait prié
 pour une opportunité d'être utile à la reconstruction de Jérusalem.

Quand le roi remarqua que Néhémie n'était pas dans sa bonne humeur habituelle,
 il lui demanda~: \og Pourquoi es-tu aussi triste? \fg{}
 Néhémie décrivit la condition de Jérusalem~; les murs sont en ruine,
 les portes ont été brûlées et les gens sont démoralisés.
 \og Comment pourrais-je être heureux après avoir reçu de telles nouvelles? \fg{}

Le roi lui demanda donc~: \og Qu'attends-tu de moi? \fg{}
 C'était l'occasion que Néhémie avait demandée.
 Conscient de l'importance capitale du moment,
 il offrit une courte prière au Dieu du ciel.
 Le roi ne remarqua sans doute pas la légère pause, mais Dieu la nota.

\dvbox{
Ce n'est pas la durée de nos prières qui les rend efficaces
 --~c'est la relation que nous avons avec Dieu qui compte.
}

\og La prière agissante du juste a une grande efficacité \fg{}
 (\bibleverse{James}(5:16)).
 La prière de Néhémie a dû être courte et muette, mais elle a été efficace.
 Dieu a entendu et a répondu.
 Quel réconfort de se rendre compte qu'Il entend nos prières faites en silence.

Si vous n'avez pas de relation avec Dieu, il n'y a qu'une seule prière
 qui l'intéresse, à savoir~:
 \og Ô Dieu, aie pitié de moi, qui suis un pécheur. \fg{}
 Mais une fois que vous avez prié cette prière, alors en tant qu'enfant de Dieu,
 vous pouvez offrir à Dieu ces prières rapides,
 chaque fois que vous êtes dans le besoin.
 Si cette juste relation existe entre vous et Lui,
 ces petites prières peuvent marcher de façon formidable. 

\dvrule

\dvprayer{
Père, nous Te remercions de ce que Tu réponds aux prières
 -- même à ces prières rapides qui s'élèvent spontanément
 par cause de faiblesse ou d'agitation. 
}{Dans le Nom de Jésus, Amen.}



%%%%%%%%%%%%
% 8 avril
%%%%%%%%%%%%

\dvday{Quand l'Opposition Apparaît}

\dvquote{
Lorsque Sanballat apprit que nous rebâtissions la muraille,
 il fut en colère et très mécontent. Il se moqua des Juifs.
}{\bibleverse{Neh}(3:33-34)}

\dvlettrine{À}{ peine} Néhémie était-il retourné à Jérusalem
 et avait-il commencé à rebâtir la muraille que de l'opposition
 s'éleva contre lui.
 Tentant de stopper le travail, Sanballat et Tobiya ridiculisèrent Néhémie.
 Mais la réponse de Néhémie à leurs moqueries fut la prière.

Voyant que la moquerie ne marchait pas, leur prochaine manoeuvre
 fut de préparer des attaques secrètes contre ceux qui construisaient
 la muraille.
 Néhémie donna donc aux ouvriers l'ordre de se protéger d'eux en instaurant
 une surveillance continue.

La première mesure est toujours la prière -- et puis nous passons à l'action.
 La prière ne doit jamais remplacer la mise en place de mesures pratiques.
 Vous pourriez dire~:
 \og Bien, j'ai prié et j'ai simplement la foi de croire que le Seigneur
 va régler la situation. \fg{}
 Mais la Bible nous dit que la foi sans les œuvres est morte.

\dvbox{
Pour que la foi soit valide, il faut qu'elle conduise à l'action.
}

Paul a prévenu Timothée que ceux qui vivent pieusement en Christ Jésus
 subiront des persécutions.
 Chaque fois que vous prenez position pour Dieu,
 ou que vous essayez de faire le travail de Dieu,
 de l'opposition va s'élever et essayer de vous mettre en échec.
 Quand cela arrive, nous devons prier et demander à Dieu de nous fixer
 un plan d'action, puis nous devons prendre des mesures pour rebâtir
 nos murs et installer des sentinelles pour les surveiller.

C'est maintenant le moment de se réveiller et de se préparer. 


\dvrule

\dvprayer{
Père, nous nous rendons compte que les murailles ont été démolies.
 Aide-nous à avoir le courage de tenir bon contre le ridicule et les moqueries.
 Et, Seigneur, puissions résoudre en nos coeurs de rebâtir
 ces murailles de défense. 
}{Dans le Nom de Jésus, Amen.}


%%%%%%%%%%%%
% 9 avril
%%%%%%%%%%%%

\dvday{La Nature de Dieu}

\dvquote{
Mais Toi, Tu es un Dieu qui pardonne, qui est compatissant et qui fait grâce,
 lent à la colère et riche en bienveillance, et tu ne les as pas abandonnés.
}{\bibleverse{Neh}(9:17)}

\dvlettrine{D}{ans} ce court verset, Néhémie décrit la nature de Dieu
 en faisant la liste de quelques-uns de ses attributs.
 Le premier qu'il mentionne est la volonté de Dieu de pardonner.
 Bien que Son peuple Lui ait été complètement infidèle,
 Dieu est resté complètement fidèle à Son peuple.
 Nous sommes tous privés de Sa gloire,
 mais le sang de Jésus est capable de nous purifier de tout péché.

\dvbox{
Tout comme Dieu était prêt a pardonné aux israélites,
 Il est prêt à vous pardonner -- quoi que vous ayez fait.
}

Dieu est plein de grâce.
 Il accorde Ses bénédictions à ceux qui ne les méritent pas.

Dieu est plein de miséricorde.
 Comme David (qui a vraiment eu besoin de la miséricorde de Dieu)
 le faisait remarquer, Dieu ne nous rétribue pas selon nos péchés.
 Non, au contraire, Il offre Sa miséricorde à ceux qui le craignent --
 une miséricorde qui est aussi haute que les cieux sont hauts
 au-dessus de la Terre.

Dieu est lent à la colère. Quelquefois, ça me dérange un peu.
 Je ressemble beaucoup à Jacques et Jean qui voulaient appeler
 le feu du ciel à tomber et à consumer les samaritains qui insultaient Jésus.
 Mais Dieu est lent à la colère. Il est endurant et patient.

Dieu est bienveillant. Il a été bienveillant envers Israël.
 Il a été bienveillant envers notre pays,
 et il a certainement été bienveillant envers moi.

Dieu voit nos échecs, nos fautes, nos faiblesses.
 Et pourtant, Il se tient toujours prêt à nous pardonner.
 Quand nous L'invoquons, Il nous pardonne,
 Il nous lave et nous renouvelle. C'est simplement Sa nature.

\dvrule

\dvprayer{
Père, nous ne remercions pour Ta miséricorde.
 Puissions-nous abandonner nos mauvaises voies
 et recevoir Ton pardon et Ta purification.
}{Amen.}


%%%%%%%%%%%%
% 10 avril
%%%%%%%%%%%%

\dvday{Le Haman dans nos Vies}

\dvquote{
Haman vit que Mardochée ne s'inclinait ni ne se prosternait en son honneur,
 et Haman fut rempli de fureur.
 Il considéra avec dédain l'idée de porter la main sur le seul Mardochée~;
 on lui avait signalé, en effet, le peuple auquel appartenait Mardochée.
 Haman entreprit d'exterminer de tout le royaume d'Assuérus tous les Juifs,
 le peuple de Mardochée.
}{\bibleverse{Est}(3:5-6)}


\dvlettrine{H}{aman} était un Agaguite, un descendant des Amalécites.
 Dieu avait ordonné à Saül de totalement exterminer les Amalécites,
 mais il avait désobéi et laissé le roi vivre.
 Le résultat fut que les Amalécites continuèrent de faire la guerre à Israël.

En typologie\NdT{\emph{typologie biblique}~:
 rapprochement entre une personne ou un événement de l'Ancien Testament,
 le \og type \fg{} ou \og préfiguration \fg{},
 et de leur \og antitype \fg{} ou \og accomplissement \fg{},
 personne ou événement du Nouveau Testament.}
 biblique, Haman est un type de la chair,
 quelque chose que Dieu hait.
 Tout comme les Amalécites faisaient constamment la guerre à Israël,
 la chair fait aussi constamment la guerre à l'Esprit.
 Comme les Amalécites attaquaient Israël à son point le plus faible,
 la chair va vous attaquer à votre point le plus faible.
 De même que les Amalécites essayaient d'empêcher Israël
 d'entrer en Terre Promise, la chair va essayer de vous empêcher
 d'être victorieux et de goûter aux promesses de la vie dans l'Esprit.

Dieu jura qu'Il serait en guerre avec Amalec de génération en génération,
 mais Il a aussi promis qu'un jour Amalec périrait et
 qu'il serait complètement oublié.

\dvbox{
Un jour, la chair sera éliminée à jamais.
}

Nous avons tous un Haman dans nos vies -- ce problème particulier de la chair
 qui n'a pas encore été réglé.
 Considérons le comme mort, crucifié avec Christ,
 afin d'avoir Sa complète victoire et de marcher dans l'Esprit.

\dvrule

\dvprayer{
Père, Merci d'avoir offert la victoire sur la vieille nature
 par l'intermédiaire de Jésus-Christ.
 Aide-nous à continuer d'avancer dans l'Esprit
 afin que nous puissions connaître Ta puissance sur la chair.
}{Amen.}



%%%%%%%%%%%%
% 11 avril
%%%%%%%%%%%%

\dvday{Vaincre la Peur}

\dvquote{
Va rassembler tous les Juifs qui se trouvent à Suse. Jeûnez à mon intention,
 sans manger ni boire pendant trois jours, vingt-quatre heures sur vingt-quatre.
 Moi aussi je jeûnerai de même avec mes jeunes servantes.
 Dans ces conditions, j'irai chez le roi malgré la loi.
 Si c'est pour ma perte, je périrai !}{\bibleverse{Est}(4:16)}

\dvlettrine{À}{ la} demande d'Haman, le roi de Perse décréta qu'à un jour fixé,
 tous les juifs du royaume seraient mis à mort.
 Personne ne savait que la Reine Esther était juive.
 Son cousin l'encouragea à se présenter à son mari et à plaider pour son peuple,
 mais faire cela allait mettre sa vie en danger.

Les desseins de Dieu vont toujours être accomplis,
 indépendamment de ce que nous faisons ou ne faisons pas.
 Cependant, Il nous permet d'être des instruments par lesquels Il les accomplit.
 Esther choisit d'être cet instrument.
 Elle reconnut que Dieu l'avait conduite jusqu'à ce moment très spécial,
 et elle surmonta sa peur.
 \og Même si cela doit me coûter la vie, \fg{} dit-elle \og je le ferai.
 Si je dois périr, je périrai. \fg{}

Jésus nous a dit de ne pas nous inquiéter du lendemain,
 car le lendemain s'inquiétera de lui-même.
 Et pourtant, beaucoup de chrétiens s'inquiètent.
 Ils s'inquiètent parce qu'ils ne se sont pas encore
 abandonnés à la volonté de Dieu.

\dvbox{
L'abandon à Dieu libère de la peur et de l'anxiété.
}

Le jour où Dieu quittera le trône, nous aurons tous de gros ennuis.
 Mais, tant que Dieu est sur le trône, nous n'avons rien à craindre.
 Après tout, si nous pouvons Lui faire confiance pour notre destinée éternelle,
 ne pourrions-nous pas Lui faire confiance pour demain?

\dvrule

\dvprayer{
Seigneur, libère Ton peuple des inquiétudes pour le futur.
 Aide-nous à croire que Tu vas accomplir Ton dessein éternel
 quand nous Te remettons notre sort,
 par l'intermédiaire de Jésus-Christ, notre Seigneur.
}{Amen.}



%%%%%%%%%%%%
% 12 avril
%%%%%%%%%%%%

\dvday{Pourquoi Suis-je Né ?}

\dvquote{
Pourquoi ne suis-je pas mort dès les entrailles de ma mère?
 Pourquoi n'ai-je pas expiré au sortir de son ventre?
}{\bibleverse{Job}(3:11)}

\dvlettrine{E}{n seulement} l'espace de quelques instants,
 Job perdit ses enfants, sa santé et toutes ses possessions terrestres.
 Il se retrouva comme une âme nue, réduite à l'expression la plus simple
 de l'existence, sans aucun support ni soutien.

\dvbox{
Quelles questions pose un homme qui a été dépouillé de tout ?
}

Job voulait savoir pourquoi il était né. Il n'a pas maudit Dieu,
 mais il a maudit le jour de sa naissance.
 \og Pourquoi suis-je donc né? Quel est le sens de ma vie? \fg{}

Si vous souscrivez à la théorie de l'évolution, la réponse est évidente.
 Votre vie n'a aucun sens, car votre existence est le fruit d'un accident.
 Votre existence est le résultat de l'occurrence fortuite
 de circonstances accidentelles survenues pendant des milliards d'années.

La Bible, toutefois, enseigne que la vie existe après la mort.
 Cet endroit-ci est simplement un endroit où Dieu me prépare
 afin que je puisse vivre avec Lui pour toujours.
 Les épreuves, les difficultés, les tribulations,
 les déceptions sont toutes destinées à me montrer
 combien les choses terrestres sont temporaires
 et à m'apprendre à vivre pour ce qui est éternel et non pour le présent.

Pourquoi suis-je né? Pourquoi êtes-vous nés?
 Dieu nous a créés pour Le connaître et Lui faire confiance,
 afin de pouvoir vivre avec Lui dans la gloire de Son royaume pour toujours. 

\dvrule

\dvprayer{
Père, apprends-nous à Te faire confiance,
 sachant que nos vies sont entre Tes mains.
 Nous savons que l'ennemi ne peut pas faire plus
 que ce que Tu lui laisses faire.
 Quand nous trébuchons ou que nous échouons à une épreuve,
 c'est seulement pour que nous puissions voir
 combien nous sommes faibles afin de Te faire davantage confiance. 
}{Dans le Nom de Jésus, Amen.}



%%%%%%%%%%%%
% 13 avril
%%%%%%%%%%%%

\dvday{Qu'est-ce que l'Homme?}

\dvquote{
Qu'est-ce que l'homme, pour que tu en fasses tant de cas,
 pour que tu le prennes tellement à cœur?
}{\bibleverse{Job}(7:17)}

\dvlettrine{N}{ous} ne sommes tous qu'un infime grain de poussière
 sur un infime grain de poussière appelé la Terre,
 qui gravite autour du Soleil dans un petit coin de la galaxie
 de la Voie Lactée.
 Cependant les Écritures disent que les pensées de Dieu pour nous
 sont si nombreuses, que si nous pouvions les compter
 leur nombre dépasserait le nombre des grains de sable de la mer.

Que Dieu pense à nous est stupéfiant. Encore plus extraordinaire
 est le fait qu'Il nous exalte --~au-dessus des plantes,
 au-dessus des animaux, et même au-dessus des anges.
 Job voulait savoir pourquoi.
 \og Qu'est-ce que l'homme, pour que Tu l'exaltes? \fg{}

Ce sont les voies de Dieu.

\dvbox{
Dieu exalte la personne qui Lui voue sa vie.
}

Dieu rend une personne plus grande qu'elle ne pourrait jamais l'être
 en dehors de Son intervention.
 Dieu agit ainsi parce qu'Il nous aime.
 Il nous aime tant qu'Il a donné Son Fils unique,
 qui est devenu ce que nous étions afin qu'Il puisse
 nous faire devenir ce qu'Il est.

Et ainsi, Dieu nous développe et nous forme.
 Il permet les épreuves et les déconvenues parce qu'Il sait
 que c'est la seule façon de nous préparer à l'éternité.
 Comme Job, qui passa par une période de mise à l'épreuve
 où il perdit tout ce qu'il avait, nous passons quelquefois
 par des moments de perte et de déconvenue.
 Mais chaque perte fait partie du plan éternel de Dieu
 qui nous prépare ainsi à être avec Lui pour toujours. 

\dvrule

\dvprayer{
Père, merci pour Ta fidélité à nous façonner pour faire de nous
 la personne que Tu veux que nous soyons et enlever tout ce qui pourrait
 nous polluer ou nous détruire.
 Combien nous sommes reconnaissants que Ton cœur se soit fixé sur nous.
 Puissions-nous marcher dans cet amour. 
}{Amen.}


%%%%%%%%%%%%
% 14 avril
%%%%%%%%%%%%

\dvday{Cri du Cœur pour un Médiateur}

\dvquote{
Il n'est pas un homme comme moi, pour que je lui réponde,
 pour que nous allions ensemble en justice.
 Il n'y a pas entre nous d'arbitre, qui pose sa main sur nous deux.
}{\bibleverse{Job}(9:32-33)}

\dvlettrine[ante=\og]{S}{i seulement} tu te mettais en règle avec Dieu \fg{},
 conseillaient les amis de Job, \og tout le reste irait bien. \fg{}
 Mais Job ne savait pas comment faire.
 Il voyait la grandeur de Dieu comparée à sa propre petitesse,
 et se rendait compte que l'écart entre le Dieu infini
 et sa créature limitée est bien trop grand pour que l'homme
 puisse le franchir lui-même.
 Reconnaissant ce dilemme, Job réclama un médiateur.

Dans le Nouveau Testament, nous lisons \og Il y a un seul Dieu,
 et aussi un seul Médiateur entre Dieu, et les hommes,
 le Christ-Jésus homme \fg{} (\bibleverse{ITim}(2:5)).

\dvbox{
Jésus est le pont entre Dieu et l'homme.
}

Parce que Jésus et le Père sont un (\bibleverse{Jn}(10:30)),
 Il est capable de toucher Dieu.
 Parce que Jésus \og s'est fait chair et a habité parmi nous, \fg{}
 (\bibleverse{Jn}(1:14)), Il nous touche aussi.
 Il comprend nos faiblesses, nos peurs, nos tentations.
 Et ainsi, Jésus est capable de nous amener à Dieu.

Ce pont qui a amené l'homme à Dieu n'a pas été créé par l'homme.
 À la différence des autres religions, dans le Christianisme,
 ce n'est pas l'homme qui essaye d'atteindre Dieu
 --~c'est Dieu qui s'abaisse pour atteindre l'homme.
 Si vous voulez trouver le vrai Dieu éternel et vivant,
 vous ne pouvez le toucher que si vous Le laissez vous toucher
 par l'intermédiaire de Son Fils, Jésus. 

\dvrule

\dvprayer{
Père, nous sommes si reconnaissants de ce que Tu as offert un moyen
 par lequel nous pouvons être justifiés, non pas par nos œuvres de justice,
 mais simplement en croyant et en faisant confiance à notre Médiateur
 Jésus-Christ, Celui qui intercède pour nous. 
}{Amen.}



