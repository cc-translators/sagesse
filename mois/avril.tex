\dvmonth{Avril}

%%%%%%%%%%%%
% 1er avril
%%%%%%%%%%%%

\dvday{Le Secret du Succès}

\dvquote{
Le Roi Ozias s'appliqua à rechercher Dieu du vivant de Zacharie,
 qui avait l'intelligence des visions de Dieu~;
 et tant qu'il rechercha l'Éternel, Dieu lui donna du succès.
}{\bibleverse{IICh}(26:5)}

\dvlettrine{L}{e secret} d'une vie réussie est dans la recherche du Seigneur.
 Jésus a dit \og Cherchez premièrement le royaume de Dieu et sa justice,
 et tout cela vous sera donné par-dessus \fg{} (\bibleverse{Matt}(6:33)).
 Autrement dit, prenez soin de votre relation avec Dieu et Il prendra soin
 de tout le reste. 

Le Roi Ozias se focalisait sur sa relation avec Dieu,
 et \og il recherchait Dieu du vivant de Zacharie qui avait l'intelligence
 des visions de Dieu \fg.
 Ceci tendrait à indiquer que Zacharie lui faisait office de modèle spirituel.
 Et en fait, tant que Zacharie a vécu, Ozias a recherché le Seigneur.
 Et tant qu'il a recherché le Seigneur, Dieu l'a fait prospérer.

\dvbox{
Dès l'instant où je cesse de rechercher Dieu,
 je suis aussi faible que les autres hommes.
}

Néanmoins, la prospérité présente souvent des dangers.
 Quelquefois, vous vous mettez à croire que votre succès est dû à votre propre
 talent et à votre génie, et vous laissez Dieu en dehors de votre vie.
 Quand Ozias a considéré la puissance et la richesse de son royaume,
 son coeur s'est enorgueilli.
 Nous lisons au verset 16 : \og Mais sa puissance le rendit orgueilleux,
 ce qui causa sa perte \fg{} (Version de \emph{la Bible en Français Courant}).

Je dois toujours me rappeler que le secret de ma force réside
 dans ma relation avec Dieu. 

\dvrule

\dvprayer{
Seigneur, aide-nous à ne pas Te combattre.
 Aide-nous à être sensible et ouvert à Ton Esprit. 
}{Dans le Nom de Jésus, Amen.}


%%%%%%%%%%%%
% 2 avril
%%%%%%%%%%%%

\dvday{Dernier Appel}

\dvquote{
Les coureurs allèrent avec les lettres du roi et de ses ministres
 dans tout Israël et Juda.
 D'après l'ordre du roi ils dirent~:
 \og Fils d'Israël, revenez à l'Éternel,
 le Dieu d'Abraham, d'Isaac et d'Israël, afin qu'Il revienne à vous,
 reste échappé de la main des rois d'Assyrie. \fg{}
}{\bibleverse{IICh}(30:6)}


\dvlettrine{Q}{uand} Ézéchias devint roi, il chercha à ramener le peuple
 à un renouveau spirituel, à une reconnaissance que Jéhovah était le Dieu
 qui régnait sur la nation.
 Il envoya donc une annonce dans tout le pays pour inviter le peuple
 à revenir au Seigneur et à célébrer la fête de la Pâque.

Les Assyriens avaient conquis le royaume du nord et emmené des captifs,
 mais beaucoup de gens avaient réussi à ne pas se faire capturer.
 L'invitation d'Ézéchias alla vers ceux qui restaient.
 Ils avaient abandonné Dieu, mais Celui-ci les rappelait.
 Il se trouve que ce fut en fait le dernier appel de Dieu en direction d'Israël.
 Le royaume du nord s'est moqué de l'invitation et en l'espace de trois ans,
 ils avaient disparus. 

\dvbox{
Il est effrayant de se rendre compte que si un homme persiste
 dans une voie contre laquelle Dieu l'a mis en garde,
 le jour viendra où Dieu lui adressera un appel final. 
}

Juda a toutefois répondu à l'invitation. Sous le règne d'Ézéchias,
 Juda est redevenu puissant et prospère parce que les gens s'étaient
 tournés vers le Seigneur d'un seul coeur.

Notre Dieu est plein de miséricorde et de grâce.
 Si vous vous tournez vers Lui, Il vous libèrera de la captivité
 de l'ennemi et de la puissance des ténèbres qui vous retient
 dans son étreinte. 

\dvrule

\dvprayer{
Père, réveille-nous pour que nous comprenions le sérieux de l'époque
 et de l'âge dans lesquels nous vivons.
 Puissions-nous conclure d'un seul cœur une alliance par laquelle nous
 nous engageons à Te servir et à T'adorer,
 et à nous soumettre à la Seigneurie de Jésus-Christ. 
}{Amen.}



%%%%%%%%%%%%
% 3 avril
%%%%%%%%%%%%

\dvday{Attention Recentrée}

\dvquote{
Fortifiez-vous et prenez courage! Soyez sans crainte et sans effroi
 devant le roi d'Assyrie et devant toute la multitude qui est avec lui~;
 car avec vous il y a plus qu'avec lui~: avec lui il y a un bras de chair,
 et avec nous l'Éternel, notre Dieu, qui nous aidera
 et qui soutiendra nos combats.
 Le peuple s'appuya sur les paroles d'Ézéchias, roi de Juda.
}{\bibleverse{IICh}(32:7-8)}


\dvlettrine{L}{es Assyriens} se mirent en route pour venir attaquer Jérusalem.
 Ézéchias fit preuve de sagesse en faisant des préparatifs.
 Il ordonna au peuple de renforcer les fortifications de la cité,
 de boucher les sources et les puits qui se trouvaient en dehors des murs
 et de stocker des lances et des boucliers.
 Mais à cause de la réputation de brutalité des Assyriens,
 les gens étaient découragés et terrifiés.
 Ézéchias les réunit donc pour les encourager.

Au lieu de simplement leur dire de ne pas avoir peur,
 il leur donna une raison de ne pas avoir peur.
 Il écarta leur attention de l'ennemi et la recentra sur le Seigneur.
 Il leur rappela que Dieu était de leur côté,
 prêt à se charger de leurs batailles.

\dvbox{
Dieu va mener nos combats.
}

Quand des situations impossibles surviennent,
 nous devons nous rappeler que Dieu peut faire infiniment au-delà
 de tout ce que nous demandons ou pensons.
 Ces batailles ne sont pas contre nous -- elles sont contre Dieu.
 Et personne ne peut infliger de défaite à notre Dieu.

Si un problème vous cause du souci voire du stress,
 c'est que vous n'êtes pas encore arrivés à faire réellement
 confiance à Dieu.
 Dès l'instant où vous remettez toute la situation entre les mains de Dieu,
 les soucis s'évanouiront~; la peur et l'anxiété disparaîtront.
 Reposez-vous en Dieu. Tout repose entre Ses mains. 

\dvrule

\dvprayer{
Père, nous sommes si reconnaissants d'entrer dans la bataille en vainqueurs.
 Aide-nous, Seigneur, à fixer les yeux sur Toi afin de pouvoir
 être forts et courageux. 
}{Dans le Nom de Jésus, Amen.}


%%%%%%%%%%%%
% 4 avril
%%%%%%%%%%%%

\dvday{Pas de Remède}

\dvquote{
L'Éternel, le Dieu de leurs pères, leur avait envoyé de bonne heure
 des avertissements par l'intermédiaire de Ses messagers,
 car Il voulait épargner Son peuple et Sa propre demeure.
 Mais ils se moquaient des messagers de Dieu,
 ils méprisaient Ses paroles et se raillaient de Ses prophètes,
 jusqu'à ce que la fureur de l'Éternel contre Son peuple
 monte et soit sans remède.
}{\bibleverse{IICh}(36:15-16)}


\dvlettrine{C}{'est} une chose terrible que d'entendre Dieu déclarer
 qu'il n'y a plus de traitement, plus de remède.
 Tel était le cas tragique de Juda.
 Ils avaient adoré d'autres dieux, avaient refusé d'écouter
 la voix du Seigneur, et s'étaient détounés de Dieu
 -- jusqu'à ce que finalement, Il se détourne d'eux
 et les laisse se faire battre par leurs ennemis. 

\dvbox{
Si vous refusez le remède de Dieu, il n'y a pas d'autre traitement.
}

Dans Sa patience et Sa compassion, Dieu avait envoyé beaucoup de prophètes
 pour mettre Juda en garde.
 Jérémie était l'un de ses prophètes.
 Quand il annonça le message que Dieu avait résolu de livrer Juda
 aux mains des Babyloniens,
 et qu'il vaudrait mieux pour eux qu'ils se soumettent aux Babyloniens,
 le Roi Sédécias l'emprisonna pour haute trahison.
 Le remède de Dieu avait été refusé, suscitant ainsi la colère de Dieu.

Chaque fois qu'il y a un problème, Dieu a toujours un remède.
 Pour le problème du péché, Dieu a prescrit un remède, un traitement.
 Il se trouve dans le sacrifice de Jésus-Christ, l'Agneau de Dieu.
 La Bible nous dit que le sang de Christ nous lave de tous nos péchés.
 Mais soyez prévenus~: si vous refusez le remède de Dieu,
 il n'y a pas d'autre traitement. 

\dvrule

\dvprayer{
Père, nous sommes reconnaissants de ce que Tu as été si patient avec nous,
 nous donnant maintes et maintes opportunités de nous détourner du monde
 pour vivre en Te suivant.
 Seigneur, aide-nous à vivre devant Toi d'une façon qui Te plaise. 
}{Dans le Nom de Jésus, Amen.}



%%%%%%%%%%%%
% 5 avril
%%%%%%%%%%%%

\dvday{Entretenus par le Roi}

\dvquote{
Or, comme nous mangeons le sel du palais
 % \footnote ne marche pas bien avec \dvquote
 \footnotemark
 et qu'il ne nous paraît pas convenable de voir mépriser le roi,
 nous envoyons au roi ces informations.
}{\bibleverse{Esd}(4:14)}


\dvlettrine{L}{es voisins} d'Israël ont envoyé cette lettre à Artaxerxès
 pendant la reconstruction du temple.
 Quand leurs offres d'aide ont été rejetées,
 ils ont cherché à entraver le travail.
 Ils ont embauché des hommes de loi pour faire échouer le projet
 et ils ont adressé un document au roi de Perse.
 Ils disaient~: \og Nous croyons que ces gens vont vous déshonorer.
 Nous ne pouvions pas simplement voir ceci arriver sans intervenir~;
 nous avons donc estimé nécessaire d'informer le roi. \fg{}

Ce verset s'applique aujourd'hui à nous. Car nous avons également un Roi,
 et nous sommes entretenus par Lui.

Certains disent que Jésus est simplement une béquille pour les gens faibles.
 Ils ont raison. Je m'appuie sur Lui tout le temps.
 S'Il n'était pas pour moi une béquille,
 cela fait longtemps que je me serais effondré.
 Il ne m'a jamais abandonné ni laissé tomber.
 Vous n'avez pas à vous faire de soucis lorsque vous êtes entretenus par le Roi.

Mais le monde qui m'entoure déshonore mon Roi.
 Chaque fois que quelqu'un utilise le nom de mon Dieu en vain,
 mon cœur frémit d'indignation. 

\dvbox{
Je n'aime pas voir mon roi déshonoré. 
}

Tout comme les voisins d'Israël ont envoyé une lettre informant
 le roi de ce qui se passait, nous devons, nous aussi, informer le nôtre.
 Nous devons nous mettre à genoux devant Dieu pour Lui parler de la condition
 de notre nation et des choses qui se passent -- et prier pour Lui demander
 qu'Il commence à agir. 

\dvrule

\dvprayer{
Père, aide-nous aujourd'hui à amener gloire et honneur à Toi, notre Roi.
 Donne-nous l'audace de prendre la défense de ce qui est juste, saint et pur. 
}{Dans le Nom de Jésus, Amen.}



\footnotetext{expression équivalente à
 \og nous sommes entretenus par le palais \fg{}.}


