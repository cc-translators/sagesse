\jrnlmonth

%%%%%%%%%%%%
% 1er juin
% M 2013/06/10
%%%%%%%%%%%%

\jrnlday{L'appel aux mères}

\mainthemeindex{mere@mère}
\themeindex{laicite@laïcité}
\themeindex{priere@prière}
\themeindex{education@éducation}
\themeindex{nation}

\dvquote{
Laisse couler tes larmes comme un torrent!\\
 Ne te donne aucun répit,
 et que la pupille de ton \oe{}il n’ait pas de repos!\\
 Lève-toi, lance une clameur au début des veilles de la nuit!\\
 Répands ton c\oe{}ur comme de l’eau devant la face du Seigneur!\\
 Lève tes mains vers Lui pour la vie de tes enfants qui défaillent
 de faim à tous les coins de rues.
}{\ibibleverse{Lm}(2:18-19)}

\typo{Guillemet fermant non ouvert}

\lettrine{S}{e croyant} à la fois forts et sophistiqués,
 les gens d'Israël ont essayé d'éliminer Dieu de leur vie nationale.
 Ils considéraient qu'Il n'était plus nécessaire.
 Le résultat? Babylone a saccagé Jérusalem. 

Israël fait le deuil de sa cité ravagée.
 Jérémie, voyant que des temps désespérés exigent des mesures
 désespérées, délivre un appel à toutes les mères de la nation. 

\dvbox{
Ne sous-estimez jamais l'influence d'une mère. Quand elle est remuée,
 elle peut utiliser cette puissance pour guider ses enfants
 vers ce qui est~juste. 
}

Notre pays a un besoin désespéré de ce genre d'influence.
 Considérez l'atmosphère de notre système d'enseignement public.
 En 1963, la Cour Suprême a estimé qu'il était illégal de prier
 ou de lire la Bible dans nos écoles.
 Depuis cette décision, que s'est-il passé?
 Sur une moyenne nationale, les notes des tests d'aptitude
 à l'enseignement supérieur ont régulièrement baissé,
 la violence et le crime se sont largement répandus
 et la consommation de drogue a explosé. 

La seule vraie réponse pour ne plus déplaire à Dieu
 est le retour à des principes qui plaisent à Dieu
 \ocadr ces principes sur lesquels notre nation a été fondée.
 Non seulement les mères, mais les pères aussi doivent enseigner
 à nos enfants ce qui est bien et mal \ocadr du point de vue de Dieu.
 Nous devons enseigner à la prochaine génération à rechercher Dieu
 par la prière. 

Voici une suggestion radicale\frcolon{} cette semaine, éteignons nos télévisions
 et tournons nos visages vers Dieu.
 Prions pour notre nation et pour nos enfants. 

\dvrule

\dvprayer{
Incite-nous, Père, à invoquer Ton nom.\\
 Amène-nous à prier pour nos enfants, pour nos écoles et pour notre nation. 
}{\Amen}


%%%%%%%%%%%%
% 2 juin
% M 2013/06/10
%%%%%%%%%%%%

\jrnlday{De la dépression à~l'espoir}

\mainthemeindex{depression@dépression}
\mainthemeindex{espoir}
\themeindex{esperance@espérance}
\themeindex{Dieu!bienveillance de \sim}
\themeindex{Dieu!fidelite@fidélité de \sim}

\dvquote{
Voici ce que je veux repasser en mon c\oe{}ur, ce pourquoi j’espère\frcolon{}\\
 C’est que la bienveillance de l’Éternel n’est pas épuisée,\\
 Et que ses compassions ne sont pas à leur terme;\\
 Elles se renouvellent chaque matin. Grande est ta fidélité !\\
 L’Éternel est mon partage, dit mon âme;\\
 C’est pourquoi je veux m’attendre à lui.\\
 L’Éternel est bon pour qui espère en lui, pour celui qui le cherche.\\
 Il est bon d’attendre en silence le salut de l’Éternel.
}{\ibibleverse{Lm}(3:21-26)}

\punct{Guillemet fermant sans guillemet ouvrant}

\lettrine{C}{omme} c'est habituellement le cas, c'est de penser à lui-même
 qui a provoqué la dépression de Jérémie.
 Mais quand il a changé sa fa\c{c}on de penser
 \ocadr détournant son attention de lui-même pour la porter sur Dieu \fcadr{}
 la dépression s'en est allée. Quand il s'est souvenu de Dieu,
 qu'il a médité sur Son caractère, la paix et l'espoir ont rempli ses pensées. 

\dvbox[0.94\textwidth]{
Quand le désespoir se dissipe, l'espoir le remplace. 
}

Au lieu de penser à lui-même, Jérémie a pensé à la nature de Dieu.
 Il a pensé à la bienveillance de l'Éternel qui n'est pas épuisée,
 et grâce à laquelle nous ne sommes pas consumés.
 Il a pensé à la fidélité de Dieu. Dieu fait toujours exactement
 ce qu'Il dit qu'Il va faire, et Il est capable de faire ressortir du bien
 de la situation la plus désastreuse. Il a pensé au fait que Dieu
 est notre partage. Que pourrions-nous vouloir de plus?
 De quoi de plus pourrions-nous avoir besoin?
 Il a pensé à la bonté de Dieu,
 qui fait concourir toutes choses à notre bien. 

Puis Jérémie a conclu qu'il est bon \Og d'attendre en silence
 le salut de l'Éternel. \Fg{}

Êtes-vous inquiets aujourd'hui? Changez vos pensées.
 Ne méditez pas sur votre douleur \ocadr méditez sur votre Sauveur.
 Fixez vos yeux sur Jésus et souvenez-vous de Son caractère.
 Son amour ne vous a jamais fait défaut.
 Ses compassions se renouvellent chaque matin. 

\dvrule

\dvprayer{
Père, aide-nous à fixer nos yeux sur Toi plutôt que sur nous-mêmes.
 Remplis nos c\oe{}urs de Ton espoir et Ton amour aujourd'hui. 
}{\Amen}

\grammar{Remplis}


%%%%%%%%%%%%
% 3 juin
% M 2013/06/10
%%%%%%%%%%%%

\jrnlday{L'appel de~Dieu}

\mainthemeindex{Dieu!appel de \sim}
\themeindex{ministere@ministère}

\dvquote{
Va trouver les déportés, les fils de ton peuple;\\
 tu leur parleras et,
 qu'ils écoutent ou qu'ils ne prennent pas garde,\\
 tu leur diras\frcolon{}
 Ainsi parle le Seigneur, l'Éternel.
}{\ibibleverse{Ez}(3:11)}

\lettrine{D}{ieu} nous a tous appelés à un ministère
 dans le corps de Christ. Tous ne sont pas apôtres,
 tous ne sont pas prophètes, tous ne sont pas pasteurs ou enseignants,
 mais nous avons tous une place de service.
 Dieu a une tâche et un but pour vous.
 Et s'Il vous appelle à un ministère, vous pouvez être certains
 qu'Il va vous équiper pour le mener à bien. 

\dvbox[0.95\textwidth]{
Dieu ne vous demandera jamais de faire quelque chose sans vous donner
 aussi les moyens de réussir.
}

Ézéchiel a re\c{c}u l'ordre de parler au peuple\frcolon{} \punct{deux-points, majuscule}
 \Og Qu'ils écoutent, ou qu'ils n'écoutent pas. \Fg{}
 Le ministère n'est pas facile. Ça peut souvent être très décourageant.
 Les gens ne vont pas toujours répondre à ce que vous leur dites.
 Quelquefois les gens se mettent en colère avec vous parce que vous osez
 leur annoncer les choses du Seigneur. 

Mais quand votre vie sera finie, il n'y aura qu'une chose qui comptera 
 \ocadr et c'est votre obéissance à la volonté de Dieu.
 Tout ce que vous aurez pu faire pour vous-mêmes, toute la richesse
 que vous aurez pu amasser, toutes les prouesses qui seront décrites
 dans votre rubrique nécrologique seront futiles. 

Consacrez votre vie à ce qui compte vraiment
 \ocadr ce qui est éternel. Vivez votre vie pour~Lui. 

\dvrule

\dvprayer{
Père, merci pour la puissance du Saint-Esprit qui nous donne les moyens
 de répondre à notre appel.\\
 Puissions-nous toujours servir d'une position
 de compassion, de compréhension et d'amour.\\
 Et cependant, Seigneur, puissions-nous ne pas être coupables de manquer
 d'avertir les gens de ce que Tu as dit. 
}{\DlNdJ}


%%%%%%%%%%%%
% 4 juin
% M 2013/06/10
%%%%%%%%%%%%

\jrnlday{Faux espoir}

\mainthemeindex{espoir}
\themeindex{esperance@espérance}

\dvquote{
J'abattrai la muraille que vous avez enduite de badigeon,\\
 je lui ferai toucher la terre, et ses fondations seront mises à nu;\\
 elles s'écrouleront, et vous périrez au milieu d'elles,\\
 et vous reconnaîtrez que je suis l'Éternel.
}{\ibibleverse{Ez}(13:14)}

\lettrine{L}{es} prophètes mentaient au peuple.
 Désirant être porteurs d'un message populaire, ils parlaient de paix,
 alors qu'en vérité, Israël était au bord de la destruction.
 Ces prophètes avaient bâti comme une \Og muraille \Fg{}
 dans l'esprit des gens, derrière laquelle \punct{pas besoin de virgule}
 ceux-ci croyaient à tort trouver le réconfort et la sécurité par rapport
 à l'ennemi. Mais la sécurité qu'ils trouvaient dans les paroles
 des prophètes n'était qu'une fausse sécurité.
 Confrontée aux tempêtes, la \Og muraille \Fg{} n'allait pas tenir. 

\dvbox{
Un vrai serviteur de Dieu n'osera jamais prêcher un message
 rassurant à ceux qui vivent dans le péché. 
}

Les gens veulent entendre un message rassurant qui dit
 \punct{pas besoin de virgule} qu'en réalité,
 peu importe à Dieu ce que vous faites, du moment que votre attitude est bonne
 ou que \c{c}a vous semble juste.
 Cependant, ceux qui vous rassurent dans votre péché sont de faux prophètes.
 Ils construisent de mauvaises murailles qui ne tiennent pas et
 derrière lesquelles \punct{pas besoin de virgule}
 vous ne trouvez qu'une fausse sécurité. 

J'ai été appelé à avertir fidèlement et honnêtement ceux qui ont choisi
 le chemin du péché. Mais je me dois aussi d'annoncer de bonnes nouvelles
 à ceux qui ont livré leur vie à Jésus-Christ.
 Il est la muraille bâtie avec du bon mortier. Mettez votre confiance en lui,
 afin que lorsque les tempêtes arriveront, vous trouviez la vraie sécurité. 

\dvrule

\dvprayer{
Seigneur, nous savons que notre espérance n'est pas vaine.\\
 Jésus-Christ nous permettra de traverser les plus grandes tempêtes
 de la vie, et nous fera paraître un jour, devant Ta gloire,
 irréprochables dans l'allégresse. 
}{\DlPNdJ}


%%%%%%%%%%%%
% 5 juin
% M 2013/06/10
%%%%%%%%%%%%

\jrnlday{Blâme mal placé}

\themeindex{souffrance}
\themeindex{responsabilite@responsabilité}
\themeindex{peche@péché}
\themeindex{jugement}

\dvquote{
Pour ma part, je jugerai chacun de vous selon sa propre conduite,\\
 je vous l'affirme, moi, le Seigneur Dieu.\\
 Changez donc de vie,
 détournez-vous de tout le mal que vous faites,
 ne laissez plus aucune faute causer votre perte.
}{\ibibleverse{Ez}(18:30)}

\lettrine{L}{es} habitants d'Israël rejetaient la responsabilité
 de leurs souffrances en captivité sur le péché de leurs pères.
 Mais Dieu a lutté contre cet état d'esprit.
 Il les a appelés à assumer la responsabilité de leurs propres péchés. 

Que cela vous plaise ou non, vous êtes responsables de la personne
 que vous êtes et des choses que vous faites.
 Chaque homme est jugé pour ses propres actes.
 Quand vous vous retrouverez devant Dieu, \punct{virgule}
 vous n'aurez à répondre que d'une seule personne
 \ocadr et cette personne c'est vous. 

\dvbox{
Cessez de blâmer les autres pour vos péchés. 
}

Vous avez peut-être connu un passé terrible.
 Vous avez peut-être eu des parents négligents ou inaptes
 et une enfance déplorable. Vous avez peut-être souffert
 d'abus inimaginables. En Jésus se trouve la guérison de tous ces maux.
 Votre passé peut être gommé et vous pouvez être tout ce que Dieu
 veut que vous soyez. Mais tant que vous cherchez \grammar{cherchez}
 quelqu'un d'autre à blâmer, vous n'allez jamais voir le besoin
 de vous repentir de vos propres actes. 

Si vous continuez dans vos fautes et que vous continuez
 à imputer la responsabilité de vos péchés à quelqu'un d'autre,
 votre péché va finir par vous perdre.
 Mais Dieu a de meilleurs plans pour vous. Il veut vous bénir.
 Jésus a dit\frcolon{} \punct{deux-points}
 \Og Moi, je suis venu pour que les gens aient la vie,
 et pour que cette vie soit abondante \Fg{} (\ibibleverse{Jn}(10:10) \BFC). 

Cessez de blâmer les autres pour vos péchés.
 Repentez-vous, tournez-vous vers Jésus,
 et recevez la nouvelle vie qu'Il veut vous donner. 

\dvrule

\dvprayer{
Père, nous te remercions de ce que nous pouvons avoir cette nouvelle vie
 et ce nouveau c\oe{}ur que Tu nous a offerts.\\
 Puissions-nous endosser la responsabilité de notre propre péché
 et nous en détourner pour trouver une vie nouvelle et bénie en Jésus-Christ. 
}{\Amen}


%%%%%%%%%%%%
% 6 juin
% M 2013/06/11
%%%%%%%%%%%%

\jrnlday{Vous reconnaîtrez}

\themeindex{prophetie@prophétie}
\themeindex{Dieu!jugement de \sim}
\index{jugement!\sim~de Dieu|see{Dieu}}

\dvquote{
Moi, l'Éternel, j'ai parlé, cela arrivera, et je l'exécuterai ;\\
 je ne reculerai pas et je n'aurai pas de pitié ni de regret\dots{}\\
 et ils reconnaîtront que je suis l'Éternel.
}{\ibibleverse{Ez}(24:14,27)}

\lettrine{D}{ieu} a révélé à Ézéchiel la destruction complète
 qui était sur le point de frapper Jérusalem. Il a dit\frcolon{} \punct{deux-points}
 \Og J'ai parlé; cela arrivera. \Fg{}
 Le jour même noté par Ézéchiel, date qui devait être le début de la fin,
 les Babyloniens ont commencé le siège de Jérusalem. Dieu a accompli Sa Parole. 

\dvbox{
Quand Dieu va accomplir Sa Parole concernant les temps de la fin
 et commencer à déverser Son jugement sur la terre,
 les gens reconnaîtront qu'Il est Dieu. 
}

Aujourd'hui, beaucoup de gens n'accordent aucun crédit à la Parole de Dieu.
 Ils se moquent des promesses de Jésus, accomplissant ainsi sans le savoir
 \suggest{Pas besoin de virgule}
 la prophétie annoncée par Pierre quand il disait\frcolon{} \punct{deux-points}
 \Og Dans les derniers jours, il viendra des moqueurs pleins de raillerie,
 qui marcheront selon leurs propres convoitises et diront\frcolon{}
 Où est la promesse de son avènement? \Fg{}
 \punct{guillemet fermant manquant}
 (\ibibleverse{IIP}(3:3-4)).
 Ces moqueurs ne voient dans la Bible qu'une collection d'histoires imaginaires
 créées par des gens qui avaient besoin de croire en quelque chose.
 Mais ils ne peuvent expliquer pourquoi les prophéties de la Bible
 se sont accomplies avec autant d'exactitude. 

Comme il est important pour nous de reconnaître Dieu et de Lui abandonner
 nos vies maintenant! Son jugement arrive, et seuls ceux qui Lui appartiennent
 échapperont à l'éclatement de Sa colère. 

\dvrule

\dvprayer{
Père, parle à nos c\oe{}urs par Ta Parole afin que nous puissions réaliser
 la certitude de ces choses que Tu as déclarées.\\
 Puissions-nous marcher à Ta suite et nous engager envers Toi
 pleinement et complètement. 
}{\NlddlNdJ}


%%%%%%%%%%%%
% 7 juin
% M 2013/06/11
%%%%%%%%%%%%

\jrnlday{L'orgueil}

\mainthemeindex{orgueil}

\dvquote{
\dots{} Son c\oe{}ur était fier de sa hauteur,\\
 c'est pourquoi je l'ai livré entre les mains du conducteur des nations,
 qui le traitera selon sa méchanceté; je l'ai chassé.
}{\ibibleverse{Ez}(31:10-11)}

\lettrine{C}{'est} Dieu qui a \Og planté \Fg{} l'Assyrie
 près de l'eau afin que leurs racines puissent trouver une abondante nutrition.
 C'est Dieu qui les avait rendus beaux et forts,
 mais ils ont commencé à s'attribuer le mérite de leur puissance
 et à se vanter de ce qu'ils avaient réalisé.

\dvbox{
S'attribuer le mérite de ce que Dieu a fait est toujours dangereux.
}

Trop souvent, les hommes que Dieu a utilisés ou à qui Il a donné des dons
 spéciaux se mettent à regarder leurs réussites comme s'ils étaient
 un peu responsables de leur succès.
 Ils commencent à se saisir de la gloire qui appartient en réalité au Seigneur.
 Mais Dieu a dit qu'Il ne permettra à aucune chair de se glorifier
 en Sa présence. Celui qui s'élève sera abaissé.

Un ministère qui a beaucoup de succès peut poser danger parce que,
 soudain, les gens veulent apprendre vos secrets.
 La tentation est grande de montrer les choses que Dieu a faites
 comme si c'est vous qui les aviez élaborées.

Tout ce que nous avons de bon ou de valable nous a été donné par Dieu.
 C'est Dieu qui nous a donné la force, les talents et l'intelligence.
 Nous devons attribuer à Dieu le mérite des choses qu'Il a faites.

À Dieu soit la gloire; Il a accompli de grandes choses!


\dvrule

\dvprayer{
Père, puissions-nous Te reconnaître comme Celui qui donne la vie et la force.\\
 Puissions-nous continuellement Te rendre la gloire que Tu mérites si richement. 
}{\DlNdJ}


%%%%%%%%%%%%
% 8 juin
% M 2013/06/11
%%%%%%%%%%%%

\jrnlday{Justice}

\mainthemeindex{Dieu!justice de \sim}

\dvquote{
Les Israélites disent\frcolon{}\\
 \Og La fa\c{c}on de faire du Seigneur n'est pas bonne. \Fg{}\\
 Mais c'est leur fa\c{c}on de faire qui n'est pas bonne!
}{\ibibleverse{Ez}(33:17)}

\lettrine{I}{l} est intéressant de noter que dans ce passage,
 les gens sont en colère avec Dieu à cause de Sa Grâce.
 Dieu avait déclaré que si un homme était destiné à mourir
 à cause de son inclination au mal, cet homme pouvait être pardonné
 s'il se détournait de ses péchés et se mettait à marcher
 selon les lois de Dieu. Dieu avait aussi promis que si un homme se repentait,
 aucun de ses péchés passés ne lui serait jamais rappelés.\grammar{pluriel}

Or, ceci avait irrité un grand nombre de gens. Ils n'appréciaient pas que Dieu
 pardonne aussi facilement le péché. Où est la justice dans tout cela?
 Assurément, l'homme doit faire quelque chose pour mériter le pardon de Dieu. 

Il y a beaucoup de choses que nous ne savons pas sur Dieu et sur Sa justice.
 Quelle sera Son verdict final concernant la personne qui a vécu
 dans un endroit reculé et n'a jamais entendu parler de l'offre de salut
 en Jésus-Christ? Personne ne le sait. Mais quand elle se tiendra
 devant le Seigneur et que nous entendrons la sentence qui lui sera réservée,
 nous dirons très certainement\frcolon{} \punct{deux-points}
 \Og Tes jugements sont véritables et justes \Fg{} (\ibibleverse{Ap}(16:7)). 

\dvbox{
Plutôt que critiquer Dieu, nous devons apprendre de Lui.
 Nous devons avoir le même amour, la même compassion et le même empressement
 à pardonner que Dieu. 
}


Qu'il est merveilleux qu'une vie consacrée au mal puisse être effacée
 en un instant! L'homme peut secouer sa tête et dire\frcolon{} \punct{deux-points}
 \Og Ça n'est pas juste \Fg{}, mais Dieu répondra\frcolon{} \punct{deux-points}
 \Og Mes voies sont plus hautes que vos voies. \Fg{}
 \ibiblephantom{Is}(55:9)

\dvrule

\dvprayer{
Père, comme nous sommes reconnaissants pour la grâce et la miséricorde
 que Tu nous a offertes!\\
 Pardonne-nous nos transgressions
 comme nous pardonnons à ceux qui nous ont offensés.
}{\DlNdJ}

\grammar{pardonnons à ceux qui nous ont offensé}


%%%%%%%%%%%%
% 9 juin
% 2013/06/11
%%%%%%%%%%%%

\jrnlday{Inspiration plutôt que~transpiration}

\themeindex{adoration}
\mainthemeindex{chair}
\themeindex{obligation}
\themeindex{amour}
\themeindex{gratitude}

\dvquote{
Ils auront des turbans de lin sur la tête,\\
 et des cale\c{c}ons de lin sur les reins ;\\
 ils ne mettront pas de ceinture qui provoque la sueur.
}{\ibibleverse{Ez}(44:18)}

\lettrine{Q}{uand} nous pensons à la sueur,
 nous pensons à une activité faite dans l'énergie de la chair.
 Et il se peut qu'il y ait des cas où en servant le Seigneur,
 nous transpirions \ocadr comme pendant des projets de construction
 et des choses de cette nature. Mais quand il s'est agi de rendre
 un culte à Dieu, d'exprimer la louange, l'adoration
 et des actions de grâce, Dieu a dit qu'Il ne voulait rien
 qui puisse causer de la sueur. 

\dvbox{
Dieu ne veut pas que vous L'adoriez ou Le serviez
 sous le coup d'une obligation ou d'un délire émotionnel.
}

\punct{Point manquant}

Le Seigneur ne veut pas que vous ayez le sentiment qu'Il a placé sur vous
 un\grammar{"tel" en trop} fardeau si intolérablement lourd que vous avez à faire des efforts
 extraordinaires afin d'accomplir Sa volonté. Certains parlent
 des grands sacrifices \grammar{auxquels ils ont consenti}
 auxquels ils ont consentis pour servir le Seigneur.
 Mais Dieu n'exige jamais de vous d'abandonner la moindre chose valable
 pour suivre Jésus \ocadr seulement le tas de choses sans valeur
 qui encombrent votre vie. 

Dieu désire que vous adoriez avec un c\oe{}ur débordant d'amour,
 simplement parce que vous réagissez à l'amour qu'Il vous manifeste.
 Pas de sueur, pas de fausse incitation,
 simplement une expression de gratitude pure et naturelle. 

Puisse notre louange être inspirée par un c\oe{}ur pur
 \ocadr avec gratitude pour ces choses que Dieu a faites dans nos vies, 
et avec une crainte mêlée d'admiration devant Sa gloire,
 Sa majesté et Sa beauté. 

\dvrule

\dvprayer{
Père, nous voulons Te servir d'une position de joie et de gratitude. 
}{\DlNdJ}


%%%%%%%%%%%%
% 10 juin
% M 2013/06/11
%%%%%%%%%%%%

\jrnlday{Il est là}

\mainthemeindex{Dieu!presence@présence de \sim}
\themeindex{Dieu!omnipresence@omniprésence de \sim}
\themeindex{nation}
\themeindex{laicite@laïcité}
\themeindex{espoir}
\themeindex{force}
\themeindex{purete@pureté}
\themeindex{tentation}
\themeindex{peche@péché}
\themeindex{desespoir@désespoir}

\dvquote{
\dots{} Le nom de la ville sera\frcolon{}
 L'Éternel est ici.
}{\ibibleverse{Ez}(48:35)}

\lettrine{E}{ssayant} de décrire la gloire future de la nation d'Israël,
 Ézéchiel a dit\frcolon{} \punct{deux-points}
 \Og Elle sera appelée Jéhovah-Shammah \ocadr l'Éternel est là. \Fg{}
 Il ne peut pas penser à quelque chose de plus glorieux
 que le fait que l'Éternel est là\dots{}

Certains souhaiteraient échapper à la présence de Dieu.
 Ils font de grands efforts pour éliminer Dieu de notre vie nationale
 \ocadr pas de mention de Dieu dans les écoles publiques,
 pas de rappels de Dieu dans les lieux publics.
 Ils pensent qu'ils apprécieraient d'être là où Dieu n'est pas,
 parce qu'alors ils n'auraient plus de restrictions.
 Ils pourraient faire tout ce qu'ils voudraient.
 Mais vous devez vous rappeler que dans un monde sans limitations,
 tout le monde a les mêmes libertés que vous.
 Et vous ne survivriez pas longtemps dans de telles conditions. 

\dvbox{
Dieu merci, nous ne pouvons jamais échapper à Sa présence. 
}

Parce que Dieu est là, nous avons l'espoir, la force et la pureté.
 Sans Lui, nous n'avons que l'obscurité. 

Bien que Dieu soit toujours là, il nous arrive de l'oublier.
 C'est le cas quand nous sommes tentés de pécher.
 C'est aussi le cas quand nous commen\c{c}ons à désespérer de nos circonstances.
 Mais c'est précisément dans ces moments que nous avons le plus besoin
 d'avoir conscience de la présence de Dieu. 

La conscience de la présence de Dieu va nous garder du péché.
 Elle va chasser notre désespoir. Elle va nous permettre de nous réjouir
 à l'heure la plus sombre de notre vie. Cultivez cette conscience.
 Rappelez-vous de Celui qui a promis d'être toujours avec nous,
 du sommet des montagnes au plus bas des vallées
 et partout ailleurs entre les deux.

\dvrule

\dvprayer{
Père, nous te remercions pour le don de Ta présence.
 Aide-nous à nous rappeler que Tu es là, afin que nos c\oe{}urs
 soient élevés du désespoir à la gratitude. 
}{\Amen}


%%%%%%%%%%%%
% 11 juin
% M 2013/06/11
%%%%%%%%%%%%

\jrnlday{Avec vous dans la~fournaise}

\mainthemeindex{epreuve@épreuve}
\themeindex{Dieu!Fils de \sim}
\themeindex{Jesus@Jésus-Christ!presence@présence de \sim}

\dvquote{
Il répondit, et dit\frcolon{}\\
 \Og Voici, je vois quatre hommes déliés, marchant au milieu du feu
 et ils n'ont aucun mal;\\
 et l'aspect du quatrième
 est comme le Fils de Dieu. \Fg{}
}{\ibibleverse{Dn}(3:25) \KJF}

\lettrine{Q}{uand} les amis de Daniel, Chadrak, Méchak et Abed-Nego,
 ont refusé de se prosterner devant la statue que Neboukadnetsar avait érigée,
 le roi est devenu furieux, et a ordonné que l'on jette les trois hommes
 dans la fournaise ardente. Mais Dieu a toujours le dernier mot.
 Au sein de la fournaise, les hommes ne furent pas touchés par les flammes.
 Quand le roi a ouvert la fournaise, il a été très surpris de ne pas y voir
 seulement trois hommes, mais quatre.
 Et le quatrième était comme \Og le Fils de Dieu. \Fg{}

Dans ce récit, nous voyons une démonstration de la présence permanente
 de Jésus-Christ, qui préserve Son peuple au travers des heures
 les plus sombres et des \typo{des épreuves} épreuves les plus féroces.
 Nous aimerions que Dieu nous préserve simplement et complètement
 de l'épreuve du feu. Ce serait notre préférence.
 Mais souvent, Il choisit de ne pas le faire. 

\dvbox{
Au lieu de nous délivrer \emph{du} feu,
 Dieu peut choisir de nous délivrer \emph{dans} le feu. 
}

Le feu raffine. Dieu désire la pureté dans Son peuple,
 aussi nous laisse-t-Il souvent passer du temps dans la fournaise
 pour que les impuretés qui sont en nous puissent brûler. 

Vous passez peut-être par une épreuve du feu aujourd'hui, mais sachez ceci
 \suggest{deux-points au lieu de tiret}
 \ocadr vous n'êtes pas dans ce feu tout seul.
 Le Seigneur est là avec vous, et quand l'épreuve aura accompli son but,
 Il vous délivrera. 

\dvrule

\dvprayer{
Père, nous Te remercions de ce que nous ne sommes jamais seuls
 \ocadr même dans la douleur la plus profonde ou l'heure la plus sombre.\\
 Comme nous Te sommes reconnaissants pour Ta présence au sein des difficultés.\\
 Comme nous sommes bénis que Tu aies choisi de traverser nos épreuves avec nous. 
}{\DlNdJ}


%%%%%%%%%%%%
% 12 juin
% M 2013/06/12
%%%%%%%%%%%%

\jrnlday{Avec facilité ou~avec~difficulté}

\themeindex{orgueil}
\themeindex{louange}

\dvquote{
Maintenant, moi, Neboukadnetsar, je loue, j'exalte et je glorifie
 le Roi des cieux, dont toutes les \oe{}uvres sont vraies et les voies justes,
 et qui peut abaisser ceux qui marchent avec orgueil.
}{\ibibleverse{Dn}(4:34)}

\lettrine{N}{eboukadnetsar} était un homme très doué
 \ocadr trop doué probablement. Il a commencé à devenir orgueilleux.
 Il a commencé à croire que tous ses succès résultaient de la brillance
 de son propre génie. 

La Bible nous dit que \Og l'orgueil précède le désastre \Fg{}
 (\ibibleverse{Pr}(16:18)).
 Dieu a averti Neboukadnetsar du chemin dangereux qu'il avait emprunté.
 Mais Neboukadnetsar a choisi d'ignorer cet avertissement. 

\dvbox[0.85\textwidth]{
Dieu est si fidèle!
 Il nous prévient toujours quand nous nous engageons sur un terrain dangereux. 
}

Quand Dieu envoie\grammar{envoie} un avertissement,
 vous pouvez apprendre la le\c{c}on
 qu'Il veut vous voir apprendre selon l'une de deux fa\c{c}ons\frcolon{}
 avec facilité ou avec difficulté. La facilité,\suggest{La facilité}
 c'est de tenir compte de Son avertissement. La difficulté,\suggest{La difficulté}
 c'est de l'ignorer et de continuer à se diriger vers les ennuis. 

Neboukadnetsar a choisi la voie de la difficulté.
 Il est de nouveau tombé dans l'orgueil et dans l'heure qui a suivi,
 il a perdu la raison. Il lui a fallu sept ans pour finalement admettre
 que c'est\punct{apostrophe manquante} Dieu qui est souverain. 

Le Seigneur nous aime trop pour nous laisser toucher à ces choses
 qui vont nous détruire sans intervenir. Parce que nous Lui appartenons,
 Il va toujours amener les avertissements et les le\c{c}ons
 dont nous avons besoin pour nous éviter les ennuis.
 Que nous les apprenions facilement ou difficilement,
 cela\suggest{cela} dépend entièrement de nous. 

\dvrule

\dvprayer{
Père, nous Te remercions de ce que Tu nous aimes assez pour nous garder
 de la folie.\\
 Donne-nous une soif de Ta Parole, dans laquelle
 nous allons trouver toute la sagesse dont nous avons besoin
 pour vivre en Te restant fidèles. 
}{\DlNdJ}


%%%%%%%%%%%%
% 13 juin
% M 2013/06/30
%%%%%%%%%%%%

\jrnlday{S'occuper des affaires du~Roi}

\themeindex{responsabilite@responsabilité}
\themeindex{service}
\mainthemeindex{Jesus@Jésus-Christ!royauté de \sim}
\index{royaute@royauté!\sim~de Jésus-Christ|see{Jésus-Christ}}

\dvquote{
Moi, Daniel, je fus plusieurs jours affaibli et malade ;\\
 puis je me levai et m'occupai des affaires du roi.\\
 J'étais dans la stupeur à cause de la vision
 et ne la comprenais point.
}{\ibibleverse{Dn}(8:27)}

\lettrine{D}{'après} ce passage,
 nous pouvons conclure que Daniel était un représentant du gouvernement
 babylonien pour le roi Belchatsar. On lui avait confié
 les affaires du roi et il s'était fidèlement engagé
 à s'acquitter de cette responsabilité. 

À nous aussi ont été confiées les affaires du Roi et nous avons été appelés
 à Le représenter où que nous allions. Mais le royaume de notre Seigneur,
 contrairement à tous les royaumes terrestres, \punct{virgule}
 va durer pour toujours.  Notre royaume est un royaume éternel. 

\dvbox{
Votre roi a des affaires dont vous devez vous occuper aujourd'hui. 
}

\Og Notre citoyenneté, écrivait Paul, \punct{virgule, pas de répétition des guillemets}
 est dans les cieux; de là nous attendons comme sauveur
 le Seigneur Jésus-Christ \Fg{} (\ibibleverse{Ph}(3:20) \NBS).
 Quel privilège pour nous que de représenter Jésus sur la terre!
 Il nous a donné l'autorité de parler pour Lui et de parler en Son nom.
 Mais si c'est un privilège, c'est aussi une formidable responsabilité,
 car où que nous allions, les gens vont Le juger selon ce qu'ils voient en nous. 

Le Roi que nous servons n'aime pas être mal représenté.
 Nous devons aimer comme Il aime, donner comme Il donne, pardonner
 comme Il nous a pardonné. Le Roi que nous représentons est bon,
 plein de compassion, miséricordieux et plein de grâce,
 et Il a demandé que nous le représentions de cette même fa\c{c}on. 

Votre Roi a des affaires dont vous devez vous occuper aujourd'hui.
 Puissiez-vous le servir avec joie et avec fidélité. 

\dvrule

\dvprayer{
Père, nous te remercions pour l'honneur et le privilège de Te représenter.
 Aide-nous, Seigneur, pour que par nos actions et par nos paroles,
 nous puissions être de dignes ambassadeurs de notre merveilleux Roi. 
}{\NpcdlNdJ}



%%%%%%%%%%%%
% 14 juin
% M 2013/06/30
%%%%%%%%%%%%

\jrnlday{L'arme de la~prière}

\mainthemeindex{priere@prière}
\mainthemeindex{combat spirituel}
\themeindex{armure}

\dvquote{
Il me dit\frcolon{} Daniel, sois sans crainte ;\\
 car dès le premier jour où tu as eu à c\oe{}ur de comprendre
 et de t'humilier devant ton Dieu, tes paroles ont été entendues,
 et c'est à cause de tes paroles que je suis venu.
}{\ibibleverse{Dn}(10:12)}

\lettrine{D}{aniel} cherchait des réponses auprès de Dieu,
 et dans ce but, il a prié pendant vingt et un jours.
 Finalement, un visiteur céleste a informé Daniel que Dieu avait entendu
 ses prières et avait envoyé le messager pour apporter les réponses
 qu'il avait demandées. 

Cependant, alors qu'il venait voir Daniel, il s'est retrouvé au milieu
 d'une bataille avec les forces des ténèbres qui contrôlaient
 la nation de la Perse. Il a combattu pendant vingt et un jours
 avant que Michel, l'un des archanges, ne vienne à son aide et le libère.
 Alors seulement a-t-il pu achever sa mission. 

\dvbox{
Un combat spirituel fait rage dans l'univers qui nous entoure. 
}

Que vous en soyez conscients ou pas, que \c{c}a vous plaise ou non,
 vous êtes au milieu d'un combat en ce moment même.
 Des forces puissantes s'affrontent pour le contrôle de vos pensées.
 Seule la puissance du Saint-Esprit peut vous aider à tenir bon contre
 les forces spirituelles qui cherchent à vous détruire. 

\ibibleverse{Ep}(6:12) explique la nature de cette bataille
 de la fa\c{c}on suivante\frcolon{} \Og Nous n'avons pas à lutter contre la chair
 et le sang, mais contre les principautés, contre les pouvoirs,
 contre les dominateurs des ténèbres d'ici-bas,
 contre les esprits du mal dans les lieux célestes. \Fg{}

\bibleverse{Ep}(6:) décrit aussi l'armure spirituelle que Dieu
 a fournie pour votre protection. Votre tâche est de vous revêtir
 de cette armure, puis de prier. Le véritable combat a lieu
 quand vous êtes sur vos genoux et c'est là, aussi,
 que la victoire va être remportée. 

\dvrule

\dvprayer{
Aide-nous, Seigneur, à rejeter tout fardeau et le péché qui menace
 de nous entraver,\\
 afin que nous puissions courir cette épreuve pour Ta gloire. 
}{\Amen}


%%%%%%%%%%%%
% 15 juin
% M 2013/06/30
%%%%%%%%%%%%

\jrnlday{Semer et~moissonner}

\mainthemeindex{semence}
\mainthemeindex{moisson}
\themeindex{vent}
\themeindex{vanite@vanité}
\themeindex{protection}

\dvquote{
Puisqu'ils ont semé du vent, ils moissonneront la tempête;
 ils n'auront pas une tige de blé; ce qui poussera ne donnera pas de farine,
 et s'il y en avait, des étrangers l'engloutiraient.
}{\ibibleverse{Os}(8:7)}

\lettrine{D}{ieu} a donné naissance à Israël, mais les gens d'Israël
 L'ont délaissé, et ils ont commencé à semer au vent.
 En conséquence, il leur a retiré Sa protection et les a laissés \suggest{laissé}
 se débrouiller tous seuls. Et maintenant, les prévient-Il,
 le vent qu'ils avaient semé était sur le point de leur faire moissonner
 une tempête venue du Nord
 \ocadr l'Assyrie qui allait fondre sur eux et disperser la nation d'Israël. 

\dvbox{
Semez en faisant bien attention.
}

\ibibleverse{Ga}(6:7) nous dit\frcolon{} \punct{deux-points}
 \Og Ce qu'un homme aura semé, il le moissonnera aussi. \Fg{}
 Et la nature nous prouve que ce principe est vrai.
 Nous moissonnons en l'espèce. Si vous semez du maïs, vous obtenez du maïs.
 Vous n'obtenez pas des haricots. Si vous semez de la miséricorde,
 on fera preuve de miséricorde envers vous. Si vous semez du pardon,
 vous recevrez le pardon. Si vous semez de l'immoralité sexuelle
 (ou comme le dit la vieille expression,
 \Og si vous jetez votre gourme \Fg{})\dots{}
 alors mieux vaut vous préparer à en récolter les conséquences. 

Le problème, c'est que nous avons tendance à ne pas être assez sélectifs.
 Trop souvent, nous semons un mélange de graines.
 Nous allons à l'église et semons la Parole de Dieu dans nos pensées,
 mais nous rentrons à la maison, nous allumons la télé et commen\c{c}ons à semer
 pour la chair. Une récolte mélangée arrive et avec elle beaucoup de confusion. 

Si vous semez pour la chair, tout ce que vous pouvez espérer moissonner,
 c'est la corruption. Mais si vous semez pour l'Esprit,
 vous allez récolter le fruit de l'Esprit et la vie éternelle. 

\dvrule

\dvprayer{
Père, garde-nous de nous méprendre en pensant que nous pouvons semer
 n'importe quelle graine de notre choix, sans avoir à souffrir
 les conséquences de ces choix.\\
 Puissions-nous semer selon la justice. 
}{\DlNdJ}


%%%%%%%%%%%%
% 16 juin
% M 2013/06/30
%%%%%%%%%%%%

\jrnlday{Maladie du~c\oe{}ur}

\mainthemeindex{maladie}
\mainthemeindex{coeur@cœur}
\themeindex{vie!\sim~chrétienne}
\themeindex{idolatrie@idolâtrie}
\themeindex{consecration@consécration}

\dvquote{
Leur c\oe{}ur est partagé\frcolon{} ils vont en porter la culpabilité.
 L'Éternel renversera leurs autels, détruira leurs stèles.
}{\ibibleverse{Os}(10:2)}

\lettrine{L}{es} gens d'Israël accordaient bien une place à Dieu
 dans leur c\oe{}ur, mais seulement une petite place.
 Ils ne pouvaient pas Lui offrir leur c\oe{}ur entier,
 parce qu'ils avaient choisi d'adorer d'autres dieux. 

Jésus nous a dit qu'il n'est pas possible de servir deux maîtres
 (\ibibleverse{Mt}(6:24)).
 C'est cependant ce que beaucoup de gens essayent de faire.
 Au lieu d'accorder à Dieu leur totale consécration,
 ils se contentent de Lui faire un petit signe de temps à autres. 

\dvbox{
Dieu n'est pas satisfait d'avoir juste une partie de votre vie.
 Il n'est pas prêt à vous partager avec d'autres dieux. 
}

Dieu désire \ocadr Dieu mérite \fcadr{} d'avoir tout votre c\oe{}ur,
 toute votre âme, toutes vos pensées. 

Comment est votre c\oe{}ur aujourd'hui ? Est-il totalement consacré à Dieu,
 où est-il partagé? Il peut être difficile d'évaluer précisément
 votre propre c\oe{}ur, car comme Jérémie le fait remarquer\frcolon{}
 \Og Le c\oe{}ur est tortueux par-dessus tout et il est incurable\frcolon{}
 qui peut le connaître? \Fg{} (\ibibleverse{Jr}(17:9)).
 La seule fa\c{c}on de connaître la vraie condition de votre c\oe{}ur,
 c'est de demander au Seigneur de le sonder pour vous. 

Que pouvez-vous faire si Dieu vous révèle que votre c\oe{}ur est partagé?
 Suivez l'exemple de David, qui a demandé à Dieu d'unir son c\oe{}ur.
 Ou bien, relevez le défi de Josué qui exhortait\frcolon{} \punct{deux-points}
 \Og Choisissez aujourd'hui qui vous voulez servir\dots{}
 Moi et ma maison, nous servirons l'Éternel \punct{point en trop} \Fg{}
 (\ibibleverse{Jos}(24:15)). 

\dvrule

\dvprayer{
Père, aide-nous à mener une vie qui Te soit véritablement consacrée.
 Enlève tout ce qui nous conduit à ne T'offrir qu'un c\oe{}ur partagé. 
}{\DlNdJ}


%%%%%%%%%%%%
% 17 juin
% M 2013/06/30
%%%%%%%%%%%%

\jrnlday{Lutter avec~Dieu}

\mainthemeindex{lutte}
\themeindex{benediction@bénédiction}
\themeindex{grace@grâce}
\mainthemeindex{reddition}

\dvquote{
Dans le sein maternel, (Jacob) saisit son frère par le talon,
 et dans son âge mûr, il lutta avec Dieu.\\
 Il lutta avec un ange, et fut vainqueur.\\
 Il pleura et lui demanda grâce.
}{\ibibleverse{Os}(12:3-4)}

\lettrine{C}{omment} un homme pourrait-il bien lutter avec Dieu
 \ocadr ou avec un ange de Dieu \fcadr{} et être vainqueur ?
 Ça ne semble pas possible.
 Cependant, Osée nous dit que c'est bien ce qui s'est passé,
 et l'explication se trouve à la fin de la phrase\frcolon{}
 \Og Il [Jacob] pleura et lui demanda grâce. \Fg{}
 En d'autres termes, Jacob a dit\frcolon{} \punct{deux-points}
 \Og S'il-te-plaît, ne pars pas sans me bénir! \Fg{}

Le combat de Jacob a eu lieu la nuit avant la rencontre qu'il devait
 faire avec son frère, Ésaü, qui était en marche à la tête de \numprint{400}~hommes.
 Cette fois là, aucune combine ou manigance ne pouvait plus le sauver.
 Jacob savait qu'il n'avait plus de sortie de secours.
 Il a donc lutté avec l'Ange, s'est accroché à Lui et L'a supplié de le bénir. 

\dvbox{
Il est temps de capituler et de pleurer devant le~Seigneur. 
}

Quelquefois, je vois des hommes qui luttent avec Dieu et je me demande\frcolon{} \punct{deux-points}
 \Og Qu'est-ce qu'il va falloir pour les faire tomber à genoux ? \Fg{}
 Dans le cas de Jacob, il a fallu qu'il soit estropié.
 Quand il ne lui a plus été possible de continuer à courir, de manigancer,
 de tricher pour se sortir d'affaire, il s'est finalement rendu à Dieu. 

Peut-être êtes-vous à la même place que Jacob où vous ne savez plus
 ce que vous allez faire. Quand vous arrivez à la fin de vos propres forces
 et que vous capitulez devant Dieu, c'est le point à partir duquel
 vous commencez à découvrir la puissance de Dieu à l'\oe{}uvre dans votre vie. 

\dvrule

\dvprayer{
Père,nous pouvons voir que la victoire arrive quand nous nous rendons à Toi.\\
 Rends-nous infirmes, Seigneur, si c'est ce qu'il faut pour nous amener
 à une complète reddition. 
}{\DlNdJ}


%%%%%%%%%%%%
% 18 juin
% M 2013/06/30
%%%%%%%%%%%%

\jrnlday{Le jeûne}

\mainthemeindex{jeune@jeûne}
\themeindex{chair}
\themeindex{esprit}
\themeindex{plaisir}

\dvquote{
Maintenant encore, \ocadr Oracle de l'Éternel \fcadr{}\\
 Revenez à Moi de tout votre c\oe{}ur, avec des jeûnes,
 avec des pleurs et des lamentations!
}{\ibibleverse{Jl}(2:12)}

\lettrine{N}{otre} nature a deux côtés\frcolon{}
 le côté de la chair et le côté de l'esprit.
 Nous nourrissons le côté de la chair assez systématiquement
 \ocadr généralement un minimum de trois fois par jour,
 avec des en-cas généreux entre les trois.
 Mais nous ne sommes généralement pas aussi sytématiques
 pour nourrir notre esprit. 

Le jeûne est une fa\c{c}on de priver la chair et de fortifier l'esprit.
 Si vous nourrissiez votre esprit aussi systématiquement
 que vous nourrissez votre chair, vous seriez spirituellement gras.
 Le jeûne renverse le processus habituel
 \ocadr il prive la chair et nourrit l'esprit. 

\dvbox[0.85\textwidth]{
Le jeûne n'était pas juste une discipline de l'Ancien Testament. 
}

Ésaïe a dit au peuple qu'il ne jeûnait pas de la fa\c{c}on dont Dieu
 voulait qu'il le fasse\frcolon{}
 \Og Au jour de votre jeûne vous trouvez du plaisir \Fg{}
 (\ibibleverse{Is}(58:3)). En d'autres termes, ils recherchaient du plaisir
 alors qu'ils auraient dû se centrer sur Dieu. C'est comme prendre
 un jour pour jeûner mais passer la journée entière à regarder la télé.
 Notre chair a terriblement envie de distractions, mais le but d'un jeûne
 est de remettre la chair à sa place et de rechercher Dieu. 

Quand vous jeûnez, que vous passez du temps dans la prière
 et la lecture de la Parole de Dieu, votre esprit devient de plus en plus
 fort et l'emprise de la chair est affaiblie. Nous nous retrouvons alors
 à remporter des victoires dans nos batailles spirituelles. 

Le jeûne n'était pas juste une discipline de l'Ancien Testament.
 Dieu appelle Son peuple à jeûner aujourd'hui.
 Notre monde est dans une situation désespérée.
 Nous qui aimons Dieu devons renoncer à notre chair,
 rechercher Son visage et fortifier nos esprits pour le combat. 

\dvrule

\dvprayer{
Père, nous désirons nous détourner de notre péché
 pour que Tu puisses manifester Ta grâce et miséricorde envers nous.
 Aide-nous à Te rechercher de tout notre c\oe{}ur
 \ocadr en jeûnant et en portant le deuil devant Toi
 pour notre nation et nos familles. 
}{\Amen}


%%%%%%%%%%%%
% 19 juin
% M 2013/06/30
%%%%%%%%%%%%

\jrnlday{Les choisis}

\mainthemeindex{election@élection}
\themeindex{Dieu!communion avec \sim}
\themeindex{Dieu!amour de \sim}
\themeindex{responsabilite@responsabilité}

\dvquote{
Je vous ai choisis, vous seuls parmi toutes les familles de la terre;
 c'est pourquoi je vous demanderai compte de tous vos errements.
 Deux hommes marchent-ils ensemble, sans en avoir convenu?
}{\ibibleverse{Am}(3:2-3)}

\lettrine{D}{ieu} a choisi Israël \ocadr non pas parce que ses gens
 étaient vertueux ou puissants. Il les a choisis simplement parce
 qu'Il les aime. Et c'est par l'intermédiaire de la nation d'Israël
 que Dieu a choisi d'amener le Sauveur dans le monde. 

Comme Israël, nous avons aussi été choisis non pas en raison de nos propres
 vertus ou parce que nous aurions quelque chose dont Dieu aurait besoin.
 Dieu nous a choisis parce qu'Il nous aime et qu'Il désire
 la communion avec nous. 

\dvbox{
La communion implique l'unité. 
}

Pour avoir la communion avec Dieu, vous devez être en accord avec Dieu.
 Cela signifie que vous devez suivre les règles qu'Il a établies.
 Nous ne sommes pas en position de négocier des arrangements
 avec le Seigneur du genre\frcolon{}
 \Og Si Tu fais \c{c}a pour moi, alors je ferai \c{c}a pour Toi. \Fg{}
 Nous sommes des pécheurs impuissants qui nous tenons en faillite devant Dieu.
 Et Il nous considère avec compassion et dit\frcolon{} \punct{deux-points}
 \Og Je vous ai choisis. \Fg{}

Être le peuple choisi de Dieu crée une grande responsabilité.
 Nous connaissons Ses voies. Nous connaissons Ses exigences.
 Nous marchons dans la lumière et nous comprenons Sa Parole;
 nous serons donc jugés selon des critères plus sévères.
 Si nous choisissons de pécher, Dieu va nous châtier
 parce que nous sommes à Lui. 

Quelle bénédiction que d'être aimés par Dieu,
 choisis par Dieu et pardonnés par Dieu!
 Marchons ensemble dans l'unité avec Lui et montrons Sa majesté
 à un monde qui ne le connaît pas encore. 

\dvrule

\dvprayer{
Père, comme nous Te sommes reconnaissants d'être à Toi!\\
 Puissions-nous ne jamais dévier de la vérité,
 mais puissions-nous marcher avec Toi dans l'obéissance. 
}{\DlNdJ}


%%%%%%%%%%%%
% 20 juin
% M 2013/06/30
%%%%%%%%%%%%

\jrnlday{Prépare-toi à~rencontrer~Dieu}

\themeindex{lutte}
\themeindex{avocat}

\dvquote{
C'est pourquoi voilà ce que Je vais te faire, ô Israël!\\
 Et puisque Je vais te faire cela,\\
 prépare-toi à la rencontre de ton Dieu, ô Israël!
}{\ibibleverse{Am}(4:12)}

\lettrine{D}{ieu} avait prévenu les gens d'Israël à maintes reprises.
 Il les avait frappés de calamités. Il essayait de les faire revenir
 du chemin de destruction qu'ils avaient emprunté.
 Malgré tout cela, ils ne voulaient pas revenir à Lui.
 Le jugement était maintenant imminent.
 Le prophète Amos lance alors un avertissement au peuple\frcolon{}
 \Og Prépare-toi à rencontrer ton Dieu! \Fg{}

Cette fois, cependant, la rencontre n'aurait pas pour but l'amour,
 la communion et l'unité. Ce serait une confrontation. 

\dvbox{
C'est de la pure folie que d'aller à la rencontre de
 Dieu en tant qu'adversaire. 
}

Les Écritures déclarent\frcolon{} \punct{deux-points}
 \Og Malheur à qui conteste avec Celui qui l'a fa\c{c}onné! \Fg{}
 (\ibibleverse{Is}(45:9)).
 Comment pourriez-vous jamais espérer remporter une bataille contre Dieu?
 Cependant certains d'entre vous luttent avec Dieu.
 Dieu vous a parlé de ces choses qui dans votre vie sont malhonnêtes,
 polluées ou qui Lui déplaisent. Le Saint-Esprit vous a déclarés coupables,
 mais vous avez ignoré Sa voix et vous vous êtes opposés à Lui. 

Un beau jour, vous allez rencontrer Dieu et vous devez être prêts.
 Rien n'est caché à Ses yeux. Il connaît toutes les pensées,
 toutes les motivations du c\oe{}ur. Nous sommes coupables.
 Mais nous sommes aussi bénis \ocadr car nous avons un Avocat qui nous
 représente devant le Père.
 \Og Qui les condamnera? Le Christ-Jésus est celui qui est mort;
 bien plus, Il est ressuscité, Il est à la droite de Dieu,
 et Il intercède pour nous! \Fg{} (\ibibleverse{Rm}(8:34)). 

Loué soit Dieu pour Son Fils! 

\dvrule

\dvprayer{
Seigneur puissions-nous avoir la joie de Te rencontrer chaque jour
 \ocadr la joie de Ta présence, la puissance de Ton Esprit
 et la gloire de marcher avec Toi. 
}{\DlNdJ}


%%%%%%%%%%%%
% 21 juin
% M 2013/06/30
%%%%%%%%%%%%

\jrnlday{Possédez vos~possessions}

\mainthemeindex{benediction@bénédiction}
\themeindex{victoire}
\themeindex{decouragement@découragement}
\themeindex{anxiete@anxiété}

\dvquote{
Mais sur la montagne de Sion il y aura des rescapés,\\
 ils seront saints,
 et la maison de Jacob reprendra ses possessions.
}{\ibibleverse{Ab}(1:17)}

\lettrine{D}{ieu} avait promis de donner à Israël tout le territoire
 jusqu'à la Mer Méditerranée. Il leur appartenait déjà
 \ocadr tout ce qu'ils avaient à faire c'était de poser le pied dessus
 et de le revendiquer. Malheureusement, ils n'ont pas réussi à le faire.
 Ils ont limité ce que Dieu leur aurait donné. 

Paul a décrit Dieu comme \Og Celui qui, par la puissance qui agit en nous,
 peut faire infiniment au-delà de tout ce que nous demandons ou pensons \Fg{}
 (\ibibleverse{Ep}(3:20)). Il n'y a pas de limites à ce que Dieu peut faire. 

\dvbox[0.93\textwidth]{
Nous sommes ceux qui limitons nos bénédictions. 
}

Avez-vous le repos, la joie et la paix que Dieu a promis?
 Si ce n'est pas le cas, vous ne possédez pas vos possessions.
 Vous n'avez pas réussi à prendre tout le terrain qui vous appartient.
 Trop de chrétiens sont découragés, anxieux et vaincus au lieu de connaître
 la plénitude de joie dont Dieu veut voir Ses enfants faire l'expérience.
 Si vous n'avez pas cette joie indescriptible, alors vous n'avez pas possédé
 toutes vos possessions. Elle est là; elle est à votre disposition
 \ocadr c'est juste que vous ne l'avez pas encore saisie. 

Le Seigneur parle du jour qui vient où Jacob possèdera ses possessions.
 Nous pouvons posséder nos possessions aujourd'hui.
 Ne vous contentez pas d'une vie médiocre en-dessous du niveau
 où Dieu veut que vous viviez. Poursuivons les choses de Dieu,
 poursuivons la puissance de Dieu;
 poursuivons une onction de l'Esprit de Dieu. 

Croyez\dots{} et prenez ce qui vous appartient. 

\dvrule

\dvprayer{
Seigneur, nous Te remercions pour Tes précieuses promesses.\\
 Aide-nous à dépendre de Ta force et à revendiquer ces endroits
 de victoire que Tu as pour nous. 
}{\Amen}


%%%%%%%%%%%%
% 22 juin
% M 2013/06/30
%%%%%%%%%%%%

\jrnlday{Vanités mensongères}

\mainthemeindex{vanite@vanité}
\mainthemeindex{mensonge}
\themeindex{misericorde@miséricorde}
\themeindex{Dieu!appel de \sim}
\themeindex{Dieu!omnipresence@omniprésence de \sim}
\themeindex{Dieu!voies de \sim}
\themeindex{Dieu!misericorde@miséricorde de \sim}
\themeindex{Dieu!sagesse de \sim}
\themeindex{Dieu!volonte@volonté de \sim}

\dvquote{
Ceux qui s'adonnent à des vanités mensongères abandonnent
 leur propre miséricorde.
}{\ibibleverse{Jon}(2:8) \KJF}

\lettrine{A}{près} trois jours et trois nuits misérables
 \ocadr des jours et des nuits qu'il avait attirés sur lui-même \fcadr{},
 Jonas a retrouvé la terre ferme et s'est enquis du Seigneur.
 Puis, il a déclaré ce qu'il avait appris dans le ventre de la baleine\frcolon{}
 \Og Ceux qui s'adonnent à des vanités mensongères abandonnent
 leur propre miséricorde. \Fg{} \punct{Point après guillemet} 

Quelles étaient les vanités mensongères auxquelles Jonas s'était adonné?
 Premièrement, il croyait qu'il pouvait échapper à l'appel de Dieu;
 deuxièmement, il croyait pouvoir trouver un endroit où Dieu
 ne se trouvait pas; et en dernier lieu, il croyait que ses voies
 étaient meilleures que les voies de Dieu. 

\dvbox{
Ceux qui s'adonnent à des vanités mensongères se retrouvent dans
 des situations exécrables. Ne les imitez pas.
}

Dieu est extrêmement miséricordieux et extrêmement sage.
 Ses voies sont toujours les meilleures et Il ne désire rien d'autre
 que du bon dans votre vie.
 \Og Je connais, moi, les desseins que je forme à votre sujet,
 \ocadr oracle de l'Éternel \fcadr{}, desseins de paix et non de malheur,
 afin de vous donner un avenir fait d'espérance \Fg{} (\ibibleverse{Jr}(29:11)).
 Il n'y a rien de mieux pour vous que de trouver la volonté de Dieu
 et de la faire. Vous ne pouvez pas faire mieux; c'est la vie à son optimum.
 C'est là où vous trouvez la pleine satisfaction et le contentement total,
 et vous ne serez jamais pleinement content ou satisfait
 avec quelque chose de moindre. 

Ne vous créez pas de problèmes. Soumettez-vous\grammar{tiret manquant}
 plutôt à Celui qui vous aime.
 Vous pouvez y arriver volontairement ou vous pouvez y arriver en\dots{}
 dégageant une odeur de poisson \ocadr mais vous allez finir par aller
 là où Dieu veut que vous alliez. 

\dvrule

\dvprayer{
Père, nous Te remercions pour le contentement qui vient de nous en remettre
 à Ta volonté pour nos vies et de trouver ainsi la joie,
 la paix et la satisfaction de savoir que Tu contrôles toutes choses. 
}{\DlNdJ}


%%%%%%%%%%%%
% 23 juin
% M 2013/08/14
%%%%%%%%%%%%

\jrnlday{Pas de restrictions}

\themeindex{Dieu!Esprit de \sim}
\themeindex{droiture}

\dvquote{
L'Esprit du Seigneur est-Il restreint?
 Est-ce que ce sont là Ses actes?
 Mes Paroles ne font-elles pas du bien à celui qui marche avec droiture?
}{\ibibleverse{Mi}(2:7) \KJF}

\lettrine{L}{'Esprit} de Dieu est-Il restreint?
 Est-Il limité à celui par qui Il peut parler, ou par ce qu'Il peut dire?
 Peut-on blâmer l'Esprit pour ce qui arrive à la nation
 ou pour l'apparente faiblesse de l'Église aujourd'hui? La réponse est non. 

\dvbox{
Dieu n'a aucunes restrictions, aucunes limitations. 
}

L'Esprit de Dieu n'est restreint en aucune fa\c{c}on.
 Si nous sommes faibles et inefficaces, pour Lui, en revanche,
 c'est tout le contraire!
 Le Saint-Esprit parle à nos c\oe{}urs et déclare\frcolon{} \punct{deux-points}
 \Og Ni par la puissance, ni par la force, mais par mon Esprit \Fg{}
 (\ibibleverse{Za}(4:6)).
 Et bien que je sois déconcerté par la condition du monde, Il ne l'est pas.
 Il a toujours autant le monde à c\oe{}ur et désire toujours autant attirer
 les perdus à Jésus-Christ. 

Vous pouvez être un instrument entre Ses mains quelles que soient
 vos propres capacités. Dieu peut vous utiliser sans un diplôme
 de séminaire et sans doctorat. Si vous vous soumettez à Lui,
 Il peut et va vous utiliser pour communiquer la vérité de Jésus-Christ
 à un monde qui se meurt. Nous pouvons être la lumière
 et le sel de la terre si nous nous soumettons à l'Esprit de Dieu
 et Le laissons accomplir Ses plans à travers nos vies. 

Puisse Dieu \oe{}uvrer par notre intermédiaire pour amener ce monde à genoux devant Lui.

\dvrule

\dvprayer{
Père, nous Te remercions pour le pouvoir du Saint-Esprit d'effectuer
 des changements dans les vies et les sociétés.
 Dieu, nous prions pour un réveil spirituel dans notre nation,
 par l'intermédiaire de l'onction et de la communication
 de puissance de Ton Esprit. 
}{\DlNdJ}


%%%%%%%%%%%%
% 24 juin
% M 2013/08/14
%%%%%%%%%%%%

\jrnlday{Qui est un Dieu comme~notre~Dieu?}

\themeindex{Dieu!pardon de \sim}
\themeindex{Dieu!colere@colère de \sim}
\themeindex{Dieu!misericorde@miséricorde de \sim}

\dvquote{
Qui est Dieu comme Toi, qui pardonnes la faute et passes sur la transgression
 en faveur du reste de ton patrimoine?
 Il~n'entretient pas sa colère à jamais,
 car il prend plaisir à la fidélité.
}{\ibibleverse{Mi}(7:18)}

\lettrine{L}{e} psalmiste décrit Dieu comme un Dieu \Og compatissant
 et qui fait grâce, lent à la colère et riche en bienveillance \Fg{};
 \punct{Problème de guillemets}
 comme un Dieu \Og qui ne nous traite pas selon nos péchés et ne nous rétribue pas
 selon nos fautes.\Fg{} (\ibibleverse{Ps}(103:8,10)).
 Sa compassion est si grande qu'Il a dit à Ézéchiel de transmettre au peuple
 ce message\frcolon{} Dis-leur\frcolon{} \punct{deux-points} \Og Je suis vivant!
 \ocadr oracle du Seigneur, l'Éternel \fcadr{}, ce que je désire,
 ce n'est pas que le méchant meure, c'est qu'il change de conduite
 et qu'il vive. Revenez, revenez de vos mauvaises voies.
 Pourquoi devriez-vous mourir, maison d'Israël?\Fg{} (\ibibleverse{Ez}(33:11)). 

\dvbox{
Il n'y pas de dieu comme notre Dieu. 
}

Qui est un dieu comme notre Dieu, qui est si miséricordieux,
 si prêt et si disposé à pardonner nos transgressions?
 Qui est un dieu comme notre Dieu, qui nous guide sur le chemin
 de la justice et jette nos péchés dans les profondeurs de la mer? 

Il n'y a pas de dieu comme notre Dieu. Il est le Père des orphelins.
 Il est le Conquérant du péché et de la mort. Il est le Rocher
 vers lequel nous pouvons courir nous réfugier.
 Il est Celui qui pardonne toutes nos offenses. 

Comme nous sommes bénis de servir le Dieu de miséricorde! 

\dvrule

\dvprayer{
Père, nous Te remercions pour Ton caractère, pour Ton amour, Ta compassion,
 Ta miséricorde, Ton pardon, Ta justice et Ta puissance.
 Oh Seigneur, nous t'aimons! Attire-nous à Toi et guide-nous. 
}{\DlNdJ}


%%%%%%%%%%%%
% 25 juin
% M 2013/08/14
%%%%%%%%%%%%

\jrnlday{Confiez-vous en~l'Éternel}

\themeindex{confiance}

\dvquote{
L'Éternel est bon; il est une forteresse au jour de la détresse, et il connaît ceux qui se confient en lui.
}{\ibibleverse{Na}(1:7) \Ostervald}

\lettrine{D}{ieu} n'a jamais promis qu'une fois que vous avez confié
 votre vie à Jésus, Il vous garderait à l'abri de la douleur et des problèmes.
 Il a seulement promis que quand vous passerez par des épreuves
 \ocadr remarquez bien qu'il est dit \Og quand \Fg{} et non pas \Og si \Fg{} \fcadr
 Il vous accompagnera dans la douleur et vous aidera à traverser l'épreuve. 

\dvbox{
Quand des épreuves vous arrivent, où allez-vous chercher de l'aide? 
}

Faites-vous de l'introspection en espérant trouver de la puissance
 en vous-mêmes? Comptez-vous sur les autres pour vous aider?
 Ou essayez-vous de vous débrouiller tout seul à l'aide de petits dictons du genre\frcolon{}
 \Og Ce que vous ne pouvez pas éviter, vous devez l'endurer \Fg{},
 \Og Garde le sourire et accroche-toi \Fg{},
 \Og Je crois que je peux, je crois que je peux, je crois que je peux \Fg{}
 comme la petite locomotive rouge du livre pour enfants? 

Tout le monde a des ennuis. C'est un aspect inévitable de la vie!
 Nous vivons dans un monde en rébellion contre Dieu, et en conséquence
 nous vivons dans un monde rempli de chaos. Au jour de la détresse,
 il n'y a qu'une forteresse vers laquelle vous pouvez vous tourner
 pour trouver la sécurité, c'est le Seigneur Jésus-Christ. 

Ne vous tournez pas vers les autres. Ne vous tournez vers vos propres
 ressources. N'allez pas penser que \Og positiver \Fg{} à coup de petits dictons
 optimistes va vous sortir d'affaires.
 Mettez, au contraire, votre confiance dans le Seigneur.
 Il vous accompagnera dans l'épreuve et quand vous en serez sortis,
 vous verrez la sagesse de Son plan. 

\dvrule

\dvprayer{
Père, quand la tempête fait rage autour de nous, attire-nous à Toi.
 Nous savons que tu es bon et plein d'amour.\\
 Apprends-nous à te faire
 confiance au sein de notre confusion. 
}{\DlNdJ}


%%%%%%%%%%%%
% 26 juin
% M 2013/08/14
%%%%%%%%%%%%

\jrnlday{Quand Dieu garde le~silence}

\themeindex{silence}
\themeindex{Dieu!voies de \sim}

\dvquote{
Jusques à quand, Éternel, appellerai-je au secours sans que Tu écoutes,
 Te crierai-je\frcolon{} Violence! sans que Tu sauves?
}{\ibibleverse{Ha}(1:2)}

\lettrine{I}{l} est difficile de supporter le silence de Dieu
 \ocadr dans ces moments où nous prions mais où Il ne semble pas
 entendre car rien ne change. 

Habacuc était frustré. Israël avait tourné le dos à Dieu.
 La nation était dans un chaos moral.
 Inquiet, Habacuc s'est écrié\frcolon{} \punct{deux-points}
 \Og Dieu, Tu ne fais rien! \Fg{} \punct{double ponctuation}
 Le Seigneur lui a répondu\frcolon{} \Og Voyez, regardez parmi les nations,
 soyez dans la stupéfaction et la stupeur, car quelqu'un est en train
 d'accomplir en votre temps une \oe{}uvre que vous ne croiriez pas
 si on la racontait \Fg{} (\ibibleverse{Ha}(1:5)).

\dvbox{
Les voies de Dieu ne sont pas nos voies. 
}

Quelquefois, nous nous méprenons en prenant le silence de Dieu
 pour de l'indifférence; la réalité, c'est que Dieu est toujours à l'\oe{}uvre. 

Dieu a alors donné une prophétie à Habacuc concernant les jours dangereux
 à venir. Il a décrit le jugement qui allait venir par la main des Chaldéens,
 qu'Il préparait dans ce but. Et puis, exhortant Habacuc à rester ferme
 dans sa confiance envers Lui, Dieu a encouragé le prophète avec des paroles
 qui ont continué à encourager les croyants au cours des siècles\frcolon{}
 \Og le juste vivra par sa foi \Fg{} (\ibibleverse{Ha}(2:4)). 

Il vous semble que peut-être aujourd'hui, Dieu ne vous entend pas, ou pire
 \ocadr qu'il ne se soucie pas de votre situation. Mais je peux vous assurer,
 d'après la Parole de Dieu, qu'Il entend vraiment,
 qu'Il s'inquiète vraiment et qu'Il est vraiment à l'\oe{}uvre pour votre compte
 \ocadr même s'Il semble silencieux. Détournez votre regard de vos problèmes.
 Ignorez les circonstances. Déchargez vous, au contraire, de vos soucis
 sur Celui qui vous aime et qui a promis de satisfaire tous vos besoins. 

\dvrule

\dvprayer{
Père, amène-nous à ce degré de foi où, quand nous ne comprenons pas
 ce que Tu es en train de faire, nous pouvons dire\frcolon{}
 \Og Seigneur, continue à le faire. \Fg{}
}{\Amen}


%%%%%%%%%%%%
% 27 juin
%%%%%%%%%%%%

\jrnlday{Dieu se réjouit à~notre~sujet}

\dvquote{
Le Seigneur, ton Dieu, au milieu de toi est puissant;
 Il~sauvera, Il se réjouira à ton sujet avec joie;
 Il~se reposera dans Son amour;\\
 Il~se réjouira à ton sujet avec des chants.
}{\ibibleverse{So}(3:17) \KJF}

\lettrine{N}{ous} qui sommes parents savons ce que c'est que de se réjouir
 au sujet de nos enfants. Dieu a les mêmes sentiments.
 Il se réjouit avec allégresse au sujet de ceux qui sont devenus
 Ses enfants par la foi en Jésus-Christ. Sophonie nous dit que
 \Og Dieu se réjouit à ton sujet avec joie. \Fg{}

Puis il nous dit que Dieu \Og se reposera dans Son amour pour toi. \Fg{}
 Quelle merveilleuse image! La Bible nous dit que Dieu nous a tant aimés
 qu'Il a donné Sa vie pour nous. \ibiblephantom{Jn}(3:16)
 L'amour de Jésus pour nous est sans fin, éternel, immuable.
 En dépit des afflictions, des épreuves, des difficultés,
 Son amour est imperturbable. Laissez les implications qui en résultent
 pénétrer \grammar{pénétrer} au fond de votre c\oe{}ur.
 Nous devons méditer sur la profondeur de la passion de Dieu pour nous. 

\dvbox[0.92\textwidth]{
Quand nous comprenons le grand amour de notre Père pour nous,
 nos c\oe{}urs sont apaisés et nous pouvons trouver le repos dans cette~vérité. 
}

Puis, Sophonie nous dit que Dieu va \Og se réjouir à notre sujet
 avec des chants. \Fg{} Pouvez-vous imaginer le Seigneur vous chantant
 un chant d'amour, accompagné par un orchestre céleste? 

La vie est incertaine. Il y a beaucoup de choses que nous ne savons pas.
 Mais grâce à la Parole de Dieu, nous savons la seule chose
 qui compte vraiment\frcolon{} le Seigneur nous aime. Que les tempêtes se déchaînent,
 que les flèches volent. Quoiqu'il advienne, nous sommes aimés. 

Loué soit Dieu pour Son Fils! 

\dvrule

\dvprayer{
Père, nous sommes émerveillés par Ton amour.\\
 Comme nous attendons ce jour où Tu demeureras au milieu de nous,
 notre Dieu puissant, notre Roi des rois ! 
}{\DlNdJ}


%%%%%%%%%%%%
% 28 juin
%%%%%%%%%%%%

\jrnlday{Le désir des~Nations}

\dvquote{
Car ainsi dit le Seigneur des armées\frcolon{} [\dots{}]\\
 J'ébranlerai les cieux et la terre, et la mer et la terre sèche.\\
 Et j’ébranlerai toutes les nations, et le Désir de Toutes les Nations
 viendra, et je remplirai cette maison de gloire, dit le \Seigneur des armées.
}{\ibibleverse{Ag}(2:6-7) \KJF}

\lettrine{C}{eux} qui étaient revenus de la captivité à Babylone
 s'étaient découragé dans la reconstruction du temple et,
 au lieu de persévérer, ils s'étaient mis à réparer leurs propres maisons.
 Aggée les a réprimandés en leur disant que c'était la raison
 pour laquelle ils ne prospéraient pas. Mais le peuple s'est plaint
 de ce que le nouveau temple semblait insignifiant et pitoyable
 en comparaison de la splendeur du temple de Salomon.
 Aussi, Aggée les encourage-t-il en leur disant qu'un beau jour,
 la gloire de l'Éternel viendra de nouveau remplir le temple. 

\dvbox{
Dieu ne va pas rester pour toujours à ne rien faire
 en laissant la méchanceté régner. 
}

L'Écriture décrit le jour du jugement de Dieu
 \ocadr La Grande Tribulation \fcadr{} où les cieux,
 la terre et les mers seront ébranlés. Après cela,
 le \Og Désir de Toutes les Nations \Fg{} arrivera. 

Jésus a dit aux juifs\frcolon{} \punct{deux-points, majuscule}
 \Og Vous ne me verrez plus désormais jusqu'à ce que vous disiez\frcolon{}
 Béni soit celui qui vient au nom du Seigneur! \Fg{} (\ibibleverse{Mt}(23:39)).
 Bien que l'Écriture nous apprenne que \Og le Désir de Toutes les Nations \Fg{}
 va venir établir un royaume de justice, de paix, de joie, les Juifs
 l'ont rejetté la première fois qu'Il est venu.
 Ils n'avaient pas de désir pour Lui. 

Et vous? Jésus est-Il le désir de votre c\oe{}ur aujourd'hui?
 Attendez-vous avec impatience l'instauration de Son royaume? 

\dvrule

\dvprayer{
Père, nous Te remercions pour la promesse de Ton royaume à venir.
 Dans un monde rempli de péché et de chaos, nous nous raccrochons
 à Ta Parole et nous attendons que Jésus-Christ,
 le Désir de Toutes les Nations, revienne pour établir Ton royaume de justice. 
}{\Amen}


%%%%%%%%%%%%
% 29 juin
%%%%%%%%%%%%

\jrnlday{Chante et réjouis-toi}

\dvquote{
\Og Chante et réjouis-toi, ô fille de Sion car voici,\\
 Je viens, et je demeurerai au milieu de toi, dit le Seigneur.\\
 Et beaucoup de nations se joindront au Seigneur en ce jour-là,
 et deviendront Mon peuple. \Fg{}
}{\ibibleverse{Za}(2:10-11)}

\suggest{Les guillemets sont-ils nécessaires?}

\lettrine{Q}{uand} le peuple est rentré de la captivité à Babylone,
 Jérusalem était dévastée et ravagée. Mais Zacharie les a encouragés
 en leur annon\c{c}ant qu'un jour, Dieu choisirait à nouveau Jérusalem.
 L'Éternel des Armées va de nouveau revenir prendre Sa place en Judée
 et demeurera au milieu de Son peuple. 

\dvbox{
La seule chose qui suscite le chant et la joie dans mon c\oe{}ur,
 c'est la promesse de la venue de Jésus-Christ. 
}

La perspective de la venue du Seigneur devrait inspirer des chants
 et de la joie dans nos c\oe{}urs. Quand je considère le monde aujourd'hui,
 je ne vois pas beaucoup de sujets de chants ou de joie.
 Mon c\oe{}ur pleure devant le mal que nous voyons abonder autour de nous.
 Rien dans ce monde n'inspire le genre de joie auquel le prophète
 se réfère dans ce passage. 

Chaque jour qui passe nous rapproche de vingt-quatre heures de ce jour
 glorieux, ce jour de chants et de joie, où le Seigneur viendra demeurer
 au milieu de Son peuple. 

Nous vivons des temps d'une importance capitale, où les évènements
 se conjuguent pour annoncer le retour de Jésus. Aussi longtemps que Dieu
 nous prête vie, nous devons avertir les autres des jours mauvais à venir,
 mais nous devons aussi proclamer le jour glorieux qui arrive
 \ocadr le jour du retour du Seigneur. 

Loué soit Dieu pour Son Fils! 

\dvrule

\dvprayer{
Seigneur, puisse Ton royaume arriver et Ta volonté être faite sur la terre
 comme aux cieux.\\
 Nos c\oe{}urs rêvent de ce jour où Tu viendras demeurer
 dans la paix et la justice au milieu de Ton peuple! 
}{\Amen}


%%%%%%%%%%%%
% 30 juin
%%%%%%%%%%%%

\jrnlday{Déplacer les~montagnes}

\dvquote{
Ce n'est ni par la puissance, ni par la force, mais c'est par Mon Esprit,
 dit l'Éternel des armées.\\
 Qui es-tu, grande montagne?\\
 Devant Zorobabel, tu seras aplanie.
}{\ibibleverse{Za}(4:6-7)}

\lettrine{L}{es juifs} n'étaient pas préparés à ce qu'ils ont trouvé
 en revenant de captivité. La dévastation était extrême.
 Là où le temple de Dieu s'était un jour dressé,
 il n'y avait plus qu'une énorme montagne de gravats. 

Une montagne énorme! Le découragement s'était abattu sur tous ceux
 qui avaient essayé de dégager les décombres. Mais la parole de Dieu
 fut adressée par Zacharie à Zorobabel, l'homme qui dirigeait
 la reconstruction du temple.
 \Og Qui es-tu, grande montagne? Tu seras aplanie! \Fg{}

Souvent, quand nous considérons une tâche que le Seigneur nous a confiée,
 elle peut nous sembler insurmontable. Nous nous demandons\frcolon{} \punct{deux-points}
 \Og Comment une telle tâche pourra-t-elle jamais être accomplie? \Fg{}
 Mais la réponse que Dieu nous fait est la même que celle faite à Zorobabel\frcolon{}
 \Og Cette montagne ne peut pas être déplacée par vos propres forces.
 Mon Esprit seul peut accomplir ce travail. \Fg{}

Vous vous trouvez peut-être confrontés à une montagne aujourd'hui
 \ocadr un problème de finances, un problème de santé,
 un problème de relations. Vous avez tout fait pour essayer
 d'enlever les décombres, mais vous êtes fatigués, épuisés, prêts d'abandonner. 

\dvbox{
Il est bon d'atteindre vos limites.
 C'est là que vous allez finalement tendre la main à Dieu. 
}

Demandez de l'aide au Saint-Esprit. Laissez Dieu se montrer fort pour vous.
 Laissez tomber votre pelle, prenez du recul et regardez
 Le déplacer cette montagne. 

\dvrule

\dvprayer{
Seigneur, nous venons à Toi aujourd'hui, las et épuisés devant
 la tâche à accomplir.\\
 Nous avons besoin de Ton aide.\\
 Nous avons besoin de la puissance de Ton Esprit Saint pour attaquer
 cette tâche.\\
 Remplis-nous, Seigneur. Sois notre force aujourd'hui. 
}{\DlNdJ}


