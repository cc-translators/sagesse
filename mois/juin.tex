\dvmonth{Juin}

%%%%%%%%%%%%
% 1er juin
%%%%%%%%%%%%

\dvday{L\ap{}Appel aux Mères}

\dvquote{
Laisse couler tes larmes comme un torrent! Ne te donne aucun répit,
 et que la pupille de ton œil n’ait pas de repos!
 Lève-toi, lance une clameur au début des veilles de la nuit!
 Répands ton cœur comme de l’eau devant la face du Seigneur!
 Lève tes mains vers Lui pour la vie de tes enfants qui défaillent
 de faim à tous les coins de rues.
}{\ibibleverse{Lm}(2:18-19)}

\typo{Guillemet fermant non ouvert}

\dvlettrine{S}{e croyant à la fois forts et sophistiqués,}
 les gens d'Israël ont essayé d'éliminer Dieu de leur vie nationale.
 Ils considéraient qu'Il n'était plus nécessaire.
 Le résultat? Babylone a saccagé Jérusalem. 

Israël fait le deuil de sa cité ravagée.
 Jérémie, voyant que des temps désespérés exigent des mesures
 désespérées, délivre un appel à toutes les mères de la nation. 

\dvbox{
Ne sous-estimez jamais l'influence d'une mère. Quand elle est remuée,
 elle peut utiliser cette puissance pour guider ses enfants
 vers ce qui est juste. 
}

Notre pays a un besoin désespéré de ce genre d'influence.
 Considérez l'atmosphère de notre système d'enseignement public.
 En 1963, la Cour Suprême a estimé qu'il était illégal de prier
 ou de lire la Bible dans nos écoles.
 Depuis cette décision, que s'est-il passé?
 Sur une moyenne nationale, les notes des tests d'aptitude
 à l'enseignement supérieur ont régulièrement baissé,
 la violence et le crime se sont largement répandus
 et la consommation de drogue a explosé. 

La seule vraie réponse pour ne plus déplaire à Dieu
 est le retour à des principes qui plaisent à Dieu
 \ocadr ces principes sur lesquels notre nation a été fondée.
 Non seulement les mères, mais les pères aussi doivent enseigner
 à nos enfants ce qui est bien et mal \ocadr du point de vue de Dieu.
 Nous devons enseigner à la prochaine génération à rechercher Dieu
 par la prière. 

Voici une suggestion radicale~: cette semaine, éteignons nos télévisions
 et tournons nos visages vers Dieu.
 Prions pour notre nation et pour nos enfants. 

\dvrule

\dvprayer{
Incite-nous, Père, à invoquer Ton nom.
 Amène-nous à prier pour nos enfants, pour nos écoles et pour notre nation. 
}{\Amen}


%%%%%%%%%%%%
% 2 juin
%%%%%%%%%%%%

\dvday{De la Dépression à l\ap{}Espoir}

\dvquote{
Voici ce que je veux repasser en mon cœur, Ce pourquoi j’espère~:
 C’est que la bienveillance de l’Éternel n’est pas épuisée,
 Et que ses compassions ne sont pas à leur terme;
 Elles se renouvellent chaque matin. Grande est ta fidélité !
 L’Éternel est mon partage, dit mon âme;
 C’est pourquoi je veux m’attendre à lui.
 L’Éternel est bon pour qui espère en lui, Pour celui qui le cherche.
 Il est bon d’attendre en silence Le salut de l’Éternel.
}{\ibibleverse{Lm}(3:21-26)}

\punct{Guillemet fermant sans guillemet ouvrant}

\dvlettrine{C}{omme c'est généralement le cas,} c'est de penser à lui-même
 qui a provoqué la dépression de Jérémie.
 Mais quand il a changé sa façon de penser
 \ocadr détournant son attention de lui-même pour la porter sur Dieu \fcadr{}
 la dépression s'en est allée. Quand il s'est souvenu de Dieu,
 qu'il a médité sur Son caractère, la paix et l'espoir ont rempli ses pensées. 

\dvbox{
Quand le désespoir se dissipe, l'espoir le remplace. 
}

Au lieu de penser à lui-même, Jérémie a pensé à la nature de Dieu.
 Il a pensé à la bienveillance de l'Éternel qui n'est pas épuisée,
 et grâce à laquelle nous ne sommes pas consumés.
 Il a pensé à la fidélité de Dieu. Dieu fait toujours exactement
 ce qu'Il dit qu'Il va faire, et Il est capable de faire ressortir du bien
 de la situation la plus désastreuse. Il a pensé au fait que Dieu
 est notre partage. Que pourrions-nous vouloir de plus?
 De quoi de plus pourrions-nous avoir besoin?
 Il a pensé à la bonté de Dieu,
 qui fait concourir toutes choses à notre bien. 

Puis Jérémie a conclu qu'il est bon \og d'attendre en silence
 le salut de l'Éternel. \fg{}

Êtes-vous inquiets aujourd'hui? Changez vos pensées.
 Né méditez pas sur votre douleur \ocadr méditez sur votre Sauveur.
 Fixez vos yeux sur Jésus et souvenez-vous de Son caractère.
 Son amour ne vous a jamais fait défaut.
 Ses compassions se renouvellent chaque matin. 

\dvrule

\dvprayer{
Père, aide-nous à fixer nos yeux sur Toi plutôt que sur nous-mêmes.
 Rempli nos cœurs de Ton espoir et Ton amour aujourd'hui. 
}{\Amen}


%%%%%%%%%%%%
% 3 juin
%%%%%%%%%%%%

\dvday{L\ap{}Appel de Dieu}

\dvquote{
Va trouver les déportés, les fils de ton peuple; tu leur parleras et,
 qu'ils écoutent ou qu'ils ne prennent pas garde, tu leur diras~:
 Ainsi parle le Seigneur, l'Éternel.
}{\ibibleverse{Ez}(3:11)}

\dvlettrine{D}{ieu nous a tous appelés} à un ministère
 dans le corps de Christ. Tous ne sont pas apôtres,
 tous ne sont pas prophètes, tous ne sont pas pasteurs ou enseignants,
 mais nous avons tous une place de service.
 Dieu a une tâche et un but pour vous.
 Et s'Il vous appelle à un ministère, vous pouvez être certains
 qu'Il va vous équiper pour le mener à bien. 

\dvbox{
Dieu ne vous demandera jamais de faire quelque chose sans vous donner
 aussi les moyens de réussir.
}

Ézéchiel a reçu l'ordre de parler au peuple~: \punct{deux-points, majuscule}
 \og Qu'ils écoutent, ou qu'ils n'écoutent pas. \fg{}
 Le ministère n'est pas facile. Ça peut souvent être très décourageant.
 Les gens ne vont pas toujours répondre à ce que vous leur dites.
 Quelquefois les gens se mettent en colère avec vous parce que vous osez
 leur annoncer les choses du Seigneur. 

Mais quand votre vie sera finie, il n'y a aura qu'une chose qui comptera 
 \ocadr et c'est votre obéissance à la volonté de Dieu.
 Tout ce que vous aurez pu faire pour vous-mêmes, toute la richesse
 que vous aurez pu amasser, toutes les prouesses qui seront décrites
 dans votre rubrique nécrologique seront futiles. 

Consacrez votre vie à ce qui compte vraiment
 \ocadr ce qui est éternel. Vivez votre vie pour Lui. 

\dvrule

\dvprayer{
Père, merci pour la puissance du Saint-Esprit qui nous donne les moyens
 de répondre à notre appel. Puissions-nous toujours servir d'une position
 de compassion, de compréhension et d'amour.
 Et cependant, Seigneur, puissions-nous ne pas être coupables de manquer
 d'avertir les gens de ce que Tu as dit. 
}{\DlNdJ}

