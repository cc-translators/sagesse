\dvmonth{Janvier}

%%%%%%%%%%%%%%%
% 1er janvier
%%%%%%%%%%%%%%%

\dvday{Créé à son image}

\dvquote{
Dieu dit~: \og Faisons l'homme à Notre image selon Notre ressemblance\dots{} \fg{} 
Dieu créa l'homme à Son image~:
 Il le créa à l'image de Dieu, homme et femme Il les créa.
}{\bibleverse{Gen}(1:26-27)}


\dvlettrine{P}{ourquoi} Dieu voudrait-Il nous créer à Son image? 

Il a agi ainsi parce qu'Il désirait une relation pleine d'amour et de sens avec
 nous. Parce qu'Il est amour, Dieu nous a donné la capacité d'aimer. Parce qu'Il
 s'auto-détermine, Il nous a donné la capacité de choisir. Il nous a rendus
 capables d'aimer et de choisir, afin que nous puissions Lui amener du plaisir
 --- non seulement en recevant Son amour, mais aussi en choisissant de L'aimer
 en retour. 

Beaucoup de gens choisissent de ne pas aimer Dieu. Ce faisant, ils ratent le but
 même de leur vie et ne devraient pas être surpris quand la vie leur paraît vide,
 frustrante et sans valeur.

\dvbox{La capacité de choix est une chose merveilleuse qui peut aussi avoir des
 conséquences dévastatrices.}

L'exercice du libre arbitre de l'homme dans le jardin a amené la chute qui l'a
 éloigné de l'image de Dieu. Cela a entraîné la mort spirituelle et a conduit
 l'homme à vivre comme un animal --- absorbé seulement par la satisfaction de ses
 appétits physiques. Mais l'homme ne pourra jamais être satisfait en ne se
 revendiquant que du royaume animal. Il ne peut seulement trouver le contentement
 réel que lorsqu'il a une relation avec Dieu. 

Jésus est venu dans ce but. Il a pris notre péché et il est mort à notre place.
 Il nous offre maintenant un autre choix --- nous pouvons vivre une vie selon
 l'Esprit en communion avec Lui, ou nous pouvons continuer la vie selon la chair,
 qui mène à l'aliénation et à la mort. 

Choisissez en faisant bien attention, car Dieu respecte et honore nos choix. 

\dvrule

\dvprayer{Père, merci car en contemplant Ta gloire, nous sommes transformés en Ton
 image par l'Esprit Saint qui agit en nous.}{Amen}


%%%%%%%%%%%%%%%
% 2 janvier
%%%%%%%%%%%%%%%

\dvday{Rupture}

\dvquote{
Alors Dieu dit à Noé~: \og J'ai décidé de mettre fin à tous les êtres vivants~; 
car la terre est pleine de violence à cause d'eux~;
Je vais donc les détruire avec la terre.\fg{}
}{\bibleverse{Gen}(6:13)}


\dvlettrine{D}{ans} les temps qui ont précédé le déluge, l'humanité s'était
 détournée de Dieu et vivait comme s'Il n'existait pas. Les mœurs étaient
 dissolues. Les pensées des hommes étaient corrompues et ils avaient adopté
 des pratiques sexuelles anormales. La violence recouvrait la terre et les
 hommes avaient commencé à vivre comme des animaux. C'est dans ces conditions,
 que Dieu ayant regardé la situation, déclara~: \og Mon esprit ne débattra pas
 toujours avec l’homme \fg{} (\bibleverse{Gen}(6:3) --- version
 \og King James Française \fg{}). 

Vous voyez, il arrive finalement un temps où la patience de Dieu arrive à son terme. 

Nous vivons, nous aussi, dans un monde rempli de violence et de corruption.
 Ce n'est qu'une question de temps avant que Dieu ne redise~: \og Ça suffit! \fg{} 
 \bibleverse{Matt}(24:37) nous dit~: \og Comme aux jours de Noé ainsi en sera-t-il
 à l'avènement du Fils de l'homme.\fg{} Comme Il l'a fait la première fois, Dieu
 dira à Son peuple~: \og Venez, entrez dans l'arche. \fg{} Merci à Dieu, car Il
 nous a offert cet endroit de refuge. Cet arche, aujourd'hui, c'est Jésus-Christ.


\dvbox{Vous faites partie, soit du système qui corrompt le monde et le fait couler,
 soit du système qui surnagera quand l'autre aura coulé.}

Dieu a établi des normes absolues définissant ce qui est juste, et nous devons
 nous soumettre à Son autorité, parce que Jésus revient bientôt pour juger le
 monde selon la justice.

\dvrule

\dvprayer{Père, puissions-nous prendre en compte ces choses que nous avons
 entendues, de peur que nous nous en éloignions en partant à la dérive.
 Aide-nous à effectuer une rupture décisive et définitive d'avec ce monde
 corrompu, afin que nous puissions tenir debout et être comptés avec ceux
 qui T'aiment et Te servent.}{Dans le Nom de Jésus, Amen}


%%%%%%%%%%%%%%%
% 3 janvier
%%%%%%%%%%%%%%%

\dvday{Toutes les Promesses de Dieu}

\dvquote{
Tant que la terre subsistera, les semailles et la moisson, le froid et la chaleur,
 l'été et l'hiver, le jour et la nuit ne cesseront pas.
}{\bibleverse{Gen}(8:22)}

\dvlettrine[ante=\og]{T}{ant} que la Terre subsistera\dots{} \fg{} Ce qui est
 impliqué par cette expression, c'est que la Terre ne va pas durer éternellement~;
 mais dans les mots qui suivent, Dieu promet que tant que la Terre durera, il y
 aura certaines constantes~: les semailles et la moisson, le froid et la chaleur,
 l'été et l'hiver, le jour et la nuit. Nous faisons confiance à ces promesses de
 Dieu. Je ne perds pas de sommeil la nuit à m'inquiéter de savoir si une nouvelle
 aube poindra ou non le lendemain matin. 

\dvbox{Si nous pouvons complètement croire à quelques unes des promesses de Dieu,
 pourquoi avons-nous du mal à croire à toutes les promesses de Dieu?}

Jésus a dit \og Je ne te délaisserai pas, ni ne t'abandonnerai \fg{}, mais
 quelques fois nous nous inquiétons de ce qu'Il puisse précisément le faire.
 Tout comme je fais entièrement confiance à Dieu pour qu'Il continue d'assurer
 la rotation de la Terre, je devrais être confiant que Dieu va pourvoir à tous
 mes besoins.

Je pense que Dieu voulait que ces promesses nous rappellent Sa fidélité
 quotidiennement. Chaque matin quand nous nous levons et que nous voyons la
 clarté du jour apparaître au dehors, nous devrions dire~: \og Eh bien, qu'est-ce
 que Dieu est fidèle à Ses promesses! \fg{} Chaque soir quand le soleil se
 couche, nous devrions nous exclamer~: \og Dis-donc, Dieu tient vraiment Ses
 promesses, n'est-ce pas? \fg{}

Dieu va tenir Ses promesses que nous y croyions ou pas. Quand nous avons des
 problèmes, nous pouvons manifester de l'anxiété et de la peur ou nous pouvons
 adopter une attitude de victoire et de joie. Nous pouvons avoir la victoire
 parce que Dieu va prendre soin de nous. Il l'a promis.

Dieu est fidèle --- vous pouvez parier tout ce que vous voulez là-dessus. 

\dvrule

\dvprayer{
Merci Père, pour Tes promesses inviolables.
 Puissions-nous commencer à
 vivre dans une confiance glorieuse parce que Tu es sur le trône.
}{Dans le Nom de Jésus, Amen.}


%%%%%%%%%%%%%%%
% 4 janvier
%%%%%%%%%%%%%%%

\dvday{Pas de Compromis}

\dvquote{
Térah prit son fils Abram, son petit-fils Loth, fils d'Harân
 et sa belle-fille Saraï, femme de son fils Abram.
 Ils sortirent ensemble d'Our-des-Chaldéens,
 pour se rendre au pays de Canaan.
 Ils arrivèrent à Harân et ils y habitèrent\dots{}
 L'Éternel dit à Abram~: Va-t'en de ton pays, de ta patrie
 et de la maison de ton père, vers le pays que je te montrerai.
}{\bibleverse{Gen}(11:31)}

\dvlettrine{D}{ésirant} développer une relation intime avec Abram,
 Dieu l'invita à faire trois choses~: quitter son pays idolâtre,
 quitter sa famille et aller dans le pays que Dieu lui montrerait.
 Mais Abram n'a pas complètement obéi.
 Il quitta Our avec son père et son neveu mais n'alla pas directement
 à Canaan.
 Ils restèrent aussi à l'intérieur des frontières de la plaine babylonienne
 --- toujours à l'intérieur du pays qu'il devait quitter. 

\dvbox{
Obéir à moitié, c'est désobéir et cela a des conséquences tragiques. 
}

Dieu est resté muet tout le temps où Abram a séjourné à Harân
 et n'a pas reparlé avant qu'Abram n'atteigne Canaan.
 Dans sa désobéissance, Abram a perdu le privilège d'une étroite
 communion avec Dieu. 

Qu'en est-il de vous ? Vivez-vous à Harân ?
 N'avez-vous jamais complètement obéi ? Avez-vous fait des compromis ?
 Vous accrochez-vous à de vieilles habitudes ?
 Dieu vous appelle à renoncer à vous-mêmes, à fuir le mal,
 à prendre votre croix et à suivre Jésus.
 Il vous invite à venir à Lui pour recevoir la bénédiction,
 la promesse, la communion et une relation intime
 --- parce qu'Il vous aime. 

\dvrule

\dvprayer{
Père, alors que Ton Esprit Saint nous conduit sur Ton chemin,
 nous prions pour aller jusqu'au bout de ce que Tu désires pour nous.
 Seigneur, puissions-nous dans l'obéissance
 prendre notre croix ---~quel que soit le prix à payer.
}{Amen.}


%%%%%%%%%%%%%%%
% 5 janvier
%%%%%%%%%%%%%%%

\dvday{Quand Dieu Attend}

\dvquote{
Lorsqu'Abram fut âgé de 99 ans, l'Éternel apparut à Abram et lui dit~:
 Je suis le Dieu Tout Puissant. Marche devant ma face et sois intègre.
 J'établirai Mon alliance avec toi, et je te multiplierai à l'extrême.
}{\bibleverse{Gn}(17:1-2)}

\dvlettrine{D}{ieu} parla à Abraham quand celui-ci avait quatre-vingt
 six ans\dots{} puis attendit treize autres années avant de lui reparler.
 Pourquoi, selon vous, Dieu a-t-Il attendu aussi longtemps? 

Dieu choisit souvent d'attendre que nous ayons épuisé toutes les ressources
 possibles qui s'offrent à nous.
 Il attend que nous ayons atteint un point de désespoir.
 C'est là qu'Abraham était arrivé, enfin.
 Si jamais Dieu avait parlé plus tôt, Abraham aurait probablement essayé
 d'aider Dieu à accomplir Sa promesse. 

Lors de ma formation de maître-nageur, j'ai appris que quand une personne
 est en train de se noyer, vous ne vous approchez pas d'elle tout de suite.
 Si vous approchez trop près d'elle quand elle a encore des forces
 et qu'elle s'agite beaucoup dans tous les sens,
 elle risque de vous faire couler avec elle.
 Vous devez attendre qu'elle se fatigue.
 Je crois que Dieu agit en suivant le même principe.
 Tant que nous nous agitons dans tous les sens, Dieu attend. 

\dvbox{
Quand, finalement, nous cessons de nous débattre,
 Dieu peut commencer à mettre en œuvre Son plan glorieux. 
}

\og Oh, que les hommes célèbrent l'Éternel pour
 Ses œuvres merveilleuses! \fg{}
 Bien souvent, nous avons tendance à donner des explications rationnelles
 ou dire que nous avons trop de chance quand les choses finissent
 par s'arranger.
 Nous ne donnons pas à Dieu le crédit du travail qu'Il a accompli. 

Vous vous trouvez, peut-être, dans une situation qui semble sans issue.
 Vous avez fait de votre mieux et pourtant vous avez échoué.
 Réalisez, qu'en fait, les choses se présentent très bien!
 Car les impossibilités de l'homme sont les opportunités de Dieu.
 Dieu va faire pour vous ce que vous ne pouvez pas faire pour vous-mêmes! 

\dvrule

\dvprayer{
Père, merci pour Ta patience.
 Nous te confions toutes nos situations désespérées.
 Amène la gloire à Ton nom, Seigneur,
 afin que les hommes Te célèbrent.
}{Amen.}


%%%%%%%%%%%%%%%
% 6 janvier
%%%%%%%%%%%%%%%

\dvday{Pas de Quoi se Plaindre !}

\dvquote{
Rébecca se leva avec ses jeunes servantes~;
 elles montèrent sur les chameaux et suivirent l'homme.
}{\bibleverse{Gen}(24:61)}

\dvlettrine{R}{ébecca} a vraiment fait preuve de foi
 en montant sur ce chameau et en s'engageant à faire un voyage
 de plus de 600 kilomètres.
 Elle devait complètement dépendre du serviteur d'Isaac
 et le croire sur parole concernant la richesse
 et la gloire du royaume qui l'attendait. 

Voyager à dos de chameau est une véritable gageure.
 Le chameau se déplace de façon saccadée.
 Si vous n'arrivez à suivre le rythme de son balancement
 en vous relaxant, les secousses de son mouvement vont vous tuer.

La vie chrétienne n'est pas facile non plus.
 \og Bien-aimés, ne soyez pas surpris de la fournaise qui sévit
 parmi vous pour vous éprouver, comme s'il vous arrivait
 quelque chose d'étrange \fg{} (\bibleverse{IPet}(4:12)).
 En chevauchant cette bête inconfortable, vous allez être éprouvés,
 secoués et mis dans des situations inconfortables.
 Pendant la traversée du désert,
 il est même possible que vous vous découragiez. 

Après être arrivée à sa destination,
 croyez-vous que Rébecca ait crié \og sale bête ! \fg{}
 et qu'elle ait battu son chameau ? Je ne le pense pas.
 Elle lui a probablement fait des caresses en disant,
 \og Ce voyage sur ton dos a été vraiment pénible,
 mais tu as fait ton travail. Tu m'as amené à mon maître et seigneur. \fg{}

\dvbox{
Les épreuves de la vie sont les instruments par lesquels Dieu nous amène à lui. 
}

Un jour, le voyage va s'achever. Les chameaux sont destinés à nous amener
 à une totale dépendance envers Jésus.
 Très bientôt, nous allons lever les yeux et voir notre Seigneur arriver.
 Et quand je descendrai de mon chameau, je ne vais pas le battre.
 Il m'aura amené à mon Seigneur et je remercierai Dieu
 d'être arrivé à destination. 

\dvrule

\dvprayer{
Père, merci pour les difficultés qui nous font venir à Toi.
 Aide-nous à considérer nos épreuves comme Tes instruments.
 Et puissions-nous ne pas nous lasser du voyage.
}{Dans le Nom de Jésus, Amen.}


%%%%%%%%%%%%%%%
% 7 janvier
%%%%%%%%%%%%%%%

\dvday{Restaurer une Relation Endommagée}

\dvquote{
[Jacob] lui-même passa devant eux et se prosterna en terre sept fois,
 jusqu'à ce qu'il soit tout près de son frère. Ésaü courut à sa rencontre~;
 il l'embrassa, se jeta à son cou et lui donna un baiser~; et ils pleurèrent.
}{\bibleverse{Gn}(33:3-4)}

\dvbox{
Quand l'amour s'est refroidi, peut-il être restauré ? Les Écritures disent que oui. 
}

\dvlettrine{L}{a première} chose que Jacob a faite a été de prier.
 La première chose que vous devriez faire est de prier et de demander
 à Dieu de changer le cœur de la personne qui s'est éloignée de vous.
 Quand nous faisons cela, Dieu nous montre souvent en quoi nous devons changer.

La nature de Jacob, c'était de prendre, non pas de donner.
 Mais après avoir prié, il décida d'envoyer des cadeaux à Ésaü.
 Trouver des façons de bénir la personne d'avec qui vous êtes séparés. 

Il a ensuite édifié son frère en le qualifiant de \og seigneur \fg{}.
 Trop souvent nos relations tournent au vinaigre parce que nous démolissons
 au lieu d'édifier. 

Puis Jacob a échangé en faisant part à Ésaü des évènements
 qui s'étaient produits durant leur séparation.
 Être ouvert ---~faire part de ses pensées et de ses sentiments~---
 guérit les relations malades.
 Puis Jacob et Ésaü se sont touchés. Ils se sont embrassés. 

Dieu peut-Il guérir vos relations ? Oui. Si vous suivez l'exemple de Jacob
 ---~en priant, en bénissant, en édifiant, en échangeant et en touchant
 celui ou celle dont vous êtes séparés, Dieu peut prendre cette relation
 qui a tourné au vinaigre et l'adoucir, la restaurer. 

\dvrule

\dvprayer{
Père, apprends-nous par Ta Parole comment
 nous aimer les uns les autres,
 afin que nous puissions voir un parfum glorieux remplir nos vies. 
}{Dans le Nom de Jésus, Amen.}

