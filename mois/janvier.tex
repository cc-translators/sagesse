\dvmonth{Janvier}

%%%%%%%%%%%%%%%
% 1er janvier
%%%%%%%%%%%%%%%

\dvday{Créé à son image}

\dvquote{
Dieu dit~: \Og Faisons l'homme à Notre image selon Notre ressemblance\dots{} \Fg{} 
Dieu créa l'homme à Son image~: \\
 Il le créa à l'image de Dieu, homme et femme Il les créa.
}{\ibibleverse{Gn}(1:26-27)}


\lettrine{P}{ourquoi} Dieu voudrait-Il nous créer à Son image? \\[1ex]
Il a agi ainsi parce qu'Il désirait une relation pleine d'amour et de sens avec
 nous. Parce qu'Il est amour, Dieu nous a donné la capacité d'aimer. Parce qu'Il
 s'autodétermine, Il nous a donné la capacité de choisir. Il nous a rendus
 capables d'aimer et de choisir, afin que nous puissions Lui amener du plaisir
 \ocadr{}non seulement en recevant Son amour, mais aussi en choisissant de L'aimer
 en retour. 

Beaucoup de gens choisissent de ne pas aimer Dieu. Ce faisant, ils ratent le but
 même de leur vie et ne devraient pas être surpris quand la vie leur paraît vide,
 frustrante et sans valeur.

\dvbox{La capacité de choix est une chose merveilleuse qui peut aussi avoir des
 conséquences dévastatrices.}

L'exercice du libre arbitre de l'homme dans le jardin a amené la chute qui l'a
 éloigné de l'image de Dieu. Cela a entraîné la mort spirituelle et a conduit
 l'homme à vivre comme un animal \ocadr{}absorbé seulement par la satisfaction de ses
 appétits physiques. Mais l'homme ne pourra jamais être satisfait en ne se
 revendiquant que du royaume animal. Il ne peut seulement trouver le contentement
 réel que lorsqu'il a une relation avec Dieu. 

Jésus est venu dans ce but. Il a pris notre péché et il est mort à notre place.
 Il nous offre maintenant un autre choix \ocadr{}nous pouvons vivre une vie selon
 l'Esprit en communion avec Lui, ou nous pouvons continuer la vie selon la chair,
 qui mène à l'aliénation et à la mort. 

Choisissez en faisant bien attention, car Dieu respecte et honore nos choix. 

\dvrule

\dvprayer{
Père, merci car en contemplant Ta gloire, nous sommes transformés
 en Ton image par l'Esprit Saint qui agit en nous.
}{\Amen}

\suggest{ou Esprit-Saint?}


%%%%%%%%%%%%%%%
% 2 janvier
%%%%%%%%%%%%%%%

\dvday{Rupture}

\dvquote{
Alors Dieu dit à Noé~: \Og J'ai décidé de mettre fin à tous les êtres vivants; 
car la terre est pleine de violence à cause d'eux;
Je vais donc les détruire avec la terre.\Fg{}
}{\ibibleverse{Gn}(6:13)}


\lettrine{D}{ans} les temps qui ont précédé le déluge, l'humanité s'était
 détournée de Dieu et vivait comme s'Il n'existait pas. Les m\oe{}urs étaient
 dissolues. Les pensées des hommes étaient corrompues et ils avaient adopté
 des pratiques sexuelles anormales. La violence recouvrait la terre et les
 hommes avaient commencé à vivre comme des animaux. C'est dans ces conditions,
 que Dieu ayant regardé la situation, déclara~: \Og Mon esprit ne débattra pas
 toujours avec l'homme \Fg{} (\ibibleverse{Gn}(6:3) \KJF{}). 
 \footnote{Ostervald: \Og Mon esprit ne contestera point dans l'homme à toujours. \Fg}

Vous voyez, il arrive finalement un temps où la patience de Dieu arrive à son terme. 

Nous vivons, nous aussi, dans un monde rempli de violence et de corruption.
 Ce n'est qu'une question de temps avant que Dieu ne redise~: \Og Ça suffit! \Fg{} 
 \ibibleverse{Mt}(24:37) nous dit~: \Og Comme aux jours de Noé ainsi en sera-t-il
 à l'avènement du Fils de l'homme.\Fg{} Comme Il l'a fait la première fois, Dieu
 dira à Son peuple~: \Og Venez, entrez dans l'arche. \Fg{} Merci à Dieu, car Il
 nous a offert cet endroit de refuge. Cet arche, aujourd'hui, c'est Jésus-Christ.


\dvbox{Vous faites partie, soit du système qui corrompt le monde et le fait couler,
 soit du système qui surnagera quand l'autre aura coulé.}

Dieu a établi des normes absolues définissant ce qui est juste, et nous devons
 nous soumettre à Son autorité, parce que Jésus revient bientôt pour juger le
 monde selon la justice.

\dvrule

\dvprayer{Père, puissions-nous prendre en compte ces choses que nous avons
 entendues, de peur que nous nous en éloignions en partant à la dérive.\\
 Aide-nous à effectuer une rupture décisive et définitive d'avec ce monde
 corrompu, afin que nous puissions tenir debout et être comptés avec ceux
 qui T'aiment et Te servent.}{\DlNdJ}

%%%%%%%%%%%%%%%
% 3 janvier
%%%%%%%%%%%%%%%

\dvday{Toutes les promesses\\ de Dieu}

\dvquote{
Tant que la terre subsistera, les semailles et la moisson, le froid et la chaleur,
 l'été et l'hiver, le jour et la nuit ne cesseront pas.
}{\ibibleverse{Gn}(8:22)}

\lettrine[ante=\Og]{T}{ant} que la Terre subsistera\dots{} \Fg{} Ce qui est
 impliqué par cette expression, c'est que la Terre ne va pas durer éternellement;
 mais dans les mots qui suivent, Dieu promet que tant que la Terre durera, il y
 aura certaines constantes~: les semailles et la moisson, le froid et la chaleur,
 l'été et l'hiver, le jour et la nuit. Nous faisons confiance à ces promesses de
 Dieu. Je ne perds pas de sommeil la nuit à m'inquiéter de savoir si une nouvelle
 aube poindra ou non le lendemain matin. 

\dvbox{Si nous pouvons complètement croire à quelques unes des promesses de Dieu,
 pourquoi avons-nous du mal à croire à toutes les promesses de Dieu?}

Jésus a dit~: \punct{deux-points}
 \Og Je ne te délaisserai pas, ni ne t'abandonnerai \Fg{}, mais
 quelques fois nous nous inquiétons de ce qu'Il puisse précisément le faire.
 Tout comme je fais entièrement confiance à Dieu pour qu'Il continue d'assurer
 la rotation de la Terre, je devrais être~confiant que Dieu va pourvoir à tous
 mes besoins.

Je pense que Dieu voulait que ces promesses nous rappellent Sa fidélité
 quotidiennement. Chaque matin quand nous nous levons et que nous voyons la
 clarté du jour apparaître au dehors, nous devrions dire~: \Og Eh bien, qu'est-ce
 que Dieu est fidèle à Ses promesses! \Fg{} Chaque soir quand le soleil se
 couche, nous devrions nous exclamer~: \Og Dis-donc, Dieu tient vraiment Ses
 promesses, n'est-ce pas? \Fg{}

Dieu va tenir Ses promesses que nous y croyions ou pas. Quand nous avons des
 problèmes, nous pouvons manifester de l'anxiété et de la peur ou nous pouvons
 adopter une attitude de victoire et de joie. Nous pouvons avoir la victoire
 parce que Dieu va prendre soin de nous. Il l'a promis.

Dieu est fidèle \ocadr{}vous pouvez parier tout ce que vous voulez là-dessus. 

\dvrule

\dvprayer{
Merci Père, pour Tes promesses inviolables.
 Puissions-nous commencer à
 vivre dans une confiance glorieuse parce que Tu es sur le trône.
}{\DlNdJ}


%%%%%%%%%%%%%%%
% 4 janvier
%%%%%%%%%%%%%%%

\dvday{Pas de compromis}

\dvquote{
Térah prit son fils Abram, son petit-fils Loth, fils d'Harân
 et sa belle-fille Saraï, femme de son fils Abram.
 Ils sortirent ensemble d'Our-des-Chaldéens,
 pour se rendre au pays de Canaan.
 Ils arrivèrent à Harân et ils y habitèrent\dots{}
 L'Éternel dit à Abram~: Va-t'en de ton pays, de ta patrie
 et de la maison de ton père, vers le pays que je te montrerai.
}{\ibibleverse{Gn}(11:31)}

\lettrine{D}{ésirant} développer une relation intime avec Abram,
 Dieu l'invita à faire trois choses~: quitter son pays idolâtre,
 quitter sa famille et aller dans le pays que Dieu lui montrerait.
 Mais Abram n'a pas complètement obéi.
 Il quitta Our avec son père et son neveu mais n'alla pas directement
 à Canaan.
 Ils restèrent aussi à l'intérieur des frontières de la plaine babylonienne
 \ocadr{}toujours à l'intérieur du pays qu'il devait quitter. 

\dvbox{
Obéir à moitié, c'est désobéir et cela a des conséquences tragiques. 
}

Dieu est resté muet tout le temps où Abram a séjourné à Harân
 et n'a pas reparlé avant qu'Abram n'atteigne Canaan.
 Dans sa désobéissance, Abram a perdu le privilège d'une étroite
 communion avec Dieu. 

Qu'en est-il de vous ? Vivez-vous à Harân ?
 N'avez-vous jamais complètement obéi ? Avez-vous fait des compromis ?
 Vous accrochez-vous à de vieilles habitudes ?
 Dieu vous appelle à renoncer à vous-mêmes, à fuir le mal,
 à prendre votre croix et à suivre Jésus.
 Il vous invite à venir à Lui pour recevoir la bénédiction,
 la promesse, la communion et une relation intime
 \ocadr{} parce qu'Il vous aime. 

\dvrule

\dvprayer{
Père, alors que Ton Esprit Saint nous conduit sur Ton chemin,
 nous prions pour aller jusqu'au bout de ce que Tu désires pour nous.
 Seigneur, puissions-nous dans l'obéissance
 prendre notre croix \ocadr{}quel que soit le prix à payer.
}{\Amen}

\suggest{ou Esprit-Saint?}


%%%%%%%%%%%%%%%
% 5 janvier
%%%%%%%%%%%%%%%

\dvday{Quand Dieu attend}

\dvquote{
Lorsqu'Abram fut âgé de 99 ans, l'Éternel apparut à Abram et lui dit~:
 Je suis le Dieu Tout Puissant. Marche devant ma face et sois intègre.
 J'établirai Mon alliance avec toi, et je te multiplierai à l'extrême.
}{\ibibleverse{Gn}(17:1-2)}

\lettrine{D}{ieu} parla à Abraham quand celui-ci avait quatre-vingt
 six ans\dots{} puis attendit treize autres années avant de lui reparler.
 Pourquoi, selon vous, Dieu a-t-Il attendu aussi longtemps? 

Dieu choisit souvent d'attendre que nous ayons épuisé toutes les ressources
 possibles qui s'offrent à nous.
 Il attend que nous ayons atteint un point de désespoir.
 C'est là qu'Abraham était arrivé, enfin.
 Si jamais Dieu avait parlé plus tôt, Abraham aurait probablement essayé
 d'aider Dieu à accomplir Sa promesse. 

Lors de ma formation de maître-nageur, j'ai appris que quand une personne
 est en train de se noyer, vous ne vous approchez pas d'elle tout de suite.
 Si vous approchez trop près d'elle quand elle a encore des forces
 et qu'elle s'agite beaucoup dans tous les sens,
 elle risque de vous faire couler avec elle.
 Vous devez attendre qu'elle se fatigue.
 Je crois que~Dieu agit en suivant le même principe.
 Tant que nous nous agitons dans tous les sens, Dieu attend. 

\dvbox{
Quand, finalement, nous cessons de nous débattre,
 Dieu peut commencer à mettre en \oe{}uvre Son plan glorieux. 
}

\Og Oh, que les hommes célèbrent l'Éternel pour
 Ses \oe{}uvres merveilleuses! \Fg{}
 Bien souvent, nous avons tendance à donner des explications rationnelles
 ou dire que nous avons trop de chance quand les choses finissent
 par s'arranger.
 Nous ne donnons pas à Dieu le crédit du travail qu'Il a accompli. 

Vous vous trouvez, peut-être, dans une situation qui semble sans issue.
 Vous avez fait de votre mieux et pourtant vous avez échoué.
 Réalisez, qu'en fait, les choses se présentent très bien!
 Car les impossibilités de l'homme sont les opportunités de Dieu.
 Dieu va faire pour vous ce que vous ne pouvez pas faire pour vous-mêmes! 

\dvrule

\dvprayer{
Père, merci pour Ta patience.
 Nous te confions toutes nos situations désespérées.
 Amène la gloire à Ton nom, Seigneur,
 afin que les hommes Te célèbrent.
}{\Amen}


%%%%%%%%%%%%%%%
% 6 janvier
%%%%%%%%%%%%%%%

\dvday{Pas de quoi se plaindre !}

\dvquote{
Rébecca se leva avec ses jeunes servantes;
 elles montèrent sur les chameaux et suivirent l'homme.
}{\ibibleverse{Gn}(24:61)}

\lettrine{R}{ébecca} a vraiment fait preuve de foi
 en montant sur ce chameau et en s'engageant à faire un voyage
 de plus de 600 kilomètres.
 Elle devait complètement dépendre du serviteur d'Isaac
 et le croire sur parole concernant la richesse
 et la gloire du royaume qui l'attendait. 

Voyager à dos de chameau est une véritable gageure.
 Le chameau se déplace de fa\c{c}on saccadée.
 Si vous n'arrivez à suivre le rythme de son balancement
 en vous relaxant, les secousses de son mouvement vont vous tuer.

La vie chrétienne n'est pas facile non plus.
 \Og Bien-aimés, ne soyez pas surpris de la fournaise qui sévit
 parmi vous pour vous éprouver, comme s'il vous arrivait
 quelque chose d'étrange \Fg{} (\ibibleverse{IP}(4:12)).
 En chevauchant cette bête inconfortable, vous allez être éprouvés,
 secoués et mis dans des situations inconfortables.
 Pendant la traversée du désert,
 il est même possible que vous vous découragiez. 

Après être arrivée à sa destination,
 croyez-vous que Rébecca ait crié \Og sale bête ! \Fg{}
 et qu'elle ait battu son chameau ? Je ne le pense pas.
 Elle lui a probablement fait des caresses en disant,
 \Og Ce voyage sur ton dos a été vraiment pénible,
 mais tu as fait ton travail. Tu m'as amené à mon maître et seigneur. \Fg{}

\dvbox{
Les épreuves de la vie sont les instruments par lesquels Dieu nous amène à lui. 
}

Un jour, le voyage va s'achever. Les chameaux sont destinés à nous amener
 à une totale dépendance envers Jésus.
 Très bientôt, nous allons lever les yeux et voir notre Seigneur arriver.
 Et quand je descendrai de mon chameau, je ne vais pas le battre.
 Il m'aura amené à mon Seigneur et je remercierai Dieu
 d'être arrivé à destination. 

\dvrule

\dvprayer{
Père, merci pour les difficultés qui nous font venir à Toi.
 Aide-nous à considérer nos épreuves comme Tes instruments.
 Et puissions-nous ne pas nous lasser du voyage.
}{\DlNdJ}


%%%%%%%%%%%%%%%
% 7 janvier
%%%%%%%%%%%%%%%

\dvday{Restaurer une relation endommagée}

\dvquote{
[Jacob] lui-même passa devant eux et se prosterna en terre sept fois,
 jusqu'à ce qu'il soit tout près de son frère. Ésaü courut à sa rencontre;
 il l'embrassa, se jeta à son cou et lui donna un baiser; et ils pleurèrent.
}{\ibibleverse{Gn}(33:3-4)}

\dvbox{
Quand l'amour s'est refroidi, peut-il être restauré ? Les Écritures disent que oui. 
}

\lettrine{L}{a première chose} que Jacob a faite a été de prier.
 La première chose que vous devriez faire est de prier et de demander
 à Dieu de changer le c\oe{}ur de la personne qui s'est éloignée de vous.
 Quand nous faisons cela, Dieu nous montre souvent en quoi nous devons changer.

La nature de Jacob, c'était de prendre, non pas de donner.
 Mais après avoir prié, il décida d'envoyer des cadeaux à Ésaü.
 Trouver des fa\c{c}ons de bénir la personne d'avec qui vous êtes séparés. 

Il a ensuite édifié son frère en le qualifiant de \Og seigneur \Fg{}.
 Trop souvent nos relations tournent au vinaigre parce que nous démolissons
 au lieu d'édifier. 

Puis Jacob a échangé en faisant part à Ésaü des évènements
 qui s'étaient produits durant leur séparation.
 Être ouvert \ocadr{}faire part de ses pensées et de ses sentiments\fcadr{}
 guérit les relations malades.
 Puis Jacob et Ésaü se sont touchés. Ils se sont embrassés. 

Dieu peut-Il guérir vos relations ? Oui. Si vous suivez l'exemple de Jacob
 \ocadr{}en priant, en bénissant, en édifiant, en échangeant et en touchant
 celui ou celle dont vous êtes séparés, Dieu peut prendre cette relation
 qui a tourné au vinaigre et l'adoucir, la restaurer. 

\dvrule

\dvprayer{
Père, apprends-nous par Ta Parole comment
 nous aimer les uns les autres,
 afin que nous puissions voir un parfum glorieux remplir nos vies. 
}{\DlNdJ}


%%%%%%%%%%%%%%%
% 8 janvier
%%%%%%%%%%%%%%%

\dvday{Surmonter la tentation}

\dvquote{
Alors elle le saisit par son vêtement en disant~:
 \Og Couche avec moi ! \Fg{}
 Il lui abandonna son vêtement dans la main et s'enfuit au dehors\dots{}
}{\ibibleverse{Gn}(39:12)}

\lettrine{N}{ous} sommes tous confrontés à des tentations quotidiennes
 \ocadr{}tentation de mentir, de tricher et de voler.
 Joseph a été confronté à une tentation puissante,
 cependant il a résisté.
 Quelles ont été les clés de son succès ? 

Premièrement, Joseph savait que quelqu'un lui faisait implicitement confiance.
 Il ne voulait pas violer la confiance que Potiphar lui avait accordée. 

Deuxièmement, Joseph savait qu'il était différent du reste du monde.
 Parce que Dieu l'avait appelé comme une race choisie,
 Joseph a dit ce qui semblait évident~: \Og Comment pourrais-je\dots{}?\Fg{}

Troisièmement, Joseph savait que la tentation était redoutable.
 Un de nos problèmes aujourd'hui, c'est notre attitude laxiste
 vis-à-vis du péché.
 Nous avons appris à être très tolérant envers le péché. 

Finalement, Joseph savait que s'il succombait, le péché serait,
 en fin de compte, contre Dieu, pas contre Potiphar.
 Puis Joseph a choisit l'action la plus sage possible
%% NdE: Use ':' instead of '---' ?
 \ocadr{}il s'est mis à courir.
 Si tout le reste échoue, partez en courant.
 Paul a dit à Timothée~:
 \Og Fuis les passions de la jeunesse \Fg{}.
 \ibiblephantom{IITm}(2:22)
 Restez à l'écart des endroits où vous vous savez faibles
 et où vous serez tentés de tomber. 

\dvbox{
Nous pouvons vivre une vie de pureté.
 Nous pouvons surmonter la tentation.
 Dieu nous a donné toutes les règles nécessaires à une vie de victoire. 
}

Comme Joseph, reconnaissez quatre points importants~:
% NdE: use itemize here?
 (1)~ les gens ont placé leur confiance en vous;
 (2)~vous êtes spécial parce que vous êtes un enfant de Dieu
  et vous ne pouvez pas faire ce que les autres peuvent faire;
 (3)~la tentation est redoutable, et quand vous cédez,
  c'est contre Dieu Lui-même que vous péchez;
% NdE: Use ',' instead of '---' ?
 (4)~si tout le reste échoue \ocadr{}courez. Fuyez vers Jésus.

\dvrule

\dvprayer{
Père, puissions-nous décider de vivre une vie sainte et juste.
 Aide-nous Seigneur, à résister aux tentations placés sur notre sentier
 jour après jour. 
}{\DlNdJ}


%%%%%%%%%%%%%%%
% 9 janvier
%%%%%%%%%%%%%%%

\dvday{Culpabilité}

\dvquote{
Ruben, prenant la parole, leur dit~:
 Ne vous disais-je pas~: Ne commettez pas un crime contre cet enfant ?
 Mais vous n'avez pas écouté. Maintenant son sang nous est réclamé. 
}{\ibibleverse{Gn}(42:22)}

\lettrine{V}{ingt ans} ont passé depuis que les frères de Joseph
 l'ont vendu comme esclave, mais ses cris sont encore enfouis
 au tréfonds de leurs mémoires. La culpabilité !
 Comme elle subsiste dans nos esprits, pour resurgir
 et nous rappeler nos actes.
 Les frères avaient réussi à cacher leur culpabilité devant leur père,
 mais elle n'avait pas disparu.
 Se libérer de la culpabilité est une tâche pratiquement impossible.
 Mais c'est une tâche absolument essentielle. 

\dvbox{
Le péché crée la culpabilité.
 Et le péché dans nos vies crée de terribles problèmes spirituels.
}

David a décrit ce qui s'est passé quand il a essayé de cacher son péché.
 \Og Car nuit et jour Ta main pesait sur moi,
 ma vigueur n'était plus que sécheresse, comme celle de l'été \Fg{}
 (\ibibleverse{Ps}(32:4)). 

Dieu a dit~: \Og Le salaire du péché, c'est la mort. \Fg{}
 \ibiblephantom{Rm}(6:23)
 Le péché qui reste non pardonné nous rend morts à Dieu.
 Nous ne Le voyons plus, ne Le sentons plus, ne Le touchons plus.
 C'est pourquoi nous devons nous débarrasser de la culpabilité.
 Elle nous sépare de Dieu et détruit nos vies. 

Mais comment faire ? Si j'ai fait quelque chose de mal,
 comment puis-je le \Og défaire \Fg{} ? 

Vous n'avez pas à être punis pour vos péchés;
 quelqu'un d'autre a re\c{c}u votre punition.
 Dieu a offert une solution en envoyant Son Fils.
 Et si vous voulez seulement croire en Lui,
 Dieu va vous pardonner et complètement vous renouveler.
 Il va vous donner un nouveau départ.

\dvrule

\dvprayer{
Père, merci pour la fa\c{c}on glorieuse dont Tu enlèves
 notre culpabilité par Jésus-Christ,
 Qui nous a lavé et nettoyé de toute notre iniquité. 
}{\DlNdJ}


%%%%%%%%%%%%%%%
% 10 janvier
%%%%%%%%%%%%%%%

\dvday{Rechercher Dieu\\ pour être dirigé}

\dvquote{
Dieu parla à Israël dans des visions nocturnes.
 Il dit~: Jacob ! Jacob !\dots{}
 Ne crains pas de descendre en Égypte,
 car c'est là que Je te ferai devenir une grande nation.
 C'est Moi qui descendrai avec toi en Égypte
 et c'est Moi qui t'en ferai aussi remonter.
}{\ibibleverse{Gn}(46:2-4)}

\lettrine{P}{endant} des années,
 Jacob avait pensé que son fils Joseph était mort.
 Mais voilà que ses fils aînés lui annoncent la nouvelle bouleversante
 que Joseph est vivant et qu'il veut que son père vienne en Égypte.
 Et ainsi, ils se mettent en chemin.
 Mais quand ils atteignent Beér-Chéba, la peur s'empare de Jacob. 

La peur est une motivation puissante.
 La peur éprouvée par Jacob l'a conduit à rechercher Dieu.
 Il vaut mieux être motivé par l'amour,
 mais si vous n'allez pas rechercher Dieu pour cette raison,
 il est possible qu'Il vous mette dans une situation de nécessité
 afin que vous Le recherchiez par besoin.
 Si \c{c}a n'aboutit pas, Il est possible qu'Il vous mette
 dans une situation qui suscite la peur afin que vous Le recherchiez par peur.
 Parce que Dieu vous aime tant et qu'Il apprécie d'être en communion avec vous,
 Il se servira de n'importe quel moyen nécessaire pour vous faire venir à Lui.

\dvbox{
Nous nous mettons souvent dans des situations vraiment épouvantables
 parce que nous avan\c{c}ons sans nous soucier de l'avis de Dieu. 
}

Quand Dieu \ocadr{}Qui sait toutes choses et vous aime de fa\c{c}on parfaite\fcadr{}
 a offert de guider votre vie, il semble plutôt ridicule d'ignorer
 cette aide et d'avancer avec votre propre sagesse bien limitée.
 Quand nous atteignons un carrefour et ne savons pas
 quel est le chemin à prendre, il est sage de rechercher Dieu. 

\dvrule

\dvprayer{
Apprends-nous, Ô Dieu, à rechercher Ta volonté,
 à marcher à Ta suite et à être conduit par Ton Esprit. 
}{\DlNdJ}


%%%%%%%%%%%%%%%
% 11 janvier
%%%%%%%%%%%%%%%

\dvday{Dieu sait ce qui se passe}

\dvquote{
J'ai bien vu la misère de Mon peuple qui est en Égypte,
 et j'ai entendu son cri à cause de ses oppresseurs,
 car Je connais ses douleurs. Je suis descendu pour le délivrer\dots{}
}{\ibibleverse{Ex}(3:7-8)}

\lettrine{L}{e peuple} de Dieu était traité cruellement
 par les Égyptiens depuis des années.
 Ils ne voyaient pas d'issue à leur esclavage.
 Ils avaient imploré Dieu mais craignaient qu'Il ne les écoute pas.

\dvbox{
Quand nous traversons des épreuves, nous nous sentons souvent très seuls.
 Nous pensons que personne ne voit ni ne comprend.
 Mais Dieu sait ce qui se passe.
}

Les Israélites ne savaient pas que Dieu était déjà à l'\oe{}uvre,
 qu'Il avait un plan. Mais c'était bien le cas.

Dieu a d'abord dit à Moïse~: \Og Je sais ce qui se passe. \Fg{}
 Puis Il a ajouté~: \Og Car Je connais ses douleurs. \Fg{}
 Bien souvent tout ce dont j'ai besoin,
 c'est de savoir que quelqu'un sait ce qui se passe,
 comprend et se préoccupe de ma situation.

Mais Dieu va encore plus loin. Il dit~:
 \Og Moïse, Je vois, J'ai entendu ses cris, Je connais ses douleurs,
 et je suis descendu pour le délivrer. \Fg{}

Je suis tellement heureux que le Dieu que je sers est un Dieu qui peut voir,
 un Dieu qui peut entendre, un Dieu qui sait,
 et un Dieu qui peut faire infiniment au-delà
 de tout ce que je demande ou de tout ce dont j'ai besoin.

Vous vous trouvez peut-être dans une situation vraiment difficile.
 Vous avez imploré le Seigneur, mais vous vous demandez s'Il entend.
 Dieu vous dit~:
 \Og J'ai vu ton affliction. J'ai entendu ton cri. Je connais ta douleur.
 Et Je suis descendu ce jour pour te délivrer. \Fg{}

\dvrule

\dvprayer{
Père, nous Te remercions de ce que Tu es un Dieu vivant, puissant,
 capable de faire infiniment au-delà de tout ce dont nous avons besoin,
 de tout ce que nous demandons ou de tout ce que nous pouvons imaginer. 
}{\Amen}


%%%%%%%%%%%%%%%
% 12 janvier
%%%%%%%%%%%%%%%

\dvday{Qui est l'Éternel ?}

\dvquote{
Moïse et Aaron se rendirent ensuite auprès du Pharaon et lui dirent~:
 Ainsi parle l'Éternel, le Dieu d'Israël~: Laisse partir Mon peuple,
 pour qu'il célèbre une fête en Mon honneur au désert.
 Le Pharaon répondit~: Qui est l'Éternel, pour que je lui obéisse,
 en laissant partir Israël ?
 Je ne connais pas l'Éternel, aussi je ne laisserai point partir Israël.
}{\ibibleverse{Ex}(3:7-8)}

\lettrine{M}{oïse et Aaron} vinrent auprès du Pharaon
 avec l'exigence formulée par l'Éternel que le Pharaon
 laisse partir leur peuple.
 Mais il répondit~: \Og Qui est l'Éternel ? Je ne connais pas l'Éternel,
 et je n'obéirai pas. \Fg{}
 Qui est l'Éternel ? Le Pharaon n'allait pas tarder
 à obtenir une réponse à sa question.

Les Égyptiens avaient choisi de vouer un culte aux forces de la nature.
 Ils faisaient une erreur que beaucoup de gens font même encore aujourd'hui
 \ocadr{}adorer et servir la création plutôt que le Créateur.
 Jéhovah, le seul vrai Dieu vivant, avait fait connaître Son exigence
 auprès du Pharaon, mais le Pharaon a prétendu
 ne pas Le connaître et a refusé d'obéir.

Quelle a été la réponse de Dieu ? Il s'est révélé au Pharaon
 par les plaies qu'Il a envoyées sur l'Égypte
 \ocadr{}ténèbres, ulcères, mouches venimeuses, et grenouilles,
 eau changée en sang, mort; des plaies qui tourmentaient les Égyptiens,
 mais épargnaient le peuple de Dieu.

\dvbox{
Qui est l'Éternel pour que vous lui obéissiez ?
 Il n'y a aucune excuse à ne pas le connaître.
}

Il est le Dieu qui s'est révélé par l'intermédiaire de la nature,
 par l'intermédiaire de Sa Parole et par l'intermédiaire de Son Fils.
 Il est le Dieu qui règne sur tout l'univers, le Créateur de toutes choses,
 le Roi des rois et le Seigneur des seigneurs,
 l'Éternel Tout-Puissant, le Dieu aimant
 qui vous invite à Le connaître mieux aujourd'hui. 

\dvrule

\dvprayer{
Nous Te remercions Père, de ce que nous en sommes venus à Te connaître.
 Nous voulons Te connaître encore plus. 
}{\Amen}


%%%%%%%%%%%%%%%
% 13 janvier
%%%%%%%%%%%%%%%

\dvday{La Pâque}

\dvquote{
Le sang vous servira de signe sur les maisons où vous serez;\\
 Je verrai le sang, Je passerai au-dessus de vous,
 et il n'y aura pas sur vous de fléau destructeur,
 quand Je frapperai le pays d'Égypte.
}{\ibibleverse{Ex}(12:13)}

\lettrine{D}{ieu} avait été si patient avec les Égyptiens.
 Ils avaient fait souffrir Son peuple pendant au moins quatre-vingt ans.
 Mais Il ne les a pas simplement éliminés d'un coup.
 Il leur a fait savoir par des avertissements préliminaires
 que le jour du jugement arrivait.
 Il~leur a donné l'opportunité de changer et d'échapper à Son jugement.

En proclamant l'arrivée de Son jugement,
 Dieu a aussi offert aux enfants d'Israël un moyen d'y échapper.
 Tant qu'ils resteraient dans la maison ayant le sang de l'agneau sur la porte,
 Dieu leur a dit~:
 \Og Quand je passerai dans le pays, ils seront en sécurité. \Fg{}

La caractéristique principale de Dieu, c'est l'amour
 \ocadr{}amour qui est au-delà de ce que nous pouvons pleinement comprendre.
 Mais Dieu est aussi parfaitement juste et le méchant doit donc être jugé.
 En fin de compte, la justice de Dieu doit prévaloir.
 Cela a mis du temps à venir, mais nous voyons déjà les premiers signes
 du jugement qui va venir.
 Êtes-vous en sécurité aujourd'hui?

\dvbox{
Dieu n'a prévu qu'un seul plan pour votre sécurité
 quand le jour du jugement arrivera, et il est en Jésus-Christ.
}

Jésus est l'Agneau de Dieu qui ôte le péché du monde.
 Nous devons dépendre entièrement de Lui. Il n'y pas d'autre refuge. 

\dvrule

\dvprayer{
Père, merci pour la merveilleuse solution que Tu as pourvue en Jésus-Christ,
 en ce qu'Il a porté nos péchés, Il a pris notre culpabilité,
 Il est mort à notre place et Son sang a été versé pour nous.\\
 Merci pour Ta patience, Ton endurance, Ta miséricorde et Ton amour. 
}{\Amen}


%%%%%%%%%%%%%%%
% 14 janvier
%%%%%%%%%%%%%%%

\dvday{Piégés}

\dvquote{
Moïse répondit au peuple~:
 \Og Soyez sans crainte, restez en place et voyez comment l'Éternel
 va vous sauver aujourd'hui; car les Égyptiens que vous voyez aujourd'hui,
 vous ne les verrez plus jamais.
 L'Éternel combattra pour vous; et vous, gardez le silence. \Fg{}
}{\ibibleverse{Ex}(14:13-14)}

\lettrine{L}{es enfants d'Israël} avaient quitté l'esclavage d'Égypte,
 leur misérable vie de captivité,
 et étaient maintenant déterminés à servir le Seigneur.
 Mais le premier endroit où Dieu les conduit se révèle être un piège pour eux.
 Il n'y a pas d'issue à cette vallée à moins de faire demi-tour.
 Ils ne peuvent franchir les montagnes qui la bordent,
 et ils ne peuvent traverser la Mer Rouge.

Pourquoi Dieu conduirait-Il Son peuple dans un piège?

\dvbox{
Dieu veut nous apprendre à Lui faire entièrement confiance
 \ocadr{}même quand nous ne pouvons voir aucune solution.
}

Il veut que nous nous rendions compte qu'Il est capable de trouver une solution
 même quand il n'y a pas de solution;
 qu'Il n'est pas limité par les ressources humaines
 ni par les capacités humaines et que ce qu'Il a promis,
 Il est capable de l'accomplir.

Le plan de délivrance de Dieu exige de la foi.
 Quand vous vous trouvez piégés, la tendance naturelle,
 c'est de partir en courant.
 Mais si nous avan\c{c}ons par la foi,
 Dieu va aller au devant de nous et ouvrir un chemin dans la mer.

Vous vous sentez peut-être coincés.
 Vous êtes peut-être dans des circonstances
 auxquelles vous ne voyez pas d'issue.
 Mais il est bien possible que Dieu vous ait conduit dans cette impasse
 afin que vous vous tourniez vers Lui aujourd'hui. 

\dvrule

\dvprayer{
Oh Seigneur, comme nous Te remercions pour Ta délivrance !
 Si souvent, Tu as séparé les eaux d'une mer de difficultés
 ou enlevé les montagnes pour nous secourir.
 Aide-nous à ne pas avoir peur mais à faire confiance. 
}{\DlNdJ}


%%%%%%%%%%%%%%%
% 15 janvier
%%%%%%%%%%%%%%%

\dvday{Sur des ailes d'aigle}

\dvquote{
Voici ce que tu diras à la maison de Jacob et que tu annonceras aux Israélites~:
 \Og Vous avez vu vous-mêmes ce que J'ai fait à l'Égypte~:\\
 Je vous ai portés sur des ailes d'aigle et fait venir vers Moi.\\
 Maintenant, si vous écoutez Ma voix et si vous gardez Mon alliance,
 vous M'appartiendrez en propre entre tous les peuples,
 car toute la terre est à Moi.\\
 Quant à vous, vous serez pour Moi un royaume de sacrificateurs
 et une nation sainte. \Fg{}
}{\ibibleverse{Ex}(19:3-6)}

\lettrine{U}{n aigle} bâtit son nid très haut
 sur le flanc d'une paroi rocheuse.
 Quand vient le temps pour un aiglon d'apprendre à voler,
 la mère volette au-dessus du nid et par le battement de ses ailes
 pousse le petit aigle dans le vide.
% NdE: "tourner rouler", "tourner-rouler" ?
 Ce petit aiglon se met à tourner rouler sans contrôle
 dans sa chute libre vers les rochers plus bas.
 Juste quand vous croyez que le pauvre oiseau va se fracasser sur les rochers,
 la mère fond en piqué plus bas que lui, le récupère sur ses ailes
 et le ramène au nid.
 La le\c{c}on numéro un est terminée.
 Mais la le\c{c}on va être répétée maintes et maintes fois,
 jusqu'à ce que l'aiglon ait appris à voler.

Nous nous sentons à l'aise et en sécurité à l'intérieur du nid.
 Nous n'aimons pas les moments où Dieu nous pousse au dehors
 et où nous commen\c{c}ons à tomber sans aucun contrôle.
 Nous pensons: \Og C'est sûr, je vais être fracassé. \Fg{}

\dvbox{
Mais dans ces moments Dieu nous relève et nous montre Sa fidélité.
}

Dieu nous a délivrés, vous et moi, de l'esclavage
 qui consistait à vivre pour la chair.
 Il nous a porté sur des ailes d'aigle, nous amenant à Lui
 et Il a fait de nous Son trésor spécial. 

\dvrule

\dvprayer{
Père, comme nous sommes reconnaissants d'être Tes enfants
 et de ce que Tu nous ais portés sur des ailes d'aigle jusqu'à Toi !
 Nous voulons être Ton trésor spécial
 et produire de bonnes \oe{}uvres pour Ta gloire. 
}{\Amen}



%%%%%%%%%%%%%%%
% 16 janvier
%%%%%%%%%%%%%%%

\dvday{Petit à petit}

\dvquote{
Je ne les chasserai pas en une seule année loin de toi,\\
 de peur que le pays ne soit désolé et que les animaux sauvages
 ne se multiplient contre toi.
 Je les chasserai peu à peu\dots{}
}{\ibibleverse{Ex}(23:29-30)}

\lettrine{L}{es Israélites} étaient confrontés à un formidable ennemi.
 Nous sommes, nous aussi, confrontés à un ennemi quotidien
 quand nous nous battons contre notre chair.
 La chair a édifié de vraies places fortes dans nos vies
 \ocadr{}des forteresses entourées de hautes murailles.
 Des géants habitent notre terre.
 Mais Dieu veut que nous entrions et prenions possession
 de toutes Ses promesses.
 Il veut que nous vivions une vie de bénédictions et de victoire.

Comment faire? Nous pouvons apprendre à combattre notre chair
 en observant les principes que Dieu a exposés pour Israël.
 Notez qu'Il n'a pas donné la victoire à Israël en une seule année.
 Il a conquis, au contraire, leurs ennemis peu à peu.

\dvbox{
Dieu ne va pas vous donner la victoire sur la chair en un an.
 Il n'existe pas de raccourcis pour le succès
 \ocadr{}c'est la bataille de toute une vie.
}

Tant que vous vivrez dans ce corps,
 vous allez être ennuyés par les choses de la chair.
 Dieu ne nous montre pas l'ensemble de la bataille d'un seul coup,
 parce qu'Il sait que \c{c}a nous découragerait.
 Il nous montre donc un seul domaine à la fois.

Tout comme Israël ne s'est jamais saisi complètement
 de tout le territoire que Dieu leur avait promis,
 nous n'aurons jamais la victoire complète sur notre chair
 jusqu'à ce que nous nous tenions enfin devant Son trône.
 Mais nous pouvons nous réjouir du territoire que nous avons déjà conquis,
 et cesser de nous inquiéter, sachant qu'Il va continuer cette \oe{}uvre
 \ocadr{}petit à petit. 

\dvrule

\dvprayer{
Seigneur, aide-nous à ne pas nous satisfaire d'une conquête incomplète
 de la terre, mais puissions-nous persévérer
 jusqu'à ce que nous ayons pris tout ce que Tu as promis. 
}{\Amen}


%%%%%%%%%%%%%%%
% 17 janvier
%%%%%%%%%%%%%%%

\dvday{Pas un Dieu lointain}

\dvquote{
Je demeurerai au milieu des Israélites et je serai leur Dieu.
 Ils reconnaîtront que je suis l'Éternel, leur Dieu\dots{}
}{\ibibleverse{Ex}(29:45:46)}

\lettrine{O}{ù} demeure Dieu ? \\[1ex]
Vous pourriez dire qu'Il demeure dans l'univers,
 mais dès que vous commencez à parler de l'univers,
 vous décrivez quelque chose de si vaste et si distant
% NdE: très _très_ lointain?
 que \c{c}a peut donner l'impression que Dieu est très très lointain.

Mais Dieu n'est pas éloigné ni lointain.
 Paul a dit que nous ne devrions pas penser à Dieu comme étant très loin
 dans le ciel où nous ne pouvons pas l'atteindre.
 La vérité, c'est qu'Il est aussi proche que nos bouches.
 \Og La parole est près de toi, dans ta bouche et dans ton c\oe{}ur. \Fg{}
 (\ibibleverse{Rm}(10:8)).

Dieu est très proche. Il nous entoure.
 Et Il nous est dit dans ce passage de l'Écriture
 qu'Il était proche de Son peuple d'Israël.
 Le tabernacle devait se trouver au centre exact de leur camp
 et les tribus devaient camper tout autour.
 La première chose qu'ils voyaient chaque matin était la fumée
 qui montait du sacrifice du matin
 \ocadr{}et ils se rappelaient que Dieu demeure en plein au milieu de Son peuple.

\dvbox{
Dieu veut que vous sachiez qu'il est assez proche pour que vous puissiez aller
 vers Lui à tout moment et communier avec Lui.
}

Quels que soient votre situation ou votre problème, Il est assez proche.
 Vous pouvez tendre les bras et prier Dieu, et par cette prière,
 être guéris, secourus et fortifiés. 

\dvrule

\dvprayer{
Père, nous Te remercions de ce que Tu es un Dieu qui est proche.
 Manifeste-Toi dans nos vies, Seigneur,
 afin que nous soyons remplis de Ta grâce et de Ton amour infinis. 
}{\Amen}


%%%%%%%%%%%%%%%
% 18 janvier
%%%%%%%%%%%%%%%

\dvday{Comment est Dieu ?}

\dvquote{
L'Éternel passa devant lui en proclamant~:
 L'Éternel, l'Éternel, Dieu compatissant et qui fait grâce,
 lent à la colère, riche en bienveillance et en fidélité\dots{}
}{\ibibleverse{Ex}(34:6)}

\lettrine{S}{elon vous}, Dieu est comment? \\[1ex]
En hébreu, le nom de Dieu est \emph{Yahvé, Yahvé El}.
 Le mot \emph{Yahvé} signifie le \Og Dieu auto-existant \Fg{}.
 Certains ont aussi traduit ce nom par \Og celui qui devient tout \Fg{}.
 C'est-à-dire que quel que puisse être votre besoin, Il est la réponse.
 Si vous avez besoin de force, Il est votre force.
 Si vous avez besoin du salut, Il est votre Sauveur.
 Si vous avez besoin de guérison, de justice, ou de paix,
 Il devient ces choses pour vous.
 Quel que puisse être le besoin d'un homme, Dieu est tout-suffisant.
 Ainsi Dieu se révèle par Son nom.

Mais Il va plus loin. Dieu se décrit comme étant par nature compatissant
 ou plein de miséricorde. Ceci veut dire que Dieu ne nous donne pas
 ce que nous méritons.
 Il dit ensuite qu'Il est plein de grâce.
 Si la miséricorde peut être vue comme une qualité négative
 (ne pas obtenir ce que vous méritez), la grâce, elle, est une qualité positive.
 C'est obtenir ce que vous ne méritez pas
 \ocadr{}la bonté et la bénédiction de Dieu.

\dvbox{
Dieu déclare aussi qu'Il est lent à la colère et patient.
 Puis Il se décrit comme étant riche en bienveillance et en fidélité.
}

Dans le Nouveau Testament, nous voyons la révélation finale de Dieu
 en Jésus-Christ qui nous montre la pleine mesure de la miséricorde,
 de la patience et de la grâce de Dieu.

Tel est le Dieu que nous servons; le Dieu qui \oe{}uvre pour nous faire grandir
 et mûrir et nous fa\c{c}onner à Son image. 

\dvrule

\dvprayer{
Merci, Père, de nous révéler la vérité sur Toi-même.
 Aide-nous à nous saisir de la vérité et nous en remettre à Toi,
 pour que nous puissions devenir comme Toi. 
}{\Amen}



%%%%%%%%%%%%%%%
% 19 janvier
%%%%%%%%%%%%%%%

\dvday{Péché involontaire}

\dvquote{
Si c'est quelqu'un du peuple qui a péché involontairement\dots{}\\
 il présentera en oblation une chèvre\dots{}
 pour le péché qu'il a commis.\\
 Il posera sa main sur la tête de la victime offerte pour le péché;\\
 il égorgera la victime offerte pour le péché à l'endroit des holocaustes.
}{\ibibleverse{Lv}(4:27-29)}

\lettrine{L}{e mot} \Og pécher \Fg{} signifie \Og rater la cible \Fg{}.
 Rater la cible n'est pas toujours intentionnel.
 Si je fais délibérément, intentionnellement, ce que je sais être mal,
 c'est une transgression. Mais quand, par faiblesse, incapacité ou ignorance,
 je fais involontairement des choses qui ne sont pas bien, \suggest{bonnes}
 alors c'est du péché. Mais même le péché doit être expié.

\dvbox[0.87\textwidth]{
Quelle est la cible visée ? C'est la perfection.
 Et nous en arrivons tous bien loin.
}

Le péché entraîne la mort spirituelle et donc l'aliénation par rapport à Dieu.
 Parce que l'homme pécheur ne peut pas avoir de communion avec un Dieu saint,
 Dieu a fourni une offrande pour le péché afin que nous puissions vivre
 une communion intime avec Lui.

Mais les Écritures nous apprennent que les sacrifices de chèvres
 ne pouvaient pas enlever les péchés, ils ne pouvaient que les couvrir.
 Les gens ne pouvaient jouir de cette glorieuse communion avec Dieu
 que jusqu'à ce qu'ils fassent de nouveau un faux pas, ce qui, malheureusement,
 ne prenait pas longtemps.
 Et bien sûr, les gens du peuples se retrouvaient vite
 à cours de chèvres à sacrifier.

Aussi Dieu a-t-Il fourni un autre sacrifice pour le péché~: Jésus-Christ,
 qui a pris ma culpabilité et ma punition.
 En mourant à ma place, Il m'a permis d'avoir la communion avec Dieu
 et de faire l'expérience de la joie, de la bénédiction, du plaisir,
 de la gloire et de la richesse de Le connaître. 

\dvrule

\dvprayer{
Père, nous Te remercions de ce que Jésus est devenu pour nous
 un sacrifice pour le péché, afin que nous puissions connaître
 la joie de vivre en communion avec Toi. 
}{\Amen}


%%%%%%%%%%%%%%%
% 20 janvier
%%%%%%%%%%%%%%%

\dvday{Feu étrange}

\dvquote{
Et Nadab et Abihu, les fils d'Aaron, prirent chacun son encensoir,
 et y mirent du feu, et y mirent de l'encens dessus,
 et offrirent un feu étrange devant le \Seigneur;
 lequel Il ne leur avait pas commandé.
 Et le feu sortit de devant le \Seigneur, et les dévora,
 et ils moururent devant le \Seigneur.
}{\ibibleverse{Lv}(10:1-2) \KJF{}}

\lettrine{C}{es deux hommes} ont offert un feu étrange à Dieu.
 Et leur service a brutalement pris fin quand le feu de Dieu les a consumés.
 Quel était ce feu étrange qu'ils ont offert?
 Leur feu venait d'une autre source que celle du feu
 qui avait été spontanément allumé par Dieu.

Il se pourrait qu'ils aient été excités et emportés par l'émotion du moment.
 Il est tout à fait possible qu'ils aient désobéi au verset
 \ibiblevs{Lv}(10:9) du chapitre 10,
 où le Seigneur ordonnait~:
 \Og Ne buvez ni vin ni boisson forte, toi et tes fils avec toi,
 lorsque vous entrerez dans le tabernacle de la congrégation,
 de peur que vous ne mouriez. \Fg{}
 Il se pourrait que ces gar\c{c}ons aient été un peu éméchés
 et pas en pleine possession de leurs facultés.
 Ou peut-être étaient-ils motivés par le désir d'attirer
 l'attention sur eux-mêmes et leur ministère plutôt que de porter gloire à Dieu.

\dvbox{
Que vous le serviez n'est pas la seule chose qui intéresse Dieu,
 pourquoi et comment vous le faites L'intéresse beaucoup plus encore.
}

Nous pouvons gagner notre vie en faisant d'autres choses,
 mais le véritable appel sur chacune de nos vies est de servir le Seigneur.

La question est~: Qu'est-ce qui vous motive?
 D'où vient ce feu allumé dans votre c\oe{}ur? 

\dvrule

\dvprayer{
Dieu, aide-nous à Te servir avec le véritable feu de Ton Esprit
 brûlant dans nos c\oe{}urs.
 Puissions-nous connaître la joie, le privilège et la bénédiction
 d'être un instrument que Tu as utilisé pour servir les autres. 
}{\Amen}


%%%%%%%%%%%%%%%
% 21 janvier
%%%%%%%%%%%%%%%

\dvday{Purifiés et restaurés}

\dvquote{
L'Éternel parla à Moïse et dit~:
 Voici quelle sera la loi concernant le lépreux,
 le jour de sa purification\dots 
}{\ibibleverse{Lv}(14:1-2)}

\lettrine{E}{n dépit} de tous nos prOgrès en science et en médecine,
 il n'y a toujours pas de remède humain connu contre la lèpre.
 Elle peut désormais être arrêtée mais toujours pas guérie.
 La maladie se propage dans le corps par un processus de pourrissement
 \ocadr{}tout d'abord, elle détruit les nerfs;
 puis graduellement, elle s'étend jusqu'à ce qu'elle atteigne
 un organe vital et alors elle tue.
 Ainsi en est-il aussi du péché \ocadr{}il attaque d'abord le système nerveux
 de l'esprit de telle sorte que vous êtes prOgressivement détruits
 sans même vous en rendre compte.
 Le péché toléré dans un domaine va s'étendre jusqu'à ce qu'il anéantisse
 toute votre vie \ocadr{}cependant vous n'êtes pas conscients
 de ce qu'il est en train de vous faire. 

Pourquoi Dieu établirait-Il une loi concernant le lépreux,
 le jour de sa purification alors que la lèpre est incurable ?
 Cela indique que Dieu se réserve la possibilité d'accomplir
 une \oe{}uvre souveraine de grâce que les lépreux ne pourraient
 jamais faire pour eux-mêmes. 

\dvbox{
Tout comme l'homme n'a pas de remède contre la lèpre,
 l'homme n'a pas de remède contre le péché.
}

Mais Dieu a offert la purification par l'intermédiaire de Jésus-Christ.
 Le sang de Jésus, le Fils de Dieu, purifie un homme de tous péchés. 

\Og Le péché ne dominera pas sur vous \Fg{} (\ibibleverse{Rm}(6:14)).
 Je n'ai plus à être dirigé par le péché.
 Je ne suis plus en sujétion \grammar{sujétion}
 au péché en raison de la puissance
 qui m'est conférée par l'entremise de Jésus-Christ
 \ocadr{}J'ai été purifié, lavé et restauré.
 Et maintenant, je peux vivre en communion avec Dieu. 

\dvrule

\dvprayer{
Père, nous Te remercions de ce que la venue de Ton Fils a amené
 la puissance de l'Esprit qui nous permet de vivre une vie libérée
 de l'esclavage du péché et de la corruption.
 Seigneur, nous nous réjouissons et nous sommes d'accord avec David~:
 \Og Heureux celui dont la transgression est enlevée,
 dont le péché est pardonné ! \Fg{}
 \ibiblephantom{Ps}(32:1)
}{\Amen}


%%%%%%%%%%%%%%%
% 22 janvier
%%%%%%%%%%%%%%%

\dvday{Le sang de l'expiation}

\dvquote{
Car la vie de la chair est dans le sang.\\
 Je vous l'ai donné sur l'autel, afin qu'il serve d'expiation pour votre vie,
 car c'est par la vie que le sang fait l'expiation.
}{\ibibleverse{Lv}(17:11)}

\lettrine{L}{a Parole de Dieu} déclare que l'expiation
 pour l'âme se fait par le sang.
 La pénalité requise par Dieu pour le péché était la mort.
 Ainsi, du fait que nous avons tous péché,
 nous sommes tous condamnés à mort.
 Et cette mort est la mort spirituelle. Exclu et séparé de Dieu,
 l'homme perd la conscience de la présence de Dieu dans sa vie. 

Notre attitude envers le péché est très désinvolte.
 Aujourd'hui, nous avons une compréhension totalement erronée du péché
 et de la sainteté de Dieu.
 L'homme demande à Dieu de venir faire partie de sa vie remplie de péché.
 Mais il est nous est dit que Dieu est si saint et si pur
 qu'Il ne peut pas regarder le péché.
 Et, à coup sûr, la croix de Jésus-Christ devrait nous convaincre
 que Dieu ne va pas accepter le péché.
 Dieu a abandonné Son propre Fils quand Jésus a pris vos péchés
 sur Lui-même.

\dvbox{
Si Dieu a abandonné Son propre Fils à cause du péché,
 comment pouvez-vous vous espérer que dieu soit en communion avec vous,
 tant que le péché imprègne votre vie ? 
}

Merci à Dieu de ce que nous voyons le plan de rédemption accompli
 quand nous contemplons Jésus-Christ.
 Plutôt que d'essayer d'amener notre propre justice à Dieu,
 nous pouvons venir à Lui par la justice de Christ. 

\dvrule

\dvprayer{
Merci, Père, pour le sang de Jésus-Christ
 qui a amené l'expiation de nos péchés.\\
 Seigneur, nous recevons cette purification aujourd'hui,
 afin de pouvoir vivre avec Toi en nouveauté de vie.
}{\DlNdJ}


%%%%%%%%%%%%%%%
% 23 janvier
%%%%%%%%%%%%%%%

\dvday{Perfection exigée}

\dvquote{
Si un homme offre à l'Éternel du gros ou du menu bétail
 en sacrifice de communion, soit pour l'accomplissement d'un v\oe{}u,
 soit comme offrande volontaire, la victime sera sans défaut,
 pour être agréée; il n'y aura en elle aucune malformation.
}{\ibibleverse{Lv}(22:21)}

\lettrine{P}{our} ce qui était des sacrifices,
 Dieu exigeait la perfection parce qu'Il connaissait la tendance
 de l'homme à Lui offrir des choses dont personne ne veut plus.
 \Og Je vais juste offrir à Dieu ma vieille vache aveugle\dots{}
 ou cet agneau qui s'est cassé la patte la semaine dernière.
 Il ne va pas survivre. Offrons-le en sacrifice à Dieu. \Fg{}

\dvbox{
Dans votre désir d'être en communion avec Dieu,
 comment vous approchez-vous de Lui ? 
}

Pensez-vous que vous pouvez vous approcher de Lui n'importe comment,
 simplement comme vous êtes ? 

Certains cherchent à s'approcher de Dieu sur le base de leurs bonnes \oe{}uvres.
 Mais vos \oe{}uvres sont-elles parfaites ?
 Non seulement, les \oe{}uvres doivent être parfaites mais les raisons
 qui les motivent doivent aussi être parfaites. Cela m'exclut immédiatement.
 Et cela élimine mes \oe{}uvres, à coup sûr. 

Certains essayent de rechercher Dieu par le moyen de l'église. \fixme{Ou l'Église?}
 Ils croient que l'église \fixme{Église, pour tout le paragraphe?}
 va les sauver. Mais l'église est pleine d'imperfections,
 et les rites de l'église sont célébrés par des hommes imparfaits. 

Notre seul espoir de communion avec Dieu réside en un parfait sacrifice.
 Jésus est ce parfait sacrifice, et en dehors de Lui,
 il n'existe aucun moyen d'entrer en communion avec Dieu.
 \Og Je suis le chemin, la vérité, la vie, a dit Jésus.
 \typo{la vie}\punct{virgule après guillemet fermant}
 \punct{Pas besoin de répéter les guillemets autour de l'incise}
 Nul ne vient au Père que par Moi \Fg{} (\ibibleverse{Jn}(14:6)). 

\dvrule

\dvprayer{
Aide-nous, Seigneur, à ne pas être assez fous pour nous approcher de Toi
 par nos efforts imparfaits.
 Puissions-nous venir, par le chemin de Jésus-Christ,
 dans cette vie de communion \ocadr{}en T'aimant,
 en marchant avec Toi et en étant un avec Toi par Son entremise.
}{\DlNdJ}


%%%%%%%%%%%%%%%
% 24 janvier
%%%%%%%%%%%%%%%

\dvday{Représenter Dieu}

\dvquote{
L'Éternel parla à Moïse et dit~:
 \Og Parle à Aaron et à ses fils et dis~:
 Vous bénirez ainsi les Israélites \Fg{}, vous leur direz:
 \Og Que l'Éternel te bénisse et te garde !
 Que l'Éternel fasse briller sa face sur toi et t'accorde sa grâce !
 Que l'Éternel lève sa face vers toi et te donne la paix ! \Fg{}
}{\ibibleverse{Nb}(6:22-27)}

\lettrine{U}{n prêtre de Dieu} avait une double responsabilité.
 Il devait, premièrement, aller devant Dieu représenter le peuple.
 Deuxièmement, il représentait Dieu auprès du peuple
 en lui déclarant la Parole de Dieu. 

\dvbox{
Quelle responsabilité énorme que de représenter Dieu
 auprès d'autres personnes ! 
}

Ceux qui observent les représentants de Dieu forment leur concept de Dieu
 d'après ce qu'ils voient chez le prêtre.
 C'est pourquoi Dieu est si concerné par la fa\c{c}on dont Il est représenté
 auprès du peuple. 

Dans ce passage, nous voyons d'abord que Dieu désirait bénir le peuple.
 Ensuite, l'idée de la face de Dieu brillant sur nous indique
 que Dieu Lui-même est la source de cette grâce.
 Enfin, Dieu veut que Son nom soit associé à la paix. 

Les gens ont soif de voir la réalité de Dieu. Ils vous observent.
 Le représentez-vous comme quelqu'un qui s'énerve et se met en colère
 à la moindre contrariété ? Puissent Son amour, Sa compassion,
 Sa tendresse et bonté se répandre de votre vie afin que les gens
 puissent se faire une idée correcte de Dieu en voyant
 Son Esprit agir par votre intermédiaire. 

\dvrule

\dvprayer{
Père, puissions-nous Te refléter fidèlement dans ce monde.
 Aide-nous Seigneur et pardonne-nous pour les fois où nous T'avons
 mal représenté et avons donné aux gens une fausse idée de notre Dieu.
}{\Amen}


%%%%%%%%%%%%%%%
% 25 janvier
%%%%%%%%%%%%%%%

\dvday{Les tombes de la convoitise}

\dvquote{
Le ramassis de gens qui se trouvait au milieu d'Israël
 fut rempli de convoitise et même les Israélites recommencèrent
 à pleurer et dirent~:
 Qui nous donnera de la viande à manger ?
}{\ibibleverse{Nb}(11:4)}

\lettrine{S}{e rappelant} leur ancienne vie,
 les étrangers d'origines diverses qui étaient sortis d'Égypte
 avec les enfants d'Israël commencèrent à être obsédés
 par l'idée de retourner à cette ancienne vie.
 Il déplaisait à Dieu qu'ils éprouvent une telle nostalgie pour l'Égypte,
 néanmoins et en dépit de leurs murmures contre Lui, Il pourvut à leurs besoins.

\Og Comme la viande était encore entre leurs dents, sans être mâchée,
 la colère de l'Éternel s'enflamma contre le peuple, et l'Éternel
 frappa le peuple d'une très grande plaie.
 On donna à cet endroit le nom de Qibroth-Hattaava
 [ce qui signifie \Og tombes de la convoitise \Fg{}],
 parce qu'on y ensevelit le peuple rempli de désir \Fg{}
 (\ibiblechvs{Nb}(11:33-34)).

Il est possible qu'étant tellement affamés,
 ils aient avalé la viande sans la mâcher,
 qu'ils se soient étranglés avec les os et qu'ils soient morts étouffés.
 Ou il est possible qu'après plus d'un an de ce régime de manne bien fade,
 leurs corps n'aient pas pu soudain assimiler toute cette viande
 et qu'en s'en gavant, ils soient morts d'indigestion.
 Dans tous les cas, leur convoitise les a fait mourir.

\dvbox{
Le Seigneur veut que nous ayons la victoire sur la chair,
 et cette victoire ne vient que quand nous nous marchons
 et vivons selon l'Esprit.
}

Paul a déclaré~:
 \Og La chair a des désirs contraires à l'Esprit,
 et l'Esprit en a de contraires à la chair \Fg{} (\ibibleverse{Ga}(5:17)).
 Céder à la chair n'amène que des souffrances spirituelles.
 Dieu n'a jamais prévu que nous devenions esclaves des désirs de notre chair.

\dvrule

\dvprayer{
Père, nous voyons le pouvoir destructeur de la chair.
 Aide-nous à bien exercer sagesse et jugement
 afin de pouvoir vivre et marcher selon l'Esprit. 
}{\Amen}


%%%%%%%%%%%%%%%
% 26 janvier
%%%%%%%%%%%%%%%

\dvday{Se tenir entre les morts et les vivants}

\dvquote{
Aaron prit le brasier, comme Moïse l'avait dit,
 et courut au milieu de l'assemblée;
 et voici que la plaie avait commencé parmi le peuple.
 Il offrit le parfum et fit l'expiation pour le peuple.
 Il se pla\c{c}a entre les morts et les vivants, et la plaie fut arrêtée. 
}{\ibibleverse{Nb}(17:12-13)}

\lettrine{P}{arce que} certains s'étaient rebellés contre l'autorité de Moïse,
 la congrégation toute entière était coupable devant Dieu.
 La seule chose qui séparait les morts des vivants était Aaron
 \ocadr{}il se tenait entre eux en offrant le parfum
 et en accomplissant l'expiation.

L'encens est associé avec les prières des saints.
 Dans le \ibibleverse{Ps}(141:), le psalmiste
 fait référence à nos prières comme de l'encens présenté à Dieu.

\dvbox{
Pour Dieu, les prières de Son peuple
 constituent un parfum de bonne odeur.
}

Tout comme Aaron intercédait entre le morts et les vivants, nous pouvons,
 nous aussi, nous tenir sur la brèche entre les morts et les vivants
 par la puissance de la prière.
 Je crois que beaucoup de gens subsistent aujourd'hui
 seulement parce que quelqu'un prie pour eux.

Parce que Dieu est juste, Il doit punir le coupable,
 sinon Il ne serait plus un Dieu juste.
 Mais Dieu déborde aussi de miséricorde et d'amour;
 en conséquence, Il est lent à la colère.
 Parce que Dieu ne veut pas punir, Il cherche à exercer Sa miséricorde.

Notre monde est dans une grande rébellion contre Dieu.
 Comme il est important que nous nous tenions sur la brèche
 pour offrir le parfum de la prière entre les morts et les vivants
 \ocadr{}afin que par nos prières nous puissions soutenir
 ceux qui méritent de mourir et arrêter la plaie infligée
 par Dieu à notre nation. 

\dvrule

\dvprayer{
Père, pose un défi à nos c\oe{}urs avec les impératifs spirituels
 présentés par Ta Parole.
 Puissions-nous nous tenir sur la brèche
 et faire une différence par nos prières. 
}{\Amen}


%%%%%%%%%%%%%%%
% 27 janvier
%%%%%%%%%%%%%%%

\dvday{La bataille de la chair}

\dvquote{
Oracle de Balaam, fils de Béor;
 Oracle de l'homme qui a l'\oe{}il clairvoyant;
 Oracle de celui qui entend les paroles de Dieu,
 de celui qui connaît les desseins du Très-Haut,
 de celui qui voit la vision du Tout-Puissant,
 de celui qui se prosterne et dont les yeux s'ouvrent.
 \Og Je le vois, mais non maintenant;
 Je le contemple, mais non de près.
 Un astre sort de Jacob, Un sceptre s'élève d'Israël.
 Il blesse les flancs de Moab et il abat tous les fils de Seth\dots{}\Fg{}
}{\ibibleverse{Nb}(24:15-17)}

\lettrine{B}{alak}, le roi de Moab, était angoissé parce que
 les enfants d'Israël voulaient traverser son territoire.
 Il envoya donc chercher Balaam pour prononcer une malédiction contre eux.
 Mais au lieu de les maudire, celui-ci prophétisa une bénédiction.

La déclaration de Balaam confirme qu'il avait bien entendu la Parole de Dieu,
 il n'empêche que c'était quand même un faux prophète.
 Bien qu'il n'ait initialement prononcé que des bénédictions,
 il finit par dire au roi comment tenter et faire tomber Israël dans le péché.

Balaam dit~:
 \Og Dans l'Esprit, je ne peux rien faire. Attrape-les dans la chair.
 Envoie les filles là-bas avec leurs petites idoles.
 Fais tout pour qu'ils soient excités dans leur chair
 et ils trébucheront et tomberont dans le péché
 \ocadr{}et puis alors Dieu s'occupera d'eux. \Fg{}

En mettant en pratique ce conseil, Moab poussa les Enfants d'Israël
 à marcher selon leur chair.
 Balaam savait que c'était leur point faible. Satan le sait aussi.

\dvbox{
Satan tente notre chair dans le but de nous détruire.
}

Quand le Sceptre d'Israël viendra, Il nous délivrera de notre chair.
 Quel jour de gloire ce sera ! 

\dvrule

\dvprayer{
Rends-nous sensibles, Seigneur, aux périls de la vie selon la chair.
 Fais que ce message pénètre au fonds de nos c\oe{}urs
 pour que nous puissions marcher selon l'Esprit. 
}{\Amen}


%%%%%%%%%%%%%%%
% 28 janvier
%%%%%%%%%%%%%%%

\dvday{Le Bon Berger}

\dvquote{
Moïse parla à l'Éternel et dit~:
 \Og Que l'Éternel, le Dieu des esprits de toute chair,
 établisse sur la communauté un homme qui sorte devant eux
 et qui entre devant eux, qui les fasse sortir et qui les fasse entrer,
 afin que la communauté de l'Éternel
 ne soit pas comme des brebis qui n'ont point de berger. \Fg{}
}{\ibibleverse{Nb}(24:15-17)}

\lettrine{D}{ieu} vient d'annoncer à Moïse que ses jours de leader
 sont terminés et qu'il ne va pas pouvoir guider les enfants d'Israël
 en Terre Promise.

Moïse savait que des brebis sans berger se disperseraient.
 Elles s'égareraient en s'éloignant du troupeau,
 mourraient de faim ou seraient dévorées par des loups.
 Moïse, qui était un vrai berger,
 aimait ces gens qu'il avait guidés à travers le désert
 pendant les quarante dernières années.

Il y a une grande différence entre un berger et un mercenaire.
 Le mercenaire ne s'intéresse pas du tout aux brebis.
 Il s'enfuira au moindre danger plutôt que de rester et de défendre les brebis.
 Mais un vrai berger a les brebis à c\oe{}ur.

\dvbox{
Jésus a dit~: \Og Je suis le bon berger;
 le bon berger donne sa vie pour les brebis \Fg{}
 (\ibibleverse{Jn}(10:11)).
}

Jésus a aussi dit que Ses brebis entendraient Sa voix et le suivraient,
 mais elles ne suivraient pas un étranger.
 Quelqu'un peut essayer d'imiter l'appel du berger,
 mais les brebis ne lèveront même pas la tête.
 Elles connaissent le sifflet ou l'appel de leur berger.

Comme il est glorieux d'avoir entendu l'appel de notre Bon Berger
 et de L'avoir suivi. 

\dvrule

\dvprayer{
Père, comme nous sommes reconnaissants que Jésus-Christ
 soit notre Berger qui surveille Son troupeau,
 qui prend soin des Siens, qui a donné Sa vie pour les brebis.\\
 Merci de nous conduire sur le bon chemin. 
}{\Amen}


%%%%%%%%%%%%%%%
% 29 janvier
%%%%%%%%%%%%%%%

\dvday{Pas de péché secret}

\dvquote{
Mais si vous n'agissez pas ainsi, vous péchez contre l'Éternel;
 sachez que votre péché vous retrouvera.
}{\ibibleverse{Nb}(32:23)}

\lettrine{Q}{uarante années} d'errance viennent de prendre fin.
 Israël est maintenant à la frontière, prêt à entrer dans le pays.
 Mais les anciens des tribus de Ruben et de Gad
 refusent de traverser la rivière;
 ils se contentent de rester sur cette rive du Jourdain.
 \Og Nous promettons d'envoyer des troupes combattre avec vous
 jusqu'à ce que vous ayez conquis tout le pays,
 \punct{Pas besoin de répéter les guillemets}
 disent-ils à Moïse. Puis nous reviendrons habiter de ce côté. \Fg{}

Moïse les a avertis que s'ils ne tenaient pas cette promesse,
 cela constituerait un péché contre le Seigneur
 \ocadr{}et que leur péché les retrouverait.

Quel péché ? Il parle du péché d'omission, du péché de ne rien faire.
 Et je suggère que c'est là le péché de la majorité silencieuse.
 Le mal va prévaloir dans notre pays simplement
 si nous nous tenons à l'écart et ne faisons rien.

\dvbox{
Le péché secret, \c{c}a n'existe pas !
}

la Bible nous dit que tout est à nu et ouvert devant Dieu.
 Il entoure tellement votre vie que rien ne peut être fait secrètement.

Vous pouvez être certains que votre péché vous retrouvera
 dans votre conscience et dans votre contenance.
 Et il vous retrouvera à la fin.
 Un jour vous vous retrouverez devant Dieu
 là où tous les secrets seront révélés.

Je suis tellement reconnaissant à Jésus. La Bible nous dit~:
 \Og Si nous confessons nos péchés, il est fidèle et juste
 pour nous pardonner nos péchés et nous purifier
 de toute injustice \Fg{} (\ibibleverse{IJn}(1:9)). 

\dvrule

\dvprayer{
Père, plante Ta Parole en profondeur dans nos c\oe{}urs.
 Aide-nous, Seigneur, à confesser nos péchés et à recevoir Ton abondant pardon. 
}{\Amen}


%%%%%%%%%%%%%%%
% 30 janvier
%%%%%%%%%%%%%%%

\dvday{Avez-vous régressé ?}

\dvquote{
C'est de là aussi que tu rechercheras l'Éternel, ton Dieu;
 tu Le trouveras, si tu Le cherches de tout ton c\oe{}ur et de toute ton âme.
 Au sein de ta détresse, tous ces évènements t'atteindront.
 Alors, dans les temps à venir, tu retourneras à l'Éternel, ton Dieu,
 et tu écouteras Sa voix; car l'Éternel, ton Dieu, est un Dieu compatissant,
 Qui ne t'abandonnera pas et ne te détruira pas;
 Il n'oubliera pas l'alliance qu'Il a jurée à tes pères.
}{\ibibleverse{Dt}(4:29-31)}

\lettrine{Q}{uelquefois,} les enfants de Dieu régressent dans leur foi
 et se retrouvent de nouveau prisonniers en territoire ennemi.
 Jésus a dit à l'église d'Éphèse~:
 \Og J'ai contre toi que tu as abandonné ton premier amour. \Fg{}
 \ibiblephantom{Ap}(2:4)

Si vous pouvez penser à un moment dans votre expérience chrétienne
 où vous marchiez avec le Seigneur d'une fa\c{c}on plus proche qu'aujourd'hui,
 à un moment où vous faisiez l'expérience de Sa présence et de Sa puissance
 dans une plus grande mesure, alors, vous avez régressé.

\dvbox{
La régression se produit quand vous laissez tout autre désir,
 toute autre ambition ou préférence prendre la première place
 dans votre c\oe{}ur et votre vie.
}

Dieu doit être le premier, au-dessus de toute autre chose.
 Et quand n'importe quelle autre chose prend Sa place,
 vous êtes sur le chemin de la régression.

Mais il y a de l'espoir pour celui qui régresse.
 Même si vous vous êtes bien éloignés de Lui, vous pouvez retrouver Dieu
 si vous Le recherchez de tout votre c\oe{}ur et de toute votre âme.
 Dieu est patient et \oe{}uvrera de nouveau dans votre vie.
 Il ne vous abandonnera pas. Il ne vous détruira pas.
 Il se rappellera de Son alliance car c'est un Dieu miséricordieux. 

\dvrule

\dvprayer{
Père, puisse Ton Esprit Saint
 défier nos c\oe{}urs avec la vérité de Ta Parole.
 Aide-nous à renverser toutes les idoles et à mettre Christ
 sur le trône de nos vies. 
}{\Amen}


\suggest{ou Esprit-Saint?}


%%%%%%%%%%%%%%%
% 31 janvier
%%%%%%%%%%%%%%%

\dvday{Preuve par l'épreuve}

\dvquote{
Tu te souviendras de tout le chemin que l'Éternel, ton Dieu,
 t'a fait faire pendant ces quarante années dans le désert,
 afin de t'humilier et de t'éprouver, pour reconnaître
 ce qu'il y avait dans ton c\oe{}ur et si tu observerais Ses commandements,
 oui ou non.
 Il t'a humilié, Il t'a fait souffrir de la faim et Il t'a nourri
 de la manne\dots{}
 Ton vêtement ne s'est pas usé sur toi, et ton pied ne s'est pas enflé pendant
 ces quarante années.
 Reconnais en ton c\oe{}ur que l'Éternel, ton Dieu,
 t'éduque comme un homme éduque son fils. 
}{\ibibleverse{Dt}(8:2-5)}

\lettrine{D}{ans} ses instructions finales,
 alors que les enfants d'Israël étaient juste sur le point
 d'entrer en Terre Promise, Moïse leur rappelle que Dieu
 a permis une mise à l'épreuve sévère pour leur enseigner trois choses.

Tout d'abord, Dieu voulait que les enfants d'Israël voient
 ce qui était dans leur c\oe{}urs.
 Face à l'épreuve, Lui obéiraient-ils ?
 Lui feraient-ils confiance ? L'adoreraient-t-ils?

\dvbox{
Dieu se sert des épreuves pour nous montrer nos c\oe{}urs,
 afin de pouvoir nous purifier et nous faire ensuite entrer
 dans le pays de la bénédiction \ocadr{}physiquement et spirituellement.
}

Deuxièmement, les Israélites devaient savoir que la vie
 est plus qu'une expérience physique.
 Dieu veut que nous vivions dans le domaine spirituel,
 en marchant selon l'Esprit et en étant guidés par l'Esprit.

Troisièmement, ils devaient savoir que Dieu punit les Siens,
 pour nous empêcher de faire des choses qui nous détruiraient.

Dieu se sert des épreuves de la même fa\c{c}on aujourd'hui.
 Il se sert des épreuves pour nous montrer nos c\oe{}urs,
 afin de pouvoir nous purifier et nous faire ensuite entrer
 dans le pays de la bénédiction. 

\dvrule

\dvprayer{
Père, puisse Ton Esprit Saint défier nos c\oe{}urs avec la vérité de Ta Parole.
 Aide-nous à renverser toutes les idoles
 et à mettre Christ sur le trône de nos vies. 
}{\Amen}

\suggest{ou Esprit-Saint?}



