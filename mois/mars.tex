\dvmonth{Mars}

%%%%%%%%%%%%%%
% 1er mars
%%%%%%%%%%%%%%

\dvday{Pratique du Tir à L\ap{}Arc}

\dvquote{
Ainsi [David] ordonna d’enseigner aux enfants de Judah,
 l’usage de l’arc\dots{}
}{\bibleverse{IISam}(1:18) \KJF}

\dvlettrine{Q}{uand} David reçu la nouvelle de la mort
 de son cher ami Jonathan et du Roi Saül qu'il admirait tant,
 il eut beaucoup de chagrin. Mais il ne se contenta pas de rester
 là à pleurer.
 Une des façons dont David réagit à son deuil fut d'ordonner
 à tous les papas de la tribu de Juda d'enseigner le tir à l'arc
 à leurs enfants.

C'était un mémorial particulièrement approprié pour Jonathan.
 David prit un des atouts de la vie de Jonathan et ordonna
 de l'enseigner à tous leurs enfants. De cette façon,
 l'influence de Jonathan perdurerait.
 Nous pouvons spiritualiser le texte en pensant à l'arc
 comme représentant l'arme spirituelle de la prière.

\dvbox{
Tout comme l'arc était capable d'envoyer des flèches et de frapper l'ennemi
 à une grande distance, nos prières sont une arme excellente par laquelle
 nous pouvons spirituellement influencer les gens à distance.
}

Quelquefois quand nous avons tant témoigné de notre foi auprès d'êtres chers,
 qu'ils commencent à ne pas apprécier (et à nous en vouloir),
 nous pouvons tirer des flèches à distance.
 L'Esprit commence à travailler dans leurs cœurs alors que nous,
 par la prière, lions l'œuvre de l'ennemi.

Tout comme David leur ordonna d'enseigner l'usage de l'arc à leurs enfants,
 nous devons apprendre à utiliser l'arc de la prière afin d'être efficaces
 dans le combat spirituel dans lequel nous sommes engagés. 

\dvrule

\dvprayer{
Père, aide-nous à devenir compétents dans l'usage des armes de combat
 que Tu nous a données, afin que nous puissions livrer bataille
 pour le compte de nos bien-aimés. 
}{\Amen}


%%%%%%%%%%%%%%
% 2 mars
%%%%%%%%%%%%%%

\dvday{Qui Suis-Je ?}

\dvquote{
Alors le roi David alla se présenter devant l'Éternel et dit~:
 \og Qui suis-je, Seigneur Éternel, et qu'est-ce que ma maison,
 pour que Tu m'aies fait parvenir jusqu'ici? \fg{}
}{\bibleverse{IISam}(7:18)}

\dvlettrine{C}{'est Nathan,} le prophète, qui rapporta à David
 les paroles de Dieu.
 \og David, tu ne peux bâtir une maison pour Moi, mais Je construirai
 une maison pour toi.
 Ton fils, qui siègera sur le trône après toi, bâtira une maison pour Moi.
 Et J'établirai ton trône pour toujours. \fg{}
 David comprit correctement que cela signifiait
 que le Messie serait son descendant. 

David en fut tellement bouleversé, qu'il pria Nathan de l'excuser
 et alla se recueillir devant Dieu.
 Il reconnut qu'il était totalement indigne d'une telle faveur
 et se rendit compte que c'était la grâce de Dieu qui lui avait été ainsi
 prodiguée dans une telle abondance. 

Comme David, vous vous dites peut-être~:
 \og Qui suis-je? J'ai fait quelque chose de mal. J'ai tout gâché. \fg{}
 Considérez la vie de David. Ce gars n'était pas un enfant de ch\oe{}ur.
 Il a fait bien pire que vous; cependant regardez ce que Dieu a fait pour lui.

\dvbox{
Il semble parfois que Dieu recherche le candidat le moins probable
 pour Lui accorder Son amour et Sa bonté. 
}

Si souvent, Dieu va dans le caniveau pour chercher le destinataire de Sa grâce.
 Il le relève, le lave et le transforme \ocadr faisant de lui un enfant de Dieu
 prêt pour Son royaume. Ça, c'est la grâce de Dieu. 

\dvrule

\dvprayer{
Merci, Père, pour Ta grâce abondante. En dépit de toutes nos carences,
 Tu nous as aimés et Tu as commencé à nous couvrir de riches bénédictions.
 Mais, comme si cela n'était pas encore suffisant,
 Tu as parlé de l'éternité à venir et de la gloire qui nous sera révélée.
}{\DlNdJ}


%%%%%%%%%%%%%%
% 3 mars
%%%%%%%%%%%%%%

\dvday{Le Dieu Qui Cherche}

\dvquote{
David lui dit~:
 \og Sois sans crainte !
 Pour sûr, je vais user de bienveillance envers toi
 à cause de ton père Jonathan.
 Je te rendrai toutes les terres de ton père Saül,
 et tu mangeras toujours à ma table. \fg{}
}{\bibleverse{IISam}(9:7)}

\dvlettrine{P}{lusieurs années auparavant,} David avait fait
 une alliance avec Jonathan, le fils de Saül.
 David avait promis, par égard pour Jonathan,
 que quand il accèderait au trône, il traiterait la famille de Saül
 avec une grande gentillesse.
 Aussi, une fois que David est arrivé sur le trône,
 qu'il a été béni et que Dieu l'a fait prospérer,
 David a recherché un des descendants de Saül
 pour lui montrer de la bienveillance. 

De même que David cherchait un descendant pour lui montrer
 de la bienveillance, Dieu recherche les perdus afin de montrer
 la richesse surabondante de Son amour et de Sa bonté envers eux
 en Jésus-Christ. C'est ce qui rend le Christianisme différent des religions.
 Dans les religions les hommes recherchent Dieu.
 Dans le Christianisme, Dieu recherche les hommes perdus. 

Mephibocheth fut en fait effrayé quand on l'amena devant David,
 parce qu'il ne connaissait pas les intentions de David.
 Si souvent, nous nous méprenons sur Dieu et nous imaginons
 qu'Il est en colère contre nous ou dégoûté de nous.
 Nous pensons qu'Il nous cherche afin de nous réprimander très sévèrement.
 Mais nous avons tort. 

David voulait bénir Mephibocheth par égard pour Jonathan.
 Dieu veut nous bénir par égard pour Christ. 

\dvbox{
Ce n'est pas parce que nous le méritons
 \ocadr c'est parce que Jésus nous désire en héritage. 
}

Tout comme Mephibocheth dînait à la table du roi
 comme un membre de la famille du roi,
 nous aussi avons été introduits dans la famille de Dieu
 et invités à venir nous asseoir à Sa table. 

\dvrule

\dvprayer{
Merci, Père, pour la joie, la bénédiction,
 la richesse dont nous faisons l'expérience
 à Ta table grâce à Jésus.
}{\DlNdJ}


%%%%%%%%%%%%%%
% 4 mars
%%%%%%%%%%%%%%

\dvday{Confession et Pardon}

\dvquote{
David dit à Nathan~:
 \og J'ai péché contre l'Éternel! \fg{}
 Et Nathan dit à David~:
 \og L'Éternel pardonne ton péché, tu ne mourras pas. \fg{}
}{\bibleverse{IISam}(12:13)}

\dvlettrine{L}{a confession} de péché de David
 était une vraie confession.
 Il avait effectivement péché contre le Seigneur.
 Sa tentative de dissimuler son péché
 n'avait que compliqué les choses. 

Mais notez bien qu'avec la confession de David,
 est arrivé un pardon immédiat.
 Oh, si seulement nous pouvions cesser d'essayer de dissimuler
 ou de justifier nos péchés!
 Le pardon ne peut venir qu'avec la confession.
 Ce n'est pas avant d'avoir reconnu et confessé votre culpabilité
 devant Dieu que vous pouvez recevoir la purification
 que Dieu veut vous donner. 

\dvbox{
Dieu prend plaisir à la miséricorde. 
}

Il veut pardonner.
 Mais tant qu'il n'y a pas eu de confession,
 Dieu n'a pas l'opportunité d'exercer Sa miséricorde et Sa grâce. 

Quel soulagement quand ce lourd fardeau de péché s'en va.
 Le mot hébreu \og béni \fg{} signifie \og oh, heureux! \fg{}.
 C'est ce que David décrit dans le \bibleverse{Ps}(32:)
 quand il écrit \og Béni fg{} ou \og Heureux \fg{} est l'homme
 dont le péché est pardonné.
 Combien est heureux l'homme qui a été restauré
 dans sa communion avec Dieu. 

Il est important de noter que bien que Dieu ait pardonné David,
 le péché a quand même bien laissé son empreinte.
 N'allez pas croire que vous pouvez pécher sans que cela ne laisse
 de marques dans votre vie.
 Dieu pardonne et la punition du péché est retirée.
 Des clous plantés dans une planche peuvent aussi être retirés,
 mais la planche restera toujours marquée par des trous.
 Votre péché peut être enlevé, mais il laisse sa marque. 

\dvrule

\dvprayer{
Père, nous prions pour que Ton Saint Esprit nous parle,
 nous qui croulons sous la culpabilité.
 Puisse-t-il y avoir de la confession, afin qu'il puisse
 aussi y avoir cette effusion de Ta grâce et de Ton pardon.
}{\Amen}


%%%%%%%%%%%%%%
% 5 mars
%%%%%%%%%%%%%%

\dvday{Ne Plus Être Bannis}

\dvquote{
\og Il nous faut certainement mourir.
 Comme des eaux répandues à terre ne se rassemblent plus,
 Dieu ne relève pas un mort, mais dans sa pensée,
 il ne faut pas que celui qui est banni loin de Lui
 le reste. \fg{}
}{\bibleverse{IISam}(14:14)}

\dvlettrine{L}{a femme} qui parlait à David était venue supplier
 le roi de se réconcilier avec son fils, Absalom, qu'il avait banni.
 Et par ces mots qui peignent une illustration imagée, elle rappelle à David
 la brièveté de la vie. Nous devons tous mourir.
 Quand cela nous arrive, nous sommes comme de l'eau qui a été répandue
 sur le sol et que l'on ne peut plus ramasser. 

Elle met en garde le roi que si l'amertume subsiste, et que lui ou son fils
 vienne à mourir, ils ne seront plus jamais en position
 de réparer leur relation.
 La mort met fin aux opportunités de réconciliation. 

Les familles se divisent souvent à propos de choses futiles.
 Et souvent, nous traitons plus durement ceux que nous aimons
 que ceux que nous ne connaissons même pas. 

N'est-il pas intéressant qu'il nous semble que certaines personnes
 soient plus pardonnables que d'autres?
 Mais Dieu va pardonner même les crimes les plus odieux.
 Dieu offre Son pardon et Sa grâce gratuitement. 

\dvbox{
Les portes du pardon sont ouvertes à tous ceux qui l'invoquent. 
}

Dieu a en effet conçu un moyen par lequel \og Ses bannis \fg{}
 \NdT{c'est-à-dire ceux qui ont été bannis de Sa présence}
 ne restent pas séparés de Lui.
 C'est précisément le sujet dont l'évangile traite \ocadr la restauration.
 Par l'intermédiaire de Son Fils, qui a pris notre culpabilité et notre honte
 sur Lui-même, Dieu a restauré ceux qui avaient été bannis en conséquence
 de leur péché. 

\dvrule

\dvprayer{
Père, Je prie pour ceux qui ont été séparés de Ton amour et de Ta communion
 par leurs actions et leurs péchés.
 Puissent-ils se tourner vers Toi et recevoir miséricorde, grâce et pardon.
}{\DlNdJ}


%%%%%%%%%%%%%%
% 6 mars
%%%%%%%%%%%%%%

\dvday{Engagement}

\dvquote{
Ittaï répondit au roi en ces mots~:
 \og L'Éternel est vivant et mon seigneur le roi est vivant!
 A l'endroit où sera mon seigneur le roi, soit pour mourir,
 soit pour vivre, là aussi sera ton serviteur. \fg{}
}{\bibleverse{IISam}(15:21)}

\dvlettrine{I}{ttaï} s'engagea totalement envers David,
 bien qu'il ne soit arrivé de Gath que la veille.
 Beaucoup des amis fidèles de David désertaient le navire
 et rejoignaient Absalom, et voici que cet étranger arrive
 et promet une loyauté indéfectible à David !
 David ne faisait aucune espèce d'offres particulièrement attrayantes
 à Ittaï, mais cet homme s'était figuré que le pire que David avait
 à offrir valait encore mieux que le meilleur qu'Absalom avait à offrir.

L'une des tragédies de notre époque est l'absence d'engagement véritable. 

\dvbox{
Les gens sont réticents à s'engager complètement pour quoi que ce soit. 
}

Dans les v\oe{}ux de mariage, nous pouvons affirmer
 \og pour le meilleur et pour le pire \fg{}, mais dès que les choses
 commencent à ne plus aller, les gens veulent tout laisser tomber.
 L'engagement qu'ils avaient pris n'était pas un engagement total. 

L'église souffre, aujourd'hui, d'une grave faiblesse.
 Les gens acceptent le Seigneur, puis ils se mettent à hésiter.
 Ils montrent qu'il n'y a pas un réel et profond engagement envers Jésus.

Le genre d'engagement qu'Ittaï a pris envers David est le genre d'engagement
 que Jésus désire de notre part aujourd'hui.
 Il n'a pas offert un évangile au rabais ou un chemin facile.
 Mais si vous vous engagez complètement envers Lui, votre vie va être pleine,
 joyeuse et richement satisfaisante \ocadr parce que vous saurez
 que vous êtes exactement à l'endroit voulu par Dieu. 

\dvrule

\dvprayer{
Père, nous nous rendons compte que nous étions à un moment des étrangers
 dans ce monde, exclus de la communion avec Toi.
 Comme nous sommes reconnaissants que Tu nous aies choisis
 pour être Tes disciples et que Tu nous aies amenés à ce point d'engagement
 où nous avons dit~: \og Seigneur, je T'appartiens \fg{} !
}{\DlNdJ}


%%%%%%%%%%%%%%
% 7 mars
%%%%%%%%%%%%%%

\dvday{Laissez-Vous Simplement Aller}

\dvquote{
Le roi dit à Tsadoq~:
 \og Rapporte l'arche de Dieu dans la ville.
 Si j'obtiens la faveur de l'Éternel,
 Il me ramènera et Il me fera voir l'arche et sa demeure.
 Mais s'Il dit~: Je ne t'agrée plus ! me voici,
 qu'Il me fasse ce qui Lui semblera bon. \fg{}
}{\bibleverse{IISam}(15:25-26)}

\dvlettrine{D}{avid} fuit Jérusalem pour échapper à Absalom et son armée.
 Pendant que les fidèles compagnons de David viennent le rejoindre,
 Tsadoq, le grand-prêtre sort des portes de la cité avec quelques uns
 des Lévites qui portent l'arche sacrée de l'alliance. 

David déclara~:
 \og Je ne vais pas compter sur une relique religieuse pour être sauvé.
 Je ne vais compter que sur Dieu. Si Dieu me sauve, très bien. Je reviendrai.
 Je reverrai l'arche. Mais si Dieu n'estime pas devoir me sauver,
 alors me voici et qu'Il me fasse ce que bon Lui semble,
 quoi que ça puisse être. \fg{}

La vie de David est complètement consacrée au dessein
 et au plan que Dieu peut avoir pour lui. 

\dvbox{
Je ne peux pas changer ce que Dieu a décrété.
 Si je cherche a lutter contre le plan de Dieu,
 alors je suis voué à l'échec.
}

Si je me bats contre Dieu, tout ce que je vais connaître, c'est l'anxiété,
 un sentiment de futilité et la frustration.
 Mais quand je me rends finalement compte qu'il ne me reste rien à faire,
 je découvre l'endroit de repos. 

Sachant que les voies de Dieu sont supérieures aux miennes,
 sachant que Dieu m'aime et qu'Il est totalement conscient des circonstances
 auxquelles je fais face, et reconnaissant que je ne sais vraiment pas comment
 me sortir de la situation, je la place entre Ses mains. 

\dvrule

\dvprayer{
Père, puissions-nous cesser de nous agiter et de nous faire du souci,
 et puissions-nous nous laisser aller sachant que Tu nous aimes
 et que Tu vas faire ce qui est le mieux.
}{\Amen}


%%%%%%%%%%%%%%
% 8 mars
%%%%%%%%%%%%%%

\dvday{Les Promesses de Dieu}

\dvquote{
Je remercie le Seigneur qui a donné la paix à Israël, son peuple,
 tout comme il l'avait promis; en effet, il a réalisé en tous ses détails
 la merveilleuse promesse qu'il avait faite par l'intermédiaire
 de son serviteur Moïse.
}{\bibleverse{IKgs}(8:56) \BFC} 

Dans le chapitre 26 du \bibleverse{Lv}, Dieu avait promis à Moïse
 qu'Il donnerait à Son peuple la terre promise et que la paix y règnerait.

Dans le verset 11 du même chapitre, Dieu avait aussi déclaré~:
 \og J'établirai Mon tabernacle au milieu de vous. \fg{}
 Ici dans \bibleverse{IKgs}, Salomon célèbre précisément
 la dédicace de ce temple, et il profite de l'occasion pour rappeler au peuple
 la promesse de Dieu.
 Bien que cela Lui ait pris 490 ans, pas un seul des détails des promesses
 de Dieu n'a manqué de se réaliser. 

\dvbox{
Dieu tient Ses promesses;
 pas une seule parole n'a manqué de se réaliser. 
}

Vous feriez bien de faire attention à ce que Dieu dit.
 Le fait que Dieu tienne ainsi Sa parole à la lettre va,
 soit grandement encourager votre c\oe{}ur, soit vous terroriser.
 Tout dépend d'où vous en êtes dans votre relation avec Lui. 

Je suis tellement excité que Dieu tienne fidèlement parole,
 car Il a promis que si je confesse de ma bouche le Seigneur Jésus
 et que si je crois dans mon c\oe{}ur que Dieu l'a ressuscité
 d'entre les morts, je vais être sauvé.
 Dieu a promis que si je confesse Jésus devant les hommes,
 Il me confessera devant Son Père.
 Dieu a promis que si je Le reçois, j'aurai la vie éternelle. 


\dvrule

\dvprayer{
Père, tout comme Salomon pouvait témoigner de Ta fidélité, nous aussi,
 quelques trois mille ans plus tard, pouvons donner le même témoignage
 et le même constat de Ta fidélité.
}{\DlNdJ}


%%%%%%%%%%%%%%
% 9 mars
%%%%%%%%%%%%%%

\dvday{Plus Que Nous Ne Demandons}

\dvquote{
L'Éternel lui dit~:
 \og J'ai écouté ta prière et la supplication que tu M'as adressées,
 Je consacre cette maison que tu as bâtie pour y mettre à jamais Mon nom
 et J'y aurai toujours Mes yeux et Mon cœur.
}{\bibleverse{IKgs}(9:3)} 

\dvlettrine{L}{e Dieu} qui a créé cet univers m'aime.
 Il pense à moi et Il m'entend quand je prie.
 Non seulement Il m'entend, mais de façon étonnante, Il me répond.

Cependant, Il ne me répond pas toujours quand je voudrais qu'Il le fasse.
 Alors je deviens impatient. Je veux que Dieu me réponde
 et agisse à ma demande immédiatement.
 Mais souvent Dieu attend parce qu'Il désire me donner davantage
 que ce que je demande.
 Ainsi donc, Il attend que je me mette en parfaite harmonie avec Son plan. 

Salomon avait prié pour que cette maison soit un lieu où le peuple de Dieu
 puisse se réunir pour rencontrer Dieu, et il avait demandé
 que les yeux du Seigneur soient fixés sur cet endroit.
 Dieu a répondu~:
 \og Non seulement j'y aurais Mes yeux,
 mais Mon c\oe{}ur sera là, aussi. \fg{}

\dvbox{
Dieu désire donner davantage que nous ne demandons.
 Trop souvent les gens recherchent les yeux de Dieu au lieu de rechercher
 le c\oe{}ur de Dieu, mais c'est dans le c\oe{}ur de Dieu
 que de vouloir nous bénir.
}

Puisse Dieu vous attirer à Lui dans une communion plus intime
 que ce que vous avez jamais connu, où non seulement Ses yeux seront sur vous
 mais Son c\oe{}ur aussi, alors que vous le recherchez
 et qu'Il vous tend Sa main.
 Puissiez-vous connaître dans une plus grande mesure le contact de l'amour,
 de la puissance et de la présence de Dieu. 

\dvrule

\dvprayer{
Père, nous te remercions de nous entendre quand nous appelons.
 Attire-nous à Toi. Puissions-nous trouver l'endroit de repos
 tranquille près de Ton c\oe{}ur.
}{\Amen}


%%%%%%%%%%%%%%
% 10 mars
%%%%%%%%%%%%%%

\dvday{Religion Facile}

\dvquote{
Le roi\dots{} fit deux veaux d'or et dit au peuple~:
 \og Vous êtes assez montés à Jérusalem !
 Israël, voici tes dieux qui t'ont fait monter du pays d'Égypte. \fg{}
 Il en plaça un à Béthel et il mit l'autre à Dan.
 Ce fut là une occasion de péché\dots{}
}{\bibleverse{IKgs}(12:28-30)}

\dvlettrine{L}{e roi Jéroboam} chercha à créer une religion facile
 pour le peuple. Sa logique était~:
 \og Il est trop difficile pour vous de monter à Jérusalem
 pour adorer le Seigneur. C'est un trop grand sacrifice.
 Aussi, nous allons vous simplifier la vie.
 Nous allons installer des dieux ici même,
 dans votre propre communauté. \fg{}

Les religions faciles plaisent toujours à notre chair
 parce qu'elles n'exigent pas que nous renoncions à nous-mêmes.
 Cependant, quelle a été la première exigence spécifiée par Jésus
 à quelqu'un qui voulait Le suivre?
 \og Renonce à toi-même. \fg{}
 Je ne peux pas marcher selon l'Esprit et selon la chair.
 Les deux sont contraires l'un à l'autre. 

\dvbox{
Les religions faciles ne vous mèneront jamais au vrai créateur,
 éternel et vivant. 
}

Il est possible que les religions faciles plaisent à ma chair,
 mais elles ne plaisent pas à Dieu.
 En cherchant mon propre plaisir, je perds le Sien.
 Si vous suivez ces religions faciles, vous allez vous retrouver aliénés
 du vrai Dieu en cherchant à faire de vous-mêmes un dieu. 

Le chemin qui mène à la vie éternelle commence à la croix;
 vous ne pouvez pas y échapper.
 C'est un chemin droit. C'est un chemin étroit.
 Mais il conduit à la vie éternelle. 

\dvrule

\dvprayer{
Père, si notre passion ou ce qui motive nos vies
 est quelque chose d'autre que Toi, aide-nous à l'amener à la croix
 et à renoncer à nous-mêmes, afin que nous puissions Te suivre
 de tout notre c\oe{}ur.
}{\Amen}


%%%%%%%%%%%%%%
% 11 mars
%%%%%%%%%%%%%%

\dvday{Nouvelles Pénibles}

\dvquote{
Lorsque Ahiya entendit le bruit de ses pas,
 au moment où elle franchissait la porte, il dit~:
 \og Entre, femme de Jéroboam; pourquoi cela?
 Tu te fais passer pour une autre!
 Je suis envoyé vers toi avec un message pénible. \fg{}
}{\bibleverse{IKgs}(14:6)}

\dvlettrine{J}{éroboam} avait à un moment créé de faux dieux
 pour que le peuple les adore.
 Mais quand son fils est tombé mortellement malade,
 Jéroboam n'a pas cherché de l'aide auprès de ces faux dieux.
 Au lieu de ça, il a déguisé sa femme en paysanne
 et l'a envoyée chercher de l'aide auprès d'Ahiya, prophète de Dieu.

\dvbox{
Comme il est ridicule de penser que nous pouvons cacher
 la vérité à Dieu! 
}

Il connaît la vérité. Dieu voit à travers tous les déguisements,
 parce qu'Il voit nos c\oe{}urs. 

Le prophète était aveugle, mais il n'était certainement pas sourd.
 Il pouvait entendre la voix du Seigneur, et il passa le message de Dieu
 à la femme de Jéroboam.  Il lui dit~:
 \og Je suis envoyé vers toi avec un message pénible. \fg{}

Sachant que Dieu est gracieux et plein de compassion,
 je ne peux m'empêcher de croire que si Jéroboam s'était vraiment repenti
 et avait cherché Dieu avec sincérité, Dieu l'aurait touché
 et aurait guéri son fils et aurait établi son trône pour toujours.
 Mais ça ne devait pas se passer ainsi. 

L'hypocrisie ne mène qu'à la destruction.
 Si votre c\oe{}ur n'est pas honnête devant Dieu,
 vous êtes engagés sur une pente glissante et dangereuse.
 Mais si vous abandonnez votre déguisement et venez à Dieu
 avec une complète sincérité, Il enlèvera rapidement ce lourd fardeau
 de vos épaules, parce qu'Il vous aime et veut bénir votre vie. 

\dvrule

\dvprayer{
Père, merci pour Ta fidélité dans la façon dont Tu nous traites.
 Seigneur, nous voulons entendre ce que Tu as à dire sur le futur.
 Aide-nous à t'écouter et t'obéir.
}{\Amen}


%%%%%%%%%%%%%%
% 12 mars
%%%%%%%%%%%%%%

\dvday{Indécision}

\dvquote{
Et Élie s’approcha de tout le peuple et dit~:
 \og Combien de temps balancerez-vous entre deux opinions ?
 Si le \Seigneur est Dieu, suivez-le;
 mais si c’est Baal, alors suivez-le.
 Et le peuple ne lui répondit pas un mot.
}{\bibleverse{IKgs}(18:21) \KJF}

\dvlettrine{Q}{uand Élie} a lancé son défi aux prophètes
 de Baal et d'Astarté, deux groupes ont rejoint ce fidèle serviteur
 sur le Mont Carmel~: les 850 \og prophètes \fg{} de ces deux faux dieux
 et la multitude hésitante
 \ocadr la foule qu'Élie haranguait dans ce passage. 

Les temps ont changé, mais pas les gens.
 Aujourd'hui, nous avons les trois mêmes catégories de gens.
 Il y a ceux qui, comme Élie, ont consacré leurs vies au Seigneur
 et le servent activement.
 Puis il y a ceux qui ont consacré leurs vies au mal.
 Ils travaillent dur pour retirer Dieu du secteur public
 et manifestent pour des causes telles que l'avortement et l'homosexualité.
 Ce sont ces gens qui cherchent à plonger notre nation
 dans les puits gluants de l'immoralité. 

\dvbox{
Ceux qui ne sont pas engagés restent les bras croisés
 en regardant les forces des ténèbres prendre le pouvoir. 
}

Et puis, il y a ce dernier groupe \ocadr la multitude hésitante.
 Ce sont les gens qui ne se sont pas encore engagés
 ni d'un côté ni de l'autre. Jésus a décrit ces gens comme étant tièdes,
 une condition qui provoque la nausée.
 Ils ne font rien du tout pour enrayer la montée du mal.
 Parce qu'ils ne défendent rien, ils permettent tout. 

Si vous êtes un attentiste assis entre deux chaises,
 aujourd'hui est un bon jour pour choisir votre camp.
 Que votre choix soit Jésus
 \ocadr que votre vie puisse avoir une influence positive pour le bien. 

\dvrule

\dvprayer{
Aide-nous, Père, à avoir le courage de défendre la justice.
 Donne-nous la même détermination, le même enthousiasme
 que ceux des ouvriers d'iniquité
 \ocadr afin que nous puissions défendre Jésus et Lui amener la gloire.
}{\NpDlNdJ}


%%%%%%%%%%%%%%
% 13 mars
%%%%%%%%%%%%%%

\dvday{Que fais-tu ?}

\dvquote{
Là-bas, il entra dans la grotte et y passa la nuit.
 Or, voici que la parole de l'Éternel
 lui fut adressée en ces mots~:
 Que fais-tu ici, Élie ?
}{\bibleverse{IKgs}(19:9)}

\dvlettrine{N}{ous avons tendance} à penser aux grands hommes de Dieu
 comme à des \og super-saints \fg{},
 des hommes d'une catégorie supérieure à la nôtre. Nous pensons~:
 \og Je ne pourrai jamais faire les choses qu'Élie a faites. \fg{}
 Mais ce n'était qu'un homme. Bien qu'il ait eu les mêmes défauts
 et faiblesses que nous, Dieu l'a utilisé d'une puissante façon.
 Et ça, c'est encourageant. Cela signifie que Dieu peut m'utiliser aussi.

\`A ce moment spécifique de sa vie, Élie était si découragé
 qu'il voulait mourir. Le sachant, Dieu est venu à lui,
 et d'une petite voix douce et subtile lui a demandé ce qu'il faisait.
 Mais au lieu de Lui répondre honnêtement~: \og Je me cache \fg{},
 Élie s'est cherché des excuses devant Dieu. 

Nous faisons la même chose. Dieu nous voit nous cacher
 et nous demande ce que nous faisons.
 Au lieu de confesser~: \og Je ne fais rien \fg{},
 nous commençons à inventer des excuses.
 Et quand vous devenez bon à inventer des excuses,
 vous n'êtes généralement plus bon à rien d'autre. 

Êtes-vous découragés? Si c'est le cas, laissez-moi vous demander~:
 Que faites-vous pour Dieu qui compte vraiment?
 Combien de temps et d'énergie avez-vous investi dans les choses éternelles?

\dvbox{
Le meilleur remède au découragement est de se mettre à travailler pour Dieu. 
}

En vous mettant au travail pour Lui, vous allez oublier vos propres problèmes
 et vous allez commencer à éprouver un sentiment de plénitude.
 Si vous écoutez cette petite voix douce et subtile aujourd'hui,
 il se pourrait que vous soyez surpris par ce que Dieu vous appelera à faire. 

\dvrule

\dvprayer{
Père, nous te remercions d'être patients avec nous.
 Puissions-nous entendre Ta petite voix douce et subtile
 et te répondre honnêtement.
}{\NpDlNdJ}


%%%%%%%%%%%%%%
% 14 mars
%%%%%%%%%%%%%%

\dvday{Envies}

\dvquote{
Sa femme Jézabel vint auprès de lui et lui dit~:
 \og Pourquoi as-tu l'esprit maussade et ne manges-tu point? \fg{}
 Il lui répondit~:
 \og J'ai parlé à Naboth de Jizréel et je lui ai dit~:
 Donne-moi ta vigne pour de l'argent; ou, si tu veux,
 je te donnerai une autre vigne à la place. \fg{}
 Mais il a dit~:
 \og Je ne te donnerai pas ma vigne! \fg{}
}{\bibleverse{IKgs}(21:5-6)}

\dvlettrine{A}{chab convoitait} la vigne de Naboth.
 Et parce qu'il la convoitait, il s'est laissé influencer par Jézabel.
 Il a laissé Naboth être condamné par des faux témoignages et lapidé\dots{}
 puis il a volé la vigne de Naboth.
 La convoitise de Naboth a mené au mensonge, au meurtre et au vol.

Il y a une vaste différence entre admiration et convoitise.
 Je peux admirer ce que vous avez. Je peux dire~:
 \og Ça alors, C'est formidable! C'est vraiment très beau.
 Je suis heureux pour vous. \fg{}
 Mais quand je commence à convoiter ce que vous avez et à penser:
 \og Je voudrais que cette chose m'appartienne \fg{},
 alors je deviens coupable de convoitise. 

\dvbox{
La seule chose que nous devrions convoiter,
 c'est une relation plus intime avec Dieu. 
}

La Bible nous dit que la convoitise est de l'idolâtrie.
 C'est parce que la chose que vous désirez devient le centre d'attention
 de votre c\oe{}ur et de vos pensées.
 Toutes vos journées sont passées à élaborer des stratagèmes
 pour acquérir cette chose. 

La Bible dit que nous ne devons désirer qu'une seule chose.
 \og Mais désirez ardemment les meilleurs dons (de l'Esprit) \fg{}
 (\bibleverse{ICor}(12:31) \KJF).
 La seule chose que nous devrions convoiter est une relation
 plus intime avec Dieu. Si vous avez à convoiter une chose,
 convoitez la puissance de l'Esprit de Dieu librement
 et pleinement à l'\oe{}uvre dans votre vie. 

\dvrule

\dvprayer{
Père, crée en nous un c\oe{}ur qui ait soif de Toi,
 afin de pouvoir connaître la joie et le contentement de vivre
 en communion avec Toi.
}{\CDlNdJqnp}


%%%%%%%%%%%%%%
% 15 mars
%%%%%%%%%%%%%%

\dvday{Où est le Dieu d'Élie ?}

\dvquote{
Il prit le manteau qu'Élie avait laissé tomber,
 il en frappa les eaux et dit~:
 \og Où est l'Éternel, le Dieu d'Élie? \fg{}
 Lui aussi, il frappa les eaux qui se partagèrent çà et là.
 Élisée passa.
}{\bibleverse{IIKgs}(2:14)}

\dvlettrine{É}{lisée avait vu} le pouvoir de Dieu sur la vie d'Élie.
 En fait, il avait même vu un chariot de feu emmener le prophète au ciel.
 Maintenant la responsabilité d'être un prophète pour le peuple
 revenait à Élisée, et il savait qu'il ne pourrait pas servir si Dieu
 ne lui conférait pas de Sa puissance. Aussi a-t-il imploré de l'aide. 

\og Où est le Dieu d'Élie? \fg{}
 Élisée a cherché de l'aide auprès du Dieu qui avait révélé
 Sa puissance au Mont Carmel. Il voulait qu'Il révèle la même puissance
 devant le peuple afin que ces gens sachent que Jéhovah
 est vivant et qu'Il est Dieu. 

\dvbox{
Dieu ne change pas.
 Il veut montrer sa puissance à cette génération. 
}

Vivant dans un âge de scepticisme, je m'écrie moi aussi~:
 \og Oh Dieu, montre-Toi. Prouve-Toi, Seigneur. \fg{}
 Mon cri est avec Élisée~: Où est le Dieu d'Élie qui peut démontrer
 Sa puissance pour prouver aux gens qu'Il est toujours vivant? 

Nous ne voyons pas la démonstration de la puissance de Dieu dans notre monde,
 mais la faute nous en incombe. Dieu est vivant et Il est aux commandes.
 Puissions-nous être un peuple qui l'implorons; un peuple par l'intermédiaire
 duquel Il est révélé au monde. 

\dvrule

\dvprayer{
Père, nous recherchons la puissance de Ton Esprit agissant
 au milieu de Ton peuple, afin de voir Ton nom glorifié sur nos lèvres.
 Puissions-nous être les instruments par lesquels Ta puissance
 et Ton amour sont révélés.
}{\Amen}


%%%%%%%%%%%%%%
% 16 mars
%%%%%%%%%%%%%%

\dvday{Des Yeux Pour Voir}

\dvquote{
Élisée pria en disant~:
 \og Éternel, ouvre ses yeux, je T'en prie, pour qu'il voie. \fg{}
 L'Éternel ouvrit les yeux du jeune serviteur qui vit ceci~:
 la montagne pleine de chevaux et de chars de feu autour d'Élisée.
}{\bibleverse{IIKgs}(6:17)}

\dvlettrine{Q}{uand Guéhazi,} le serviteur d'Élisée, se réveilla
 et alla regarder dehors, il vit les chars de l'armée syrienne
 qui encerclaient la cité de Dotân.
 Rentrant en courant, il dit à Élisée~:
 \og Nous sommes pris au piège. Nous sommes encerclés. \fg{}

Élisée répondit~:
 \og N'aie pas peur, car ceux qui sont avec nous sont plus nombreux
 que ceux qui sont avec eux. \fg{}
 Puis il pria pour que les yeux de son serviteur soient ouverts
 afin qu'il puisse entrevoir le monde spirituel.
 Et quand les yeux de Guéhazi furent ouverts,
 il vit des chars de feu qui encerclaient l'ennemi. 

\dvbox{
Comme les choses apparaissent différentes
 quand nos yeux spirituels sont ouverts! 
}

Quand nous nous contentons de regarder nos circonstances physiques,
 nous désespérons. Mais si vous pouvez regarder la vérité spirituelle,
 vous obtenez alors une perspective complètement nouvelle.
 Plutôt qu'une défaite certaine, vous voyez une victoire certaine.
 Avec mes yeux spirituels et la Parole de Dieu pour guide
 dans les choses spirituelles, je peux voir ce que Dieu fait,
 et je peux me réjouir parce que je sais que Christ a déjà gagné
 la bataille sur les forces des ténèbres et du mal. 

Puisse Dieu ouvrir nos yeux pour voir la puissance qui est à notre
 disposition par l'intermédiaire de Jésus-Christ et de l'Esprit Saint,
 afin que nous puissions connaître et avoir la pleine victoire de Dieu
 dans nos vies. 

\dvrule

\dvprayer{
Père, nous Te remercions de ce que nous ne sommes pas obligés
 d'être vaincus dans aucun domaine de nos vies.
 Dieu, ouvre nos yeux afin que nous puissions voir la vérité
 et la réalité du monde spirituel.
}{\DlNdJ}


%%%%%%%%%%%%%%
% 17 mars
%%%%%%%%%%%%%%

\dvday{Le Respect Ne Suffit Pas}

\dvquote{
Lorsqu'Élisée fut atteint de la maladie dont il mourut,
 Joas, roi d'Israël, descendit vers lui,
 pleura sur son visage et dit~:
 \og Mon père! Mon père! Char d'Israël et sa cavalerie! \fg{}
}{\bibleverse{IIKgs}(13:14)}

\dvlettrine{B}{ien que le Roi Joas} vive dans l'idolâtrie,
 il respectait le prophète de Dieu.
 Élisée avait changé le cours de l'histoire d'Israël.
 Par son intermédiaire, Dieu avait réalisé plus de miracles
 que par l'intermédiaire d'aucun autre homme de la période
 de l'Ancien Testament.
 Ainsi donc, quand le roi apprit qu'Élisée était mourant,
 il vint lui rendre visite et se mit à pleurer de chagrin.
 Dans le passage cité, Joas reconnaissait que cet homme était
 la vraie force d'Israël, car il faisait mention de lui comme étant
 \og le Char d'Israël et sa cavalerie. \fg{}

À cette époque, le char était l'armement le plus redoutable
 dont une nation pouvait disposer sur le champ de bataille.
 C'était le facteur décisif dans une guerre.
 Ce terme représentait donc la force et la puissance.
 Et considérant ce prophète, le roi reconnaissait que cet homme
 en communion avec Dieu était la vraie force et la puissance
 de la nation, le \og Char d'Israël. \fg{}

Au fond de lui, Joas savait que le seul espoir pour la nation
 résidait en Dieu. Et pourtant, il persistait dans ses mauvaises voies.
 Il est comme tant de personnes aujourd'hui qui reconnaissent
 l'existence de Dieu et le respectent, mais qui pourtant refusent de le servir.

\dvbox{
Un engagement partiel ne peut jamais conduire à la victoire totale. 
}

Le respect de Dieu ne suffit pas. Même les démons le respectent.
 Mais le respect n'amène pas le salut.
 La victoire totale n'arrive que par l'engagement total. 

\dvrule

\dvprayer{
Père, nous Te remettons nos vies. Fais de nous Tes serviteurs
 et Tes instruments, afin que, par nous,
 Tu puisses accomplir Tes desseins.
}{\DlNdJ}


%%%%%%%%%%%%%%
% 18 mars
%%%%%%%%%%%%%%

\dvday{Se mêler de ce qu'il ne faut pas}

\dvquote{
Bien sûr, tu as battu les Édomites, et ton cœur s'élève.
 Jouis de ta gloire et reste chez toi.
 Pourquoi t'engager dans une malheureuse entreprise?
 Tu tomberas, et Juda avec toi!
}{\bibleverse{IIKgs}(14:10)}

\dvlettrine{A}{matsia} venait juste de battre Édom.
 Rempli d'orgueil, il mit Joas, le roi d'Israël,
 au défi de sortir le combattre.
 Le conseil de Joas au plus jeune roi fut~:
 \og Pourquoi te mêlerais-tu de ce qui te va faire du mal? \fg{}

Mais Amatsia a persisté dans son défi jusqu'à ce que Joas
 amène ses troupes, batte l'armée d'Amatsia, et arrive
 jusqu'à la cité de Jérusalem, où il a ouvert une brèche
 dans une section de la muraille et a pris l'or et l'argent
 qu'il y a trouvé.
 Il a alors pris des otages avant de rentrer en Israël. 

\dvbox{
Se mêler de choses inappropriées n'amènera que la défaite.
}

Les grandes victoires spirituelles peuvent nous rendre vulnérables,
 parce que nous sortons de ces victoires avec le sentiment de pouvoir
 conquérir le monde. En vérité, sans Christ, nous sommes impuissants
 dans ce combat spirituel. 

Comme Amatsia, certains d'entre vous se mêlent de choses
 dont vous ne devriez pas vous mêler. Peut-être, de drogue ou d'alcool.
 Peut-être êtes-vous mariés, mais vous flirtez avec quelqu'un,
 ou peut-être n'êtes vous pas mariés et vous avez commencé
 à vous mêler de sexe. Ne vous y trompez pas.
 Se mêler de ce genre de choses détruit les murs et les défenses,
 ce qui fait qu'il est plus facile à l'ennemi de vous attaquer
 de la même façon plus tard. Et se mêler de choses inappropriées
 conduit à la perte de l'innocence et de la pureté. 

Mais Dieu est un maître pour restaurer les choses.
 Si vous vous tournez vers Lui aujourd'hui,
 Il peut reconstruire vos défenses et restaurer votre pureté. 

\dvrule

\dvprayer{
Merci, Seigneur, de nous relever quand nous avons chuté.
 Merci de restaurer les trésors qui ont été dérobés
 par notre rencontre avec le monde.
}{\DlNdJ}


%%%%%%%%%%%%%%
% 19 mars
%%%%%%%%%%%%%%

\dvday{Raison du Déclin des Nations}

\dvquote{
Aussi l'Éternel a-t-Il éprouvé une vive colère contre Israël,
 et les a-t-Il écartés de Sa face.
 Il n'est resté que la seule tribu de Juda.
}{\bibleverse{IIKgs}(17:18)}

\dvlettrine{D}{ans cette brève phrase,} nous trouvons le
 \og certificat de décès \fg{} de la nation d'Israël.
 Après avoir épuisé la patience de Dieu,
 Israël est éliminé en tant que nation de la surface de la terre.

Quand Il les avait sortis d'Égypte, Dieu avait conclu
 une alliance avec Israël. Il leur avait promis
 que s'ils gardaient Ses lois,
 Il serait leur Dieu et ils seraient Son peuple.
 Il avait continué en leur promettant que s'ils gardaient Ses commandements,
 Il les bénirait, les protègerait et les ferait prospérer en tant que nation.

Le peuple accepta l'alliance \ocadr mais ils ne la respectèrent pas.
 Au lieu de la respecter, ils se détournèrent de Dieu
 et adorèrent des idoles. Et pour cette raison, Dieu les a éliminés. 

\dvbox{
Poursuivre le vide ne peut que vous rendre vide. 
}

Je suis toujours stupéfait par la patience de Dieu.
 Il a habité au milieu de son peuple pendant 240 ans,
 et envoyé prophètes après prophètes pour les mettre en garde
 contre leur folie. Comme notre Dieu est patient! 

Et comme l'homme est ridicule quand il se détourne du Dieu vivant.
 Voyez-vous, quand vous vous détournez de Dieu,
 vous créez un vide dans votre vie. La nature a horreur du vide.
 Inévitablement, vous allez vous saisir de quelque chose pour combler
 ce vide intérieur. C'est exactement ce qui s'est passé avec Israël.
 Ils se sont tournés vers de faux dieux païens pour combler le vide,
 et en retour, ils n'ont absolument rien reçu. 

C'est parce que rien d'autre ne peut combler la place destinée à Dieu. 

\dvrule

\dvprayer{
Père, nous Te remercions pour Ta patience et Ta miséricorde.
 Nous reconnaissons que Toi seul peut combler le vide dans nos vies.
}{\DlNdJ}


%%%%%%%%%%%%%%
% 20 mars
%%%%%%%%%%%%%%

\dvday{Détruire les Idoles}

\dvquote{
Il fit disparaître les hauts lieux, brisa les stèles,
 coupa le poteau d'Achéra et mit en pièces le serpent de bronze
 que Moïse avait fait, car les Israélites avaient jusqu'alors
 brûlé des parfums devant lui~: on l'appelait Nehouchtân.
}{\bibleverse{IIKgs}(18:14)}

\dvlettrine{P}{endant son périple dans le désert,} Israël s'est plaint
 de Dieu et de Moïse.
 Dieu a donc envoyé des serpents venimeux qui ont commencé à mordre
 et à faire mourir les gens.
 Se rendant compte qu'ils avaient péché contre Dieu,
 les gens ont demandé à Moïse de prier pour eux.
 Le Seigneur a alors dit à Moïse de façonner un serpent de bronze
 et de le mettre sur un poteau au milieu du camp.
 Quand les gens étaient mordus, ils devaient regarder le serpent
 sur le poteau. S'ils le faisaient, ils étaient sauvés. 

La foi dans les promesses de Dieu guérissait ceux qui avaient été mordus.
 Mais plus tard, Israël s'est mis à adorer le serpent
 que Moïse avait façonné. Quand Ézéchias s'en rendit compte,
 il mit en pièces le serpent de bronze et l'appella \og Nehouchtân \fg{},
 ce qui, en hébreu, signifie \og chose en bronze \fg{}.
 Il voulait rappeler au peuple que le serpent ne devait pas faire
 l'objet d'une adoration; ce n'était qu'un objet en bronze. 

Quand l'homme fait de reliques religieuses des idoles,
 cela signifie qu'il a perdu la conscience de la présence de Dieu dans sa vie.
 Cela signifie qu'il essaye de retrouver ce qui a été perdu. 

\dvbox{
Notre relation avec Dieu devrait toujours progresser. 
}

Si vous pouvez vous souvenir d'un temps où vous étiez plus proches de Dieu
 que vous ne l'êtes aujourd'hui, alors vous êtes retombés en arrière.
 Il est temps de briser ces serpents de bronze et de remettre votre foi en Jésus. 

\dvrule

\dvprayer{
Père, aide-nous à avoir une relation plus proche avec Toi,
 où Jésus signifie plus pour nous que toute autre chose au monde.
}{\DlNdJ}


%%%%%%%%%%%%%%
% 21 mars
%%%%%%%%%%%%%%

\dvday{Mettre Notre Maison en Ordre}

\dvquote{
En ce temps-là, Ézéchias fut malade à la mort;
 et le prophète Ésaïe, fils d'Amots, vint vers lui, et lui dit~:
 Ainsi a dit le Seigneur~:
 \og Mets ta maison en ordre; car tu vas mourir, et tu ne vivras plus. \fg{}
}{\bibleverse{IIKgs}(20:1) \NKJF}

\dvlettrine{I}{l y a tellement de choses} que j'ai l'intention de faire
 avant de mourir ! Il y a des gens que j'ai l'intention d'appeler,
 des cartes que j'ai l'intention d'écrire, des paroles que j'ai l'intention
 de prononcer. Et je sais que si je devais recevoir le même message
 de la part du Seigneur qu'Ézéchias, ma réaction initiale serait très semblable
 à la sienne~: \og Seigneur, peux-Tu attendre un petit moment? \fg{}

\dvbox{
Je veux que ma maison soit en ordre
 avant que le Seigneur ne me rappelle à la maison. 
}

Le jour où je comparaîtrai devant Dieu, je veux pouvoir dire avec Paul~:
 \og J'ai combattu le bon combat, j'ai achevé la course,
 j'ai gardé la foi \fg{} (\bibleverse{IITim}(4:7)). 

D'ici peu, nous comparaîtrons tous devant Dieu. Quand cela arrivera,
 vous serez soit coupables dans vos péchés, soit innocents en Christ.
 Votre maison spirituelle est-elle en ordre? Êtes-vous prêts?
 Si ce n'est pas le cas, vous pouvez mettre votre maison en ordre
 en établissant une juste relation avec Dieu, et par Son intermédiaire,
 une juste relation avec votre famille et vos amis. 

Rappelez-vous bien que demain n'est pas garanti. Je prie pour que quand Dieu
 nous rappellera, nous ne laissions pas derrière nous des choses à régler. 

\dvrule

\dvprayer{
Père, aide-nous à ajuster nos vies et nos agendas afin
 que nous accomplissions les choses qui sont les plus importantes à Tes yeux.
 Puissions-nous prononcer cette parole d'amour ou cette parole d'excuse
 que quelqu'un d'autre a besoin d'entendre.
}{\DlNdJ}


%%%%%%%%%%%%%%
% 22 mars
%%%%%%%%%%%%%%

\dvday{La Prière de Jabez}

\dvquote{
Et Jabez appela le Dieu d’Israël, disant~:
 \og Oh si Tu voulais indubitablement me bénir
 et agrandir mon territoire; et que Ta main puisse être avec moi,
 et si Tu voulais me préserver du mal,
 en sorte que je ne sois pas affligé! \fg{}
 Et Dieu lui accorda ce qu’il avait demandé.
}{\bibleverse{ICh}(4:10) \KJF}

\dvlettrine{L}{a prière de Jabez} est mal comprise par beaucoup de gens.
 Ils pensent que la prière est un moyen d'obtenir ce que nous voulons.
 Dieu n'existe pas pour notre plaisir, Il n'existe pas non plus pour faire
 nos quatre volontés.
 Quand mon c\oe{}ur est en harmonie avec celui de Dieu,
 mes prières ne vont pas se préoccuper de mes désirs mais des Siens. 

\dvbox{
Il s'agit avec la prière, d'obtenir que la volonté de Dieu soit accomplie,
 non pas la nôtre. 
}

Jabez comprenait que Dieu voulait le bénir. Dieu veut aussi vous bénir,
 mais il y a des choses que vous devez faire pour recevoir ces bénédictions.
 Vous devez sérieusement prendre en compte la volonté de Dieu.
 Vous devez marcher dans les voies de Dieu.
 Vous devez faire les choses que Dieu vous a commandées de faire. 

Si comme Israël, vous ignorez la voix de Dieu, marchez dans vos propres voies
 et désobéissez à Ses commandements, vous vous retirez vous-mêmes
 de la position où vous pouvez être bénis. Mais si seulement
 vous écoutez Sa voix, si vous marchez dans Ses voies,
 si vous observez les choses qu'Il vous a commandées de faire, alors,
 les bénédictions vous appartiendront. 

\dvrule

\dvprayer{
Père, aide-nous à vraiment comprendre le but de la prière,
 et puissions-nous ne pas chercher à l'utiliser comme un moyen de gratifier
 nos propres désirs. Nous prions pour que nous marchions dans Tes voies,
 que nous observions les choses que Tu nous a commandées
 afin de pouvoir connaître Tes bénédictions sur nos vies.
}{\DlNdJ}


%%%%%%%%%%%%%%
% 23 mars
%%%%%%%%%%%%%%

\dvday{Un Cœur Résolu}

\dvquote{
\dots{} \numprint{50000}, en état d'aller à l'armée,
 munis pour le combat de toutes les armes de guerre
 et prêts à se ranger au combat d'un cœur résolu.
}{\bibleverse{ICh}(12:34)}

\dvlettrine{L}{es hommes qui avaient rejoint David}
 étaient d'habiles guerriers, mais leur force véritable
 tenait au fait qu'ils avaient des c\oe{}urs résolus. 

David a prié~:
 \og Unifie mon c\oe{}ur \fg{} (\bibleverse{Ps}(86:11) \NBS).
 Il connaissait la tendance du c\oe{}ur à dire~:
 \og Oui, j'aime le Seigneur. Mais j'ai aussi cette attirance
 envers les choses du monde. \fg{}
 Quand, seule, une partie de mon c\oe{}ur désire servir le Seigneur,
 cela signifie qu'une autre partie de mon c\oe{}ur désire servir la chair. 

\dvbox{
Un c\oe{}ur divisé entrave notre service pour le Seigneur. 
}

Beaucoup de gens veulent être comptés comme des serviteurs de Dieu,
 mais ils ont aussi à c\oe{}ur les choses du monde.
 Ils sont pris par des activités du monde.
 Ils ont donné au Seigneur une place dans leur c\oe{}ur,
 mais ils ne lui ont pas complètement donné leurs c\oe{}urs.
 Ils sont attirés par l'Esprit vers les choses du Seigneur,
 mais ils sont aussi attirés par leur chair vers les choses du monde. 

Vous les trouverez à l'église la plupart des dimanches,
 sauf si c'est la finale de la Coupe du Monde. Le reste de la semaine,
 il y a peu de place pour Dieu dans leurs vies,
 et ils communiquent rarement avec Lui. 

Votre c\oe{}ur est-il divisé ou déchiré?
 Suivez-vous passionnément Jésus ou les choses du monde?
 Est-Il vraiment le Seigneur et Maître de votre vie?
 S'Il ne l'est pas, demandez de l'aide. Demandez à Dieu de vous donner
 une consécration sans réserve à son égard. 

\dvrule

\dvprayer{
Seigneur, défie nos c\oe{}urs, aujourd'hui, de vraiment s'engager,
 afin que nous n'aimions pas en paroles seulement mais vraiment
 et concrètement. Aide-nous à Te donner Ta juste place dans nos c\oe{}urs.
}{\DlNdJ}


%%%%%%%%%%%%%%
% 24 mars
%%%%%%%%%%%%%%

\dvday{Invoquez l'Éternel}

\dvquote{
Invoquez son nom\dots{}
}{\bibleverse{ICh}(16:8)}

\dvlettrine{Q}{uel est le Nom} de notre Créateur?
 \og Dieu \fg{} n'est pas Son nom; c'est Son titre.
 Quand le nom hébreu pour Dieu est traduit en français,
 il est parfois traduit par \og Seigneur \fg{}.
 Mais \og Seigneur \fg{} est aussi un titre,
 ce qui fait que les gens sont un peu perdus.
 Ils pensent que \og Seigneur \fg{} est son titre
 et que \og Dieu \fg{} est Son Nom.
 Mais en fait Son nom est Yahvé ou Jéhovah. 

Le nom Jéhovah est un verbe hébreu qui signifie \og être \fg{}
 ou littéralement \og Le devenant \fg{}.
 C'est le nom par lequel Dieu a choisi de se révéler Lui-même à nous,
 le \og Je suis \fg{}, le \og Devenant \fg{} car Dieu devient pour vous
 quoi que ce soit dont vous ayez besoin.
 Je suis Celui qui vous guérit \ocadr Jéhovah-rapha.
 Je suis Celui qui pourvoit \ocadr Jéhovah-jireh.
 Je suis votre paix \ocadr Jéhovah-shalom.
 Je suis votre salut \ocadr Jéhovah-shua. 

Le nom Jéhovah-shua est devenu en raccourci Jéhoshua,
 puis en l'abrégeant davantage encore, Yashua.
 En grec, Yashua est \og Jésus \fg{}.
 Ainsi, Jésus est le nom de notre Dieu. Il est Jéhovah-shua.
 Il est celui qui nous a sauvés de nos péchés. Il est devenu notre Sauveur.

\dvbox{
Dieu Lui a donné un nom qui est au-dessus de tout nom,
 afin qu'au nom de Jésus, Yashua, tout genou fléchisse
 et que toute langue confesse qu'Il est Seigneur. 
}

Invoquez Son nom \ocadr mais pas seulement dans les moments de besoin.
 Invoquez Le continuellement, jour après jour, parce qu'Il est Seigneur
 de votre vie, et parce que vous voulez faire Sa volonté, et par dessus
 tout, parce que vous L'aimez.

\dvrule

\dvprayer{
Père, nous sommes si reconnaissants que Tu nous aies appelés
 à cette relation d'alliance avec Toi.
 Puissions-nous glorifier Ton nom précieux.
}{\Amen}


%%%%%%%%%%%%%%
% 25 mars
%%%%%%%%%%%%%%

\dvday{Fais de Ton Mieux et Confie le Reste à Dieu}

\dvquote{
Sois fort, fortifions-nous pour notre peuple
 et pour les villes de notre Dieu,
 et que l'Éternel fasse ce qui lui semblera bon!
}{\bibleverse{ICh}(19:13)}

\dvlettrine{A}{lors que Joab} et les armées d'Israël se préparaient
 à combattre les Ammonites, les Syriens arrivèrent derrière eux.
 Joab se rendit compte qu'ils étaient tombés dans un piège.
 Avec des ennemis devant eux et des ennemis derrière eux,
 Joab décida que lui et son frère allaient diviser
 l'armée en deux régiments. 

Puis Joab prononça son exhortation à l'attention d'Abishaï.
 Ce qu'il disait, en essence, c'était~:
 \og Sois courageux et nous allons laisser les conséquences au Seigneur.
 Nous allons faire de notre mieux et puis confier le résultat à Dieu. \fg{}

\dvbox{
Dieu n'exige pas que je fasse ce qui est le mieux,
 seulement que je fasse de mon mieux. 
}

Nous nous excusons souvent de ne pas servir le Seigneur
 en invoquant nos insuffisances. Nous quittons le combat parce que nous savons
 que d'autres sont plus qualifiés ou plus doués
 et pourraient mieux faire le travail.
 Mais Dieu n'appelle pas toujours la personne la plus qualifiée
 pour faire Son travail. Il ne cherche pas toujours les capacités,
 seulement la disponibilité. 

Quand nous nous trouvons confrontés à des problèmes
 qui semblent insurmontables, nous devons nous rappeler des paroles de Joab
 à son jeune frère~:
 \og Sois fort, fortifions-nous pour notre peuple
 et pour les villes de notre Dieu,
 et que l'Éternel fasse ce qui lui semblera bon! \fg{}

Fais de Ton Mieux et Confie le Résultat à Dieu. 

\dvrule

\dvprayer{
Seigneur, pardonne-nous pour les nombreuses fois où nous avons capitulé
 sans même engager la bataille.
 Puissions-nous répondre à Ton amour en faisant de notre mieux pour Toi.
}{\Amen}


%%%%%%%%%%%%%%
% 26 mars
%%%%%%%%%%%%%%

\dvday{Que l'Éternel Soit Avec Toi !}

\dvquote{
Maintenant, mon fils, que l'Éternel soit avec toi,
 afin que tu aies du succès et que tu bâtisses
 la maison de l'Éternel, ton Dieu,
 comme Il l'a déclaré à ton égard !
}{\bibleverse{ICh}(22:11)}

\dvlettrine{A}{lors que Salomon} entreprend la construction
 de ce magnifique temple pour le Seigneur, son père vient l'encourager.
 Et la première chose que David lui dit est~:
 \og Que l'Éternel soit avec toi. \fg{} C'est très significatif. 

Au cours des temps, l'église a entrepris de faire l'\oe{}uvre de Dieu
 avec la sagesse de l'homme. Le désir de l'homme d'être indépendant de Dieu
 est un désir courant, mais il ne devrait jamais s'exercer
 à l'intérieur de l'église.
 Nous avons besoin que le Seigneur soit avec nous dans chacun de nos projets.
 Il ne suffit pas de savoir ce que Dieu veut voir accomplir.
 Il est important de savoir comment Dieu veut le faire accomplir. 

\dvbox{
Aucun travail entrepris n'aura la moindre valeur
 si le Seigneur n'est pas avec nous.
}

Je ne sais pas ce que Dieu vous appelle à faire.
 Mais en considérant ce ministère ou cette tâche,
 vous êtes peut-être remplis de désarroi. Peut-être vous dites-vous~:
 \og Je ne pourrais jamais y arriver, Seigneur.
 Tu as certainement choisi la mauvaise personne. \fg{}

Souvenez-vous de Moïse qui disait~:
 \og Seigneur, Je ne peux pas très bien parler. Je bredouille et je bégaye,
 et les gens ne vont pas s'intéresser à ce que j'ai à dire. \fg{}
 Qu'a répondu le Seigneur? \og Je serai avec toi. \fg{}
 C'est la seule chose nécessaire.

La réponse de Dieu est toujours~: \og Je serai avec toi. \fg{}
 Sa présence vous précèdera.
 Et quand Dieu guide, Dieu fournit tout la sagesse, toutes les directions,
 tout ce qui vous sera nécessaire pour accomplir
 la mission qu'il vous met à c\oe{}ur.

\dvrule

\dvprayer{
Seigneur, comme nous Te sommes reconnaissants
 d'être toujours présent auprès de nous. Nous en dépendons.
 Nous savons Seigneur que, sans Toi, nous ne pouvons rien faire.
}{\DlNdJ}


%%%%%%%%%%%%%%
% 27 mars
%%%%%%%%%%%%%%

\dvday{Donner à Dieu}

\dvquote{
Le peuple se réjouit de leurs offrandes volontaires,
 car c'était avec un cœur sans partage qu'ils s'étaient
 portés volontaires pour l'Éternel;
 et le roi David en eut aussi une grande joie.
}{\bibleverse{ICh}(29:9)}

\dvlettrine{L}{es gens offraient et donnaient} de leur plein gré.
 Ce faisant, ils commençaient à se réjouir et à louer Dieu.
 Et c'est la façon dont les choses devraient toujours se passer,
 que vous donniez de votre temps, de votre énergie, de votre argent
 ou de vos possessions.
 Quoi que vous donniez à Dieu, devrait être offert de votre plein gré,
 de bon c\oe{}ur. 

Donner à Dieu devrait être la plus grande joie et le plus grand bonheur
 que vous connaissiez.
 Tout ce que je donne à Dieu doit venir du fond de mon c\oe{}ur
 pour avoir une quelconque valeur. 

\dvbox{
Si vous ne pouvez pas donner avec joie,
 il vaudrait mieux ne pas donner du tout. 
}

Avant sa visite à l'église de Corinthe, Paul a écrit aux fidèles
 en leur demandant de rassembler une collecte pour l'offrir
 aux saint de Jérusalem, qui souffraient d'une véritable pauvreté.
 Et il a ajouté~:
 \og Que chacun donne comme il l'a résolu en son cœur, sans tristesse
 ni contrainte; car Dieu aime celui qui donne avec joie \fg{}
 (\bibleverse{IICor}(9:7)).
 En grec, l'expression rendue par \og avec joie \fg{} signifie
 \og hilare \fg{}. Autrement dit, Dieu apprécie un \og donneur hilare \fg{}.
 Il est béni quand nous donnons de façon \og hilare \fg{},
 c'est-à-dire d'un c\oe{}ur joyeux, comme en riant. 

Puissions-nous ne jamais oublier que tout ce que nous avons vient de Dieu.
 Tout ce que j'ai est en réalité à Lui; ça Lui appartient.
 Mais nous avons le privilège de Lui faire des dons en retour
 \ocadr du fond de notre c\oe{}ur \fcadr{} comme marque de notre amour
 et de notre gratitude. 

\dvrule

\dvprayer{
Père, Merci pour le privilège, la bénédiction et la joie
 de Te faire des dons. Prends nos vies, Seigneur.
 Nous Te les offrons pour que Tu les utilises comme bon Te semblera.
}{\DlNdJ}


%%%%%%%%%%%%%%
% 28 mars
%%%%%%%%%%%%%%

\dvday{Plus Grand Que Tous les Autres Dieux}

\dvquote{
La maison que je vais bâtir doit être grande,
 car notre Dieu est plus grand que tous les dieux.
 Mais qui a le pouvoir de Lui bâtir une maison,
 puisque les cieux des cieux ne peuvent Le contenir?
 Et qui suis-je pour Lui bâtir une maison?
}{\bibleverse{IICh}(2:4-5)} 

\dvlettrine{Q}{uelle est la taille} de l'univers?
 Est-ce que l'espace continue sans s'arrêter jusqu'à l'infini,
 ou y a-t-il un panneau quelque part qui dit~:
 \og Vous êtes arrivés à la fin \fg{} ? 

Quand nous considérons l'immensité de notre univers,
 nous commençons à voir la grandeur du Dieu que nous servons.
 Si nos intelligences limitées ne peuvent pas appréhender
 la grandeur d'un univers fini, comment pouvons-nous espérer
 comprendre le Dieu infini qui a créé l'univers? 

David reconnaissait que les païens se fabriquaient des petits dieux
 à partir de bois, de pierre, d'argent ou d'or \ocadr des dieux avec des yeux
 qui ne pouvaient pas voir, des oreilles qui ne pouvaient pas entendre,
 des bouches qui ne pouvaient pas parler et des pieds
 qui ne pouvaient pas courir. Mais le nôtre est un grand Dieu.
 C'est un Dieu qui aime \ocadr un Dieu vivant. 

\dvbox{
Quand nous considérons qui est Dieu et ce qu'Il a fait pour nous,
 comment pourrions-nous faire moins que Lui donner
 ce que nous avons de meilleur? 
}

Pour cette raison, tout ce que nous faisons pour Dieu
 doit être le meilleur possible.
 Comment pouvons-nous ne pas vivre complètement pour Lui? 

De la plus simple des cellules au grandiose univers,
 notre grand Dieu domine et règne sur toutes choses.
 La question est~: domine-t-Il et règne-t-Il sur votre vie
 \ocadr ou adorez-vous un des ces moindres petits dieux? 

\dvrule

\dvprayer{
Nous Te remercions, Seigneur, pour la bénédiction et le privilège
 de Te connaître et de Te servir. 
}

\manque{Fin de prière?}


%%%%%%%%%%%%%%
% 29 mars
%%%%%%%%%%%%%%

\dvday{Choisissez Votre Chemin Avec Soin}

\dvquote{
Toutefois, ils lui seront asservis, et ils reconnaîtront
 ce que c'est que Me servir ou servir les royaumes
 des autres pays.
}{\bibleverse{IICh}(12:8)}

\dvlettrine{Q}{uand Réhoboam} amena le peuple à abandonner le Seigneur
 et à servir d'autres dieux, la réponse de Dieu fut de dire~:
 \og Vous ne voulez plus Me servir? Très bien!
 Je vais vous laisser voir ce que c'est que de servir les royaumes
 des pays qui vous entourent. Je vais vous laisser découvrir l'esclavage
 avilissant qui résulte de servir d'autres dieux. \fg{}

\dvbox{
Beaucoup de gens aujourd'hui vivent uniquement pour servir leur propre plaisir. 
}

Les gens n'ont pas changé. Vivant pour l'instant présent, ils sont devenus
 esclaves des passions de leur chair. Si seulement ces gens savaient
 où le chemin qu'ils ont choisi va les mener, je crois qu'ils en choisiraient
 un autre. Quand Jésus nous a appelés à Le suivre, Il a dit~:
 \og Mon joug est aisé, et Mon fardeau léger. \fg{}
 Ça ne peut pas être dit du joug qui vient enserrer le cou de la personne
 qui a choisi de vivre pour sa propre chair. 

Dieu ne m'a jamais demandé de faire une seule chose qui me ferait du mal.
 Il ne m'a jamais demandé d'abandonner quelque chose qui m'était bénéfique.
 Les seules choses que Dieu m'a demandé d'abandonner ont été ces choses
 qui auraient fini par me détruire. Et quand je sers Dieu,
 Il n'exige jamais rien de ma part qu'Il ne m'ait déjà donné le pouvoir
 et la capacité d'accomplir. 

Vous choisissez le dieu que vous allez servir. Regardez bien la destination
 du chemin que vous avez emprunté. Mène-t-il à la vie ou à la mort ? 

\dvrule

\dvprayer{
Père, donne-nous cette sagesse qui va nous faire regarder plus loin
 sur la route pour voir où le chemin conduit avant de nous lancer
 dans notre voyage.
}{\DlNdJ}


%%%%%%%%%%%%%%
% 30 mars
%%%%%%%%%%%%%%

\dvday{La Victoire de Dieu}

\dvquote{
Vous n'aurez pas à y combattre~:
 présentez-vous, tenez-vous là, et vous verrez le salut de l'Éternel
 en votre faveur. Juda et Jérusalem, soyez sans crainte et sans effroi~:
 demain, sortez à leur rencontre, et l'Éternel sera avec vous!
}{\bibleverse{IICh}(20:17)}

\dvlettrine{L}{es trois nations à l'est} ont uni leurs forces et envoyé
 leur armée pour détruire Juda. Josaphat sait que Juda n'est pas de taille
 à résister à ces forces combinées.
 Humainement parlant, c'est comme s'ils étaient déjà perdus. 

Tôt ou tard, nous rencontrons tous des situations qui nous dépassent.
 La Bible nous dit qu'en tant que chrétiens,
 nous avons trois adversaires redoutables~: les forces du monde,
 de la chair et du diable. Chacune d'entre elles constitue un réel problème
 pour nous, mais si elles sont combinées, il nous est humainement impossible
 de connaître la victoire. 

\dvbox{
Quand nous restons tranquilles et laissons Dieu agir,
 c'est Lui qui reçoit toute la gloire une fois que la bataille est gagnée. 
}

Quand ces problèmes monumentaux apparaissent, nous souhaiterions souvent
 que Dieu nous donne une formule qui règlerait le problème.
 Mais si nous avions une formule, alors nous voudrions la commercialiser.
 Puis nous irions nous vanter partout d'avoir bien géré la situation. 

Dieu a tant de façons intéressantes de résoudre nos problèmes!
 Comme il est important de nous rappeler de Lui faire appel quand nous sommes
 confrontés à ces dilemmes. Nous devons adresser des prières
 au vrai Dieu vivant, le Créateur des cieux et de la terre.
 Il est capable de faire infiniment au-delà de tout ce que nous demandons
 ou pensons. La bataille, après tout, Lui appartient. 

\dvrule

\dvprayer{
Père, aide-nous aujourd'hui à reconnaître et à confesser nos limites
 afin de ne pouvoir compter que sur Toi. Nous nous réjouissons, Seigneur,
 dans Ta promesse d'aide. 
}{\DlNdJ}


%%%%%%%%%%%%%%
% 31 mars
%%%%%%%%%%%%%%

\dvday{Une Sainte Influence}

\dvquote{
Il marcha dans la voie des rois d'Israël,
 comme avait fait la maison d'Achab,
 car il avait pour femme une fille d'Achab
 et il fit ce qui est mal aux yeux de l'Éternel.
}{\bibleverse{IICh}(21:6)}

\dvlettrine{J}{osaphat fut un bon roi} et il amena le peuple
 à servir le Seigneur. Mais son fils, Yoram, fut un des rois
 les plus pervers de l'histoire de Juda.
 Comment un homme si proche de Dieu a-t-il pu avoir un fils
 aussi mauvais? 

Bien que Josaphat ait été un homme qui cherchait à plaire à Dieu,
 il était curieusement fasciné par le mal.
 Cette curiosité l'a attiré vers le Nord maintes et maintes fois,
 jusqu'à ce qu'il finisse par s'aligner avec le roi Achab.
 Bien que Josaphat lui-même ne se soit jamais impliqué personnellement
 dans les pratiques perverses du roi Achab, il fit l'erreur d'exposer
 son fils à ces pratiques. Et Yoram, qui n'avait pas une consécration
 affermie envers Dieu, est tombé sous l'influence de cette perversité. 

\dvbox{
Beaucoup de gens qui se disent chrétiens semblent aujourd'hui
 avoir la même fascination avec le mal.
}

Même s'ils ne penseraient jamais à faire ces choses-là eux-mêmes,
 ils aiment bien lire les récits de personnes qui se livrent au mal
 et voir d'autres personnes se livrer au mal dans des films
 et à la télévision. Leur logique semble être
 \og simplement regarder ne fait pas de mal. \fg{}

Vous pouvez prétendre que regarder les obscénités du monde n'affecte pas
 votre relation avec Dieu. J'en doute. Mais pensez à ceci~:
 Quel effet cela produit-il sur vos enfants? 

Quand je me retrouverai devant Dieu, je veux me tenir devant Lui
 dans la sainteté, la pureté et la justice de Jésus-Christ.
 Et je veux que mes enfants se tiennent devant Lui avec moi,
 complets en Lui. 

\dvrule

\dvprayer{
Dieu, fais que nos vies soient la lumière du monde parce que
 nous reflétons Ta sainteté, Ta beauté, Ta grâce et Ton amour.
}{\DlNdJ}


