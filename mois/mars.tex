\dvmonth{Mars}

%%%%%%%%%%%%%%
% 1er mars
%%%%%%%%%%%%%%

\dvday{Pratique du Tir à L\ap{}Arc}

\dvquote{
Ainsi [David] ordonna d’enseigner aux enfants de Judah,
 l’usage de l’arc\dots{}
}{\bibleverse{IISam}(1:18) \KJF}

\dvlettrine{Q}{uand} David reçu la nouvelle de la mort
 de son cher ami Jonathan et du Roi Saül qu'il admirait tant,
 il eut beaucoup de chagrin. Mais il ne se contenta pas de rester
 là à pleurer.
 Une des façons dont David réagit à son deuil fut d'ordonner
 à tous les papas de la tribu de Juda d'enseigner le tir à l'arc
 à leurs enfants.

C'était un mémorial particulièrement approprié pour Jonathan.
 David prit un des atouts de la vie de Jonathan et ordonna
 de l'enseigner à tous leurs enfants. De cette façon,
 l'influence de Jonathan perdurerait.
 Nous pouvons spiritualiser le texte en pensant à l'arc
 comme représentant l'arme spirituelle de la prière.

\dvbox{
Tout comme l'arc était capable d'envoyer des flèches et de frapper l'ennemi
 à une grande distance, nos prières sont une arme excellente par laquelle
 nous pouvons spirituellement influencer les gens à distance.
}

Quelquefois quand nous avons tant témoigné de notre foi auprès d'êtres chers,
 qu'ils commencent à ne pas apprécier (et à nous en vouloir),
 nous pouvons tirer des flèches à distance.
 L'Esprit commence à travailler dans leurs cœurs alors que nous,
 par la prière, lions l'œuvre de l'ennemi.

Tout comme David leur ordonna d'enseigner l'usage de l'arc à leurs enfants,
 nous devons apprendre à utiliser l'arc de la prière afin d'être efficaces
 dans le combat spirituel dans lequel nous sommes engagés. 

\dvrule

\dvprayer{
Père, aide-nous à devenir compétents dans l'usage des armes de combat
 que Tu nous a données, afin que nous puissions livrer bataille
 pour le compte de nos bien-aimés. 
}{\Amen}

