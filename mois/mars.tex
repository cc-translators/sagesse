\dvmonth{Mars}

%%%%%%%%%%%%%%
% 1er mars
%%%%%%%%%%%%%%

\dvday{Pratique du Tir à L\ap{}Arc}

\dvquote{
Ainsi [David] ordonna d’enseigner aux enfants de Judah,
 l’usage de l’arc\dots{}
}{\bibleverse{IISam}(1:18) \KJF}

\dvlettrine{Q}{uand} David reçu la nouvelle de la mort
 de son cher ami Jonathan et du Roi Saül qu'il admirait tant,
 il eut beaucoup de chagrin. Mais il ne se contenta pas de rester
 là à pleurer.
 Une des façons dont David réagit à son deuil fut d'ordonner
 à tous les papas de la tribu de Juda d'enseigner le tir à l'arc
 à leurs enfants.

C'était un mémorial particulièrement approprié pour Jonathan.
 David prit un des atouts de la vie de Jonathan et ordonna
 de l'enseigner à tous leurs enfants. De cette façon,
 l'influence de Jonathan perdurerait.
 Nous pouvons spiritualiser le texte en pensant à l'arc
 comme représentant l'arme spirituelle de la prière.

\dvbox{
Tout comme l'arc était capable d'envoyer des flèches et de frapper l'ennemi
 à une grande distance, nos prières sont une arme excellente par laquelle
 nous pouvons spirituellement influencer les gens à distance.
}

Quelquefois quand nous avons tant témoigné de notre foi auprès d'êtres chers,
 qu'ils commencent à ne pas apprécier (et à nous en vouloir),
 nous pouvons tirer des flèches à distance.
 L'Esprit commence à travailler dans leurs cœurs alors que nous,
 par la prière, lions l'œuvre de l'ennemi.

Tout comme David leur ordonna d'enseigner l'usage de l'arc à leurs enfants,
 nous devons apprendre à utiliser l'arc de la prière afin d'être efficaces
 dans le combat spirituel dans lequel nous sommes engagés. 

\dvrule

\dvprayer{
Père, aide-nous à devenir compétents dans l'usage des armes de combat
 que Tu nous a données, afin que nous puissions livrer bataille
 pour le compte de nos bien-aimés. 
}{\Amen}


%%%%%%%%%%%%%%
% 2 mars
%%%%%%%%%%%%%%

\dvday{Qui Suis-Je ?}

\dvquote{
Alors le roi David alla se présenter devant l'Éternel et dit~:
 \og Qui suis-je, Seigneur Éternel, et qu'est-ce que ma maison,
 pour que Tu m'aies fait parvenir jusqu'ici? \fg{}
}{\bibleverse{IISam}(7:18)}

\dvlettrine{C}{'est Nathan,} le prophète, qui rapporta à David
 les paroles de Dieu.
 \og David, tu ne peux bâtir une maison pour Moi, mais Je construirai
 une maison pour toi.
 Ton fils, qui siègera sur le trône après toi, bâtira une maison pour Moi.
 Et J'établirai ton trône pour toujours. \fg{}
 David comprit correctement que cela signifiait
 que le Messie serait son descendant. 

David en fut tellement bouleversé, qu'il pria Nathan de l'excuser
 et alla se recueillir devant Dieu.
 Il reconnut qu'il était totalement indigne d'une telle faveur
 et se rendit compte que c'était la grâce de Dieu qui lui avait été ainsi
 prodiguée dans une telle abondance. 

Comme David, vous vous dites peut-être~:
 \og Qui suis-je? J'ai fait quelque chose de mal. J'ai tout gâché. \fg{}
 Considérez la vie de David. Ce gars n'était pas un enfant de ch\oe{}ur.
 Il a fait bien pire que vous; cependant regardez ce que Dieu a fait pour lui.

\dvbox{
Il semble parfois que Dieu recherche le candidat le moins probable
 pour Lui accorder Son amour et Sa bonté. 
}

Si souvent, Dieu va dans le caniveau pour chercher le destinataire de Sa grâce.
 Il le relève, le lave et le transforme \ocadr faisant de lui un enfant de Dieu
 prêt pour Son royaume. Ça, c'est la grâce de Dieu. 

\dvrule

\dvprayer{
Merci, Père, pour Ta grâce abondante. En dépit de toutes nos carences,
 Tu nous as aimés et Tu as commencé à nous couvrir de riches bénédictions.
 Mais, comme si cela n'était pas encore suffisant,
 Tu as parlé de l'éternité à venir et de la gloire qui nous sera révélée.
}{\DlNdJ}


%%%%%%%%%%%%%%
% 3 mars
%%%%%%%%%%%%%%

\dvday{Le Dieu Qui Cherche}

\dvquote{
David lui dit~:
 \og Sois sans crainte !
 Pour sûr, je vais user de bienveillance envers toi
 à cause de ton père Jonathan.
 Je te rendrai toutes les terres de ton père Saül,
 et tu mangeras toujours à ma table. \fg{}
}{\bibleverse{IISam}(9:7)}

\dvlettrine{P}{lusieurs années auparavant,} David avait fait
 une alliance avec Jonathan, le fils de Saül.
 David avait promis, par égard pour Jonathan,
 que quand il accèderait au trône, il traiterait la famille de Saül
 avec une grande gentillesse.
 Aussi, une fois que David est arrivé sur le trône,
 qu'il a été béni et que Dieu l'a fait prospérer,
 David a recherché un des descendants de Saül
 pour lui montrer de la bienveillance. 

De même que David cherchait un descendant pour lui montrer
 de la bienveillance, Dieu recherche les perdus afin de montrer
 la richesse surabondante de Son amour et de Sa bonté envers eux
 en Jésus-Christ. C'est ce qui rend le Christianisme différent des religions.
 Dans les religions les hommes recherchent Dieu.
 Dans le Christianisme, Dieu recherche les hommes perdus. 

Mephibocheth fut en fait effrayé quand on l'amena devant David,
 parce qu'il ne connaissait pas les intentions de David.
 Si souvent, nous nous méprenons sur Dieu et nous imaginons
 qu'Il est en colère contre nous ou dégoûté de nous.
 Nous pensons qu'Il nous cherche afin de nous réprimander très sévèrement.
 Mais nous avons tort. 

David voulait bénir Mephibocheth par égard pour Jonathan.
 Dieu veut nous bénir par égard pour Christ. 

\dvbox{
Ce n'est pas parce que nous le méritons
 \ocadr c'est parce que Jésus nous désire en héritage. 
}

Tout comme Mephibocheth dînait à la table du roi
 comme un membre de la famille du roi,
 nous aussi avons été introduits dans la famille de Dieu
 et invités à venir nous asseoir à Sa table. 

\dvrule

\dvprayer{
Merci, Père, pour la joie, la bénédiction,
 la richesse dont nous faisons l'expérience
 à Ta table grâce à Jésus.
}{\DlNdJ}


%%%%%%%%%%%%%%
% 4 mars
%%%%%%%%%%%%%%

\dvday{Confession et Pardon}

\dvquote{
David dit à Nathan~:
 \og J'ai péché contre l'Éternel! \fg{}
 Et Nathan dit à David~:
 \og L'Éternel pardonne ton péché, tu ne mourras pas. \fg{}
}{\bibleverse{IISam}(12:13)}

\dvlettrine{L}{a confession} de péché de David
 était une vraie confession.
 Il avait effectivement péché contre le Seigneur.
 Sa tentative de dissimuler son péché
 n'avait que compliqué les choses. 

Mais notez bien qu'avec la confession de David,
 est arrivé un pardon immédiat.
 Oh, si seulement nous pouvions cesser d'essayer de dissimuler
 ou de justifier nos péchés!
 Le pardon ne peut venir qu'avec la confession.
 Ce n'est pas avant d'avoir reconnu et confessé votre culpabilité
 devant Dieu que vous pouvez recevoir la purification
 que Dieu veut vous donner. 

\dvbox{
Dieu prend plaisir à la miséricorde. 
}

Il veut pardonner.
 Mais tant qu'il n'y a pas eu de confession,
 Dieu n'a pas l'opportunité d'exercer Sa miséricorde et Sa grâce. 

Quel soulagement quand ce lourd fardeau de péché s'en va.
 Le mot hébreu \og béni \fg{} signifie \og oh, heureux! \fg{}.
 C'est ce que David décrit dans le \bibleverse{Ps}(32:)
 quand il écrit \og Béni fg{} ou \og Heureux \fg{} est l'homme
 dont le péché est pardonné.
 Combien est heureux l'homme qui a été restauré
 dans sa communion avec Dieu. 

Il est important de noter que bien que Dieu ait pardonné David,
 le péché a quand même bien laissé son empreinte.
 N'allez pas croire que vous pouvez pécher sans que cela ne laisse
 de marques dans votre vie.
 Dieu pardonne et la punition du péché est retirée.
 Des clous plantés dans une planche peuvent aussi être retirés,
 mais la planche restera toujours marquée par des trous.
 Votre péché peut être enlevé, mais il laisse sa marque. 

\dvrule

\dvprayer{
Père, nous prions pour que Ton Saint Esprit nous parle,
 nous qui croulons sous la culpabilité.
 Puisse-t-il y avoir de la confession, afin qu'il puisse
 aussi y avoir cette effusion de Ta grâce et de Ton pardon.
}{\Amen}


%%%%%%%%%%%%%%
% 5 mars
%%%%%%%%%%%%%%

\dvday{Ne Plus Être Bannis}

\dvquote{
\og Il nous faut certainement mourir.
 Comme des eaux répandues à terre ne se rassemblent plus,
 Dieu ne relève pas un mort, mais dans sa pensée,
 il ne faut pas que celui qui est banni loin de Lui
 le reste. \fg{}
}{\bibleverse{IISam}(14:14)}

\dvlettrine{L}{a femme} qui parlait à David était venue supplier
 le roi de se réconcilier avec son fils, Absalom, qu'il avait banni.
 Et par ces mots qui peignent une illustration imagée, elle rappelle à David
 la brièveté de la vie. Nous devons tous mourir.
 Quand cela nous arrive, nous sommes comme de l'eau qui a été répandue
 sur le sol et que l'on ne peut plus ramasser. 

Elle met en garde le roi que si l'amertume subsiste, et que lui ou son fils
 vienne à mourir, ils ne seront plus jamais en position
 de réparer leur relation.
 La mort met fin aux opportunités de réconciliation. 

Les familles se divisent souvent à propos de choses futiles.
 Et souvent, nous traitons plus durement ceux que nous aimons
 que ceux que nous ne connaissons même pas. 

N'est-il pas intéressant qu'il nous semble que certaines personnes
 soient plus pardonnables que d'autres?
 Mais Dieu va pardonner même les crimes les plus odieux.
 Dieu offre Son pardon et Sa grâce gratuitement. 

\dvbox{
Les portes du pardon sont ouvertes à tous ceux qui l'invoquent. 
}

Dieu a en effet conçu un moyen par lequel \og Ses bannis \fg{}
 \NdT{c'est-à-dire ceux qui ont été bannis de Sa présence}
 ne restent pas séparés de Lui.
 C'est précisément le sujet dont l'évangile traite \ocadr la restauration.
 Par l'intermédiaire de Son Fils, qui a pris notre culpabilité et notre honte
 sur Lui-même, Dieu a restauré ceux qui avaient été bannis en conséquence
 de leur péché. 

\dvrule

\dvprayer{
Père, Je prie pour ceux qui ont été séparés de Ton amour et de Ta communion
 par leurs actions et leurs péchés.
 Puissent-ils se tourner vers Toi et recevoir miséricorde, grâce et pardon.
}{\DlNdJ}


%%%%%%%%%%%%%%
% 6 mars
%%%%%%%%%%%%%%

\dvday{Engagement}

\dvquote{
Ittaï répondit au roi en ces mots~:
 \og L'Éternel est vivant et mon seigneur le roi est vivant!
 A l'endroit où sera mon seigneur le roi, soit pour mourir,
 soit pour vivre, là aussi sera ton serviteur. \fg{}
}{\bibleverse{IISam}(15:21)}

\dvlettrine{I}{ttaï} s'engagea totalement envers David,
 bien qu'il ne soit arrivé de Gath que la veille.
 Beaucoup des amis fidèles de David désertaient le navire
 et rejoignaient Absalom, et voici que cet étranger arrive
 et promet une loyauté indéfectible à David !
 David ne faisait aucune espèce d'offres particulièrement attrayantes
 à Ittaï, mais cet homme s'était figuré que le pire que David avait
 à offrir valait encore mieux que le meilleur qu'Absalom avait à offrir.

L'une des tragédies de notre époque est l'absence d'engagement véritable. 

\dvbox{
Les gens sont réticents à s'engager complètement pour quoi que ce soit. 
}

Dans les v\oe{}ux de mariage, nous pouvons affirmer
 \og pour le meilleur et pour le pire \fg{}, mais dès que les choses
 commencent à ne plus aller, les gens veulent tout laisser tomber.
 L'engagement qu'ils avaient pris n'était pas un engagement total. 

L'église souffre, aujourd'hui, d'une grave faiblesse.
 Les gens acceptent le Seigneur, puis ils se mettent à hésiter.
 Ils montrent qu'il n'y a pas un réel et profond engagement envers Jésus.

Le genre d'engagement qu'Ittaï a pris envers David est le genre d'engagement
 que Jésus désire de notre part aujourd'hui.
 Il n'a pas offert un évangile au rabais ou un chemin facile.
 Mais si vous vous engagez complètement envers Lui, votre vie va être pleine,
 joyeuse et richement satisfaisante \ocadr parce que vous saurez
 que vous êtes exactement à l'endroit voulu par Dieu. 

\dvrule

\dvprayer{
Père, nous nous rendons compte que nous étions à un moment des étrangers
 dans ce monde, exclus de la communion avec Toi.
 Comme nous sommes reconnaissants que Tu nous aies choisis
 pour être Tes disciples et que Tu nous aies amenés à ce point d'engagement
 où nous avons dit~: \og Seigneur, je T'appartiens \fg{} !
}{\DlNdJ}







