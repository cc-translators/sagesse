% We can't use chapters and sections here, because they've been
% hugely redefined in devotional.sty, so let's do things manually

\chapter{Introduction}

\typeout{Date on page \thepage:0:Introduction}

\dvlettrine{L}{e monde dans lequel nous vivons} accorde une grande
 importance au savoir. Mais les gens ne s'attachent pas toujours
 à connaître les choses les plus importantes.
 La vérité, c'est que vous pouvez être un expert en insectes,
 en étoiles ou en sciences. Vous pouvez avoir des connaissances extrêmes
 dans ces domaines et les gens peuvent réellement vous admirer pour ce savoir.
 C'est bien beau, mais si vous n'avez pas la sagesse qui vient de Dieu,
 il vous manque le savoir le plus important qu'un homme puisse espérer atteindre.

Comment acquérir la sagesse de Dieu? Le Livre des Proverbes nous dit que
 \og Le début de la sagesse, c'est la crainte de l'Eternel;
 et connaître Celui qui est Saint, c'est l'intelligence \fg{}
 (\ibiblechvs{Pr}(9:10)).
 La sagesse commence dès que nous nous mettons à révérer Dieu au-dessus
 de toutes autres choses. Et quand ce profond respect nous conduit
 à chercher Dieu au travers de Sa Parole, nous acquérons l'intelligence.

Mais dans cette quête de la sagesse, Dieu ne nous laisse pas tous seuls.
 Il a prévu quelque chose pour nous aider. En cela, c'est bien un Père
 merveilleusement bon, qui nous aide toujours dans nos faiblesses et fournit
 toujours ce qui nous manque. Il désire tellement que nous atteignons cette
 sagesse, qu'Il nous a envoyé une bénédiction spéciale pour nous assister
 dans cette quête. 

Songeant à Sa propre mort, Jésus prépara Ses disciples en leur promettant
 qu'Il ne les laisserait pas seuls au monde comme des orphelins,
 mais qu'Il leur enverrait un Consolateur et un Guide.
 \og Et Moi, je prierai le Père, et Il vous donnera un autre Consolateur
 Qui soit éternellement avec vous, l'Esprit de vérité, que le monde
 ne peut pas recevoir, parce qu'il ne Le voit pas et ne Le connaît pas;
 mais vous, vous Le connaissez, parce qu'Il demeure près de vous
 et qu'Il sera en vous \fg{} (\ibibleverse{Jn}(14:6-17)).
 Puis Jésus nomma le Consolateur et fit cette promesse Le concernant~:
 \og Mais le Consolateur, le Saint-Esprit que le Père enverra en mon nom,
 c'est Lui qui vous enseignera toutes choses et vous rappellera tout ce que
 Moi Je vous ai dit \fg{} (\ibibleverse{Jn}(14:26)).

Quelle merveilleuse promesse nous a été ainsi faite! Dieu a placé Son Esprit
 Saint en nous. Il va demeurer en nous pour toujours \ocadr et Il va nous
 apprendre tout ce que nous devons savoir. En fait, Jean écrit dans un autre
 passage~:
 \og Vous-mêmes, vous avez une onction de la part de Celui qui est saint,
 et tous, vous avez la connaissance \fg{} (\ibibleverse{IJn}(2:20)).
 Le Saint-Esprit, qui demeure en nous, révèle à nos c\oe{}urs les vérités de Dieu.
 Quand nous lisons la Parole de Dieu, Il la rend vivante.
 C'est peut-être un verset que vous avez déjà lu des douzaines de fois avant,
 mais au moment où vous en avez le plus besoin, le Saint-Esprit lui donne vie
 et le fait, pour ainsi dire, sauter de la page.
 
Quelle bénédiction est pour nous le Saint-Esprit!
 Comme Il nous fait du bien! Son \oe{}uvre dans nos vies individuelles et dans
 l'Église est tellement importante! Mais tragiquement, les hommes ont essayé de
 substituer la sagesse des hommes à la sagesse du Saint-Esprit.
 Nous nous sommes convaincus nous-mêmes que la sagesse spirituelle ne peut
 s'atteindre qu'après avoir étudié le grec et l'hébreu. Nous pensons que ce qui
 qualifie un homme pour instruire les gens dans la Parole de vérité est
 un diplôme obtenu dans un séminaire.
 Mais je ne peux pas être d'accord avec cette théorie.
 Ce n'est pas un diplôme encadré sur un mur qui qualifie un homme,
 c'est le Saint Esprit dans le c\oe{}ur.
 Je préfère écouter un homme sans éducation particulière mais
 qui est rempli de l'Esprit Saint plutôt qu'un quelconque docteur
 en théologie qui n'est pas rempli de l'Esprit,
 et qui n'aborde la Bible que comme une \oe{}uvre littéraire intéressante
 sur le sujet des valeurs morales. 

Au cours des années, j'ai rencontré beaucoup de gens que le monde
 qualifierait de simples voire d'ignorants, mais qui, en réalité,
 possédaient une grande sagesse en raison de leur relation avec le Seigneur.
 Je pense en particulier à une femme que Kay et moi avons rencontrée
 voilà des années alors que nous étions en vacances à Bass Lake.
 La ville la plus proche du lac s'appelait North Fork, et quand le dimanche
 est arrivé et que nous avons voulu assister à un culte,
 nous nous sommes rendus dans cette petite ville. Il s'est trouvé que ce
 dimanche-là, le pasteur était en vacances et avait demandé
 à une femme de la région des collines du Kentucky à \grammar{de prêcher}
 prêcher en son absence.
 Je dois vous dire, que quand cette femme a pris la parole, on aurait dit
 que sa bouche était remplie de gravier. Elle massacrait la langue anglaise
 avec son dialecte de péquenaud des collines.
 Mais l'Esprit Saint agissait en elle avec une telle puissance dynamique
 que Kay et moi furent extrêmement bénis ce matin-là.

Et puis il y a eu une chère sainte de Dieu que nous avons connu quand
 j'étais pasteur de l'église de Huntington Beach.
 C'était une petite femme du genre bonne grand mère qui semblait
 toujours discerner les moments où j'étais découragé
 pour une raison quelconque. Quelquefois, je regardais de trop près à la
 condition du monde et à toutes les choses mauvaises qui s'y passaient,
 et je me laissais démoraliser par ces pensées.
 Dieu envoyait alors cette petite femme dans ma direction,
 et elle me disait~: \og Rappelle-toi, Charles, que le Seigneur est toujours
 sur le trône. \fg{}
 Rien que cette simple phrase mais, oh, la sagesse qu'elle renfermait!
 Peu importait comment le monde devenait obscur ou comment les choses
 paraissaient désespérées, cette seule phrase avait le pouvoir d'éclaircir
 ma vision et de chasser le découragement.
 \og Le Seigneur est sur le trône. \fg{}

Je prie pour que vous deveniez sages dans toutes les choses qui importent
 le plus. Et je prie pour que vous vous rappeliez l'onction que vous avez
 reçue de Dieu. Chérissez cette onction. Chérissez le ministère du
 Saint-Esprit dans votre c\oe{}ur alors que vous étudiez la Parole
 et que vous vous rapprochez du Seigneur.

La Sagesse est à votre portée. Dieu désire vous la donner. Honorez Le,
 méditez Sa Parole et obéissez à tout ce qu'Il vous demande de faire.
 Et en ouvrant les pages de ce livre, invitez le Saint-Esprit à vous guider
 chaque jour dans toute la vérité.
 C'est la\dots{} \emph{Sagesse pour Aujourd'hui}.

\signature{Chuck Smith}
\typeout{Date on page \thepage:0:Introduction}


