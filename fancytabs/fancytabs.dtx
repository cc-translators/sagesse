% \iffalse meta-comment
%
% Copyright (C) 2011 by Raphaël Pinson
%
% This file may be distributed and/or modified under the
% conditions of the LaTeX Project Public License, either
% version 1.2 of this license or (at your option) any later
% version. The latest version of this license is in:
%
% http://www.latex-project.org/lppl.txt
%
% and version 1.2 or later is part of all distributions of
% LaTeX version 1999/12/01 or later.
%
% \fi
%
% \iffalse
%<package>\NeedsTeXFormat{LaTeX2e}[1999/12/01]
%<package>\ProvidesPackage{fancytabs}
%<package>   [2011/04/14 0.1 Fancy page border tabs]
%
%<*driver>
\documentclass{ltxdoc}
\EnableCrossrefs
\CodelineIndex
\RecordChanges
\OnlyDescription
\begin{document}
 \DocInput{fancytabs.dtx}
\end{document}
%</driver>
% \fi
%
% \CheckSum{0}
%
% \changes{v1.0}{2011/04/14}{Initial version}
%
% \GetFileInfo{fancytabs.dtx}
%
% \title{The \textsf{fancytabs} package\thanks{This document
%  corresponds to \textsf{fancytabs}~\fileversion,
%  dated~\filedate.}}
% \author{Rapha\"el Pinson \\ \texttt{raphink@gmail.com}}
%
% \maketitle
% 
% \begin{abstract}
%  The \textsf{fancytabs} package allows to insert tabs
%  on the border of pages.
% \end{abstract}
%
% \section{Usage}
%
% \subsection{Adding a tab}
%
% \DescribeMacro{\fancytab}
%  The |\fancytab| macro adds a new tab.
%  It takes two arguments:
%  |\fancytab{<text>}{<position>}|
%  and displays the text in the tab, on given vertical position.
%
% \subsection{Adding a tab to all odd pages with scrpage2}
%
% The following code adds a tab with the chapter name, positioned by chapter order, on each odd page.
%
% \begin{verbatim}
% \documentclass{scrbook}
% \usepackage{scrpage2}
% \usepackage{fancytabs}
% \usepackage{lipsum}
% 
% % Capture chapter head
% \let\oldchapter\chapter
% \newcommand\temphead{}
% \newcommand\chaphead{}
% \renewcommand\chapter[2][\temphead]{%
%    \renewcommand\temphead{#2}%
%    \renewcommand\chaphead{#2}%
%    \oldchapter[#1]{#2}}
%
% % Set central right header to display tab
% % for both scrplain and scrheadings
% \cohead[\fancytab{\chaphead}{\thechapter}]
%        {\fancytab{\chaphead}{\thechapter}}
% \renewcommand*{\chapterpagestyle}{scrplain}
% \pagestyle{scrheadings}
% 
% \begin{document}
% \chapter{My sample chapter}
% 
% \lipsum
% \lipsum
%
% \end{document}
% \end{verbatim}
%
% \subsection{Parameters}
%
% \DescribeMacro{\tabstyle}
%  The style of for the text in the tab.
%  Default is |\Large\scshape|.
%  You can set this value with:
%  |\renewcommand{\tabstyle}{<value>}|.
%
% \DescribeMacro{\tabheight}
%  Define the height of the tab.
%  Default is |4cm|.
%  You can set this value with
%  |\renewcommand{\tabheight}{<value>}|.
%
% \DescribeMacro{\tabwidth}
%  Define the height of the tab.
%  Default is |1cm|.
%  You can set this value with
%  |\renewcommand{\tabwidth}{<value>}|.
%
% \DescribeMacro{\tabcount}
%  Define the maximum amount of tabs on a page.
%  After |\tabcount| tabs are displayed, the next tab is displayed on top of the page again.
%  Default is |6|.
%  You can set this value with
%  |\renewcommand{\tabcount}{<value>}|.
%
% \DescribeMacro{\tableftcolor}
%  Define the left color for the tab gradient.
%  Default is |white|.
%  You can set this value with
%  |\renewcommand{\tableftcolor}{<value>}|.
%
% \DescribeMacro{\tabrightcolor}
%  Define the right color for the tab gradient.
%  Default is |gray!50|.
%  You can set this value with
%  |\renewcommand{\tabrightcolor}{<value>}|.
%
% \DescribeMacro{\tabtop}
%  Define margin on top of the top tab.
%  Default is |\tabheight|.
%  You can set this value with
%  |\renewcommand{\tabtop}{<value>}|.
%
% \DescribeMacro{\tabtextvpos}
%  Define the relative vertical position
%  of the text in the tab.
%  Default is |0.5|.
%  You can set this value with
%  |\renewcommand{\tabtextvpos}{<value>}|.
%
% \DescribeMacro{\tabtexthpos}
%  Define the relative horizontal position
%  of the text in the tab.
%  Default is |\tabtextvpos|.
%  You can set this value with
%  |\renewcommand{\tabtexthpos}{<value>}|.
%
% \DescribeMacro{\tabgap}
%  Define the gap between two tabs.
%  Default is |0in|.
%  You can set this value with
%  |\renewcommand{\tabgap}{<value>}|.
%
% \StopEventually{\PrintIndex}
%
% \section{Implementation}
% 
%  \begin{macrocode}
\RequirePackage{tikz}
\newcommand{\tabstyle}{\Large\scshape}
\newcommand{\tabheight}{4cm}
\newcommand{\tabwidth}{1cm}
\newcommand{\tabcount}{6}
\newcommand{\tableftcolor}{white}
\newcommand{\tabrightcolor}{gray!50}
\newcommand{\tabtop}{\tabheight}
\newcommand{\tabtextvpos}{0.5}
\newcommand{\tabtexthpos}{\tabtextvpos}
\newcommand{\tabgap}{0in}
%  \end{macrocode}
%
% \begin{macro}{\fancytab}
% We define the |\fancytab| macro to generate a new tab.
%  \begin{macrocode}
\newcommand{\fancytab}[2]{%
  \begin{tikzpicture}[remember picture,overlay]
    \node[yshift={-1*(\tabtop+\tabheight/2+mod(#2-1,\tabcount)*(\tabheight+\tabgap))},
                  xshift=-0.5*\tabwidth]
      at (current page.north east) {
      \tikz\shade[shading=axis,bottom color=\tableftcolor,
                  top color=\tabrightcolor,shading angle=-90] 
        (0,0) rectangle (\tabwidth,\tabheight)
           node[rotate=90]
                  at (\tabtexthpos*\tabwidth,\tabtextvpos*\tabheight)
                      {\tabstyle#1};
    };
  \end{tikzpicture}%
}
%  \end{macrocode}
% \end{macro}
%
% \Finale
\endinput


